* Multiple Linear Regression 						:mlr:
:PROPERTIES:
:tangle: R/12-mlr.R
:CUSTOM_ID: cha-multiple-linear-regression
:END:

#+BEGIN_SRC R :exports none :eval never
#    IPSUR: Introduction to Probability and Statistics Using R
#    Copyright (C) 2014  G. Jay Kerns
#
#    Chapter: Multiple Linear Regression
#
#    This file is part of IPSUR.
#
#    IPSUR is free software: you can redistribute it and/or modify
#    it under the terms of the GNU General Public License as published by
#    the Free Software Foundation, either version 3 of the License, or
#    (at your option) any later version.
#
#    IPSUR is distributed in the hope that it will be useful,
#    but WITHOUT ANY WARRANTY; without even the implied warranty of
#    MERCHANTABILITY or FITNESS FOR A PARTICULAR PURPOSE.  See the
#    GNU General Public License for more details.
#
#    You should have received a copy of the GNU General Public License
#    along with IPSUR.  If not, see <http://www.gnu.org/licenses/>.
#+END_SRC

#+BEGIN_SRC R :exports none :eval no-export
# This chapter's package dependencies
library(ggplot2)
library(scatterplot3d)
library(lattice)
#+END_SRC

#+LaTeX: \noindent 
We know a lot about simple linear regression models, and a next step
is to study multiple regression models that have more than one
independent (explanatory) variable. In the discussion that follows we
will assume that we have \(p\) explanatory variables, where \(p > 1\).

The language is phrased in matrix terms -- for two reasons. First, it
is quicker to write and (arguably) more pleasant to read. Second, the
matrix approach will be required for later study of the subject; the
reader might as well be introduced to it now.

Most of the results are stated without proof or with only a cursory
justification. Those yearning for more should consult an advanced text
in linear regression for details, such as /Applied Linear Regression
Models/ \cite{Neter1996} or /Linear Models: Least Squares and
Alternatives/ \cite{Rao1999}.

*What do I want them to know?*
- the basic MLR model, and how it relates to the SLR
- how to estimate the parameters and use those estimates to make
  predictions
- basic strategies to determine whether or not the model is doing a
  good job
- a few thoughts about selected applications of the MLR, such as
  polynomial, interaction, and dummy variable models
- some of the uses of residuals to diagnose problems
- hints about what will be coming later

** The Multiple Linear Regression Model
:PROPERTIES:
:CUSTOM_ID: sec-The-MLR-Model
:END:

The first thing to do is get some better notation. We will write
\begin{equation}
\mathbf{Y}_{\mathrm{n}\times1}=
\begin{bmatrix}y_{1}\\
y_{2}\\
\vdots\\
y_{n}
\end{bmatrix},
\quad \mbox{and}\quad \mathbf{X}_{\mathrm{n}\times(\mathrm{p}+1)}=
\begin{bmatrix}1 & x_{11} & x_{21} & \cdots & x_{p1}\\
1 & x_{12} & x_{22} & \cdots & x_{p2}\\
\vdots & \vdots & \vdots & \ddots & \vdots\\
1 & x_{1n} & x_{2n} & \cdots & x_{pn}
\end{bmatrix}.
\end{equation}
The vector \(\mathbf{Y}\) is called the /response vector/
@@latex:\index{response vector}@@ and the matrix \(\mathbf{X}\) is called the
/model matrix/ @@latex:\index{model matrix}@@. As in Chapter [[#cha-simple-linear-regression]], the most general assumption that relates \(\mathbf{Y}\) to
\(\mathbf{X}\) is
\begin{equation}
\mathbf{Y}=\mu(\mathbf{X})+\upepsilon,
\end{equation}
where \(\mu\) is some function (the /signal/) and \(\upepsilon\) is
the /noise/ (everything else). We usually impose some structure on
\(\mu\) and \(\upepsilon\). In particular, the standard multiple
linear regression model @@latex:\index{model!multiple linear regression}@@
assumes
\begin{equation}
\mathbf{Y}=\mathbf{X}\upbeta+\upepsilon,
\end{equation}
where the parameter vector \(\upbeta\) looks like 
\begin{equation}
\upbeta_{(\mathrm{p}+1)\times1}=\begin{bmatrix}\beta_{0} & \beta_{1} & \cdots & \beta_{p}\end{bmatrix}^{\mathrm{T}},
\end{equation}
and the random vector
\(\upepsilon_{\mathrm{n}\times1}=\begin{bmatrix}\epsilon_{1} &
\epsilon_{2} & \cdots & \epsilon_{n}\end{bmatrix}^{\mathrm{T}}\) is
assumed to be distributed
\begin{equation}
\upepsilon\sim\mathsf{mvnorm}\left(\mathtt{mean}=\mathbf{0}_{\mathrm{n}\times1},\,\mathtt{sigma}=\sigma^{2}\mathbf{I}_{\mathrm{n}\times\mathrm{n}}\right).
\end{equation}

The assumption on \(\upepsilon\) is equivalent to the assumption that
\(\epsilon_{1}\), \(\epsilon_{2}\), ..., \(\epsilon_{n}\) are IID
\(\mathsf{norm}(\mathtt{mean}=0,\,\mathtt{sd}=\sigma)\). It is a
linear model because the quantity
\(\mu(\mathbf{X})=\mathbf{X}\upbeta\) is linear in the parameters
\(\beta_{0}\), \(\beta_{1}\), ..., \(\beta_{p}\). It may be helpful to
see the model in expanded form; the above matrix formulation is
equivalent to the more lengthy
\begin{equation} 
Y_{i}=\beta_{0}+\beta_{1}x_{1i}+\beta_{2}x_{2i}+\cdots+\beta_{p}x_{pi}+\epsilon_{i},\quad i=1,2,\ldots,n.
\end{equation}

# +BEGIN_exampletoo

*Girth, Height, and Volume for Black Cherry trees.* @@latex:\index{Data
sets!trees@\texttt{trees}}@@ Measurements were made of the girth,
height, and volume of timber in 31 felled black cherry trees. Note
that girth is the diameter of the tree (in inches) measured at 4 ft 6
in above the ground. The variables are

1. =Girth=: tree diameter in inches (denoted \(x_{1}\))
2. =Height=: tree height in feet (\(x_{2}\)).
3. =Volume=: volume of the tree in cubic feet. (\(y\))

The data are in the =datasets= package \cite{datasets} and are already
on the search path; they can be viewed with

#+BEGIN_SRC R :exports both :results output pp 
head(trees)
#+END_SRC

#+RESULTS:
:   Girth Height Volume
: 1   8.3     70   10.3
: 2   8.6     65   10.3
: 3   8.8     63   10.2
: 4  10.5     72   16.4
: 5  10.7     81   18.8
: 6  10.8     83   19.7

Let us take a look at a visual display of the data. For multiple
variables, instead of a simple scatterplot we use a scatterplot matrix
which is made with the =splom= function in the =lattice= package
\cite{lattice} as shown below. The plot is shown in Figure
[[fig-splom-trees]].

#+NAME: splom-trees
#+BEGIN_SRC R :exports both :results graphics :file fig/mlr-splom-trees.ps
splom(trees)
#+END_SRC

#+NAME: fig-splom-trees
#+CAPTION[Scatterplot matrix of =trees= data]: \small A scatterplot matrix of the =trees= data.
#+ATTR_LaTeX: :width 0.9\textwidth :placement [ht!]
#+RESULTS: splom-trees
[[file:fig/mlr-splom-trees.ps]]

The dependent (response) variable =Volume= is listed in the first row
of the scatterplot matrix. Moving from left to right, we see an
approximately linear relationship between =Volume= and the independent
(explanatory) variables =Height= and =Girth=. A first guess at a model
for these data might be
\begin{equation}
Y=\beta_{0}+\beta_{1}x_{1}+\beta_{2}x_{2}+\epsilon,
\end{equation}
in which case the quantity
\(\mu(x_{1},x_{2})=\beta_{0}+\beta_{1}x_{1}+\beta_{2}x_{2}\) would
represent the mean value of \(Y\) at the point \((x_{1},x_{2})\).
# +END_exampletoo

*** What does it mean?

The interpretation is simple. The intercept \(\beta_{0}\) represents
the mean =Volume= when all other independent variables are zero. The
parameter \(\beta_{i}\) represents the change in mean =Volume= when
there is a unit increase in \(x_{i}\), while the other independent
variable is held constant. For the =trees= data, \(\beta_{1}\)
represents the change in average =Volume= as =Girth= increases by one
unit when the =Height= is held constant, and \(\beta_{2}\) represents
the change in average =Volume= as =Height= increases by one unit when
the =Girth= is held constant.


In simple linear regression, we had one independent variable and our
linear regression surface was 1D, simply a line. In multiple
regression there are many independent variables and so our linear
regression surface will be many-D... in general, a hyperplane. But
when there are only two explanatory variables the hyperplane is just
an ordinary plane and we can look at it with a 3D scatterplot.

One way to do this is with the \(\mathsf{R}\) Commander in the =Rcmdr=
package \cite{Rcmdr}. It has a 3D scatterplot option under the
=Graphs= menu. It is especially great because the resulting graph is
dynamic; it can be moved around with the mouse, zoomed, /etc/. But
that particular display does not translate well to a printed book.

Another way to do it is with the =scatterplot3d= function in the
=scatterplot3d= package \cite{scatterplot3d}. The code follows, and
the result is shown in Figure [[fig-3D-scatterplot-trees]].

#+NAME: 3D-scatterplot-trees
#+BEGIN_SRC R :exports both :results graphics :file fig/mlr-3D-scatterplot-trees.ps
s3d <- with(trees, scatterplot3d(Girth, Height, Volume, pch = 16, 
                                 highlight.3d = TRUE, angle = 60))
fit <- lm(Volume ~ Girth + Height, data = trees)
#+END_SRC

#+NAME: fig-3D-scatterplot-trees
#+CAPTION[3D scatterplot with regression plane for the =trees= data]: \small A 3D scatterplot with regression plane for the =trees= data.
#+ATTR_LaTeX: :width 0.9\textwidth :placement [ht!]
#+RESULTS: 3D-scatterplot-trees
[[file:fig/mlr-3D-scatterplot-trees.ps]]

Looking at the graph we see that the data points fall close to a plane
in three dimensional space. (The plot looks remarkably good. In the
author's experience it is rare to see points fit so well to the plane
without some additional work.)

** Estimation and Prediction
:PROPERTIES:
:CUSTOM_ID: sec-Estimation-and-Prediction-MLR
:END:

*** Parameter estimates
:PROPERTIES:
:CUSTOM_ID: sub-mlr-parameter-estimates
:END:

We will proceed exactly like we did in Section [[#sec-SLR-Estimation]]. We know
\begin{equation}
\upepsilon\sim\mathsf{mvnorm}\left(\mathtt{mean}=\mathbf{0}_{\mathrm{n}\times1},\,\mathtt{sigma}=\sigma^{2}\mathbf{I}_{\mathrm{n}\times\mathrm{n}}\right),
\end{equation}
which means that \(\mathbf{Y}=\mathbf{X}\upbeta+\upepsilon\) has an \(\mathsf{mvnorm}\left(\mathtt{mean}=\mathbf{X}\upbeta,\,\mathtt{sigma}=\sigma^{2}\mathbf{I}_{\mathrm{n}\times\mathrm{n}}\right)\) distribution. Therefore, the likelihood function @@latex:\index{likelihood function}@@ is
\begin{equation}
L(\upbeta,\sigma)=\frac{1}{2\pi^{n/2}\sigma}\exp\left\{ -\frac{1}{2\sigma^{2}}\left(\mathbf{Y}-\mathbf{X}\upbeta\right)^{\mathrm{T}}\left(\mathbf{Y}-\mathbf{X}\upbeta\right)\right\}.
\end{equation}

To /maximize/ the likelihood @@latex:\index{maximum likelihood}@@ in \(\upbeta\),
we need to /minimize/ the quantity
\(g(\upbeta)=\left(\mathbf{Y}-\mathbf{X}\upbeta\right)^{\mathrm{T}}\left(\mathbf{Y}-\mathbf{X}\upbeta\right)\). We
do this by differentiating \(g\) with respect to \(\upbeta\). (It may
be a good idea to brush up on the material in Appendices [[#sec-Linear-Algebra]]] and [[#sec-Multivariable-Calculus]].) First we will rewrite \(g\):
\begin{equation}
g(\upbeta)=\mathbf{Y}^{\mathrm{T}}\mathbf{Y}-\mathbf{Y}^{\mathrm{T}}\mathbf{X}\upbeta-\upbeta^{\mathrm{T}}\mathbf{X}^{\mathrm{T}}\mathbf{Y}+\upbeta^{\mathrm{T}}\mathbf{X}^{\mathrm{T}}\mathbf{X}\upbeta,
\end{equation}
which can be further simplified to
\(g(\upbeta)=\mathbf{Y}^{\mathrm{T}}\mathbf{Y}-2\upbeta^{\mathrm{T}}\mathbf{X}^{\mathrm{T}}\mathbf{Y}+\upbeta^{\mathrm{T}}\mathbf{X}^{\mathrm{T}}\mathbf{X}\upbeta\)
since \(\upbeta^{\mathrm{T}}\mathbf{X}^{\mathrm{T}}\mathbf{Y}\) is
\(1\times1\) and thus equal to its transpose. Now we differentiate to
get
\begin{equation}
\frac{\partial g}{\partial\upbeta}=\mathbf{0}-2\mathbf{X}^{\mathrm{T}}\mathbf{Y}+2\mathbf{X}^{\mathrm{T}}\mathbf{X}\upbeta,
\end{equation}
since \(\mathbf{X}^{\mathrm{T}}\mathbf{X}\) is symmetric. Setting the
derivative equal to the zero vector yields the so called "normal
equations" @@latex:\index{normal equations}@@
\begin{equation}
\mathbf{X}^{\mathrm{T}}\mathbf{X}\upbeta=\mathbf{X}^{\mathrm{T}}\mathbf{Y}.
\end{equation}

In the case that \(\mathbf{X}^{\mathrm{T}}\mathbf{X}\) is
invertible[fn:fn-invert], we may solve the equation for \(\upbeta\) to
get the maximum likelihood estimator of \(\upbeta\) which we denote by
\(\mathbf{b}\):
\begin{equation}
\label{eq-b-formula-matrix}
\mathbf{b}=\left(\mathbf{X}^{\mathrm{T}}\mathbf{X}\right)^{-1}\mathbf{X}^{\mathrm{T}}\mathbf{Y}.
\end{equation}

[fn:fn-invert] We can find solutions of the normal equations even when
\(\mathbf{X}^{\mathrm{T}}\mathbf{X}\) is not of full rank, but the
topic falls outside the scope of this book. The interested reader can
consult an advanced text such as Rao \cite{Rao1999}.

#+BEGIN_rem
The formula in Equation \eqref{eq-b-formula-matrix} is convenient for
mathematical study but is inconvenient for numerical
computation. Researchers have devised much more efficient algorithms
for the actual calculation of the parameter estimates, and we do not
explore them here.
#+END_rem

#+BEGIN_rem
We have only found a critical value, and have not actually shown that
the critical value is a minimum. We omit the details and refer the
interested reader to \cite{Rao1999}.
#+END_rem

**** How to do it with \(\mathsf{R}\)

We do all of the above just as we would in simple linear
regression. The powerhouse is the =lm= @@latex:\index{lm@\texttt{lm}}@@
function. Everything else is based on it. We separate explanatory
variables in the model formula by a plus sign.

#+BEGIN_SRC R :exports both :results output pp 
trees.lm <- lm(Volume ~ Girth + Height, data = trees)
trees.lm
#+END_SRC

#+RESULTS:
: 
: Call:
: lm(formula = Volume ~ Girth + Height, data = trees)
: 
: Coefficients:
: (Intercept)        Girth       Height  
:    -57.9877       4.7082       0.3393

We see from the output that for the =trees= data our parameter
estimates are \[ \mathbf{b}=\begin{bmatrix}-58.0 & 4.7 &
0.3\end{bmatrix}, \] and consequently our estimate of the mean
response is \(\hat{\mu}\) given by
\begin{alignat}{1} 
\hat{\mu}(x_{1},x_{2}) = & \ b_{0} + b_{1} x_{1} + b_{2}x_{2},\\ \approx & -58.0 + 4.7 x_{1} + 0.3 x_{2}.
\end{alignat} 

We could see the entire model matrix \(\mathbf{X}\) with the
=model.matrix= @@latex:\index{model.matrix@\texttt{model.matrix}}@@
function, but in the interest of brevity we only show the first few
rows.

#+BEGIN_SRC R :exports both :results output pp 
head(model.matrix(trees.lm))
#+END_SRC

#+RESULTS:
:   (Intercept) Girth Height
: 1           1   8.3     70
: 2           1   8.6     65
: 3           1   8.8     63
: 4           1  10.5     72
: 5           1  10.7     81
: 6           1  10.8     83

*** Point Estimates of the Regression Surface
:PROPERTIES:
:CUSTOM_ID: sub-mlr-point-est-regsurface
:END:

The parameter estimates \(\mathbf{b}\) make it easy to find the fitted
values @@latex:\index{fitted values}@@, \(\hat{\mathbf{Y}}\). We write them
individually as \(\hat{Y}_{i}\), \(i=1,2,\ldots,n\), and recall that
they are defined by
\begin{eqnarray}
\hat{Y}_{i} & = & \hat{\mu}(x_{1i},x_{2i}),\\
 & = & b_{0}+b_{1}x_{1i}+b_{2}x_{2i},\quad i=1,2,\ldots,n.
\end{eqnarray}
They are expressed more compactly by the matrix equation
\begin{equation}
\hat{\mathbf{Y}}=\mathbf{X}\mathbf{b}.
\end{equation}
From Equation \eqref{eq-b-formula-matrix} we know that
\(\mathbf{b}=\left(\mathbf{X}^{\mathrm{T}}\mathbf{X}\right)^{-1}\mathbf{X}^{\mathrm{T}}\mathbf{Y}\),
so we can rewrite
\begin{eqnarray}
\hat{\mathbf{Y}} & = & \mathbf{X}\left[\left(\mathbf{X}^{\mathrm{T}}\mathbf{X}\right)^{-1}\mathbf{X}^{\mathrm{T}}\mathbf{Y}\right],\\
 & = & \mathbf{H}\mathbf{Y},
\end{eqnarray}
where
\(\mathbf{H}=\mathbf{X}\left(\mathbf{X}^{\mathrm{T}}\mathbf{X}\right)^{-1}\mathbf{X}^{\mathrm{T}}\)
is appropriately named /the hat matrix/ @@latex:\index{hat matrix}@@ because it
"puts the hat on \(\mathbf{Y}\)". The hat matrix is very important in
later courses. Some facts about \(\mathbf{H}\) are
- \(\mathbf{H}\) is a symmetric square matrix, of dimension
  \(\mathrm{n}\times\mathrm{n}\).
- The diagonal entries \(h_{ii}\) satisfy \(0\leq h_{ii}\leq1\)
  (compare to Equation \eqref{eq-slr-leverage-between}).
- The trace is \(\mathrm{tr}(\mathbf{H})=p\).
- \(\mathbf{H}\) is /idempotent/ (also known as a /projection matrix/)
  which means that \(\mathbf{H}^{2}=\mathbf{H}\). The same is true of
  \(\mathbf{I}-\mathbf{H}\).


Now let us write a column vector
\(\mathbf{x}_{0}=(x_{10},x_{20})^{\mathrm{T}}\) to denote given values
of the explanatory variables =Girth == \(x_{10}\) and =Height ==
\(x_{20}\). These values may match those of the collected data, or
they may be completely new values not observed in the original data
set. We may use the parameter estimates to find
\(\hat{Y}(\mathbf{x}_{0})\), which will give us
1. an estimate of \(\mu(\mathbf{x}_{0})\), the mean value of a future
   observation at \(\mathbf{x}_{0}\), and
2. a prediction for \(Y(\mathbf{x}_{0})\), the actual value of a
   future observation at \(\mathbf{x}_{0}\).

We can represent \(\hat{Y}(\mathbf{x}_{0})\) by the matrix equation
\begin{equation}
\label{eq-mlr-single-yhat-matrix}
\hat{Y}(\mathbf{x}_{0})=\mathbf{x}_{0}^{\mathrm{T}}\mathbf{b},
\end{equation}
which is just a fancy way to write
\begin{equation}
\hat{Y}(x_{10},x_{20})=b_{0}+b_{1}x_{10}+b_{2}x_{20}.
\end{equation}
 
# +BEGIN_exampletoo

If we wanted to predict the average volume of black cherry trees that
have =Girth = 15= in and are =Height = 77= ft tall then we would use
the estimate
\begin{alignat*}{1}
\hat{\mu}(15,\,77)= & -58+4.7(15)+0.3(77),\\
\approx & 35.6\mbox{\,\ ft}^{3}.
\end{alignat*}

We would use the same estimate \(\hat{Y}=35.6\) to predict the
measured =Volume= of another black cherry tree -- yet to be observed
-- that has =Girth = 15= in and is =Height = 77= ft tall.
# +END_exampletoo

**** How to do it with \(\mathsf{R}\)

The fitted values are stored inside =trees.lm= and may be accessed
with the =fitted= function. We only show the first five fitted values.

#+BEGIN_SRC R :exports both :results output pp 
fitted(trees.lm)[1:5]
#+END_SRC

#+RESULTS:
:         1         2         3         4         5 
:  4.837660  4.553852  4.816981 15.874115 19.869008

The syntax for general prediction does not change much from simple
linear regression. The computations are done with the =predict=
function as described below.

The only difference from SLR is in the way we tell \(\mathsf{R}\) the
values of the explanatory variables for which we want predictions. In
SLR we had only one independent variable but in MLR we have many (for
the =trees= data we have two). We will store values for the
independent variables in the data frame =new=, which has two columns
(one for each independent variable) and three rows (we shall make
predictions at three different locations).

#+BEGIN_SRC R :exports code :results silent 
new <- data.frame(Girth = c(9.1, 11.6, 12.5), Height = c(69, 74, 87))
#+END_SRC

We can view the locations at which we will predict:

#+BEGIN_SRC R :exports both :results output pp 
new
#+END_SRC

#+RESULTS:
:   Girth Height
: 1   9.1     69
: 2  11.6     74
: 3  12.5     87

We continue just like we would have done in SLR.

#+BEGIN_SRC R :exports both :results output pp 
predict(trees.lm, newdata = new)
#+END_SRC

#+RESULTS:
:         1         2         3 
:  8.264937 21.731594 30.379205

#+BEGIN_SRC R :exports none :results silent
treesFIT <- round(predict(trees.lm, newdata = new), 1)
#+END_SRC

# +BEGIN_exampletoo

Using the =trees= data,
1. Report a point estimate of the mean =Volume= of a tree of =Girth=
   9.1 in and =Height= 69 ft.  The fitted value for \(x_{1}=9.1\) and
   \(x_{2} = 69\) is SRC_R[:eval no-export]{treesFIT[ 1 ]} 8.3, so a point estimate
   would be SRC_R[:eval no-export]{treesFIT[ 1 ]} 8.3 cubic feet.
2. Report a point prediction for and a 95% prediction interval for the
   =Volume= of a hypothetical tree of =Girth= 12.5 in and =Height= 87
   ft.  The fitted value for \(x_{1} = 12.5\) and \(x_{2} = 87\) is
   SRC_R[:eval no-export]{treesFIT[ 3 ]} 30.4, so a point prediction for the =Volume=
   is SRC_R[:eval no-export]{treesFIT[ 3 ]} 30.4 cubic feet.
# +END_exampletoo

*** Mean Square Error and Standard Error
:PROPERTIES:
:CUSTOM_ID: sub-mlr-mse-se
:END:

The residuals are given by
\begin{equation}
\mathbf{E}=\mathbf{Y}-\hat{\mathbf{Y}}=\mathbf{Y}-\mathbf{H}\mathbf{Y}=(\mathbf{I}-\mathbf{H})\mathbf{Y}.
\end{equation}
Now we can use Theorem [[thm-mvnorm-dist-matrix-prod]] to see that the
residuals are distributed
\begin{equation}
\mathbf{E}\sim\mathsf{mvnorm}(\mathtt{mean}=\mathbf{0},\,\mathtt{sigma}=\sigma^{2}(\mathbf{I}-\mathbf{H})),
\end{equation}
since
\((\mathbf{I}-\mathbf{H})\mathbf{X}\upbeta=\mathbf{X}\upbeta-\mathbf{X}\upbeta=\mathbf{0}\)
and
\((\mathbf{I}-\mathbf{H})\,(\sigma^{2}\mathbf{I})\,(\mathbf{I}-\mathbf{H})^{\mathrm{T}}=\sigma^{2}(\mathbf{I}-\mathbf{H})^{2}=\sigma^{2}(\mathbf{I}-\mathbf{H})\). Thesum
of squared errors \(SSE\) is just
\begin{equation}
SSE=\mathbf{E}^{\mathrm{T}}\mathbf{E}=\mathbf{Y}^{\mathrm{T}}(\mathbf{I}-\mathbf{H})(\mathbf{I}-\mathbf{H})\mathbf{Y}=\mathbf{Y}^{\mathrm{T}}(\mathbf{I}-\mathbf{H})\mathbf{Y}.
\end{equation}
Recall that in SLR we had two parameters (\(\beta_{0}\) and
\(\beta_{1}\)) in our regression model and we estimated \(\sigma^{2}\)
with \(s^{2}=SSE/(n-2)\). In MLR, we have \(p+1\) parameters in our
regression model and we might guess that to estimate \(\sigma^{2}\) we
would use the /mean square error/ \(S^{2}\) defined by
\begin{equation}
S^{2}=\frac{SSE}{n-(p+1)}.
\end{equation}
That would be a good guess. The /residual standard error/ is
\(S=\sqrt{S^{2}}\).

**** How to do it with \(\mathsf{R}\)

The residuals are also stored with =trees.lm= and may be accessed with
the =residuals= function. We only show the first five residuals.

#+BEGIN_SRC R :exports both :results output pp 
residuals(trees.lm)[1:5]
#+END_SRC

#+RESULTS:
:          1          2          3          4          5 
:  5.4623403  5.7461484  5.3830187  0.5258848 -1.0690084

The =summary= function output (shown later) lists the =Residual
Standard Error= which is just \(S=\sqrt{S^{2}}\). It is stored in the
=sigma= component of the =summary= object.

#+BEGIN_SRC R :exports both :results output pp 
treesumry <- summary(trees.lm)
treesumry$sigma
#+END_SRC

#+RESULTS:
: [1] 3.881832

For the =trees= data we find \(s \approx\) SRC_R[:eval no-export]{round(treesumry$sigma,3)}.

*** Interval Estimates of the Parameters
:PROPERTIES:
:CUSTOM_ID: sub-mlr-interval-est-params
:END:

We showed in Section [[#sub-mlr-parameter-estimates]] that
\(\mathbf{b}=\left(\mathbf{X}^{\mathrm{T}}\mathbf{X}\right)^{-1}\mathbf{X}^{\mathrm{T}}\mathbf{Y}\),
which is really just a big matrix -- namely
\(\left(\mathbf{X}^{\mathrm{T}}\mathbf{X}\right)^{-1}\mathbf{X}^{\mathrm{T}}\)
-- multiplied by \(\mathbf{Y}\). It stands to reason that the sampling
distribution of \(\mathbf{b}\) would be intimately related to the
distribution of \(\mathbf{Y}\), which we assumed to be
\begin{equation}
\mathbf{Y}\sim\mathsf{mvnorm}\left(\mathtt{mean}=\mathbf{X}\upbeta,\,\mathtt{sigma}=\sigma^{2}\mathbf{I}\right).
\end{equation}
Now recall Theorem [[thm-mvnorm-dist-matrix-prod]] that we said we were
going to need eventually (the time is now). That proposition
guarantees that
\begin{equation}
\label{eq-distn-b-mlr}
\mathbf{b}\sim\mathsf{mvnorm}\left(\mathtt{mean}=\upbeta,\,\mathtt{sigma}=\sigma^{2}\left(\mathbf{X}^{\mathrm{T}}\mathbf{X}\right)^{-1}\right),
\end{equation}
since
\begin{equation}
\mathbb{E}\mathbf{b}=\left(\mathbf{X}^{\mathrm{T}}\mathbf{X}\right)^{-1}\mathbf{X}^{\mathrm{T}}(\mathbf{X}\upbeta)=\upbeta,
\end{equation}
and
\begin{equation}
\mbox{Var}(\mathbf{b})=\left(\mathbf{X}^{\mathrm{T}}\mathbf{X}\right)^{-1}\mathbf{X}^{\mathrm{T}}(\sigma^{2}\mathbf{I})\mathbf{X}\left(\mathbf{X}^{\mathrm{T}}\mathbf{X}\right)^{-1}=\sigma^{2}\left(\mathbf{X}^{\mathrm{T}}\mathbf{X}\right)^{-1},
\end{equation}
the first equality following because the matrix
\(\left(\mathbf{X}^{\mathrm{T}}\mathbf{X}\right)^{-1}\) is symmetric.

There is a lot that we can glean from Equation \eqref{eq-distn-b-mlr}. First,
it follows that the estimator \(\mathbf{b}\) is unbiased (see Section
[[#sec-Point-Estimation-1]]). Second, the variances of \(b_{0}\), \(b_{1}\),
..., \(b_{n}\) are exactly the diagonal elements of
\(\sigma^{2}\left(\mathbf{X}^{\mathrm{T}}\mathbf{X}\right)^{-1}\),
which is completely known except for that pesky parameter
\(\sigma^{2}\). Third, we can estimate the standard error of \(b_{i}\)
(denoted \(S_{b_{i}}\)) with the mean square error \(S\) (defined in
the previous section) multiplied by the corresponding diagonal element
of \(\left(\mathbf{X}^{\mathrm{T}}\mathbf{X}\right)^{-1}\). Finally,
given estimates of the standard errors we may construct confidence
intervals for \(\beta_{i}\) with an interval that looks like
\begin{equation}
b_{i}\pm\mathsf{t}_{\alpha/2}(\mathtt{df}=n-p-1)S_{b_{i}}.
\end{equation}
The degrees of freedom for the Student's \(t\) distribution[fn:fn-tdf]
are the same as the denominator of \(S^{2}\).

[fn:fn-tdf] We are taking great leaps over the mathematical
details. In particular, we have yet to show that \(s^{2}\) has a
chi-square distribution and we have not even come close to showing
that \(b_{i}\) and \(s_{b_{i}}\) are independent. But these are
entirely outside the scope of the present book and the reader may rest
assured that the proofs await in later classes. See C.R. Rao for more.

**** How to do it with \(\mathsf{R}\)

To get confidence intervals for the parameters we need only use
=confint= @@latex:\index{confint@\texttt{confint}}@@:

#+BEGIN_SRC R :exports both :results output pp 
confint(trees.lm)
#+END_SRC

#+RESULTS:
:                    2.5 %      97.5 %
: (Intercept) -75.68226247 -40.2930554
: Girth         4.16683899   5.2494820
: Height        0.07264863   0.6058538

#+BEGIN_SRC R :exports none :results silent
treesPAR <- round(confint(trees.lm), 1)
#+END_SRC

For example, using the calculations above we say that for the
regression model =Volume ~ Girth + Height= we are 95% confident that
the parameter \(\beta_{1}\) lies somewhere in the interval
SRC_R[:eval no-export]{treesPAR[2, 1]} 4.2 to SRC_R[:eval no-export]{treesPAR[2, 2]} 5.2.

*** Confidence and Prediction Intervals

We saw in Section [[#sub-mlr-point-est-regsurface]] how to make point estimates
of the mean value of additional observations and predict values of
future observations, but how good are our estimates? We need
confidence and prediction intervals to gauge their accuracy, and lucky
for us the formulas look similar to the ones we saw in SLR.

In Equation \eqref{eq-mlr-single-yhat-matrix} we wrote \(
\hat{Y}(\mathbf{x}_{0})=\mathbf{x}_{0}^{\mathrm{T}}\mathbf{b} \), and
in Equation \eqref{eq-distn-b-mlr} we saw that
\begin{equation}
\mathbf{b}\sim\mathsf{mvnorm}\left(\mathtt{mean} = \upbeta,\,\mathtt{sigma}=\sigma^{2}\left(\mathbf{X}^{\mathrm{T}}\mathbf{X}\right)^{-1}\right).
\end{equation}
The following is therefore immediate from Theorem
[[thm-mvnorm-dist-matrix-prod]]:
\begin{equation}
\hat{Y}(\mathbf{x}_{0})\sim\mathsf{mvnorm}\left(\mathtt{mean}=\mathbf{x}_{0}^{\mathrm{T}}\upbeta,\,\mathtt{sigma}=\sigma^{2}\mathbf{x}_{0}^{\mathrm{T}}\left(\mathbf{X}^{\mathrm{T}}\mathbf{X}\right)^{-1}\mathbf{x}_{0}\right).
\end{equation}
It should be no surprise that confidence intervals for the mean value
of a future observation at the location
\(\mathbf{x}_{0}=\begin{bmatrix}x_{10} & x_{20} & \ldots &
x_{p0}\end{bmatrix}^{\mathrm{T}}\) are given by
\begin{equation}
\hat{Y}(\mathbf{x}_{0})\pm\mathsf{t}_{\alpha/2}(\mathtt{df}=n-p-1)\, S\sqrt{\mathbf{x}_{0}^{\mathrm{T}}\left(\mathbf{X}^{\mathrm{T}}\mathbf{X}\right)^{-1}\mathbf{x}_{0}}.
\end{equation}
Intuitively,
\(\mathbf{x}_{0}^{\mathrm{T}}\left(\mathbf{X}^{\mathrm{T}}\mathbf{X}\right)^{-1}\mathbf{x}_{0}\)
measures the distance of \(\mathbf{x}_{0}\) from the center of the
data. The degrees of freedom in the Student's \(t\) critical value are
\(n-(p+1)\) because we need to estimate \(p+1\) parameters.

Prediction intervals for a new observation at \(\mathbf{x}_{0}\) are
given by
\begin{equation}
\hat{Y}(\mathbf{x}_{0})\pm\mathsf{t}_{\alpha/2}(\mathtt{df}=n-p-1)\, S\sqrt{1+\mathbf{x}_{0}^{\mathrm{T}}\left(\mathbf{X}^{\mathrm{T}}\mathbf{X}\right)^{-1}\mathbf{x}_{0}}.
\end{equation}
The prediction intervals are wider than the confidence intervals, just as in Section [[#sub-slr-interval-est-regline]].

**** How to do it with \(\mathsf{R}\)

The syntax is identical to that used in SLR, with the proviso that we
need to specify values of the independent variables in the data frame
=new= as we did in Section [[#sub-slr-interval-est-regline]] (which we
repeat here for illustration).

#+BEGIN_SRC R :exports code :results silent 
new <- data.frame(Girth = c(9.1, 11.6, 12.5), Height = c(69, 74, 87))
#+END_SRC

Confidence intervals are given by

#+BEGIN_SRC R :exports both :results output pp 
predict(trees.lm, newdata = new, interval = "confidence")
#+END_SRC

#+RESULTS:
:         fit      lwr      upr
: 1  8.264937  5.77240 10.75747
: 2 21.731594 20.11110 23.35208
: 3 30.379205 26.90964 33.84877

#+BEGIN_SRC R :exports none :results silent
treesCI <- round(predict(trees.lm, newdata = new, interval = "confidence"), 1)
#+END_SRC

Prediction intervals are given by

#+BEGIN_SRC R :exports both :results output pp 
predict(trees.lm, newdata = new, interval = "prediction")
#+END_SRC

#+RESULTS:
:         fit         lwr      upr
: 1  8.264937 -0.06814444 16.59802
: 2 21.731594 13.61657775 29.84661
: 3 30.379205 21.70364103 39.05477

#+BEGIN_SRC R :exports none :results silent
treesPI <- round(predict(trees.lm, newdata = new, interval = "prediction"), 1)
#+END_SRC

As before, the interval type is decided by the =interval= argument and
the default confidence level is 95% (which can be changed with the
=level= argument).

# +BEGIN_exampletoo

Using the =trees= data, 

1. Report a 95% confidence interval for the mean =Volume= of a tree of
   =Girth= 9.1 in and =Height= 69 ft. The 95% CI is given by
   SRC_R[:eval no-export]{treesCI[1, 2]} 5.8 to SRC_R[:eval no-export]{treesCI[1, 3]} 10.8, so with 95%
   confidence the mean =Volume= lies somewhere between
   SRC_R[:eval no-export]{treesCI[1, 2]} 5.8 cubic feet and SRC_R[:eval no-export]{treesCI[1, 3]} 10.8 cubic feet.
2. Report a 95% prediction interval for the =Volume= of a hypothetical
   tree of =Girth= 12.5 in and =Height= 87 ft. The 95% prediction
   interval is given by SRC_R[:eval no-export]{treesCI[3, 2]} 26.9 to
   SRC_R[:eval no-export]{treesCI[3,3]} 33.8, so with 95% confidence we may assert that
   the hypothetical =Volume= of a tree of =Girth= 12.5 in and =Height=
   87 ft would lie somewhere between SRC_R[:eval no-export]{treesCI[3, 2]} 26.9
   cubic feet and SRC_R[:eval no-export]{treesCI[3, 3]} 33.8 feet.

# +END_exampletoo

** Model Utility and Inference
:PROPERTIES:
:CUSTOM_ID: sec-Model-Utility-and-MLR
:END:

*** Multiple Coefficient of Determination

We saw in Section [[#sub-mlr-mse-se]] that the error sum of squares \(SSE\) can be conveniently written in MLR as 
\begin{equation}
\label{eq-mlr-sse-matrix}
SSE=\mathbf{Y}^{\mathrm{T}}(\mathbf{I}-\mathbf{H})\mathbf{Y}.
\end{equation}
It turns out that there are equally convenient formulas for the total sum of squares \(SSTO\) and the regression sum of squares \(SSR\). They are:
\begin{alignat}{1}
\label{eq-mlr-ssto-matrix}
SSTO= & \mathbf{Y}^{\mathrm{T}}\left(\mathbf{I}-\frac{1}{n}\mathbf{J}\right)\mathbf{Y}
\end{alignat}
and
\begin{alignat}{1}
\label{eq-mlr-ssr-matrix}
SSR= & \mathbf{Y}^{\mathrm{T}}\left(\mathbf{H}-\frac{1}{n}\mathbf{J}\right)\mathbf{Y}.
\end{alignat}
(The matrix \(\mathbf{J}\) is defined in Appendix
[[#sec-Linear-Algebra]].) Immediately from Equations
\eqref{eq-mlr-sse-matrix}, \eqref{eq-mlr-ssto-matrix}, and
\eqref{eq-mlr-ssr-matrix} we get the /Anova Equality/
\begin{equation} 
SSTO=SSE+SSR.
\end{equation}
(See Exercise [[xca-anova-equality]].) We define the /multiple coefficient of determination/ by the formula
\begin{equation} 
R^{2}=1-\frac{SSE}{SSTO}.
\end{equation}

We interpret \(R^{2}\) as the proportion of total variation that is
explained by the multiple regression model. In MLR we must be careful,
however, because the value of \(R^{2}\) can be artificially inflated
by the addition of explanatory variables to the model, regardless of
whether or not the added variables are useful with respect to
prediction of the response variable. In fact, it can be proved that
the addition of a single explanatory variable to a regression model
will increase the value of \(R^{2}\), /no matter how worthless/ the
explanatory variable is. We could model the height of the ocean tides,
then add a variable for the length of cheetah tongues on the Serengeti
plain, and our \(R^{2}\) would inevitably increase.

This is a problem, because as the philosopher, Occam, once said:
"causes should not be multiplied beyond necessity". We address the
problem by penalizing \(R^{2}\) when parameters are added to the
model. The result is an /adjusted/ \(R^{2}\) which we denote by
\(\overline{R}^{2}\).
\begin{equation}
\overline{R}^{2}=\left(R^{2}-\frac{p}{n-1}\right)\left(\frac{n-1}{n-p-1}\right).
\end{equation}
It is good practice for the statistician to weigh both \(R^{2}\) and
\(\overline{R}^{2}\) during assessment of model utility. In many cases
their values will be very close to each other. If their values differ
substantially, or if one changes dramatically when an explanatory
variable is added, then (s)he should take a closer look at the
explanatory variables in the model.

**** How to do it with \(\mathsf{R}\)
For the =trees= data, we can get \(R^{2}\) and \(\overline{R}^{2}\)
from the =summary= output or access the values directly by name as
shown (recall that we stored the =summary= object in =treesumry=).

#+BEGIN_SRC R :exports both :results output pp 
treesumry$r.squared
#+END_SRC

#+RESULTS:
: [1] 0.94795

#+BEGIN_SRC R :exports both :results output pp 
treesumry$adj.r.squared
#+END_SRC

#+RESULTS:
: [1] 0.9442322

High values of \(R^{2}\) and \( \overline{R}^2 \) such as these
indicate that the model fits very well, which agrees with what we saw
in Figure [[fig-3D-scatterplot-trees]].

*** Overall /F/-Test
:PROPERTIES:
:CUSTOM_ID: sub-mlr-Overall-F-Test
:END:

Another way to assess the model's utility is to to test the hypothesis
\[ H_{0}:\beta_{1}=\beta_{2}=\cdots=\beta_{p}=0\mbox{ versus
}H_{1}:\mbox{ at least one $\beta_{i}\neq0$}.  \] The idea is that if
all \(\beta_{i}\)'s were zero, then the explanatory variables
\(X_{1},\ldots,X_{p}\) would be worthless predictors for the response
variable \(Y\). We can test the above hypothesis with the overall
\(F\) statistic, which in MLR is defined by
\begin{equation}
F=\frac{SSR/p}{SSE/(n-p-1)}.
\end{equation}
When the regression assumptions hold and under \(H_{0}\), it can be
shown that
\(F\sim\mathsf{f}(\mathtt{df1}=p,\,\mathtt{df2}=n-p-1)\). We reject
\(H_{0}\) when \(F\) is large, that is, when the explained variation
is large relative to the unexplained variation.

**** How to do it with \(\mathsf{R}\)

The overall \(F\) statistic and its associated /p/-value is listed at
the bottom of the =summary= output, or we can access it directly by
name; it is stored in the =fstatistic= component of the =summary=
object.

#+BEGIN_SRC R :exports both :results output pp 
treesumry$fstatistic
#+END_SRC

#+RESULTS:
:    value    numdf    dendf 
: 254.9723   2.0000  28.0000

For the =trees= data, we see that \( F = \)
SRC_R[:eval no-export]{treesumry$fstatistic[ 1 ]} 254.972337410669 with a /p/-value =<
2.2e-16=. Consequently we reject \(H_{0}\), that is, the data provide
strong evidence that not all \(\beta_{i}\)'s are zero.

*** Student's /t/ Tests
:PROPERTIES:
:CUSTOM_ID: sub-mlr-Students-t-Tests
:END:

We know that
\begin{equation}
\mathbf{b}\sim\mathsf{mvnorm}\left(\mathtt{mean}=\upbeta,\,\mathtt{sigma}=\sigma^{2}\left(\mathbf{X}^{\mathrm{T}}\mathbf{X}\right)^{-1}\right)
\end{equation}
and we have seen how to test the hypothesis \(H_{0}:\beta_{1}=\beta_{2}=\cdots=\beta_{p}=0\), but let us now consider the test
\begin{equation}
H_{0}:\beta_{i}=0\mbox{ versus }H_{1}:\beta_{i}\neq0,
\end{equation}
where \(\beta_{i}\) is the coefficient for the \(i^{\textrm{th}}\)
independent variable. We test the hypothesis by calculating a
statistic, examining it's null distribution, and rejecting \(H_{0}\)
if the /p-value/ is small. If \(H_{0}\) is rejected, then we conclude
that there is a significant relationship between \(Y\) and \(x_{i}\)
/in the regression model/ \(Y\sim(x_{1},\ldots,x_{p})\). This last
part of the sentence is very important because the significance of the
variable \(x_{i}\) sometimes depends on the presence of other
independent variables in the model[fn:fn-multic].

[fn:fn-multic] In other words, a variable might be highly significant
one moment but then fail to be significant when another variable is
added to the model. When this happens it often indicates a problem
with the explanatory variables, such as /multicollinearity/. See
Section \ref{sub-Multicollinearity}.

To test the hypothesis we go to find the sampling distribution of \(
b_{i} \), the estimator of the corresponding parameter \( \beta_{i}
\), when the null hypothesis is true. We saw in Section [[#sub-mlr-interval-est-params]] that
\begin{equation}
T_{i}=\frac{b_{i}-\beta_{i}}{S_{b_{i}}}
\end{equation}
has a Student's \(t\) distribution with \(n-(p+1)\) degrees of
freedom. (Remember, we are estimating \(p+1\) parameters.)
Consequently, under the null hypothesis \(H_{0}:\beta_{i}=0\) the
statistic \(t_{i}=b_{i}/S_{b_{i}}\) has a
\(\mathsf{t}(\mathtt{df}=n-p-1)\) distribution.

**** How to do it with \(\mathsf{R}\)

The Student's \(t\) tests for significance of the individual explanatory variables are shown in the =summary= output.

#+BEGIN_SRC R :exports both :results output pp 
treesumry
#+END_SRC

#+RESULTS:
#+BEGIN_example

Call:
lm(formula = Volume ~ Girth + Height, data = trees)

Residuals:
    Min      1Q  Median      3Q     Max 
-6.4065 -2.6493 -0.2876  2.2003  8.4847 

Coefficients:
            Estimate Std. Error t value Pr(>|t|)    
(Intercept) -57.9877     8.6382  -6.713 2.75e-07 ***
Girth         4.7082     0.2643  17.816  < 2e-16 ***
Height        0.3393     0.1302   2.607   0.0145 *  
---
Signif. codes:  0 '***' 0.001 '**' 0.01 '*' 0.05 '.' 0.1 ' ' 1

Residual standard error: 3.882 on 28 degrees of freedom
Multiple R-squared:  0.948,	Adjusted R-squared:  0.9442 
F-statistic:   255 on 2 and 28 DF,  p-value: < 2.2e-16
#+END_example

We see from the /p-values/ that there is a significant linear
relationship between =Volume= and =Girth= and between =Volume= and
=Height= in the regression model =Volume ~ Girth + Height=. Further,
it appears that the =Intercept= is significant in the aforementioned
model.

** Polynomial Regression
:PROPERTIES:
:CUSTOM_ID: sec-Polynomial-Regression
:END:

*** Quadratic Regression Model

In each of the previous sections we assumed that \(\mu\) was a linear
function of the explanatory variables. For example, in SLR we assumed
that \(\mu(x)=\beta_{0}+\beta_{1}x\), and in our previous MLR examples
we assumed \(\mu(x_{1},x_{2}) = \beta_{0}+\beta_{1}x_{1} +
\beta_{2}x_{2}\). In every case the scatterplots indicated that our
assumption was reasonable. Sometimes, however, plots of the data
suggest that the linear model is incomplete and should be modified.

#+NAME: Scatterplot-Volume-Girth-trees
#+BEGIN_SRC R :exports both :results graphics :file fig/mlr-Scatterplot-Volume-Girth-trees.ps
qplot(Girth, Volume, data = trees)
#+END_SRC

#+NAME: fig-Scatterplot-Volume-Girth-trees
#+CAPTION[Scatterplot of =Volume= versus =Girth= for the =trees= data]: \small Scatterplot of =Volume= versus =Girth= for the =trees= data.
#+ATTR_LaTeX: :width 0.9\textwidth :placement [ht!]
#+RESULTS: Scatterplot-Volume-Girth-trees
[[file:fig/mlr-Scatterplot-Volume-Girth-trees.ps]]

For example, let us examine a scatterplot of =Volume= versus =Girth= a
little more closely. See Figure [[fig-Scatterplot-Volume-Girth-trees]]. There
might be a slight curvature to the data; the volume curves ever so
slightly upward as the girth increases. After looking at the plot we
might try to capture the curvature with a mean response such as
\begin{equation}
\mu(x_{1})=\beta_{0}+\beta_{1}x_{1}+\beta_{2}x_{1}^{2}.
\end{equation}
The model associated with this choice of \(\mu\) is
\begin{equation}
Y=\beta_{0}+\beta_{1}x_{1}+\beta_{2}x_{1}^{2}+\epsilon.
\end{equation}
The regression assumptions are the same. Almost everything indeed is
the same. In fact, it is still called a "linear regression model",
since the mean response \(\mu\) is linear /in the parameters/
\(\beta_{0}\), \(\beta_{1}\), and \(\beta_{2}\).

*However, there is one important difference.* When we introduce the
squared variable in the model we inadvertently also introduce strong
dependence between the terms which can cause significant numerical
problems when it comes time to calculate the parameter
estimates. Therefore, we should usually rescale the independent
variable to have mean zero (and even variance one if we wish) *before*
fitting the model. That is, we replace the \(x_{i}\)'s with
\(x_{i}-\overline{x}\) (or \((x_{i}-\overline{x})/s\)) before fitting
the model[fn:fn-ortho].

[fn:fn-ortho] Rescaling the data gets the job done but a better way to
avoid the multicollinearity introduced by the higher order terms is
with /orthogonal polynomials/, whose coefficients are chosen just
right so that the polynomials are not correlated with each other. This
is beginning to linger outside the scope of this book, however, so we
will content ourselves with a brief mention and then stick with the
rescaling approach in the discussion that follows. A nice example of
orthogonal polynomials in action can be run with =example(cars)=.

**** How to do it with \(\mathsf{R}\)

There are multiple ways to fit a quadratic model to the variables
=Volume= and =Girth= using \(\mathsf{R}\).
1. One way would be to square the values for =Girth= and save them in
   a vector =Girthsq=. Next, fit the linear model =Volume ~ Girth +
   Girthsq=.
2. A second way would be to use the /insulate/ function in
   \(\mathsf{R}\), denoted by =I=:
   : Volume ~ Girth + I(Girth^2)
   The second method is shorter than the first but the end result is
   the same. And once we calculate and store the fitted model (in,
   say, =treesquad.lm=) all of the previous comments regarding
   \(\mathsf{R}\) apply.
3. A third and "right" way to do it is with orthogonal polynomials:
   :  Volume ~ poly(Girth, degree = 2)
   See =?poly= and =?cars= for more information. Note that we can
   recover the approach in 2 with =poly(Girth, degree = 2, raw =
   TRUE)=.

# +BEGIN_exampletoo

We will fit the quadratic model to the =trees= data and display the
results with =summary=, being careful to rescale the data before
fitting the model. We may rescale the =Girth= variable to have zero
mean and unit variance on-the-fly with the =scale= function.

#+BEGIN_SRC R :exports both :results output pp 
treesquad.lm <- lm(Volume ~ scale(Girth) + I(scale(Girth)^2), data = trees)
summary(treesquad.lm)
#+END_SRC

#+RESULTS:
#+BEGIN_example

Call:
lm(formula = Volume ~ scale(Girth) + I(scale(Girth)^2), data = trees)

Residuals:
    Min      1Q  Median      3Q     Max 
-5.4889 -2.4293 -0.3718  2.0764  7.6447 

Coefficients:
                  Estimate Std. Error t value Pr(>|t|)    
(Intercept)        27.7452     0.8161  33.996  < 2e-16 ***
scale(Girth)       14.5995     0.6773  21.557  < 2e-16 ***
I(scale(Girth)^2)   2.5067     0.5729   4.376 0.000152 ***
---
Signif. codes:  0 '***' 0.001 '**' 0.01 '*' 0.05 '.' 0.1 ' ' 1

Residual standard error: 3.335 on 28 degrees of freedom
Multiple R-squared:  0.9616,	Adjusted R-squared:  0.9588 
F-statistic: 350.5 on 2 and 28 DF,  p-value: < 2.2e-16
#+END_example

We see that the \(F\) statistic indicates the overall model including
=Girth= and =Girth^2= is significant. Further, there is strong
evidence that both =Girth= and =Girth^2= are significantly related to
=Volume=. We may examine a scatterplot together with the fitted
quadratic function using the =lines= function, which adds a line to
the plot tracing the estimated mean response.

#+NAME: Fitting-the-Quadratic
#+BEGIN_SRC R :exports both :results graphics :file fig/mlr-Fitting-the-Quadratic.ps
a <- ggplot(trees, aes(scale(Girth), Volume))
a + stat_smooth(method = lm, formula = y ~ poly(x, 2)) + geom_point()
#+END_SRC

#+NAME: fig-Fitting-the-Quadratic
#+CAPTION[Quadratic model for the =trees= data]: \small A quadratic model for the =trees= data.
#+ATTR_LaTeX: :width 0.9\textwidth :placement [ht!]
#+RESULTS: Fitting-the-Quadratic
[[file:fig/mlr-Fitting-the-Quadratic.ps]]

The plot is shown in Figure [[fig-Fitting-the-Quadratic]]. Pay attention to
the scale on the \(x\)-axis: it is on the scale of the transformed
=Girth= data and not on the original scale.

# +END_exampletoo



#+BEGIN_rem
When a model includes a quadratic term for an independent variable, it
is customary to also include the linear term in the model. The
principle is called /parsimony/. More generally, if the researcher
decides to include \(x^{m}\) as a term in the model, then (s)he should
also include all lower order terms \(x\), \(x^{2}\), ...,\(x^{m-1}\)
in the model.
#+END_rem

We do estimation/prediction the same way that we did in Section
[[#sub-mlr-point-est-regsurface]], except we do not need a =Height= column
in the dataframe =new= since the variable is not included in the
quadratic model.

#+BEGIN_SRC R :exports both :results output pp 
new <- data.frame(Girth = c(9.1, 11.6, 12.5))
predict(treesquad.lm, newdata = new, interval = "prediction")
#+END_SRC

#+RESULTS:
:        fit       lwr      upr
: 1 11.56982  4.347426 18.79221
: 2 20.30615 13.299050 27.31325
: 3 25.92290 18.972934 32.87286

The predictions and intervals are slightly different from what they
were previously. Notice that it was not necessary to rescale the
=Girth= prediction data before input to the =predict= function; the
model did the rescaling for us automatically.

#+BEGIN_rem
We have mentioned on several occasions that it is important to rescale
the explanatory variables for polynomial regression. Watch what
happens if we ignore this advice:

#+BEGIN_SRC R :exports both :results output pp 
summary(lm(Volume ~ Girth + I(Girth^2), data = trees))
#+END_SRC

#+RESULTS:
#+BEGIN_example

Call:
lm(formula = Volume ~ Girth + I(Girth^2), data = trees)

Residuals:
    Min      1Q  Median      3Q     Max 
-5.4889 -2.4293 -0.3718  2.0764  7.6447 

Coefficients:
            Estimate Std. Error t value Pr(>|t|)    
(Intercept) 10.78627   11.22282   0.961 0.344728    
Girth       -2.09214    1.64734  -1.270 0.214534    
I(Girth^2)   0.25454    0.05817   4.376 0.000152 ***
---
Signif. codes:  0 '***' 0.001 '**' 0.01 '*' 0.05 '.' 0.1 ' ' 1

Residual standard error: 3.335 on 28 degrees of freedom
Multiple R-squared:  0.9616,	Adjusted R-squared:  0.9588 
F-statistic: 350.5 on 2 and 28 DF,  p-value: < 2.2e-16
#+END_example

Now nothing is significant in the model except =Girth^2=. We could
delete the =Intercept= and =Girth= from the model, but the model would
no longer be /parsimonious/. A novice may see the output and be
confused about how to proceed, while the seasoned statistician
recognizes immediately that =Girth= and =Girth^2= are highly
correlated (see Section [[#sub-Multicollinearity]]). The only remedy to this
ailment is to rescale =Girth=, which we should have done in the first
place.

In Example [[exa-mlr-trees-poly-no-rescale]] of Section [[#sec-Partial-F-Statistic]].

*Note:* The =trees= data do not have any qualitative explanatory
variables, so we will construct one for illustrative
purposes[fn:fn-binning].  We will leave the =Girth= variable alone,
but we will replace the variable =Height= by a new variable =Tall=
which indicates whether or not the cherry tree is taller than a
certain threshold (which for the sake of argument will be the sample
median height of 76 ft). That is, =Tall= will be defined by
@@latex:\begin{equation} \mathtt{Tall} = \begin{cases} \mathtt{yes}, & \mbox{if }\mathtt{Height} > 76,\\ \mathtt{no}, & \mbox{if }\mathtt{Height}\leq 76. \end{cases} \end{equation}@@

We can construct =Tall= very quickly in \(\mathsf{R}\) with the =cut=
function:

#+BEGIN_SRC R :exports both :results output pp 
trees$Tall <- cut(trees$Height, breaks = c(-Inf, 76, Inf), 
                  labels = c("no","yes"))
trees$Tall[1:5]
#+END_SRC

#+RESULTS:
: [1] no  no  no  no  yes
: Levels: no yes

Note that =Tall= is automatically generated to be a factor with the
labels in the correct order. See =?cut= for more.

[fn:fn-binning] This procedure of replacing a continuous variable by a
discrete/qualitative one is called /binning/, and is almost /never/
the right thing to do. We are in a bind at this point, however,
because we have invested this chapter in the =trees= data and I do not
want to switch mid-discussion. I am currently searching for a data set
with pre-existing qualitative variables that also conveys the same
points present in the trees data, and when I find it I will update
this chapter accordingly.

Once we have =Tall=, we include it in the regression model just like
we would any other variable. It is handled internally in a special
way. Define a "dummy variable" =Tallyes= that takes values
@@latex:\begin{equation} \mathtt{Tallyes} = \begin{cases} 1, & \mbox{if }\mathtt{Tall}=\mathtt{yes},\\ 0, & \mbox{otherwise.} \end{cases} \end{equation}@@
That is, =Tallyes= is an /indicator variable/ which indicates when a
respective tree is tall. The model may now be written as
\begin{equation}
\mathtt{Volume}=\beta_{0}+\beta_{1}\mathtt{Girth}+\beta_{2}\mathtt{Tallyes}+\epsilon.
\end{equation}
Let us take a look at what this definition does to the mean
response. Trees with =Tall = yes= will have the mean response
\begin{equation}
\mu(\mathtt{Girth})=(\beta_{0}+\beta_{2})+\beta_{1}\mathtt{Girth},
\end{equation}
while trees with =Tall = no= will have the mean response
\begin{equation} 
\mu(\mathtt{Girth})=\beta_{0}+\beta_{1}\mathtt{Girth}.
\end{equation}
In essence, we are fitting two regression lines: one for tall trees,
and one for short trees. The regression lines have the same slope but
they have different \(y\) intercepts (which are exactly
\(|\beta_{2}|\) far apart).

*** How to do it with \(\mathsf{R}\)

The important thing is to double check that the qualitative variable
in question is stored as a factor. The way to check is with the
=class= command. For example,

#+BEGIN_SRC R :exports both :results output pp 
class(trees$Tall)
#+END_SRC

#+RESULTS:
: [1] "factor"

If the qualitative variable is not yet stored as a factor then we may
convert it to one with the =factor= command. See Section
[[#sub-Qualitative-Data]]. Other than this we perform MLR as we
normally would.

#+BEGIN_SRC R :exports both :results output pp 
treesdummy.lm <- lm(Volume ~ Girth + Tall, data = trees)
summary(treesdummy.lm)
#+END_SRC

#+RESULTS:
#+begin_example

Call:
lm(formula = Volume ~ Girth + Tall, data = trees)

Residuals:
    Min      1Q  Median      3Q     Max 
-5.7788 -3.1710  0.4888  2.6737 10.0619 

Coefficients:
            Estimate Std. Error t value Pr(>|t|)    
(Intercept) -34.1652     3.2438  -10.53 3.02e-11 ***
Girth         4.6988     0.2652   17.72  < 2e-16 ***
Tall[T.yes]   4.3072     1.6380    2.63   0.0137 *  
---
Signif. codes:  0 '***' 0.001 '**' 0.01 '*' 0.05 '.' 0.1 ' ' 1

Residual standard error: 3.875 on 28 degrees of freedom
Multiple R-squared:  0.9481,	Adjusted R-squared:  0.9444 
F-statistic: 255.9 on 2 and 28 DF,  p-value: < 2.2e-16
#+end_example

From the output we see that all parameter estimates are statistically
significant and we conclude that the mean response differs for trees
with =Tall = yes= and trees with =Tall = no=.

#+BEGIN_rem
We were somewhat disingenuous when we defined the dummy variable
=Tallyes= because, in truth, \(\mathsf{R}\) defines =Tallyes=
automatically without input from the user[fn:fn-contrast]. Indeed, the
author fit the model beforehand and wrote the discussion afterward
with the knowledge of what \(\mathsf{R}\) would do so that the output
the reader saw would match what (s)he had previously read. The way
that \(\mathsf{R}\) handles factors internally is part of a much
larger topic concerning /contrasts/, which falls outside the scope of
this book. The interested reader should see Neter et al
\cite{Neter1996} or Fox \cite{Fox1997} for more.
#+END_rem

[fn:fn-contrast] That is, \(\mathsf{R}\) by default handles contrasts
according to its internal settings which may be customized by the user
for fine control. Given that we will not investigate contrasts further
in this book it does not serve the discussion to delve into those
settings, either. The interested reader should check =?contrasts= for
details.

#+BEGIN_rem
In general, if an explanatory variable =foo= is qualitative with \(n\)
levels =bar1=, =bar2=, ..., =barn= then \(\mathsf{R}\) will by default
automatically define \(n-1\) indicator variables in the following way:
@@latex:\begin{eqnarray*} \mathtt{foobar2} & = & \begin{cases} 1, & \mbox{if }\mathtt{foo}=\mathtt{"bar2"},\\ 0, & \mbox{otherwise.}\end{cases},\,\ldots,\,\mathtt{foobarn}=\begin{cases} 1, & \mbox{if }\mathtt{foo}=\mathtt{"barn"},\\ 0, & \mbox{otherwise.}\end{cases} \end{eqnarray*}@@
The level =bar1= is represented by
\(\mathtt{foobar2}=\cdots=\mathtt{foobarn}=0\). We just need to make
sure that =foo= is stored as a factor and \(\mathsf{R}\) will take
care of the rest.
#+END_rem

*** Graphing the Regression Lines

We can see a plot of the two regression lines with the following
mouthful of code.

#+NAME: dummy-variable-trees
#+BEGIN_SRC R :exports both :results graphics :file fig/mlr-dummy-variable-trees.ps
treesTall <- split(trees, trees$Tall)
treesTall[["yes"]]$Fit <- predict(treesdummy.lm, treesTall[["yes"]])
treesTall[["no"]]$Fit <- predict(treesdummy.lm, treesTall[["no"]])
plot(Volume ~ Girth, data = trees)
points(Volume ~ Girth, data = treesTall[["yes"]], pch = 1)
points(Volume ~ Girth, data = treesTall[["no"]], pch = 2)
lines(Fit ~ Girth, data = treesTall[["yes"]])
lines(Fit ~ Girth, data = treesTall[["no"]])
#+END_SRC

#+NAME: fig-dummy-variable-trees
#+CAPTION[A dummy variable model for the =trees= data]: \small A dummy variable model for the =trees= data.
#+ATTR_LaTeX: :width 0.9\textwidth :placement [ht!]
#+RESULTS: dummy-variable-trees
[[file:fig/mlr-dummy-variable-trees.ps]]

It may look intimidating but there is reason to the madness. First we
=split= the =trees= data into two pieces, with groups determined by
the =Tall= variable. Next we add the fitted values to each piece via
=predict=. Then we set up a =plot= for the variables =Volume= versus
=Girth=, but we do not plot anything yet (=type = n=) because we want
to use different symbols for the two groups. Next we add =points= to
the plot for the =Tall = yes= trees and use an open circle for a plot
character (=pch = 1=), followed by =points= for the =Tall = no= trees
with a triangle character (=pch = 2=). Finally, we add regression
=lines= to the plot, one for each group.

There are other -- shorter -- ways to plot regression lines by groups,
namely the =scatterplot= function in the =car= package \cite{car} and
the =xyplot= function in the =lattice= package \cite{lattice}. We
elected to introduce the reader to the above approach since many
advanced plots in \(\mathsf{R}\) are done in a similar, consecutive
fashion.

** Partial /F/ Statistic
:PROPERTIES:
:CUSTOM_ID: sec-Partial-F-Statistic
:END:

We saw in Section [[#sub-mlr-Overall-F-Test]] how to test
\(H_{0}:\beta_{0}=\beta_{1}=\cdots=\beta_{p}=0\) with the overall
\(F\) statistic and we saw in Section [[#sub-mlr-Students-t-Tests]] how to
test \(H_{0}:\beta_{i}=0\) that a particular coefficient \(\beta_{i}\)
is zero. Sometimes, however, we would like to test whether a certain
part of the model is significant. Consider the regression model
\begin{equation}
Y=\beta_{0}+\beta_{1}x_{1}+\cdots+\beta_{j}x_{j}+\beta_{j+1}x_{j+1}+\cdots+\beta_{p}x_{p}+\epsilon,
\end{equation}
where \(j\geq1\) and \(p\geq2\). Now we wish to test the hypothesis
\begin{equation}
H_{0}:\beta_{j+1}=\beta_{j+2}=\cdots=\beta_{p}=0
\end{equation}
versus the alternative 
\begin{equation}
H_{1}:\mbox{at least one of $\beta_{j+1},\ \beta_{j+2},\ ,\ldots,\beta_{p}\neq0$}.
\end{equation}

The interpretation of \(H_{0}\) is that none of the variables
\(x_{j+1}\), ...,\(x_{p}\) is significantly related to \(Y\) and the
interpretation of \(H_{1}\) is that at least one of \(x_{j+1}\),
...,\(x_{p}\) is significantly related to \(Y\). In essence, for this
hypothesis test there are two competing models under consideration:
\begin{align}
\mbox{the full model:} & \quad y=\beta_{0}+\beta_{1}x_{1}+\cdots+\beta_{p}x_{p}+\epsilon,\\
\mbox{the reduced model:} & \quad y=\beta_{0}+\beta_{1}x_{1}+\cdots+\beta_{j}x_{j}+\epsilon,
\end{align}

Of course, the full model will always explain the data /better/ than
the reduced model, but does the full model explain the data
/significantly better/ than the reduced model? This question is
exactly what the partial \(F\) statistic is designed to answer.

We first calculate \(SSE_{f}\), the unexplained variation in the full
model, and \(SSE_{r}\), the unexplained variation in the reduced
model. We base our test on the difference \(SSE_{r}-SSE_{f}\) which
measures the reduction in unexplained variation attributable to the
variables \(x_{j+1}\), ..., \(x_{p}\). In the full model there are
\(p+1\) parameters and in the reduced model there are \(j+1\)
parameters, which gives a difference of \(p-j\) parameters (hence
degrees of freedom). The partial /F/ statistic is
\begin{equation}
F=\frac{(SSE_{r}-SSE_{f})/(p-j)}{SSE_{f}/(n-p-1)}.
\end{equation}
It can be shown when the regression assumptions hold under \(H_{0}\)
that the partial \(F\) statistic has an
\(\mathsf{f}(\mathtt{df1}=p-j,\,\mathtt{df2}=n-p-1)\) distribution. We
calculate the \(p\)-value of the observed partial \(F\) statistic and
reject \(H_{0}\) if the \(p\)-value is small.

*** How to do it with \(\mathsf{R}\)

The key ingredient above is that the two competing models are /nested/
in the sense that the reduced model is entirely contained within the
complete model. The way to test whether the improvement is significant
is to compute =lm= objects both for the complete model and the reduced
model then compare the answers with the =anova= function.

# +BEGIN_exampletoo
<<exa-mlr-trees-poly-no-rescale>> For the =trees= data, let us fit a
polynomial regression model and for the sake of argument we will
ignore our own good advice and fail to rescale the explanatory
variables.

#+BEGIN_SRC R :exports both :results output pp 
treesfull.lm <- lm(Volume ~ Girth + I(Girth^2) + Height + 
                   I(Height^2), data = trees)
summary(treesfull.lm)
#+END_SRC

#+RESULTS:
#+BEGIN_example

Call:
lm(formula = Volume ~ Girth + I(Girth^2) + Height + I(Height^2), 
    data = trees)

Residuals:
   Min     1Q Median     3Q    Max 
-4.368 -1.670 -0.158  1.792  4.358 

Coefficients:
             Estimate Std. Error t value Pr(>|t|)    
(Intercept) -0.955101  63.013630  -0.015    0.988    
Girth       -2.796569   1.468677  -1.904    0.068 .  
I(Girth^2)   0.265446   0.051689   5.135 2.35e-05 ***
Height       0.119372   1.784588   0.067    0.947    
I(Height^2)  0.001717   0.011905   0.144    0.886    
---
Signif. codes:  0 '***' 0.001 '**' 0.01 '*' 0.05 '.' 0.1 ' ' 1

Residual standard error: 2.674 on 26 degrees of freedom
Multiple R-squared:  0.9771,	Adjusted R-squared:  0.9735 
F-statistic:   277 on 4 and 26 DF,  p-value: < 2.2e-16
#+END_example

In this ill-formed model nothing is significant except =Girth= and
=Girth^2=. Let us continue down this path and suppose that we would
like to try a reduced model which contains nothing but =Girth= and
=Girth^2= (not even an =Intercept=). Our two models are now
\begin{align*} 
\mbox{the full model:} & \quad Y=\beta_{0}+\beta_{1}x_{1}+\beta_{2}x_{1}^{2}+\beta_{3}x_{2}+\beta_{4}x_{2}^{2}+\epsilon,\\
\mbox{the reduced model:} & \quad Y=\beta_{1}x_{1}+\beta_{2}x_{1}^{2}+\epsilon,
\end{align*}
We fit the reduced model with =lm= and store the results:

#+BEGIN_SRC R :exports code :results silent
treesreduced.lm <- lm(Volume ~ -1 + Girth + I(Girth^2), data = trees)
#+END_SRC

To delete the intercept from the model we used =-1= in the model
formula. Next we compare the two models with the =anova= function. The
convention is to list the models from smallest to largest.

#+BEGIN_SRC R :exports both :results output pp 
anova(treesreduced.lm, treesfull.lm)
#+END_SRC

#+RESULTS:
: Analysis of Variance Table
: 
: Model 1: Volume ~ -1 + Girth + I(Girth^2)
: Model 2: Volume ~ Girth + I(Girth^2) + Height + I(Height^2)
:   Res.Df    RSS Df Sum of Sq      F   Pr(>F)   
: 1     29 321.65                                
: 2     26 185.86  3    135.79 6.3319 0.002279 **
: ---
: Signif. codes:  0 '***' 0.001 '**' 0.01 '*' 0.05 '.' 0.1 ' ' 1

We see from the output that the complete model is highly significant
compared to the model that does not incorporate =Height= or the
=Intercept=. We wonder (with our tongue in our cheek) if the
=Height^2= term in the full model is causing all of the trouble. We
will fit an alternative reduced model that only deletes =Height^2=.

#+BEGIN_SRC R :exports both :results output pp 
treesreduced2.lm <- lm(Volume ~ Girth + I(Girth^2) + Height, 
                       data = trees)
anova(treesreduced2.lm, treesfull.lm)
#+END_SRC

#+RESULTS:
: Analysis of Variance Table
: 
: Model 1: Volume ~ Girth + I(Girth^2) + Height
: Model 2: Volume ~ Girth + I(Girth^2) + Height + I(Height^2)
:   Res.Df    RSS Df Sum of Sq      F Pr(>F)
: 1     27 186.01                           
: 2     26 185.86  1   0.14865 0.0208 0.8865

In this case, the improvement to the reduced model that is
attributable to =Height^2= is not significant, so we can delete
=Height^2= from the model with a clear conscience. We notice that the
/p-value/ for this latest partial \(F\) test is 0.8865, which seems to
be remarkably close to the /p-value/ we saw for the univariate /t/
test of =Height^2= at the beginning of this example. In fact, the
/p-values/ are /exactly/ the same. Perhaps now we gain some insight
into the true meaning of the univariate tests.

# +END_exampletoo

** Residual Analysis and Diagnostic Tools
:PROPERTIES:
:CUSTOM_ID: sec-Residual-Analysis-MLR
:END:

We encountered many, many diagnostic measures for simple linear
regression in Sections [[#sec-Residual-Analysis-SLR]] and [[#sec-Other-Diagnostic-Tools-SLR]]. All of these are valid in multiple linear regression, too, but
there are some slight changes that we need to make for the
multivariate case. We list these below, and apply them to the trees
example.

- Shapiro-Wilk, Breusch-Pagan, Durbin-Watson: :: unchanged from SLR,
     but we are now equipped to talk about the Shapiro-Wilk test
     statistic for the residuals. It is defined by the formula
     \begin{equation}
     W=\frac{\mathbf{a}^{\mathrm{T}}\mathbf{E}^{\ast}}{\mathbf{E}^{\mathrm{T}}\mathbf{E}},
     \end{equation}
     where \(\mathbf{E}^{\ast}\) is the sorted residuals and
     \(\mathbf{a}_{1\times\mathrm{n}}\) is defined by
     \begin{equation}
     \mathbf{a}=\frac{\mathbf{m}^{\mathrm{T}}\mathbf{V}^{-1}}{\sqrt{\mathbf{m}^{\mathrm{T}}\mathbf{V}^{-1}\mathbf{V}^{-1}\mathbf{m}}},
     \end{equation}
     where \(\mathbf{m}_{\mathrm{n}\times1}\) and
     \(\mathbf{V}_{\mathrm{n}\times\mathrm{n}}\) are the mean and
     covariance matrix, respectively, of the order statistics from an
     \(\mathsf{mvnorm}\left(\mathtt{mean}=\mathbf{0},\,\mathtt{sigma}=\mathbf{I}\right)\)
     distribution.
- Leverages: :: are defined to be the diagonal entries of the hat
                matrix \(\mathbf{H}\) (which is why we called them
                \(h_{ii}\) in Section
                [[#sub-mlr-point-est-regsurface]]). The sum of the
                leverages is \(\mbox{tr}(\mathbf{H})=p+1\). One rule
                of thumb considers a leverage extreme if it is larger
                than double the mean leverage value, which is
                \(2(p+1)/n\), and another rule of thumb considers
                leverages bigger than 0.5 to indicate high leverage,
                while values between 0.3 and 0.5 indicate moderate
                leverage.
- Standardized residuals: :: unchanged. Considered extreme if
     \(|R_{i}|>2\).
- Studentized residuals: :: compared to a
     \(\mathsf{t}(\mathtt{df}=n-p-2)\) distribution.
- DFBETAS: :: The formula is generalized to
   \begin{equation}
   (DFBETAS)_{j(i)}=\frac{b_{j}-b_{j(i)}}{S_{(i)}\sqrt{c_{jj}}},\quad j=0,\ldots p,\ i=1,\ldots,n,
   \end{equation}
   where \(c_{jj}\) is the \(j^{\mathrm{th}}\) diagonal entry of
   \((\mathbf{X}^{\mathrm{T}}\mathbf{X})^{-1}\). Values
   larger than one for small data sets or \(2/\sqrt{n}\)
   for large data sets should be investigated.
- DFFITS: :: unchanged. Larger than one in absolute value is
             considered extreme.
- Cook's D: :: compared to an \(\mathsf{f}(\mathtt{df1} = p +
               1,\,\mathtt{df2} = n - p - 1)\)
               distribution. Observations falling higher than the
               50\(^{\textrm{th}}\) percentile are extreme.  Note that
               plugging the value \(p=1\) into the formulas will
               recover all of the ones we saw in Chapter
               [[#cha-simple-linear-regression]].

** Additional Topics
:PROPERTIES:
:CUSTOM_ID: sec-Additional-Topics-MLR
:END:

*** Nonlinear Regression

We spent the entire chapter talking about the =trees= data, and all of
our models looked like =Volume ~ Girth + Height= or a variant of this
model. But let us think again: we know from elementary school that the
volume of a rectangle is \(V=lwh\) and the volume of a cylinder (which
is closer to what a black cherry tree looks like) is
\begin{equation}
V=\pi r^{2}h\quad \mbox{or}\quad V=4\pi dh,
\end{equation}
where \(r\) and \(d\) represent the radius and diameter of the tree,
respectively. With this in mind, it would seem that a more appropriate
model for \(\mu\) might be
\begin{equation}
\label{eq-trees-nonlin-reg}
\mu(x_{1},x_{2})=\beta_{0}x_{1}^{\beta_{1}}x_{2}^{\beta_{2}},
\end{equation}
where \(\beta_{1}\) and \(\beta_{2}\) are parameters to adjust for the
fact that a black cherry tree is not a perfect cylinder.

How can we fit this model? The model is not linear in the parameters
any more, so our linear regression methods will not work... or will
they? In the =trees= example we may take the logarithm of both sides
of Equation \eqref{eq-trees-nonlin-reg} to get
\begin{equation}
\mu^{\ast}(x_{1},x_{2})=\ln\left[\mu(x_{1},x_{2})\right]=\ln\beta_{0}+\beta_{1}\ln x_{1}+\beta_{2}\ln x_{2},
\end{equation}
and this new model \(\mu^{\ast}\) is linear in the parameters
\(\beta_{0}^{\ast}=\ln\beta_{0}\), \(\beta_{1}^{\ast}=\beta_{1}\) and
\(\beta_{2}^{\ast}=\beta_{2}\). We can use what we have learned to fit
a linear model =log(Volume) ~ log(Girth) + log(Height)=, and
everything will proceed as before, with one exception: we will need to
be mindful when it comes time to make predictions because the model
will have been fit on the log scale, and we will need to transform our
predictions back to the original scale (by exponentiating with =exp=)
to make sense.

#+BEGIN_SRC R :exports both :results output pp 
treesNonlin.lm <- lm(log(Volume) ~ log(Girth) + log(Height), data = trees)
summary(treesNonlin.lm)
#+END_SRC

#+RESULTS:
#+BEGIN_example

Call:
lm(formula = log(Volume) ~ log(Girth) + log(Height), data = trees)

Residuals:
      Min        1Q    Median        3Q       Max 
-0.168561 -0.048488  0.002431  0.063637  0.129223 

Coefficients:
            Estimate Std. Error t value Pr(>|t|)    
(Intercept) -6.63162    0.79979  -8.292 5.06e-09 ***
log(Girth)   1.98265    0.07501  26.432  < 2e-16 ***
log(Height)  1.11712    0.20444   5.464 7.81e-06 ***
---
Signif. codes:  0 '***' 0.001 '**' 0.01 '*' 0.05 '.' 0.1 ' ' 1

Residual standard error: 0.08139 on 28 degrees of freedom
Multiple R-squared:  0.9777,	Adjusted R-squared:  0.9761 
F-statistic: 613.2 on 2 and 28 DF,  p-value: < 2.2e-16
#+END_example

This is our best model yet (judging by \(R^{2}\) and
\(\overline{R}^{2}\)), all of the parameters are significant, it is
simpler than the quadratic or interaction models, and it even makes
theoretical sense. It rarely gets any better than that.

We may get confidence intervals for the parameters, but remember that
it is usually better to transform back to the original scale for
interpretation purposes:

#+BEGIN_SRC R :exports both :results output pp 
exp(confint(treesNonlin.lm))
#+END_SRC

#+RESULTS:
:                    2.5 %      97.5 %
: (Intercept) 0.0002561078 0.006783093
: log(Girth)  6.2276411645 8.468066317
: log(Height) 2.0104387829 4.645475188

(Note that we did not update the row labels of the matrix to show that
we exponentiated and so they are misleading as written.) We do
predictions just as before. Remember to transform the response
variable back to the original scale after prediction.

#+BEGIN_SRC R :exports both :results output pp 
new <- data.frame(Girth = c(9.1, 11.6, 12.5), Height = c(69, 74, 87))
exp(predict(treesNonlin.lm, newdata = new, interval = "confidence"))
#+END_SRC

#+RESULTS:
:        fit      lwr      upr
: 1 11.90117 11.25908 12.57989
: 2 20.82261 20.14652 21.52139
: 3 28.93317 27.03755 30.96169

The predictions and intervals are slightly different from those
calculated earlier, but they are close. Note that we did not need to
transform the =Girth= and =Height= arguments in the dataframe
=new=. All transformations are done for us automatically.

*** Real Nonlinear Regression

We saw with the =trees= data that a nonlinear model might be more
appropriate for the data based on theoretical considerations, and we
were lucky because the functional form of \(\mu\) allowed us to take
logarithms to transform the nonlinear model to a linear one. The same
trick will not work in other circumstances, however. We need
techniques to fit general models of the form
\begin{equation}
\mathbf{Y}=\mu(\mathbf{X})+\epsilon,
\end{equation}
where \(\mu\) is some crazy function that does not lend itself to
linear transformations.

There are a host of methods to address problems like these which are
studied in advanced regression classes. The interested reader should
see Neter /et al/ \cite{Neter1996} or Tabachnick and Fidell
\cite{Tabachnick2006}.

It turns out that John Fox has posted an Appendix to his book
\cite{Fox2002} which discusses some of the methods and issues
associated with nonlinear regression; see [[http://cran.r-project.org/doc/contrib/Fox-Companion/appendix.html][here]] for more.  Here is an
example of how it works, based on a question from R-help.

#+BEGIN_SRC R :exports both :results output pp
# fake data 
set.seed(1) 
x <- seq(from = 0, to = 1000, length.out = 200) 
y <- 1 + 2*(sin((2*pi*x/360) - 3))^2 + rnorm(200, sd = 2)
# plot(x, y)
acc.nls <- nls(y ~ a + b*(sin((2*pi*x/360) - c))^2, 
               start = list(a = 0.9, b = 2.3, c = 2.9))
summary(acc.nls)
#plot(x, fitted(acc.nls))
#+END_SRC

#+RESULTS:
#+BEGIN_example

Formula: y ~ a + b * (sin((2 * pi * x/360) - c))^2

Parameters:
  Estimate Std. Error t value Pr(>|t|)    
a  0.95884    0.23097   4.151 4.92e-05 ***
b  2.22868    0.37114   6.005 9.07e-09 ***
c  3.04343    0.08434  36.084  < 2e-16 ***
---
Signif. codes:  0 '***' 0.001 '**' 0.01 '*' 0.05 '.' 0.1 ' ' 1

Residual standard error: 1.865 on 197 degrees of freedom

Number of iterations to convergence: 3 
Achieved convergence tolerance: 6.508e-08
#+END_example

*** Multicollinearity
:PROPERTIES:
:CUSTOM_ID: sub-Multicollinearity
:END:

A multiple regression model exhibits /multicollinearity/ when two or
more of the explanatory variables are substantially correlated with
each other. We can measure multicollinearity by having one of the
explanatory play the role of "dependent variable" and regress it on
the remaining explanatory variables. The the \(R^{2}\) of the
resulting model is near one, then we say that the model is
multicollinear or shows multicollinearity.

Multicollinearity is a problem because it causes instability in the
regression model. The instability is a consequence of redundancy in
the explanatory variables: a high \(R^{2}\) indicates a strong
dependence between the selected independent variable and the
others. The redundant information inflates the variance of the
parameter estimates which can cause them to be statistically
insignificant when they would have been significant otherwise. To wit,
multicollinearity is usually measured by what are called /variance
inflation factors/.

Once multicollinearity has been diagnosed there are several approaches
to remediate it. Here are a couple of important ones.
- Principal Components Analysis. :: This approach casts out two or
     more of the original explanatory variables and replaces them with
     new variables, derived from the original ones, that are by design
     uncorrelated with one another. The redundancy is thus eliminated
     and we may proceed as usual with the new variables in
     hand. Principal Components Analysis is important for other
     reasons, too, not just for fixing multicollinearity problems.
- Ridge Regression. :: The idea of this approach is to replace the
     original parameter estimates with a different type of parameter
     estimate which is more stable under multicollinearity. The
     estimators are not found by ordinary least squares but rather a
     different optimization procedure which incorporates the variance
     inflation factor information.

We decided to omit a thorough discussion of multicollinearity because
we are not equipped to handle the mathematical details. Perhaps the
topic will receive more attention in a later edition.

- What to do when data are not normal
   - Bootstrap (see Chapter [[#cha-resampling-methods]]).

*** Akaike's Information Criterion

\[
AIC = -2\ln L + 2(p + 1)
\]

#+LaTeX: \newpage{}

** Exercises

#+LaTeX: \setcounter{thm}{0}

#+BEGIN_xca
<<xca-anova-equality>> Use Equations \eqref{eq-mlr-sse-matrix},
\eqref{eq-mlr-ssto-matrix}, and \eqref{eq-mlr-ssr-matrix} to prove the
Anova Equality: \[ SSTO = SSE + SSR.\]
#+END_xca
