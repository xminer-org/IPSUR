\chapter{Simple Linear Regression}
\label{sec-11}

\textbf{What do I want them to know?}

\begin{itemize}
\item basic philosophy of SLR and the regression assumptions
\item point and interval estimation of the model parameters, and how to
use it to make predictions
\item point and interval estimation of future observations from the model
\item regression diagnostics, including \( R^{2} \) and basic residual
analysis
\item the concept of influential versus outlying observations, and how to
tell the difference
\end{itemize}

\section{Basic Philosophy}
\label{sec-11-1}

Here we have two variables \(X\) and \(Y\). For our purposes, \(X\) is
not random (so we will write \(x\)), but \(Y\) is random. We believe
that \(Y\) depends in \emph{some} way on \(x\). Some typical examples of \(
(x,Y) \) pairs are

\begin{itemize}
\item \( x = \) study time and \( Y = \) score on a test.
\item \( x = \) height and \( Y = \) weight.
\item \( x = \) smoking frequency and \( Y = \) age of first heart attack.
\end{itemize}

Given information about the relationship between \(x\) and \(Y\), we
would like to \emph{predict} future values of \(Y\) for particular values
of \(x\). This turns out to be a difficult problem, so
instead we first tackle an easier problem: we estimate \( \mathbb{E} Y
\). How can we accomplish this? Well, we know that \(Y\) depends
somehow on \(x\), so it stands to reason that
\begin{equation}
\mathbb{E} Y = \mu(x),\ \mbox{a function of }x.
\end{equation}

\subsection{What does it all mean?}
\label{sec-11-1-1}
See Figure \texttt{fig-philosophy}. Shown in the figure is a solid line, the
regression line \index{regression line} \(\mu\), which in
this display has slope 0.5 and \emph{y}-intercept 2.5, that is, \( \mu(x) =
2.5 + 0.5x \). The intuition is that for each given value of \(x\), we
observe a random value of \(Y\) which is normally distributed with a
mean equal to the height of the regression line at that \(x\)
value. Normal densities are superimposed on the plot to drive this
point home; in principle, the densities stand outside of the page,
perpendicular to the plane of the paper. The figure shows three such
values of \(x\), namely, \( x = 1 \), \( x = 2.5 \), and \( x = 4
\). Not only do we assume that the observations at the three locations
are independent, but we also assume that their distributions have the
same spread. In mathematical terms this means that the normal
densities all along the line have identical standard deviations --
there is no "fanning out" or "scrunching in" of the normal densities
as \(x\) increases.

\label{exa-Speed-and-Stopping} \textbf{Speed and stopping distance of cars.} We
will use the data frame \texttt{cars} \index{Data
sets!cars@\texttt{cars}} from the \texttt{datasets} package
\cite{datasets}. It has two variables: \texttt{speed} and \texttt{dist}. We can take
a look at some of the values in the data frame:
\begin{verbatim}
head(cars)
\end{verbatim}

\begin{verbatim}
:   speed dist
: 1     4    2
: 2     4   10
: 3     7    4
: 4     7   22
: 5     8   16
: 6     9   10
\end{verbatim}

The \texttt{speed} represents how fast the car was going (\(x\)) in miles per
hour and \texttt{dist} (\(Y\)) measures how far it took the car to stop, in
feet. We can make a simple scatterplot of the data with the \texttt{qplot}
command in the \texttt{ggplot2} package \cite{ggplot2}.

\begin{verbatim}
plot(dist ~ speed, data = cars)
\end{verbatim}

\begin{figure}[ht!]
\centering
\includegraphics[width=0.9\textwidth]{fig/slr-carscatter.ps}
\caption[Scatterplot of \texttt{dist} versus \texttt{speed} for the \texttt{cars} data]{\label{fig-carscatter}\small A scatterplot of \texttt{dist} versus \texttt{speed} for the \texttt{cars} data.  There is clearly an upward trend to the plot which is approximately linear.}
\end{figure}

You can see the output in Figure \texttt{fig-Scatter-cars}, which was produced by the
following code.

\begin{verbatim}
qplot(speed, dist, data = cars)
\end{verbatim}

There is a pronounced upward trend to the data points, and the pattern
looks approximately linear. There does not appear to be substantial
fanning out of the points or extreme values.

\section{Estimation}
\label{sec-11-2}

\subsection{Point Estimates of the Parameters}
\label{sec-11-2-1}

Where is \( \mu(x) \)? In essence, we would like to "fit" a line to
the points. But how do we determine a "good" line? Is there a \emph{best}
line? We will use maximum likelihood \index{maximum
likelihood} to find it. We know:
\begin{equation}
Y_{i} = \beta_{0} + \beta_{1}x_{i} + \epsilon_{i},\quad i=1,\ldots,n,
\end{equation}
where the \( \epsilon_{i} \) are IID
\(\mathsf{norm}(\mathtt{mean}=0,\,\mathtt{sd}=\sigma) \). Thus \(
Y_{i}\sim\mathsf{norm}(\mathtt{mean}=\beta_{0}+\beta_{1}x_{i},\,\mathtt{sd}=\sigma),\
i=1,\ldots,n \). Furthermore, \( Y_{1},\ldots,Y_{n} \) are independent
-- but not identically distributed. The likelihood
function \index{likelihood function} is:
\begin{alignat}{1}
L(\beta_{0},\beta_{1},\sigma)= & \prod_{i=1}^{n}f_{Y_{i}}(y_{i}),\\
= & \prod_{i=1}^{n}(2\pi\sigma^{2})^{-1/2}\exp\left\{ \frac{-(y_{i}-\beta_{0}-\beta_{1}x_{i})^{2}}{2\sigma^{2}}\right\} ,\\
= & (2\pi\sigma^{2})^{-n/2}\exp\left\{ \frac{-\sum_{i=1}^{n}(y_{i}-\beta_{0}-\beta_{1}x_{i})^{2}}{2\sigma^{2}}\right\} .
\end{alignat}
We take the natural logarithm to get
\begin{equation}
\label{eq-regML-lnL}
\ln L(\beta_{0},\beta_{1},\sigma)=-\frac{n}{2}\ln(2\pi\sigma^{2})-\frac{\sum_{i=1}^{n}(y_{i}-\beta_{0}-\beta_{1}x_{i})^{2}}{2\sigma^{2}}.
\end{equation}
We would like to maximize this function of \( \beta_{0} \) and \(
\beta_{1} \). See Appendix \ref{sec-21-6} which tells us that
we should find critical points by means of the partial
derivatives. Let us start by differentiating with respect to
\(\beta_{0} \):
\begin{equation}
\frac{\partial}{\partial\beta_{0}}\ln L=0-\frac{1}{2\sigma^{2}}\sum_{i=1}^{n}2(y_{i}-\beta_{0}-\beta_{1}x_{i})(-1),
\end{equation}
and the partial derivative equals zero when \(
\sum_{i=1}^{n}(y_{i}-\beta_{0}-\beta_{1}x_{i}) = 0 \), that is, when
\begin{equation}
\label{eq-regML-a}
n \beta_{0} + \beta_{1} \sum_{i=1}^{n} x_{i} = \sum_{i = 1}^{n}y_{i}.
\end{equation}
Moving on, we next take the partial derivative of \( \ln L \)
(Equation \eqref{eq-regML-lnL}) with respect to \( \beta_{1} \) to get

\begin{alignat}{1}
\frac{\partial}{\partial \beta_{1}} \ln L = \ & 0 - \frac{1}{2\sigma^{2}} \sum_{i=1}^{n} 2 (y_{i} - \beta_{0} - \beta_{1} x_{i})(-x_{i}),\\ = & \frac{1}{\sigma^{2}}\sum_{i = 1}^{n}\left(x_{i} y_{i} - \beta_{0}x_{i} - \beta_{1}x_{i}^{2}\right),
\end{alignat}
and this equals zero when the last sum equals zero, that is, when
\begin{equation}
\label{eq-regML-b}
\beta_{0} \sum_{i = 1}^{n}x_{i} + \beta_{1} \sum_{i = 1}^{n}x_{i}^{2} = \sum_{i = 1}^{n}x_{i}y_{i}.
\end{equation}
Solving the system of equations \eqref{eq-regML-a} and \eqref{eq-regML-b}
\begin{eqnarray}
n\beta_{0} + \beta_{1}\sum_{i = 1}^{n}x_{i} & = & \sum_{i = 1}^{n}y_{i}\\
\beta_{0}\sum_{i = 1}^{n}x_{i}+\beta_{1}\sum_{i = 1}^{n}x_{i}^{2} & = & \sum_{i = 1}^{n}x_{i}y_{i}
\end{eqnarray}
for \( \beta_{0} \) and \( \beta_{1} \) (in Exercise \ref{xca-find-mles-SLR}) gives
\begin{equation}
\label{eq-regline-slope-formula}
\hat{\beta}_{1} = \frac{\sum_{i = 1}^{n}x_{i}y_{i} - \left.\left(\sum_{i = 1}^{n}x_{i}\right)\left(\sum_{i = 1}^{n}y_{i}\right)\right] n}{\sum_{i = 1}^{n}x_{i}^{2} - \left.\left(\sum_{i = 1}^{n}x_{i}\right)^{2}\right/ n}
\end{equation}
and
\begin{equation}
\hat{\beta}_{0} = \overline{y} - \hat{\beta}_{1}\overline{x}.
\end{equation}

The conclusion? To estimate the mean line 
\begin{equation}
\mu(x) = \beta_{0} + \beta_{1}x,
\end{equation}
we use the "line of best fit"
\begin{equation}
\hat{\mu}(x) = \hat{\beta}_{0} + \hat{\beta}_{1}x,
\end{equation}
where \(\hat{\beta}_{0}\) and \(\hat{\beta}_{1}\) are given as
above. For notation we will usually write \( b_{0} = \hat{\beta_{0}}
\) and \( b_{1}=\hat{\beta_{1}} \) so that \( \hat{\mu}(x) = b_{0} +
b_{1}x \).

\begin{rem}
The formula for \( b_{1} \) in Equation \eqref{eq-regline-slope-formula} gets
the job done but does not really make any sense. There are many
equivalent formulas for \( b_{1} \) that are more intuitive, or at the
least are easier to remember. One of the author's favorites is
\begin{equation}
\label{eq-sample-correlation-formula}
b_{1} = r\frac{s_{y}}{s_{x}},
\end{equation}
where \(r\), \( s_{y} \), and \( s_{x} \) are the sample correlation
coefficient and the sample standard deviations of the \(Y\) and \(x\)
data, respectively. See Exercise
\ref{xca-show-alternate-slope-formula}. Also, notice the similarity between
Equation \eqref{eq-sample-correlation-formula} and Equation
\eqref{eq-population-slope-slr}.
\end{rem}

\subsubsection{How to do it with \(\mathsf{R}\)}
\label{sec-11-2-1-1}

Here we go. \(\mathsf{R}\) will calculate the linear regression line
with the \texttt{lm} function. We will store the result in an object which we
will call \texttt{cars.lm}. Here is how it works:

\begin{verbatim}
cars.lm <- lm(dist ~ speed, data = cars)
\end{verbatim}

The first part of the input to the \texttt{lm} function, \texttt{dist \textasciitilde{} speed}, is a
\emph{model formula}, read like this: \texttt{dist} is described (or modeled) by
\texttt{speed}. The \texttt{data = cars} argument tells \(\mathsf{R}\) where to look
for the variables quoted in the model formula. The output object
\texttt{cars.lm} contains a multitude of information. Let's first take a look
at the coefficients of the fitted regression line, which are extracted
by the \texttt{coef} function (alternatively, we could just type \texttt{cars.lm} to
see the same thing):

\begin{verbatim}
coef(cars.lm)
\end{verbatim}

\begin{verbatim}
: (Intercept)       speed 
:  -17.579095    3.932409
\end{verbatim}

The parameter estimates \( b_{0} \) and \( b_{1} \) for the intercept
and slope, respectively, are shown above. 

It is good practice to visually inspect the data with the regression
line added to the plot. To do this we first scatterplot the original
data and then follow with a call to the \texttt{abline} function. The inputs
to \texttt{abline} are the coefficients of \texttt{cars.lm}; see Figure
\texttt{fig-Scatter-cars-regline}.

\begin{verbatim}
ggplot(cars, aes(x = speed, y = dist)) + 
  geom_point(shape = 19) + 
  geom_smooth(method = lm, se = FALSE)
\end{verbatim}

\begin{figure}[ht!]
\centering
\includegraphics[width=0.9\textwidth]{fig/slr-carline.ps}
\caption[Scatterplot with added regression line for the \texttt{cars} data]{\label{fig-carline}\small A scatterplot with an added regression line for the \texttt{cars} data.}
\end{figure}

To calculate points on the regression line we may simply plug the
desired \(x\) value(s) into \( \hat{\mu} \), either by hand, or with
the \texttt{predict} function. The inputs to \texttt{predict} are the fitted linear
model object, \texttt{cars.lm}, and the desired \(x\) value(s) represented by
a data frame. See the example below.

\label{exa-regline-cars-interpret} Using the regression line for the
\texttt{cars} data:
\begin{enumerate}
\item What is the meaning of \( \mu(60) = \beta_{0} + \beta_{1}(8) \)?
This represents the average stopping distance (in feet) for a car
going 8 mph.
\item Interpret the slope \(\beta_{1}\). The true slope \(\beta_{1}\)
represents the increase in average stopping distance for each mile
per hour faster that the car drives. In this case, we estimate the
car to take approximately 3.93 additional feet
to stop for each additional mph increase in speed.
\item Interpret the intercept \( \beta_{0} \). This would represent the
mean stopping distance for a car traveling 0 mph (which our
regression line estimates to be $\backslash$( -17.58. Of
course, this interpretation does not make any sense for this
example, because a car travelling 0 mph takes 0 ft to stop (it was
not moving in the first place)! What went wrong? Looking at the
data, we notice that the smallest speed for which we have measured
data is 4 mph. Therefore, if we predict what would happen for
slower speeds then we would be \emph{extrapolating}, a dangerous
practice which often gives nonsensical results.
\end{enumerate}

\subsection{Point Estimates of the Regression Line}
\label{sec-11-2-2}

We said at the beginning of the chapter that our goal was to estimate
\( \mu = \mathbb{E} Y \), and the arguments in Section
\ref{sec-11-2-1} showed how to obtain an estimate \(
\hat{\mu} \) of \( \mu \) when the regression assumptions hold. Now we
will reap the benefits of our work in more ways than we previously
disclosed. Given a particular value \(x_{0}\), there are two values we
would like to estimate:
\begin{enumerate}
\item the mean value of \(Y\) at \(x_{0}\), and
\item a future value of \(Y\) at \(x_{0}\). The first is a number,
\(\mu(x_{0})\), and the second is a random variable, \(Y(x_{0})\),
but our point estimate is the same for both: \(\hat{\mu}(x_{0})\).
\end{enumerate}

\label{exa-regline-cars-pe-8mph} We may use the regression line to obtain
a point estimate of the mean stopping distance for a car traveling 8
mph: \( \hat{\mu}(15) = b_{0} + (8) (b_{1})\) which is approximately
13.88. We would also use 13.88 as a point estimate for the stopping
distance of a future car traveling 8 mph.


Note that we actually have observed data for a car traveling 8 mph;
its stopping distance was 16 ft as listed in the fifth row of the
\texttt{cars} data (which we saw in Example \ref{exa-Speed-and-Stopping}).

\begin{verbatim}
cars[5, ]
\end{verbatim}

\begin{verbatim}
:   speed dist
: 5     8   16
\end{verbatim}

There is a special name for estimates \( \hat{\mu}(x_{0}) \) when \(
x_{0} \) matches an observed value \(x_{i}\) from the data set. They
are called \emph{fitted values}, they are denoted by \(\hat{Y}_{1}\),
\(\hat{Y}_{2}\), \ldots{}, \(\hat{Y}_{n}\) (ignoring repetition), and they
play an important role in the sections that follow.

In an abuse of notation we will sometimes write \(\hat{Y}\) or
\(\hat{Y}(x_{0})\) to denote a point on the regression line even when
\(x_{0}\) does not belong to the original data if the context of the
statement obviates any danger of confusion.

We saw in Example \ref{exa-regline-cars-interpret} that spooky things can
happen when we are cavalier about point estimation. While it is
usually acceptable to predict/estimate at values of \(x_{0}\) that
fall within the range of the original \(x\) data, it is reckless to
use \(\hat{\mu}\) for point estimates at locations outside that
range. Such estimates are usually worthless. \emph{Do not extrapolate}
unless there are compelling external reasons, and even then, temper it
with a good deal of caution.

\subsubsection{How to do it with \(\mathsf{R}\)}
\label{sec-11-2-2-1}

The fitted values are automatically computed as a byproduct of the
model fitting procedure and are already stored as a component of the
\texttt{cars.lm} object. We may access them with the \texttt{fitted} function (we
only show the first five entries):

\begin{verbatim}
fitted(cars.lm)[1:5]
\end{verbatim}

\begin{verbatim}
:         1         2         3         4         5 
: -1.849460 -1.849460  9.947766  9.947766 13.880175
\end{verbatim}

Predictions at \(x\) values that are not necessarily part of the
original data are done with the \texttt{predict} function. The first argument
is the original \texttt{cars.lm} object and the second argument \texttt{newdata}
accepts a dataframe (in the same form that was used to fit \texttt{cars.lm})
that contains the locations at which we are seeking predictions. Let
us predict the average stopping distances of cars traveling 6 mph, 8
mph, and 21 mph:

\begin{verbatim}
predict(cars.lm, newdata = data.frame(speed = c(6, 8, 21)))
\end{verbatim}

\begin{verbatim}
:         1         2         3 
:  6.015358 13.880175 65.001489
\end{verbatim}

Note that there were no observed cars that traveled 6 mph or 21
mph. Also note that our estimate for a car traveling 8 mph matches the
value we computed by hand in Example \ref{exa-regline-cars-pe-8mph}.

\subsection{Mean Square Error and Standard Error}
\label{sec-11-2-3}

To find the MLE of \(\sigma^{2}\) we consider the partial derivative
\begin{equation}
\frac{\partial}{\partial\sigma^{2}}\ln L=\frac{n}{2\sigma^{2}}-\frac{1}{2(\sigma^{2})^{2}}\sum_{i=1}^{n}(y_{i}-\beta_{0}-\beta_{1}x_{i})^{2},
\end{equation}
and after plugging in \(\hat{\beta}_{0}\) and \(\hat{\beta}_{1}\) and
setting equal to zero we get
\begin{equation}
\hat{\sigma^{2}}=\frac{1}{n}\sum_{i=1}^{n}(y_{i}-\hat{\beta}_{0}-\hat{\beta}_{1}x_{i})^{2}=\frac{1}{n}\sum_{i=1}^{n}[y_{i}-\hat{\mu}(x_{i})]^{2}.
\end{equation}
We write \(\hat{Yi}=\hat{\mu}(x_{i})\), and we let
\(E_{i}=Y_{i}-\hat{Y_{i}}\) be the \(i^{\mathrm{th}}\) \emph{residual}. We
see
\begin{equation}
n\hat{\sigma^{2}}=\sum_{i=1}^{n}E_{i}^{2}=SSE=\mbox{ the sum of squared errors.}
\end{equation}
For a point estimate of \(\sigma^{2}\) we use the \emph{mean square error}
\(S^{2}\) defined by
\begin{equation}
S^{2}=\frac{SSE}{n-2},
\end{equation}
and we estimate \(\sigma\) with the \emph{standard error}
\(S=\sqrt{S^{2}}\).

\subsubsection{How to do it with \(\mathsf{R}\)}
\label{sec-11-2-3-1}

The residuals for the model may be obtained with the \texttt{residuals}
function; we only show the first few entries in the interest of space:

\begin{verbatim}
residuals(cars.lm)[1:5]
\end{verbatim}

\begin{verbatim}
:         1         2         3         4         5 
:  3.849460 11.849460 -5.947766 12.052234  2.119825
\end{verbatim}

In the last section, we calculated the fitted value for \(x=8\) and
found it to be approximately \( \hat{\mu}(8) \approx\) 
13.88. Now, it turns out that there was only one recorded observation
at \(x = 8\), and we have seen this value in the output of
\texttt{head(cars)} in Example \ref{exa-Speed-and-Stopping}; it was \(\mathtt{dist}
= 16\) ft for a car with \( \mathtt{speed} = 8 \) mph. Therefore, the
residual should be \(E = Y - \hat{Y}\) which is \(E \approx 16 - \)
13.88. Now take a look at the last entry of
\texttt{residuals(cars.lm)}, above. It is not a coincidence.

The estimate \(S\) for \(\sigma\) is called the \texttt{Residual standard
error} and for the \texttt{cars} data is shown a few lines up on the
\texttt{summary(cars.lm)} output (see How to do it with \(\mathsf{R}\) in
Section \ref{sec-11-2-4}). We may read it from there to be \(
S\approx\) 15.38, or we can access it directly from the
\texttt{summary} object.

\begin{verbatim}
carsumry <- summary(cars.lm)
carsumry$sigma
\end{verbatim}

\begin{verbatim}
: [1] 15.37959
\end{verbatim}

\subsection{Interval Estimates of the Parameters}
\label{sec-11-2-4}

We discussed general interval estimation in Chapter \ref{sec-9}. There
we found that we could use what we know about the sampling
distribution of certain statistics to construct confidence intervals
for the parameter being estimated. We will continue in that vein, and
to get started we will determine the sampling distributions of the
parameter estimates, \(b_{1}\) and \(b_{0}\).

To that end, we can see from Equation \eqref{eq-regline-slope-formula} (and it
is made clear in Chapter \ref{sec-12}) that \(b_{1}\) is
just a linear combination of normally distributed random variables, so
\(b_{1}\) is normally distributed too. Further, it can be shown that
\begin{equation}
b_{1}\sim\mathsf{norm}\left(\mathtt{mean}=\beta_{1},\,\mathtt{sd}=\sigma_{b_{1}}\right)
\end{equation}
where
\begin{equation}
\sigma_{b_{1}}=\frac{\sigma}{\sqrt{\sum_{i=1}^{n}(x_{i}-\overline{x})^{2}}}
\end{equation}
is called \emph{the standard error of} \(b_{1}\) which unfortunately
depends on the unknown value of \(\sigma\). We do not lose heart,
though, because we can estimate \(\sigma\) with the standard error
\(S\) from the last section. This gives us an estimate \(S_{b_{1}}\)
for \(\sigma_{b_{1}}\) defined by
\begin{equation}
S_{b_{1}}=\frac{S}{\sqrt{\sum_{i=1}^{n}(x_{i}-\overline{x})^{2}}}.
\end{equation}

Now, it turns out that \(b_{0}\), \(b_{1}\), and \(S\) are mutually
independent (see the footnote in Section \ref{sec-12-2-4}). Therefore, the quantity
\begin{equation}
T=\frac{b_{1}-\beta_{1}}{S_{b_{1}}}
\end{equation}
has a \(\mathsf{t}(\mathtt{df}=n-2)\) distribution and a \(100(1 -
\alpha)\% \) confidence interval for \(\beta_{1}\) is given by
\begin{equation}
b_{1}\pm\mathsf{t}_{\alpha/2}(\mathtt{df}=n-1)\, S_{b_{1}.}
\end{equation}

It is also sometimes of interest to construct a confidence interval
for \(\beta_{0}\) in which case we will need the sampling distribution
of \(b_{0}\). It is shown in Chapter \ref{sec-12} that
\begin{equation}
b_{0}\sim\mathsf{norm}\left(\mathtt{mean}=\beta_{0},\,\mathtt{sd}=\sigma_{b_{0}}\right),
\end{equation}
where \(\sigma_{b_{0}}\) is given by
\begin{equation}
\sigma_{b_{0}}=\sigma\sqrt{\frac{1}{n}+\frac{\overline{x}^{2}}{\sum_{i=1}^{n}(x_{i}-\overline{x})^{2}}},
\end{equation}
and which we estimate with the \(S_{b_{0}}\) defined by
\begin{equation}
S_{b_{0}}=S\sqrt{\frac{1}{n}+\frac{\overline{x}^{2}}{\sum_{i=1}^{n}(x_{i}-\overline{x})^{2}}}.
\end{equation}
Thus the quantity
\begin{equation}
T=\frac{b_{0}-\beta_{0}}{S_{b_{0}}}
\end{equation}
has a \(\mathsf{t}(\mathtt{df}=n-2)\) distribution and a
\(100(1-\alpha)\%\) confidence interval for \(\beta_{0}\) is given by
\begin{equation}
b_{0}\pm\mathsf{t}_{\alpha/2}(\mathtt{df}=n-1)\, S_{b_{0}}.
\end{equation}

\subsubsection{How to do it with \(\mathsf{R}\)}
\label{sec-11-2-4-1}

Let us take a look at the output from \texttt{summary(cars.lm)}:

\begin{verbatim}
summary(cars.lm)
\end{verbatim}

\begin{verbatim}
Call:
lm(formula = dist ~ speed, data = cars)

Residuals:
    Min      1Q  Median      3Q     Max 
-29.069  -9.525  -2.272   9.215  43.201 

Coefficients:
            Estimate Std. Error t value Pr(>|t|)    
(Intercept) -17.5791     6.7584  -2.601   0.0123 *  
speed         3.9324     0.4155   9.464 1.49e-12 ***
---
Signif. codes:  0 '***' 0.001 '**' 0.01 '*' 0.05 '.' 0.1 ' ' 1

Residual standard error: 15.38 on 48 degrees of freedom
Multiple R-squared:  0.6511,	Adjusted R-squared:  0.6438 
F-statistic: 89.57 on 1 and 48 DF,  p-value: 1.49e-12
\end{verbatim}

In the \texttt{Coefficients} section we find the parameter estimates and
their respective standard errors in the second and third columns; the
other columns are discussed in Section \ref{sec-11-3}. If we
wanted, say, a 95\% confidence interval for \(\beta_{1}\) we could use
\( b_{1} = \) 3.932 and \( S_{b_{1}} = \) 
0.416 together with a \( \mathsf{t}_{0.025}(\mathtt{df}=23) \)
critical value to calculate \( b_{1} \pm
\mathsf{t}_{0.025}(\mathtt{df} = 23) \cdot S_{b_{1}} \).  Or, we could
use the \texttt{confint} function.

\begin{verbatim}
confint(cars.lm)
\end{verbatim}

\begin{verbatim}
:                  2.5 %    97.5 %
: (Intercept) -31.167850 -3.990340
: speed         3.096964  4.767853
\end{verbatim}

With 95\% confidence, the random interval 3.097 to
4.768 covers the parameter \(\beta_{1}\).

\subsection{Interval Estimates of the Regression Line}
\label{sec-11-2-5}

We have seen how to estimate the coefficients of regression line with
both point estimates and confidence intervals. We even saw how to
estimate a value \(\hat{\mu}(x)\) on the regression line for a given
value of \(x\), such as \(x=15\).

But how good is our estimate \(\hat{\mu}(15)\)? How much confidence do
we have in \emph{this} estimate? Furthermore, suppose we were going to
observe another value of \(Y\) at \(x=15\). What could we say?

Intuitively, it should be easier to get bounds on the mean (average)
value of \(Y\) at \(x_{0}\) -- called a \emph{confidence interval for the
mean value of} \(Y\) \emph{at} \(x_{0}\) -- than it is to get bounds on a
future observation of \(Y\) (called a \emph{prediction interval for} \(Y\)
\emph{at} \(x_{0}\)). As we shall see, the intuition serves us well and
confidence intervals are shorter for the mean value, longer for the
individual value.

Our point estimate of \(\mu(x_{0})\) is of course
\(\hat{Y}=\hat{Y}(x_{0})\), so for a confidence interval we will need
to know \(\hat{Y}\)'s sampling distribution. It turns out (see Section
) that \(\hat{Y}=\hat{\mu}(x_{0})\) is distributed
\begin{equation}
\hat{Y}\sim\mathsf{norm}\left(\mathtt{mean}=\mu(x_{0}),\:\mathtt{sd}=\sigma\sqrt{\frac{1}{n}+\frac{(x_{0}-\overline{x})^{2}}{\sum_{i=1}^{n}(x_{i}-\overline{x})^{2}}}\right).
\end{equation}

Since \(\sigma\) is unknown we estimate it with \(S\) (we should
expect the appearance of a \(\mathsf{t}(\mathtt{df}=n-2)\)
distribution in the near future).

A \( 100(1-\alpha)\% \) \emph{confidence interval (CI) for} \(\mu(x_{0})\)
is given by
\begin{equation}
\label{eq-SLR-conf-int-formula}
\hat{Y}\pm\mathsf{t}_{\alpha/2}(\mathtt{df}=n-2)\, S\sqrt{\frac{1}{n}+\frac{(x_{0}-\overline{x}^{2})}{\sum_{i=1}^{n}(x_{i}-\overline{x})^{2}}}.
\end{equation}
Prediction intervals are a little bit different. In order to find
confidence bounds for a new observation of \(Y\) (we will denote it
\(Y_{\mbox{new}}\)) we use the fact that
\begin{equation}
Y_{\mbox{new}}\sim\mathtt{norm}\left(\mathtt{mean}=\mu(x_{0}),\,\mathtt{sd}=\sigma\sqrt{1+\frac{1}{n}+\frac{(x_{0}-\overline{x})^{2}}{\sum_{i=1}^{n}(x_{i}-\overline{x})^{2}}}\right).
\end{equation}
Of course, \(\sigma\) is unknown so we estimate it with \(S\) and a \(
100(1-\alpha)\% \) prediction interval (PI) for a future value of
\(Y\) at \(x_{0}\) is given by
\begin{equation}
\label{eq-SLR-pred-int-formula}
\hat{Y}(x_{0})\pm\mathsf{t}_{\alpha/2}(\mathtt{df}=n-1)\: S\,\sqrt{1+\frac{1}{n}+\frac{(x_{0}-\overline{x})^{2}}{\sum_{i=1}^{n}(x_{i}-\overline{x})^{2}}}.
\end{equation}
We notice that the prediction interval in Equation
\eqref{eq-SLR-pred-int-formula} is wider than the confidence interval
in Equation \eqref{eq-SLR-conf-int-formula}, as we expected at the
beginning of the section.

\subsubsection{How to do it with \(\mathsf{R}\)}
\label{sec-11-2-5-1}

Confidence and prediction intervals are calculated in \(\mathsf{R}\)
with the \texttt{predict} \index{predict@\texttt{predict}} function, which we
encountered in Section \ref{sec-11-2-2}. There we neglected to
take advantage of its additional \texttt{interval} argument. The general
syntax follows.


We will find confidence and prediction intervals for the stopping
distance of a car travelling 5, 6, and 21 mph (note from the graph
that there are no collected data for these speeds). We have computed
\texttt{cars.lm} earlier, and we will use this for input to the \texttt{predict}
function. Also, we need to tell \(\mathsf{R}\) the values of \(x_{0}\)
at which we want the predictions made, and store the \(x_{0}\) values
in a data frame whose variable is labeled with the correct name. \emph{This
is important}.

\begin{verbatim}
new <- data.frame(speed = c(5, 6, 21))
\end{verbatim}

Next we instruct \(\mathsf{R}\) to calculate the intervals. Confidence
intervals are given by

\begin{verbatim}
predict(cars.lm, newdata = new, interval = "confidence")
\end{verbatim}

\begin{verbatim}
:         fit       lwr      upr
: 1  2.082949 -7.644150 11.81005
: 2  6.015358 -2.973341 15.00406
: 3 65.001489 58.597384 71.40559
\end{verbatim}

Prediction intervals are given by

\begin{verbatim}
predict(cars.lm, newdata = new, interval = "prediction")
\end{verbatim}

\begin{verbatim}
:         fit       lwr      upr
: 1  2.082949 -30.33359 34.49948
: 2  6.015358 -26.18731 38.21803
: 3 65.001489  33.42257 96.58040
\end{verbatim}



The type of interval is dictated by the \texttt{interval} argument (which is
\texttt{none} by default), and the default confidence level is 95\% (which
can be changed with the \texttt{level} argument).


Using the \texttt{cars} data,
\begin{enumerate}
\item Report a point estimate of and a 95\% confidence interval for the
mean stopping distance for a car travelling 5 mph.  The fitted
value for \(x = 5\) is 2.08, so a point
estimate would be 2.08 ft. The 95\% CI is
given by -7.64 to 11.81, so
with 95\% confidence the mean stopping distance lies somewhere
between -7.64 ft and 11.81 ft.
\item Report a point prediction for and a 95\% prediction interval for the
stopping distance of a hypothetical car travelling 21 mph.  The
fitted value for \(x = 21\) is 65, so a point
prediction for the stopping distance is 65 ft. The
95\% PI is 33.42 to 96.58, so with
95\% confidence we may assert that the hypothetical stopping
distance for a car travelling 21 mph would lie somewhere between
33.42 ft and 96.58 ft.
\end{enumerate}

\subsection{Graphing the Confidence and Prediction Bands}
\label{sec-11-2-6}

We earlier guessed that a bound on the value of a single new
observation would be inherently less certain than a bound for an
average (mean) value; therefore, we expect the CIs for the mean to be
tighter than the PIs for a new observation. A close look at the
standard deviations in Equations \eqref{eq-SLR-conf-int-formula} and
\eqref{eq-SLR-pred-int-formula} confirms our guess, but we would like
to see a picture to drive the point home.

We may plot the confidence and prediction intervals with one fell
swoop using the \texttt{ci.plot} function from the \texttt{HH} package
\cite{HH}. The graph is displayed in Figure \ref{fig-carscipi}.

\begin{verbatim}
library(HH)
ci.plot(cars.lm)
\end{verbatim}

Notice that the bands curve outward from the regression line as the
\(x\) values move away from the center. This is expected once we
notice the \((x_{0}-\overline{x})^{2}\) term in the standard deviation
formulas in Equations \eqref{eq-SLR-conf-int-formula} and
\eqref{eq-SLR-pred-int-formula}.

\begin{verbatim}
print(ci.plot(cars.lm))
\end{verbatim}

\begin{figure}[ht!]
\centering
\includegraphics[width=0.9\textwidth]{fig/slr-carscipi.ps}
\caption[Scatterplot with confidence/prediction bands for the \texttt{cars} data]{\label{fig-carscipi}\small A scatterplot with confidence/prediction bands for the \texttt{cars} data.}
\end{figure}

\section{Model Utility and Inference}
\label{sec-11-3}

\subsection{Hypothesis Tests for the Parameters}
\label{sec-11-3-1}

Much of the attention of SLR is directed toward \(\beta_{1}\) because
when \( \beta_{1}\neq 0 \) the mean value of \(Y\) increases (or
decreases) as \(x\) increases. It is really boring when
\(\beta_{1}=0\), because in that case the mean value of \(Y\) remains
the same, regardless of the value of \(x\) (when the regression
assumptions hold, of course). It is thus very important to decide
whether or not \( \beta_{1} = 0 \). We address the question with a
statistical test of the null hypothesis \(H_{0}:\beta_{1}=0\) versus
the alternative hypothesis \(H_{1}:\beta_{1}\neq0\), and to do that we
need to know the sampling distribution of \(b_{1}\) when the null
hypothesis is true.

To this end we already know from Section \ref{sec-11-2-4} that
the quantity

\begin{equation} 
T=\frac{b_{1}-\beta_{1}}{S_{b_{1}}}
\end{equation}
has a \(\mathsf{t}(\mathtt{df}=n-2)\) distribution; therefore, when
\(\beta_{1}=0\) the quantity \(b_{1}/S_{b_{1}}\) has a
\(\mathsf{t}(\mathtt{df}=n-2)\) distribution and we can compute a
\(p\)-value by comparing the observed value of \(b_{1}/S{}_{b_{1}}\)
with values under a \(\mathsf{t}(\mathtt{df}=n-2)\) curve.

Similarly, we may test the hypothesis \(H_{0}:\beta_{0}=0\) versus the
alternative \(H_{1}:\beta_{0}\neq0\) with the statistic
\(T=b_{0}/S_{b_{0}}\), where \(S_{b_{0}}\) is given in Section \ref{sec-11-2-4}. The test is conducted the same way as for
\(\beta_{1}\).

\subsubsection{How to do it with \(\mathsf{R}\)}
\label{sec-11-3-1-1}

Let us take another look at the output from \texttt{summary(cars.lm)}:

\begin{verbatim}
summary(cars.lm)
\end{verbatim}

\begin{verbatim}
Call:
lm(formula = dist ~ speed, data = cars)

Residuals:
    Min      1Q  Median      3Q     Max 
-29.069  -9.525  -2.272   9.215  43.201 

Coefficients:
            Estimate Std. Error t value Pr(>|t|)    
(Intercept) -17.5791     6.7584  -2.601   0.0123 *  
speed         3.9324     0.4155   9.464 1.49e-12 ***
---
Signif. codes:  0 '***' 0.001 '**' 0.01 '*' 0.05 '.' 0.1 ' ' 1

Residual standard error: 15.38 on 48 degrees of freedom
Multiple R-squared:  0.6511,	Adjusted R-squared:  0.6438 
F-statistic: 89.57 on 1 and 48 DF,  p-value: 1.49e-12
\end{verbatim}

In the \texttt{Coefficients} section we find the \(t\) statistics and the
\(p\)-values associated with the tests that the respective parameters
are zero in the fourth and fifth columns. Since the \(p\)-values are
(much) less than 0.05, we conclude that there is strong evidence that
the parameters \(\beta_{1}\neq0\) and \(\beta_{0}\neq0\), and as such,
we say that there is a statistically significant linear relationship
between \texttt{dist} and \texttt{speed}.

\subsection{Simple Coefficient of Determination}
\label{sec-11-3-2}

It would be nice to have a single number that indicates how well our
linear regression model is doing, and the \emph{simple coefficient of
determination} is designed for that purpose. In what follows, we
observe the values \(Y_{1}\), \(Y_{2}\), \ldots{},\(Y_{n}\), and the goal
is to estimate \(\mu(x_{0})\), the mean value of \(Y\) at the location
\(x_{0}\).

If we disregard the dependence of \(Y\) and \(x\) and base our
estimate only on the \(Y\) values then a reasonable choice for an
estimator is just the MLE of \(\mu\), which is \(\overline{Y}\). Then
the errors incurred by the estimate are just \(Y_{i}-\overline{Y}\)
and the variation about the estimate as measured by the sample
variance is proportional to
\begin{equation}
SSTO=\sum_{i=1}^{n}(Y_{i}-\overline{Y})^{2}.
\end{equation}
The acronym \(SSTO\) stands for \emph{total sum of squares}.  And we have
additional information, namely, we have values \(x_{i}\) associated
with each value of \(Y_{i}\). We have seen that this information leads
us to the estimate \(\hat{Y_{i}}\) and the errors incurred are just
the residuals, \(E_{i}=Y_{i}-\hat{Y_{i}}\). The variation associated
with these errors can be measured with
\begin{equation}
SSE=\sum_{i=1}^{n}(Y_{i}-\hat{Y_{i}})^{2}.
\end{equation}
We have seen the \(SSE\) before, which stands for the \emph{sum of squared
errors} or \emph{error sum of squares}. Of course, we would expect the
error to be less in the latter case, since we have used more
information. The improvement in our estimation as a result of the
linear regression model can be measured with the difference \[
(Y_{i}-\overline{Y})-(Y_{i}-\hat{Y_{i}})=\hat{Y_{i}}-\overline{Y}, \]
and we measure the variation in these errors with
\begin{equation}
SSR=\sum_{i=1}^{n}(\hat{Y_{i}}-\overline{Y})^{2},
\end{equation}
also known as the \emph{regression sum of squares}. It is not obvious, but
some algebra proved a famous result known as the \textbf{ANOVA Equality}:
\begin{equation}
\label{eq-anovaeq}
\sum_{i=1}^{n}(Y_{i}-\overline{Y})^{2}=\sum_{i=1}^{n}(\hat{Y_{i}}-\overline{Y})^{2}+\sum_{i=1}^{n}(Y_{i}-\hat{Y_{i}})^{2}
\end{equation}
or in other words,
\begin{equation}
SSTO=SSR+SSE.
\end{equation}
This equality has a nice interpretation. Consider \(SSTO\) to be the
\emph{total variation} of the errors. Think of a decomposition of the total
variation into pieces: one piece measuring the reduction of error from
using the linear regression model, or \emph{explained variation} (\(SSR\)),
while the other represents what is left over, that is, the errors that
the linear regression model doesn't explain, or \emph{unexplained
variation} (\(SSE\)). In this way we see that the ANOVA equality
merely partitions the variation into \[ \mbox{total
variation}=\mbox{explained variation}+\mbox{unexplained variation}.
\] For a single number to summarize how well our model is doing we use
the \emph{simple coefficient of determination} \(r^{2}\), defined by
\begin{equation}
r^{2}=1-\frac{SSE}{SSTO}.
\end{equation}
We interpret \(r^{2}\) as the proportion of total variation that is
explained by the simple linear regression model. When \(r^{2}\) is
large, the model is doing a good job; when \(r^{2}\) is small, the
model is not doing a good job.

Related to the simple coefficient of determination is the sample
correlation coefficient, \(r\). As you can guess, the way we get \(r\)
is by the formula \(|r|=\sqrt{r^{2}}\). The sign of \(r\) is equal the
sign of the slope estimate \(b_{1}\). That is, if the regression line
\(\hat{\mu}(x)\) has positive slope, then
\(r=\sqrt{r^{2}}\). Likewise, if the slope of \(\hat{\mu}(x)\) is
negative, then \(r=-\sqrt{r^{2}}\).

\subsubsection{How to do it with \(\mathsf{R}\)}
\label{sec-11-3-2-1}

The primary method to display partitioned sums of squared errors is
with an \emph{ANOVA table}. The command in \(\mathsf{R}\) to produce such a
table is \texttt{anova}. The input to \texttt{anova} is the result of an \texttt{lm} call
which for the \texttt{cars} data is \texttt{cars.lm}.

\begin{verbatim}
anova(cars.lm)
\end{verbatim}

\begin{verbatim}
: Analysis of Variance Table
: 
: Response: dist
:           Df Sum Sq Mean Sq F value   Pr(>F)    
: speed      1  21186 21185.5  89.567 1.49e-12 ***
: Residuals 48  11354   236.5                     
: ---
: Signif. codes:  0 '***' 0.001 '**' 0.01 '*' 0.05 '.' 0.1 ' ' 1
\end{verbatim}

The output gives
\[
r^{2}=1-\frac{SSE}{SSR+SSE}=1-\frac{11353.5}{21185.5+11353.5}\approx0.65.
\]

The interpretation should be: "The linear regression line accounts for
approximately 65\% of the variation of \texttt{dist} as explained by \texttt{speed}".

The value of \(r^{2}\) is stored in the \texttt{r.squared} component of
\texttt{summary(cars.lm)}, which we called \texttt{carsumry}.

\begin{verbatim}
carsumry$r.squared
\end{verbatim}

\begin{verbatim}
: [1] 0.6510794
\end{verbatim}

We already knew this. We saw it in the next to the last line of the
\texttt{summary(cars.lm)} output where it was called \texttt{Multiple
R-squared}. Listed right beside it is the \texttt{Adjusted R-squared} which
we will discuss in Chapter \ref{sec-12}.  For the \texttt{cars}
data, we find \(r\) to be

\begin{verbatim}
sqrt(carsumry$r.squared)
\end{verbatim}

\begin{verbatim}
: [1] 0.8068949
\end{verbatim}

We choose the principal square root because the slope of the
regression line is positive.

\subsection{Overall \emph{F} statistic}
\label{sec-11-3-3}

There is another way to test the significance of the linear regression
model. In SLR, the new way also tests the hypothesis
\(H_{0}:\beta_{1}=0\) versus \(H_{1}:\beta_{1}\neq0\), but it is done
with a new test statistic called the \emph{overall F statistic}. It is
defined by
\begin{equation}
\label{eq-slr-overall-F-statistic}
F=\frac{SSR}{SSE/(n-2)}.
\end{equation}

Under the regression assumptions and when \(H_{0}\) is true, the \(F\)
statistic has an \(\mathtt{f}(\mathtt{df1}=1,\,\mathtt{df2}=n-2)\)
distribution. We reject \(H_{0}\) when \(F\) is large -- that is, when
the explained variation is large relative to the unexplained
variation.

All this being said, we have not yet gained much from the overall
\(F\) statistic because we already knew from Section
\ref{sec-11-3-1} how to test \(H_{0}:\beta_{1} =
0\)\ldots{} we use the Student's \(t\) statistic. What is worse is that (in
the simple linear regression model) it can be proved that the \(F\) in
Equation \eqref{eq-slr-overall-F-statistic} is exactly the Student's
\(t\) statistic for \(\beta_{1}\) squared,

\begin{equation}
F=\left(\frac{b_{1}}{S_{b_{1}}}\right)^{2}.
\end{equation}

So why bother to define the \(F\) statistic? Why not just square the
\(t\) statistic and be done with it? The answer is that the \(F\)
statistic has a more complicated interpretation and plays a more
important role in the multiple linear regression model which we will
study in Chapter \ref{sec-12}. See Section
\ref{sec-12-3-2} for details.

\subsubsection{How to do it with \(\mathsf{R}\)}
\label{sec-11-3-3-1}

The overall \(F\) statistic and \(p\)-value are displayed in the
bottom line of the \texttt{summary(cars.lm)} output. It is also shown in the
final columns of \texttt{anova(cars.lm)}:

\begin{verbatim}
anova(cars.lm)
\end{verbatim}

\begin{verbatim}
: Analysis of Variance Table
: 
: Response: dist
:           Df Sum Sq Mean Sq F value   Pr(>F)    
: speed      1  21186 21185.5  89.567 1.49e-12 ***
: Residuals 48  11354   236.5                     
: ---
: Signif. codes:  0 '***' 0.001 '**' 0.01 '*' 0.05 '.' 0.1 ' ' 1
\end{verbatim}

Here we see that the \(F\) statistic is 89.57 with a
\(p\)-value very close to zero. The conclusion: there is very strong
evidence that \(H_{0}:\beta_{1} = 0 \) is false, that is, there is
strong evidence that \( \beta_{1} \neq 0 \). Moreover, we conclude
that the regression relationship between \texttt{dist} and \texttt{speed} is
significant.

Note that the value of the \(F\) statistic is the same as the
Student's \(t\) statistic for \texttt{speed} squared.

\section{Residual Analysis}
\label{sec-11-4}

We know from our model that \(Y=\mu(x)+\epsilon\), or in other words,
\(\epsilon=Y-\mu(x)\). Further, we know that
\(\epsilon\sim\mathsf{norm}(\mathtt{mean}=0,\,\mathtt{sd}=\sigma)\). We
may estimate \(\epsilon_{i}\) with the \emph{residual}
\(E_{i}=Y_{i}-\hat{Y_{i}}\), where
\(\hat{Y_{i}}=\hat{\mu}(x_{i})\). If the regression assumptions hold,
then the residuals should be normally distributed. We check this in
Section \ref{sec-11-4-1}. Further, the residuals should have mean
zero with constant variance \(\sigma^{2}\), and we check this in
Section \ref{sec-11-4-2}. Last, the residuals should be
independent, and we check this in Section \ref{sec-11-4-3}.

In every case, we will begin by looking at residual plots -- that is,
scatterplots of the residuals \(E_{i}\) versus index or predicted
values \(\hat{Y_{i}}\) -- and follow up with hypothesis tests.

\subsection{Normality Assumption}
\label{sec-11-4-1}

We can assess the normality of the residuals with graphical methods
and hypothesis tests. To check graphically whether the residuals are
normally distributed we may look at histograms or \emph{q-q} plots. We
first examine a histogram in Figure \ref{fig-Normal-q-q-plot-cars}. There we see
that the distribution of the residuals appears to be mound shaped, for
the most part. We can plot the order statistics of the sample versus
quantiles from a \(\mathsf{norm}(\mathtt{mean}=0,\,\mathtt{sd}=1)\)
distribution with the command \texttt{plot(cars.lm, which = 2)}, and the
results are in Figure \ref{fig-Normal-q-q-plot-cars}. If the assumption of
normality were true, then we would expect points randomly scattered
about the dotted straight line displayed in the figure. In this case,
we see a slight departure from normality in that the dots show
systematic clustering on one side or the other of the line. The points
on the upper end of the plot also appear begin to stray from the
line. We would say there is some evidence that the residuals are not
perfectly normal.

\begin{verbatim}
plot(cars.lm, which = 2)
\end{verbatim}

\begin{figure}[ht!]
\centering
\includegraphics[width=0.9\textwidth]{fig/slr-Normal-q-q-plot-cars.ps}
\caption[Normal q-q plot of the residuals for the \texttt{cars} data]{\label{fig-Normal-q-q-plot-cars}\small Used for checking the normality assumption. Look out for any curvature or substantial departures from the straight line; hopefully the dots hug the line closely.}
\end{figure}

\subsubsection{Testing the Normality Assumption}
\label{sec-11-4-1-1}

Even though we may be concerned about the plots, we can use tests to
determine if the evidence present is statistically significant, or if
it could have happened merely by chance. There are many statistical
tests of normality. We will use the Shapiro-Wilk test, since it is
known to be a good test and to be quite powerful. However, there are
many other fine tests of normality including the Anderson-Darling test
and the Lillefors test, just to mention two of them.

The Shapiro-Wilk test is based on the statistic
\begin{equation}
W=\frac{\left(\sum_{i=1}^{n}a_{i}E_{(i)}\right)^{2}}{\sum_{j=1}^{n}E_{j}^{2}},
\end{equation}
where the \(E_{(i)}\) are the ordered residuals and the \(a_{i}\) are
constants derived from the order statistics of a sample of size \(n\)
from a normal distribution. See Section \ref{sec-10-5-2}.
We perform the Shapiro-Wilk test below, using the \texttt{shapiro.test}
function from the \texttt{stats} package \cite{stats}. The hypotheses are \[
H_{0}:\mbox{ the residuals are normally distributed } \] versus \[
H_{1}:\mbox{ the residuals are not normally distributed.}  \] The
results from \(\mathsf{R}\) are

\begin{verbatim}
shapiro.test(residuals(cars.lm))
\end{verbatim}

\begin{verbatim}
: 
: 	Shapiro-Wilk normality test
: 
: data:  residuals(cars.lm)
: W = 0.9451, p-value = 0.02152
\end{verbatim}

For these data we would reject the assumption of normality of the
residuals at the \(\alpha=0.05\) significance level, but do not lose
heart, because the regression model is reasonably robust to departures
from the normality assumption. As long as the residual distribution is
not highly skewed, then the regression estimators will perform
reasonably well. Moreover, departures from constant variance and
independence will sometimes affect the quantile plots and histograms,
therefore it is wise to delay final decisions regarding normality
until all diagnostic measures have been investigated.

\subsection{Constant Variance Assumption}
\label{sec-11-4-2}

We will again go to residual plots to try and determine if the spread
of the residuals is changing over time (or index). However, it is
unfortunately not that easy because the residuals do not have constant
variance! In fact, it can be shown that the variance of the residual
\(E_{i}\) is
\begin{equation}
\mbox{Var$(E_{i})$}=\sigma^{2}(1-h_{ii}),\quad i=1,2,\ldots,n,
\end{equation}
where \(h_{ii}\) is a quantity called the \emph{leverage} which is defined
below. Consequently, in order to check the constant variance
assumption we must standardize the residuals before plotting. We
estimate the standard error of \(E_{i}\) with
\(s_{E_{i}}=s\sqrt{(1-h_{ii})}\) and define the \emph{standardized
residuals} \(R_{i}\), \(i=1,2,\ldots,n\), by
\begin{equation} 
R_{i}=\frac{E_{i}}{s\,\sqrt{1-h_{ii}}},\quad i=1,2,\ldots,n.
\end{equation}
For the constant variance assumption we do not need the sign of the
residual so we will plot \(\sqrt{|R_{i}|}\) versus the fitted
values. As we look at a scatterplot of \(\sqrt{|R_{i}|}\) versus
\(\hat{Y}_{i}\) we would expect under the regression assumptions to
see a constant band of observations, indicating no change in the
magnitude of the observed distance from the line. We want to watch out
for a fanning-out of the residuals, or a less common funneling-in of
the residuals. Both patterns indicate a change in the residual
variance and a consequent departure from the regression assumptions,
the first an increase, the second a decrease.

In this case, we plot the standardized residuals versus the fitted
values. The graph may be seen in Figure \ref{fig-std-resids-fitted-cars}. For
these data there does appear to be somewhat of a slight fanning-out of
the residuals.

\begin{verbatim}
plot(cars.lm, which = 3)
\end{verbatim}

\begin{figure}[ht!]
\centering
\includegraphics[width=0.9\textwidth]{fig/slr-std-resids-fitted-cars.ps}
\caption[Plot of standardized residuals against the fitted values for the \texttt{cars} data]{\label{fig-std-resids-fitted-cars}\small Used for checking the constant variance assumption. Watch out for any fanning out (or in) of the dots; hopefully they fall in a constant band.}
\end{figure}

\subsubsection{Testing the Constant Variance Assumption}
\label{sec-11-4-2-1}

We will use the Breusch-Pagan test to decide whether the variance of
the residuals is nonconstant. The null hypothesis is that the variance
is the same for all observations, and the alternative hypothesis is
that the variance is not the same for all observations. The test
statistic is found by fitting a linear model to the centered squared
residuals,
\begin{equation}
W_{i} = E_{i}^{2} - \frac{SSE}{n}, \quad i=1,2,\ldots,n.
\end{equation}

By default the same explanatory variables are used in the new model
which produces fitted values \(\hat{W}_{i}\), \(i=1,2,\ldots,n\). The
Breusch-Pagan test statistic in \(\mathsf{R}\) is then calculated with
\begin{equation}
BP=n\sum_{i=1}^{n}\hat{W}_{i}^{2}\div\sum_{i=1}^{n}W_{i}^{2}.
\end{equation}
We reject the null hypothesis if \(BP\) is too large, which happens
when the explained variation i the new model is large relative to the
unexplained variation in the original model.  We do it in
\(\mathsf{R}\) with the \texttt{bptest} function from the \texttt{lmtest} package
\cite{lmtest}.
\begin{verbatim}
bptest(cars.lm)
\end{verbatim}

\begin{verbatim}
: 
: 	studentized Breusch-Pagan test
: 
: data:  cars.lm
: BP = 3.2149, df = 1, p-value = 0.07297
\end{verbatim}

For these data we would not reject the null hypothesis at the
\(\alpha=0.05\) level. There is relatively weak evidence against the
assumption of constant variance.

\subsection{Independence Assumption}
\label{sec-11-4-3}

One of the strongest of the regression assumptions is the one
regarding independence. Departures from the independence assumption
are often exhibited by correlation (or autocorrelation, literally,
self-correlation) present in the residuals. There can be positive or
negative correlation.

Positive correlation is displayed by positive residuals followed by
positive residuals, and negative residuals followed by negative
residuals. Looking from left to right, this is exhibited by a cyclical
feature in the residual plots, with long sequences of positive
residuals being followed by long sequences of negative ones.

On the other hand, negative correlation implies positive residuals
followed by negative residuals, which are then followed by positive
residuals, \emph{etc}. Consequently, negatively correlated residuals are
often associated with an alternating pattern in the residual plots. We
examine the residual plot in Figure \ref{fig-resids-fitted-cars}. There is no
obvious cyclical wave pattern or structure to the residual plot.

\begin{verbatim}
plot(cars.lm, which = 1)
\end{verbatim}

\begin{figure}[ht!]
\centering
\includegraphics[width=0.9\textwidth]{fig/slr-resids-fitted-cars.ps}
\caption[Plot of the residuals versus the fitted values for the \texttt{cars} data]{\label{fig-resids-fitted-cars}\small Used for checking the independence assumption. Watch out for any patterns or structure; hopefully the points are randomly scattered on the plot.}
\end{figure}

\subsubsection{Testing the Independence Assumption}
\label{sec-11-4-3-1}

We may statistically test whether there is evidence of autocorrelation
in the residuals with the Durbin-Watson test. The test is based on the
statistic
\begin{equation}
D=\frac{\sum_{i=2}^{n}(E_{i}-E_{i-1})^{2}}{\sum_{j=1}^{n}E_{j}^{2}}.
\end{equation}
Exact critical values are difficult to obtain, but \(\mathsf{R}\) will
calculate the \emph{p-value} to great accuracy. It is performed with the
\texttt{dwtest} function from the \texttt{lmtest} package \cite{lmtest}. We will
conduct a two sided test that the correlation is not zero, which is
not the default (the default is to test that the autocorrelation is
positive).

\begin{verbatim}
dwtest(cars.lm, alternative = "two.sided")
\end{verbatim}

\begin{verbatim}
: 
: 	Durbin-Watson test
: 
: data:  cars.lm
: DW = 1.6762, p-value = 0.1904
: alternative hypothesis: true autocorrelation is not 0
\end{verbatim}

In this case we do not reject the null hypothesis at the
\(\alpha=0.05\) significance level; there is very little evidence of
nonzero autocorrelation in the residuals.

\subsection{Remedial Measures}
\label{sec-11-4-4}

We often find problems with our model that suggest that at least one
of the three regression assumptions is violated. What do we do then?
There are many measures at the statistician's disposal, and we mention
specific steps one can take to improve the model under certain types
of violation.

\begin{description}
\item[{Mean response is not linear}] We can directly modify the model to
better approximate the mean response. In particular, perhaps a
polynomial regression function of the form \[ \mu(x) =
     \beta_{0} + \beta_{1}x_{1} + \beta_{2}x_{1}^{2} \] would be
appropriate. Alternatively, we could have a function of the form
\[ \mu(x)=\beta_{0}\mathrm{e}^{\beta_{1}x}.  \] Models like these
are studied in nonlinear regression courses.
\item[{Error variance is not constant}] Sometimes a transformation of the
dependent variable will take care of the problem. There is a
large class of them called \emph{Box-Cox transformations}. They take
the form
\begin{equation}
Y^{\ast}=Y^{\lambda},
\end{equation}
where \(\lambda\) is a constant. (The method proposed by Box and
Cox will determine a suitable value of \(\lambda\) automatically
by maximum likelihood). The class contains the transformations
\begin{alignat*}{1} \lambda=2,\quad &
     Y^{\ast}=Y^{2}\\ \lambda=0.5,\quad &
     Y^{\ast}=\sqrt{Y}\\ \lambda=0,\quad & Y^{\ast}=\ln\:
     Y\\ \lambda=-1,\quad & Y^{\ast}= 1/Y \end{alignat*}
Alternatively, we can use the method of \emph{weighted least
squares}. This is studied in more detail in later classes.
\item[{Error distribution is not normal}] The same transformations for
stabilizing the variance are equally appropriate for smoothing
the residuals to a more Gaussian form. In fact, often we will
kill two birds with one stone.
\item[{Errors are not independent}] There is a large class of
autoregressive models to be used in this situation which occupy
the latter part of Chapter \ref{sec-16}.
\end{description}

\section{Other Diagnostic Tools}
\label{sec-11-5}

There are two types of observations with which we must be especially
careful:
\begin{description}
\item[{Influential observations}] are those that have a substantial effect
on our estimates, predictions, or inferences. A small change in
an influential observation is followed by a large change in the
parameter estimates or inferences.
\item[{Outlying observations}] are those that fall fall far from the rest
of the data. They may be indicating a lack of fit for our
regression model, or they may just be a mistake or typographical
error that should be corrected. Regardless, special attention
should be given to these observations. An outlying observation
may or may not be influential.
\end{description}

We will discuss outliers first because the notation builds
sequentially in that order.
\subsection{Outliers}
\label{sec-11-5-1}
There are three ways that an observation \((x_{i},y_{i})\) might be
identified as an outlier: it can have an \(x_{i}\) value which falls
far from the other \(x\) values, it can have a \(y_{i}\) value which
falls far from the other \(y\) values, or it can have both its
\(x_{i}\) and \(y_{i}\) values falling far from the other \(x\) and
\(y\) values.
\subsection{Leverage}
\label{sec-11-5-2}
Leverage statistics are designed to identify observations which have
\(x\) values that are far away from the rest of the data. In the
simple linear regression model the leverage of \(x_{i}\) is denoted by
\(h_{ii}\) and defined by
\begin{equation}
h_{ii}=\frac{1}{n}+\frac{(x_{i}-\overline{x})^{2}}{\sum_{k=1}^{n}(x_{k}-\overline{x})^{2}},\quad i=1,2,\ldots,n.
\end{equation}
The formula has a nice interpretation in the SLR model: if the
distance from \(x_{i}\) to \(\overline{x}\) is large relative to the
other \(x\)'s then \(h_{ii}\) will be close to 1.

Leverages have nice mathematical properties; for example, they satisfy
\begin{equation}
\label{eq-slr-leverage-between}
0\leq h_{ii}\leq1,
\end{equation}
and their sum is
\begin{eqnarray}
\label{eq-slr-average-leverage}
\sum_{i=1}^{n}h_{ii} & = & \sum_{i=1}^{n}\left[\frac{1}{n}+\frac{(x_{i}-\overline{x})^{2}}{\sum_{k=1}^{n}(x_{k}-\overline{x})^{2}}\right],\\
 & = & \frac{n}{n}+\frac{\sum_{i}(x_{i}-\overline{x})^{2}}{\sum_{k}(x_{k}-\overline{x})^{2}},\\
 & = & 2.
\end{eqnarray}

A rule of thumb is to consider leverage values to be large if they are
more than double their average size (which is \(2/n\) according to
Equation \eqref{eq-slr-average-leverage}). So leverages larger than \(4/n\)
are suspect. Another rule of thumb is to say that values bigger than
0.5 indicate high leverage, while values between 0.3 and 0.5 indicate
moderate leverage.

\subsection{Standardized and Studentized Deleted Residuals}
\label{sec-11-5-3}

We have already encountered the \emph{standardized residuals} \(r_{i}\) in
Section \ref{sec-11-4-2}; they are merely residuals that
have been divided by their respective standard deviations:
\begin{equation}
R_{i}=\frac{E_{i}}{S\sqrt{1-h_{ii}}},\quad i=1,2,\ldots,n.
\end{equation}
Values of \(|R_{i}| > 2\) are extreme and suggest that the observation has an outlying \(y\)-value. 

Now delete the \(i^{\mathrm{th}}\) case and fit the regression
function to the remaining \(n - 1\) cases, producing a fitted value
\(\hat{Y}_{(i)}\) with \emph{deleted residual}
\(D_{i}=Y_{i}-\hat{Y}_{(i)}\). It is shown in later classes that
\begin{equation}
\mbox{Var $(D_{i})$}=\frac{S_{(i)}^{2}}{1-h_{ii}},\quad i=1,2,\ldots,n,
\end{equation}
so that the \emph{studentized deleted residuals} \(t_{i}\) defined by
\begin{equation}
\label{eq-slr-studentized-deleted-resids}
t_{i}=\frac{D_{i}}{S_{(i)}/(1-h_{ii})},\quad i=1,2,\ldots,n,
\end{equation}
have a \(\mathsf{t}(\mathtt{df}=n-3)\) distribution and we compare observed values of \(t_{i}\) to this distribution to decide whether or not an observation is extreme. 

The folklore in regression classes is that a test based on the
statistic in Equation \eqref{eq-slr-studentized-deleted-resids} can be
too liberal. A rule of thumb is if we suspect an observation to be an
outlier \emph{before} seeing the data then we say it is significantly
outlying if its two-tailed \(p\)-value is less than \(\alpha\), but if
we suspect an observation to be an outlier \emph{after} seeing the data
then we should only say it is significantly outlying if its two-tailed
\(p\)-value is less than \(\alpha/n\). The latter rule of thumb is
called the \emph{Bonferroni approach} and can be overly conservative for
large data sets. The responsible statistician should look at the data
and use his/her best judgement, in every case.

\subsubsection{How to do it with \(\mathsf{R}\)}
\label{sec-11-5-3-1}

We can calculate the standardized residuals with the \texttt{rstandard}
function. The input is the \texttt{lm} object, which is \texttt{cars.lm}.

\begin{verbatim}
sres <- rstandard(cars.lm)
sres[1:5]
\end{verbatim}

\begin{verbatim}
:          1          2          3          4          5 
:  0.2660415  0.8189327 -0.4013462  0.8132663  0.1421624
\end{verbatim}

We can find out which observations have studentized residuals larger
than two with the command

\begin{verbatim}
sres[which(abs(sres) > 2)]
\end{verbatim}

\begin{verbatim}
:       23       35       49 
: 2.795166 2.027818 2.919060
\end{verbatim}

In this case, we see that observations 23, 35, and 49 are potential
outliers with respect to their \(y\)-value.  We can compute the
studentized deleted residuals with \texttt{rstudent}:

\begin{verbatim}
sdelres <- rstudent(cars.lm)
sdelres[1:5]
\end{verbatim}

\begin{verbatim}
:          1          2          3          4          5 
:  0.2634500  0.8160784 -0.3978115  0.8103526  0.1407033
\end{verbatim}

We should compare these values with critical values from a
\(\mathsf{t}(\mathtt{df}=n-3)\) distribution, which in this case is
\(\mathsf{t}(\mathtt{df}=50-3=47)\). We can calculate a 0.005 quantile
and check with

\begin{verbatim}
t0.005 <- qt(0.005, df = 47, lower.tail = FALSE)
sdelres[which(abs(sdelres) > t0.005)]
\end{verbatim}

\begin{verbatim}
:       23       49 
: 3.022829 3.184993
\end{verbatim}

This means that observations 23 and 49 have a large studentized
deleted residual. The leverages can be found with the \texttt{hatvalues}
function:

\begin{verbatim}
leverage <- hatvalues(cars.lm)
leverage[which(leverage > 4/50)]
\end{verbatim}

\begin{verbatim}
:          1          2         50 
: 0.11486131 0.11486131 0.08727007
\end{verbatim}

Here we see that observations 1, 2, and 50 have leverages bigger than
double their mean value. These observations would be considered
outlying with respect to their \(x\) value (although they may or may
not be influential).

\subsection{Influential Observations}
\label{sec-11-5-4}

\subsubsection{\(DFBETAS\) and \(DFFITS\)}
\label{sec-11-5-4-1}

Any time we do a statistical analysis, we are confronted with the
variability of data. It is always a concern when an observation plays
too large a role in our regression model, and we would not like or
procedures to be overly influenced by the value of a single
observation. Hence, it becomes desirable to check to see how much our
estimates and predictions would change if one of the observations were
not included in the analysis. If an observation changes the
estimates/predictions a large amount, then the observation is
influential and should be subjected to a higher level of scrutiny.

We measure the change in the parameter estimates as a result of
deleting an observation with \(DFBETAS\). The \(DFBETAS\) for the
intercept \(b_{0}\) are given by
\begin{equation}
(DFBETAS)_{0(i)}=\frac{b_{0}-b_{0(i)}}{S_{(i)}\sqrt{\frac{1}{n}+\frac{\overline{x}^{2}}{\sum_{i=1}^{n}(x_{i}-\overline{x})^{2}}}},\quad i=1,2,\ldots,n.
\end{equation}
and the \(DFBETAS\) for the slope \(b_{1}\) are given by
\begin{equation}
(DFBETAS)_{1(i)}=\frac{b_{1}-b_{1(i)}}{S_{(i)}\left[\sum_{i=1}^{n}(x_{i}-\overline{x})^{2}\right]^{-1/2}},\quad i=1,2,\ldots,n.
\end{equation}
See Section \ref{sec-12-6} for a better way to write these. The
signs of the \(DFBETAS\) indicate whether the coefficients would
increase or decrease as a result of including the observation. If the
\(DFBETAS\) are large, then the observation has a large impact on
those regression coefficients. We label observations as suspicious if
their \(DFBETAS\) have magnitude greater 1 for small data or
\(2/\sqrt{n}\) for large data sets.  We can calculate the \(DFBETAS\)
with the \texttt{dfbetas} function (some output has been omitted):

\begin{verbatim}
dfb <- dfbetas(cars.lm)
head(dfb)
\end{verbatim}

\begin{verbatim}
:   (Intercept)       speed
: 1  0.09440188 -0.08624563
: 2  0.29242487 -0.26715961
: 3 -0.10749794  0.09369281
: 4  0.21897614 -0.19085472
: 5  0.03407516 -0.02901384
: 6 -0.11100703  0.09174024
\end{verbatim}

We see that the inclusion of the first observation slightly increases
the \texttt{Intercept} and slightly decreases the coefficient on \texttt{speed}.

We can measure the influence that an observation has on its fitted
value with \(DFFITS\). These are calculated by deleting an
observation, refitting the model, recalculating the fit, then
standardizing. The formula is
\begin{equation}
(DFFITS)_{i}=\frac{\hat{Y_{i}}-\hat{Y}_{(i)}}{S_{(i)}\sqrt{h_{ii}}},\quad i=1,2,\ldots,n.
\end{equation}
The value represents the number of standard deviations of
\(\hat{Y_{i}}\) that the fitted value \(\hat{Y_{i}}\) increases or
decreases with the inclusion of the \(i^{\textrm{th}}\)
observation. We can compute them with the \texttt{dffits} function.

\begin{verbatim}
dff <- dffits(cars.lm)
dff[1:5]
\end{verbatim}

\begin{verbatim}
:           1           2           3           4           5 
:  0.09490289  0.29397684 -0.11039550  0.22487854  0.03553887
\end{verbatim}

A rule of thumb is to flag observations whose \(DFFIT\) exceeds one in
absolute value, but there are none of those in this data set.

\subsubsection{Cook's Distance}
\label{sec-11-5-4-2}

The \(DFFITS\) are good for measuring the influence on a single fitted
value, but we may want to measure the influence an observation has on
all of the fitted values simultaneously. The statistics used for
measuring this are Cook's distances which may be
calculatedby the formula
\begin{equation}
\label{eq-slr-cooks-distance}
D_{i}=\frac{E_{i}^{2}}{(p+1)S^{2}}\cdot\frac{h_{ii}}{(1-h_{ii})^{2}},\quad i=1,2,\ldots,n.
\end{equation}
It shows that Cook's distance depends both on the residual \(E_{i}\)
and the leverage \(h_{ii}\) and in this way \(D_{i}\) contains
information about outlying \(x\) and \(y\) values.

To assess the significance of \(D\), we compare to quantiles of an
\(\mathsf{f}(\mathtt{df1}=2,\,\mathtt{df2}=n-2)\) distribution. A rule
of thumb is to classify observations falling higher than the
\(50^{\mathrm{th}}\) percentile as being extreme.

\subsubsection{How to do it with \(\mathsf{R}\)}
\label{sec-11-5-4-3}

We can calculate the Cook's Distances with the \texttt{cooks.distance}
function.

\begin{verbatim}
cooksD <- cooks.distance(cars.lm)
cooksD[1:4]
\end{verbatim}

\begin{verbatim}
:           1           2           3           4 
: 0.004592312 0.043513991 0.006202350 0.025467338
\end{verbatim}

We can look at a plot of the Cook's distances with the command
\texttt{plot(cars.lm, which = 4)}.

\begin{verbatim}
plot(cars.lm, which = 4)
\end{verbatim}

\begin{figure}[ht!]
\centering
\includegraphics[width=0.9\textwidth]{fig/slr-Cooks-distance-cars.ps}
\caption[Cook's distances for the \texttt{cars} data]{\label{fig-Cooks-distance-cars}\small Used for checking for influential and/our outlying observations. Values with large Cook's distance merit further investigation.}
\end{figure}

Observations with the largest Cook's D values are labeled, hence we
see that observations 23, 39, and 49 are suspicious. However, we need
to compare to the quantiles of an \( \mathsf{f}(\mathtt{df1} = 2, \,
\mathtt{df2} = 48) \) distribution:

\begin{verbatim}
F0.50 <- qf(0.5, df1 = 2, df2 = 48)
any(cooksD > F0.50)
\end{verbatim}

\begin{verbatim}
: [1] FALSE
\end{verbatim}

We see that with this data set there are no observations with extreme
Cook's distance, after all.

\subsection{All Influence Measures Simultaneously}
\label{sec-11-5-5}

We can display the result of diagnostic checking all at once in one
table, with potentially influential points displayed. We do it with
the command \texttt{influence.measures(cars.lm)}:

\begin{verbatim}
influence.measures(cars.lm)
\end{verbatim}

The output is a huge matrix display, which we have omitted in the
interest of brevity. A point is identified if it is classified to be
influential with respect to any of the diagnostic measures. Here we
see that observations 2, 11, 15, and 18 merit further investigation.

We can also look at all diagnostic plots at once with the commands

\begin{verbatim}
plot(cars.lm)
\end{verbatim}

The \texttt{par} command is used so that \(2\times 2 = 4\) plots will be
shown on the same display. The diagnostic plots for the \texttt{cars} data
are shown in Figure \ref{fig-Diagnostic-plots-cars}:

\begin{figure}[ht!]
\centering
\includegraphics[width=0.9\textwidth]{fig/slr-Diagnostic-plots-cars.ps}
\caption[Diagnostic plots for the \texttt{cars} data]{\label{fig-Diagnostic-plots-cars}\small Diagnostic plots for the \texttt{cars} data.}
\end{figure}

We have discussed all of the plots except the last, which is possibly
the most interesting. It shows Residuals vs. Leverage, which will
identify outlying \(y\) values versus outlying \(x\) values. Here we
see that observation 23 has a high residual, but low leverage, and it
turns out that observations 1 and 2 have relatively high leverage but
low/moderate leverage (they are on the right side of the plot, just
above the horizontal line). Observation 49 has a large residual with a
comparatively large leverage.

We can identify the observations with the \texttt{identify} command; it
allows us to display the observation number of dots on the
plot. First, we plot the graph, then we call \texttt{identify}:

\begin{verbatim}
plot(cars.lm, which = 5)          # std'd resids vs lev plot
identify(leverage, sres, n = 4)   # identify 4 points
\end{verbatim}

The graph with the identified points is omitted (but the plain plot is
shown in the bottom right corner of Figure
\ref{fig-Diagnostic-plots-cars}). Observations 1 and 2 fall on the far right
side of the plot, near the horizontal axis.

\newpage{}

\section{Exercises}
\label{sec-11-6}
\setcounter{thm}{0}

\begin{xca}
Prove the ANOVA equality, Equation \eqref{eq-anovaeq}. \emph{Hint}:
show that
\[
\sum_{i=1}^{n}(Y_{i}-\hat{Y_{i}})(\hat{Y_{i}}-\overline{Y})=0.
\]
\end{xca}

\begin{xca}
\label{xca-find-mles-SLR} Solve the following system of equations for
\(\beta_{1}\) and \(\beta_{0}\) to find the MLEs for slope and
intercept in the simple linear regression model.
\begin{eqnarray*}
n\beta_{0}+\beta_{1}\sum_{i=1}^{n}x_{i} & = & \sum_{i=1}^{n}y_{i}\\
\beta_{0}\sum_{i=1}^{n}x_{i}+\beta_{1}\sum_{i=1}^{n}x_{i}^{2} & = & \sum_{i=1}^{n}x_{i}y_{i}
\end{eqnarray*}
\end{xca}

\begin{xca}
\label{xca-show-alternate-slope-formula} Show that the formula given in
Equation \eqref{eq-sample-correlation-formula} is equivalent to \[
\hat{\beta}_{1} =
\frac{\sum_{i=1}^{n}x_{i}y_{i}-\left.\left(\sum_{i=1}^{n}x_{i}\right)\left(\sum_{i=1}^{n}y_{i}\right)\right/
n}{\sum_{i=1}^{n}x_{i}^{2}-\left.\left(\sum_{i=1}^{n}x_{i}\right)^{2}\right/
n}.  \]
\end{xca}
