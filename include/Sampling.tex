\chapter{Sampling Distributions}
\label{sec-8}

\noindent
This is an important chapter; it is the bridge from probability and
descriptive statistics that we studied in Chapters
\ref{sec-3} through
\ref{sec-7} to inferential statistics which
forms the latter part of this book.

Here is the link: we are presented with a \emph{population} about which we
would like to learn. And while it would be desirable to examine every
single member of the population, we find that it is either impossible
or infeasible to for us to do so, thus, we resort to collecting a
\emph{sample} instead. We do not lose heart. Our method will suffice,
provided the sample is \emph{representative} of the population. A good way
to achieve this is to sample \emph{randomly} from the population.

Supposing for the sake of argument that we have collected a random
sample, the next task is to make some \emph{sense} out of the data because
the complete list of sample information is usually cumbersome,
unwieldy. We summarize the data set with a descriptive \emph{statistic}, a
quantity calculated from the data (we saw many examples of these in
Chapter \ref{sec-3}). But our sample was
random\ldots{} therefore, it stands to reason that our statistic will be
random, too. How is the statistic distributed?

The probability distribution associated with the population (from
which we sample) is called the \emph{population distribution}, and the
probability distribution associated with our statistic is called its
\emph{sampling distribution}; clearly, the two are interrelated. To learn
about the population distribution, it is imperative to know everything
we can about the sampling distribution. Such is the goal of this
chapter.

We begin by introducing the notion of simple random samples and
cataloguing some of their more convenient mathematical
properties. Next we focus on what happens in the special case of
sampling from the normal distribution (which, again, has several
convenient mathematical properties), and in particular, we meet the
sampling distribution of \(\overline{X}\) and \(S^{2}\). Then we
explore what happens to \(\overline{X}\)'s sampling distribution when
the population is not normal and prove one of the most remarkable
theorems in statistics, the Central Limit Theorem (CLT).

With the CLT in hand, we then investigate the sampling distributions
of several other popular statistics, taking full advantage of those
with a tractable form. We finish the chapter with an exploration of
statistics whose sampling distributions are not quite so tractable,
and to accomplish this goal we will use simulation methods that are
grounded in all of our work in the previous four chapters.

\textbf{What do I want them to know?}
\begin{itemize}
\item the notion of population versus simple random sample, parameter
versus statistic, and population distribution versus sampling
distribution
\item the classical sampling distributions of the standard one and two
sample statistics
\item how to generate a simulated sampling distribution when the statistic
is crazy
\item the Central Limit Theorem, period.
\item some basic concepts related to sampling distribution utility, such
as bias and variance
\end{itemize}

\section{Simple Random Samples}
\label{sec-8-1}

\begin{defn}
If \(X_{1}\), \(X_{2}\), \ldots{}, \(X_{n}\) are independent with
\(X_{i}\sim f\) for \(i=1,2,\ldots,n\), then we say that \(X_{1}\),
\(X_{2}\), \ldots{}, \(X_{n}\) are \emph{independent and identically
distributed} (IID) from the population \(f\) or alternatively we say
that \(X_{1}\), \(X_{2}\), \ldots{}, \(X_{n}\) are a \emph{simple random sample
of size} \(n\), denoted \(SRS(n)\), from the population \(f\).
\end{defn}

\begin{prop}
\label{pro-mean-sd-xbar} Let \(X_{1}\), \(X_{2}\), \ldots{}, \(X_{n}\) be a
\(SRS(n)\) from a population distribution with mean \(\mu\) and finite
standard deviation \(\sigma\). Then the mean and standard deviation of
\(\overline{X}\) are given by the formulas \(\mu_{\overline{X}}=\mu\)
and \(\sigma_{\overline{X}}=\sigma/\sqrt{n}\).
\end{prop}

\begin{proof}
Plug in \(a_{1}=a_{2}=\cdots=a_{n}=1/n\) in Proposition
\ref{pro-mean-sd-lin-comb}.
\end{proof}

The next fact will be useful to us when it comes time to prove the
Central Limit Theorem in Section \ref{sec-8-3}.

\begin{prop}
\label{pro-mgf-xbar} Let \(X_{1}\), \(X_{2}\), \ldots{}, \(X_{n}\) be a
\(SRS(n)\) from a population distribution with MGF \(M(t)\). Then the
MGF of \(\overline{X}\) is given by
\begin{equation}
M_{\overline{X}}(t)=\left[M\left(\frac{t}{n}\right)\right]^{n}.
\end{equation}
\end{prop}

\begin{proof}
Go from the definition:
\begin{eqnarray*}
M_{\overline{X}}(t) & = & \mathbb{E}\,\mathrm{e}^{t\overline{X}},\\
 & = & \mathbb{E}\,\mathrm{e}^{t(X_{1}+\cdots+X_{n})/n},\\
 & = & \mathbb{E}\,\mathrm{e}^{tX_{1}/n}\mathrm{e}^{tX_{2}/n}\cdots\mathrm{e}^{tX_{n}/n}.
\end{eqnarray*}
And because \(X_{1}\), \(X_{2}\), \ldots{}, \(X_{n}\) are independent,
Proposition \ref{pro-indep-implies-prodexpect} allows us to distribute the
expectation among each term in the product, which is \[
\mathbb{E}\mathrm{e}^{tX_{1}/n}\,\mathbb{E}\mathrm{e}^{tX_{2}/n}\cdots\mathbb{E}\mathrm{e}^{tX_{n}/n}.
\] The last step is to recognize that each term in last product above
is exactly \(M(t/n)\).
\end{proof}

\section{Sampling from a Normal Distribution}
\label{sec-8-2}

\subsection{The Distribution of the Sample Mean}
\label{sec-8-2-1}

\begin{prop}
Let \(X_{1}\), \(X_{2}\), \ldots{}, \(X_{n}\) be a \(SRS(n)\) from a
\(\mathsf{norm}(\mathtt{mean}=\mu,\,\mathtt{sd}=\sigma)\)
distribution. Then the sample mean \(\overline{X}\) has a
\(\mathsf{norm}(\mathtt{mean}=\mu,\,\mathtt{sd}=\sigma/\sqrt{n})\)
sampling distribution.
\end{prop}

\begin{proof}
The mean and standard deviation of \(\overline{X}\) follow directly
from Proposition \ref{pro-mean-sd-xbar}. To address the shape, first
remember from Section \ref{sec-6-3} that the
\(\mathsf{norm}(\mathtt{mean}=\mu,\,\mathtt{sd}=\sigma)\) MGF is of
the form \[ M(t)=\exp\left[ \mu t+\sigma^{2}t^{2}/2\right] .  \] Now
use Proposition \ref{pro-mgf-xbar} to find
\begin{eqnarray*}
M_{\overline{X}}(t) & = & \left[M\left(\frac{t}{n}\right)\right]^{n},\\
 & = & \left[\exp\left( \mu(t/n)+\sigma^{2}(t/n)^{2}/2\right) \right]^{n},\\
 & = & \exp\left( \, n\cdot\left[\mu(t/n)+\sigma^{2}(t/n)^{2}/2\right]\right) ,\\
 & = & \exp\left( \mu t+(\sigma/\sqrt{n})^{2}t^{2}/2\right),
\end{eqnarray*}
and we recognize this last quantity as the MGF of a
\(\mathsf{norm}(\mathtt{mean}=\mu,\,\mathtt{sd}=\sigma/\sqrt{n})\)
distribution.
\end{proof}

\subsection{The Distribution of the Sample Variance}
\label{sec-8-2-2}

\begin{thm}
\label{thm-Xbar-andS} Let \(X_{1}\), \(X_{2}\), \ldots{}, \(X_{n}\) be a
\(SRS(n)\) from a
\(\mathsf{norm}(\mathtt{mean}=\mu,\,\mathtt{sd}=\sigma)\)
distribution, and let
\begin{equation}
\overline{X}=\sum_{i=1}^{n}X_{i}\quad \mbox{and}\quad S^{2}=\frac{1}{n-1}\sum_{i=1}^{n}(X_{i}-\overline{X})^{2}.
\end{equation}
Then
\begin{enumerate}
\item \(\overline{X}\) and \(S^{2}\) are independent, and
\item The rescaled sample variance
\begin{equation}
\frac{(n-1)}{\sigma^{2}}S^{2}=\frac{\sum_{i=1}^{n}(X_{i}-\overline{X})^{2}}{\sigma^{2}}
\end{equation}
has a \(\mathsf{chisq}(\mathtt{df}=n-1)\) sampling distribution.
\end{enumerate}
\end{thm}

\begin{proof}
The proof is beyond the scope of the present book, but the theorem is
simply too important to be omitted. The interested reader could
consult Casella and Berger \cite{Casella2002}, or Hogg \emph{et al}
\cite{Hogg2005}.
\end{proof}

\subsection{The Distribution of Student's \(T\) Statistic}
\label{sec-8-2-3}

\begin{prop}
Let \(X_{1}\), \(X_{2}\), \ldots{}, \(X_{n}\) be a \(SRS(n)\) from a
\(\mathsf{norm}(\mathtt{mean}=\mu,\,\mathtt{sd}=\sigma)\)
distribution. Then the quantity
\begin{equation}
T=\frac{\overline{X}-\mu}{S/\sqrt{n}}
\end{equation}
has a \(\mathsf{t}(\mathtt{df}=n-1)\) sampling distribution.
\end{prop}

\begin{proof}
Divide the numerator and denominator by \(\sigma\) and rewrite \[
T=\frac{\frac{\overline{X}-\mu}{\sigma/\sqrt{n}}}{S/\sigma}=\frac{\frac{\overline{X}-\mu}{\sigma/\sqrt{n}}}{\sqrt{\left.\frac{(n-1)S^{2}}{\sigma^{2}}\right/
(n-1)}}.  \] Now let \[
Z=\frac{\overline{X}-\mu}{\sigma/\sqrt{n}}\quad \mbox{and}\quad
V=\frac{(n-1)S^{2}}{\sigma^{2}}, \] so that
\begin{equation}
T=\frac{Z}{\sqrt{V/r}},
\end{equation}
where \(r=n-1\).

We know from Section \ref{sec-8-2-1} that
\(Z\sim\mathsf{norm}(\mathtt{mean}=0,\,\mathtt{sd}=1)\) and we know
from Section \ref{sec-8-2-2} that
\(V\sim\mathsf{chisq}(\mathtt{df}=n-1)\). Further, since we are
sampling from a normal distribution, Theorem \ref{thm-Xbar-andS} gives
that \(\overline{X}\) and \(S^{2}\) are independent and by Fact
\ref{fac-indep-then-function-indep} so are \(Z\) and \(V\). In summary,
the distribution of \(T\) is the same as the distribution of the
quantity \(Z/\sqrt{V/r}\), where
\(Z\sim\mathsf{norm}(\mathtt{mean}=0,\,\mathtt{sd}=1)\) and
\(V\sim\mathsf{chisq}(\mathtt{df}=r)\) are independent. This is in
fact the definition of Student's \(t\) distribution.
\end{proof}

This distribution was first published by W. S. Gosset (1900) under the
pseudonym Student, and the distribution has consequently come to be
known as Student's \(t\) distribution. The PDF of \(T\) can be derived
explicitly using the techniques of Section \ref{sec-6-4}; it takes the form
\begin{equation}
f_{X}(x)=\frac{\Gamma[(r+1)/2]}{\sqrt{r\pi}\ \Gamma(r/2)}\left(1+\frac{x^{2}}{r}\right)^{-(r+1)/2},\quad -\infty < x < \infty.
\end{equation}
Any random variable \(X\) with the preceding PDF is said to have
Student's \(t\) distribution with \(r\) \emph{degrees of freedom}, and we
write \(X\sim\mathsf{t}(\mathtt{df}=r)\). The shape of the PDF is
similar to the normal, but the tails are considerably heavier. See
Figure \ref{fig-Students-t-dist-vary-df}. As with the normal distribution,
there are four functions in \(\mathsf{R}\) associated with the \(t\)
distribution, namely \texttt{dt}, \texttt{pt},=qt=, and \texttt{rt}, which compute the PDF,
CDF, quantile function, and generate random variates, respectively.

The code to produce Figure \ref{fig-Students-t-dist-vary-df} is

\begin{verbatim}
curve(dt(x, df = 30), from = -3, to = 3, lwd = 3, ylab = "y")
ind <- c(1, 2, 3, 5, 10)
for (i in ind) curve(dt(x, df = i), -3, 3, add = TRUE)
\end{verbatim}

\begin{figure}[ht!]
\centering
\includegraphics[width=0.9\textwidth]{fig/sampdist-Students-t-dist-vary-df.ps}
\caption[Student's \(t\) distribution for various degrees of freedom]{\label{fig-Students-t-dist-vary-df}\small A plot of Student's \(t\) distribution for various degrees of freedom.}
\end{figure}

Similar to that done for the normal we may define
\(\mathsf{t}_{\alpha}(\mathtt{df}=n-1)\) as the number on the
\(x\)-axis such that there is exactly \(\alpha\) area under the
\(\mathsf{t}(\mathtt{df}=n-1)\) curve to its right.


Find \(\mathsf{t}{}_{0.01}(\mathtt{df}=23)\) with the quantile
function.

\begin{verbatim}
qt(0.01, df = 23, lower.tail = FALSE)
\end{verbatim}

\begin{verbatim}
: [1] 2.499867
\end{verbatim}

\begin{rem}
There are a few things to note about the \(\mathtt{t}(\mathtt{df}=r)\)
distribution.
\begin{enumerate}
\item The \(\mathtt{t}(\mathtt{df}=1)\) distribution is the same as the
\(\mathsf{cauchy}(\mathtt{location}=0,\,\mathtt{scale}=1)\)
distribution. The Cauchy distribution is rather pathological and is
a counterexample to many famous results.
\item The standard deviation of \(\mathsf{t}(\mathtt{df}=r)\) is
undefined (that is, infinite) unless \(r>2\). When \(r\) is more
than 2, the standard deviation is always bigger than one, but
decreases to 1 as \(r\to\infty\).
\item As \(r\to\infty\), the \(\mathtt{t}(\mathtt{df}=r)\) distribution
approaches the \(\mathsf{norm}(\mathtt{mean}=0,\,\mathtt{sd}=1)\)
distribution.
\end{enumerate}
\end{rem}

\section{The Central Limit Theorem}
\label{sec-8-3}

In this section we study the distribution of the sample mean when the
underlying distribution is \emph{not} normal. We saw in Section \ref{sec-8-2} that when \(X_{1}\), \(X_{2}\), \ldots{} , \(X_{n}\) is a
\(SRS(n)\) from a
\(\mathsf{norm}(\mathtt{mean}=\mu,\,\mathtt{sd}=\sigma)\) distribution
then \(\overline{X} \sim \mathsf{norm}(\mathtt{mean} =
\mu,\,\mathtt{sd} = \sigma/\sqrt{n})\). In other words, we may say
(owing to Fact \ref{fac-lin-trans-norm-is-norm}) when the underlying
population is normal that the sampling distribution of \(Z\) defined
by
\begin{equation}
Z=\frac{\overline{X}-\mu}{\sigma/\sqrt{n}}
\end{equation}
is \(\mathsf{norm}(\mathtt{mean}=0,\,\mathtt{sd}=1)\). 

However, there are many populations that are \emph{not} normal \ldots{} and the
statistician often finds herself sampling from such populations. What
can be said in this case? The surprising answer is contained in the
following theorem.

\begin{thm}[\textbf{The Central Limit Theorem}]
\label{thm-central-limit-thrm} Let \(X_{1}\), \(X_{2}\), \ldots{}, \(X_{n}\) be
a \(SRS(n)\) from a population distribution with mean \(\mu\) and
finite standard deviation \(\sigma\). Then the sampling distribution
of
\begin{equation}
Z=\frac{\overline{X}-\mu}{\sigma/\sqrt{n}}
\end{equation}
approaches a \(\mathsf{norm}(\mathtt{mean}=0,\,\mathtt{sd}=1)\) distribution as \(n\to\infty\).
\end{thm}

\begin{rem}
We suppose that \(X_{1}\), \(X_{2}\), \ldots{} , \(X_{n}\) are IID, and we
learned in Section \ref{sec-8-1} that \(\overline{X}\) has
mean \(\mu\) and standard deviation \(\sigma/\sqrt{n}\), so we already
knew that \(Z\) has mean zero and standard deviation one. The beauty
of the CLT is that it addresses the \emph{shape} of \(Z\)'s distribution
when the sample size is large.
\end{rem}

\begin{rem}
Notice that the shape of the underlying population's distribution is
not mentioned in Theorem \ref{thm-central-limit-thrm}; indeed, the result is true for any
population that is well-behaved enough to have a finite standard
deviation. In particular, if the population is normally distributed
then we know from Section \ref{sec-8-2-1} that the distribution
of \(\overline{X}\) (and \(Z\) by extension) is \emph{exactly} normal, for
\emph{every} \(n\).
\end{rem}

\begin{rem}
How large is "sufficiently large"? It is here that the shape of the
underlying population distribution plays a role. For populations with
distributions that are approximately symmetric and mound-shaped, the
samples may need to be only of size four or five, while for highly
skewed or heavy-tailed populations the samples may need to be much
larger for the distribution of the sample means to begin to show a
bell-shape. Regardless, for a given population distribution (with
finite standard deviation) the approximation tends to be better for
larger sample sizes.
\end{rem}

\subsection{How to do it with \(\mathsf{R}\)}
\label{sec-8-3-1}

The \texttt{TeachingDemos} package \cite{TeachingDemos} has \texttt{clt.examp} and
the \texttt{distrTeach} \cite{distrTeach} package has \texttt{illustrateCLT}. Try
the following at the command line (output omitted):
\begin{verbatim}
example(clt.examp)
\end{verbatim}
and
\begin{verbatim}
example(illustrateCLT)
\end{verbatim}

The \texttt{IPSUR} package \cite{IPSUR} has the functions \texttt{clt1}, \texttt{clt2}, and
\texttt{clt3} (see Exercise \ref{xca-clt123} at the end of this chapter). Its purpose
is to investigate what happens to the sampling distribution of
\(\overline{X}\) when the population distribution is mound shaped,
finite support, and skewed, namely \(\mathsf{t}(\mathtt{df}=3)\),
\(\mathsf{unif}(\mathtt{a}=0,\,\mathtt{b}=10)\), and
\(\mathsf{gamma}(\mathtt{shape}=1.21,\,\mathtt{scale}=1/2.37)\),
respectively.

For example, when the command \texttt{clt1()} is issued a plot window opens
to show a graph of the PDF of a \(\mathsf{t}(\mathtt{df}=3)\)
distribution. On the display are shown numerical values of the
population mean and variance. While the students examine the graph the
computer is simulating random samples of size \texttt{sample.size = 2} from
the population distribution \texttt{rt} a total of \texttt{N.iter = 100000} times,
and sample means are calculated of each sample. Next follows a
histogram of the simulated sample means, which closely approximates
the sampling distribution of \(\overline{X}\), see Section
\ref{sec-8-5}. Also shown are the sample
mean and sample variance of all of the simulated \( \overline{X} \)
values. As a final step, when the student clicks the second plot, a
normal curve with the same mean and variance as the simulated \(
\overline{X} \) values is superimposed over the histogram. Students
should compare the population theoretical mean and variance to the
simulated mean and variance of the sampling distribution. They should
also compare the shape of the simulated sampling distribution to the
shape of the normal distribution.

The three separate \texttt{clt1}, \texttt{clt2}, and \texttt{clt3} functions were written
so that students could compare what happens overall when the shape of
the population distribution changes. It would be possible to combine
all three into one big function, \texttt{clt} which covers all three cases
(and more).

\section{Sampling Distributions of Two-Sample Statistics}
\label{sec-8-4}

There are often two populations under consideration, and it sometimes
of interest to compare properties between groups. To do so we take
independent samples from each population and calculate respective
sample statistics for comparison. In some simple cases the sampling
distribution of the comparison is known and easy to derive; such cases
are the subject of the present section.

\subsection{Difference of Independent Sample Means}
\label{sec-8-4-1}

\begin{prop}
Let \(X_{1}\), \(X_{2}\), \ldots{} , \(X_{n_{1}}\) be an \(SRS(n_{1})\)
from a
\(\mathsf{norm}(\mathtt{mean}=\mu_{X},\,\mathtt{sd}=\sigma_{X})\)
distribution and let \(Y_{1}\), \(Y_{2}\), \ldots{} , \(Y_{n_{2}}\) be an
\(SRS(n_{2})\) from a
\(\mathsf{norm}(\mathtt{mean}=\mu_{Y},\,\mathtt{sd}=\sigma_{Y})\)
distribution. Suppose that \(X_{1}\), \(X_{2}\), \ldots{} , \(X_{n_{1}}\)
and \(Y_{1}\), \(Y_{2}\), \ldots{} , \(Y_{n_{2}}\) are independent
samples. Then the quantity
\begin{equation}
\label{eq-diff-indep-sample-means}
\frac{\overline{X}-\overline{Y}-(\mu_{X}-\mu_{Y})}{\sqrt{\left.\sigma_{X}^{2}\right/ n_{1}+\left.\sigma_{Y}^{2}\right/ n_{2}}}
\end{equation}
has a \(\mathsf{norm}(\mathtt{mean}=0,\,\mathtt{sd}=1)\) sampling
distribution. Equivalently, \(\overline{X}-\overline{Y}\) has a
\(\mathsf{norm}(\mathtt{mean}=\mu_{X}-\mu_{Y},\,\mathtt{sd}=\sqrt{\left.\sigma_{X}^{2}\right/
n_{1}+\left.\sigma_{Y}^{2}\right/ n_{2}})\) sampling distribution.
\end{prop}

\begin{proof}
We know that \(\overline{X}\) is
\(\mathsf{norm}(\mathtt{mean}=\mu_{X},\,\mathtt{sd}=\sigma_{X}/\sqrt{n_{1}})\)
and we also know that \(\overline{Y}\) is
\(\mathsf{norm}(\mathtt{mean}=\mu_{Y},\,\mathtt{sd}=\sigma_{Y}/\sqrt{n_{2}})\). And
since the samples \(X_{1}\), \(X_{2}\), \ldots{}, \(X_{n_{1}}\) and
\(Y_{1}\), \(Y_{2}\), \ldots{}, \(Y_{n_{2}}\) are independent, so too are
\(\overline{X}\) and \(\overline{Y}\). The distribution of their
difference is thus normal as well, and the mean and standard deviation
are given by Proposition \ref{pro-mean-sd-lin-comb-two}.
\end{proof}

\begin{rem}
Even if the distribution of one or both of the samples is not normal,
the quantity in Equation \eqref{eq-diff-indep-sample-means} will be
approximately normal provided both sample sizes are large.
\end{rem}

\textbf{Remark:} For the special case of \(\mu_{X}=\mu_{Y}\) we have shown
 that
\begin{equation}
\frac{\overline{X} - \overline{Y}}{\sqrt{ \sigma_{X}^{2}/ n_{1} + \sigma_{Y}^{2}/n_{2}}}
\end{equation}

has a \(\mathsf{norm}(\mathtt{mean}=0,\,\mathtt{sd}=1)\) sampling
distribution, or in other words, \(\overline{X} - \overline{Y}\) has a
\(\mathsf{norm}(\mathtt{mean} = 0,\,\mathtt{sd} = \sqrt{\sigma_{X}^{2}
/ n_{1} + \sigma_{Y}^{2} / n_{2}})\) sampling distribution. This will
be important when it comes time to do hypothesis tests; see Section
\ref{sec-9-3}.


\subsection{Difference of Independent Sample Proportions}
\label{sec-8-4-2}

\begin{prop}
Let \(X_{1}\), \(X_{2}\), \ldots{}, \(X_{n_{1}}\) be an \(SRS(n_{1})\) from
a \(\mathsf{binom}(\mathtt{size}=1,\,\mathtt{prob}=p_{1})\)
distribution and let \(Y_{1}\), \(Y_{2}\), \ldots{}, \(Y_{n_{2}}\) be an
\(SRS(n_{2})\) from a
\(\mathsf{binom}(\mathtt{size}=1,\,\mathtt{prob}=p_{2})\)
distribution. Suppose that \(X_{1}\), \(X_{2}\), \ldots{} , \(X_{n_{1}}\)
and \(Y_{1}\), \(Y_{2}\), \ldots{} , \(Y_{n_{2}}\) are independent
samples. Define
\begin{equation}
\hat{p}_{1}=\frac{1}{n_{1}}\sum_{i=1}^{n_{1}}X_{i}\quad \mbox{and}\quad \hat{p}_{2}=\frac{1}{n_{2}}\sum_{j=1}^{n_{2}}Y_{j}.
\end{equation}
Then the sampling distribution of
\begin{equation}
\frac{\hat{p}_{1}-\hat{p}_{2}-(p_{1}-p_{2})}{\sqrt{\frac{p_{1}(1-p_{1})}{n_{1}}+\frac{p_{2}(1-p_{2})}{n_{2}}}}
\end{equation}
approaches a \(\mathsf{norm}(\mathtt{mean}=0,\,\mathtt{sd}=1)\) distribution as both \(n_{1},\, n_{2}\to\infty\). In other words, the sampling distribution of \(\hat{p}_{1}-\hat{p}_{2}\) is approximately
\begin{equation}
\mathsf{norm}\left(\mathtt{mean}=p_{1}-p_{2},\,\mathtt{sd}=\sqrt{\frac{p_{1}(1-p_{1})}{n_{1}}+\frac{p_{2}(1-p_{2})}{n_{2}}}\right),
\end{equation}
provided both \(n_{1}\) and \(n_{2}\) are sufficiently large.
\end{prop}

\begin{proof}
We know that \(\hat{p}_{1}\) is approximately normal for \(n_{1}\)
sufficiently large by the CLT, and we know that \(\hat{p}_{2}\) is
approximately normal for \(n_{2}\) sufficiently large, also by the
CLT. Further, \(\hat{p}_{1}\) and \(\hat{p}_{2}\) are independent
since they are derived from independent samples. And a difference of
independent (approximately) normal distributions is (approximately)
normal, by Exercise \ref{xca-diff-indep-norm}.

The expressions for the mean and standard deviation follow immediately
from Proposition \ref{pro-mean-sd-lin-comb-two} combined with the formulas
for the \(\mathsf{binom}(\mathtt{size}=1,\,\mathtt{prob}=p)\)
distribution from Chapter \ref{sec-5}.
\end{proof}

\subsection{Ratio of Independent Sample Variances}
\label{sec-8-4-3}

\begin{prop}
Let \(X_{1}\), \(X_{2}\), \ldots{}, \(X_{n_{1}}\) be an \(SRS(n_{1})\) from
a \(\mathsf{norm}(\mathtt{mean}=\mu_{X},\,\mathtt{sd}=\sigma_{X})\)
distribution and let \(Y_{1}\), \(Y_{2}\), \ldots{} , \(Y_{n_{2}}\) be an
\(SRS(n_{2})\) from a
\(\mathsf{norm}(\mathtt{mean}=\mu_{Y},\,\mathtt{sd}=\sigma_{Y})\)
distribution. Suppose that \(X_{1}\), \(X_{2}\), \ldots{} , \(X_{n_{1}}\)
and \(Y_{1}\), \(Y_{2}\), \ldots{} , \(Y_{n_{2}}\) are independent
samples. Then the ratio
\begin{equation}
F=\frac{\sigma_{Y}^{2}S_{X}^{2}}{\sigma_{X}^{2}S_{Y}^{2}}
\end{equation}
has an \(\mathsf{f}(\mathtt{df1}=n_{1}-1,\,\mathtt{df2}=n_{2}-1)\)
sampling distribution.
\end{prop}

\begin{proof}
We know from Theorem \ref{thm-Xbar-andS} that
\((n_{1}-1)S_{X}^{2}/\sigma_{X}^{2}\) is distributed
\(\mathsf{chisq}(\mathtt{df}=n_{1}-1)\) and
\((n_{2}-1)S_{Y}^{2}/\sigma_{Y}^{2}\) is distributed
\(\mathsf{chisq}(\mathtt{df}=n_{2}-1)\). Now write \[
F=\frac{\sigma_{Y}^{2}S_{X}^{2}}{\sigma_{X}^{2}S_{Y}^{2}}=\frac{\left.(n_{1}-1)S_{X}^{2}\right/
(n_{1}-1)}{\left.(n_{2}-1)S_{Y}^{2}\right/
(n_{2}-1)}\cdot\frac{\left.1\right/ \sigma_{X}^{2}}{\left.1\right/
\sigma_{Y}^{2}}, \] by multiplying and dividing the numerator with
\(n_{1}-1\) and doing likewise for the denominator with
\(n_{2}-1\). Now we may regroup the terms into \[
F=\frac{\left.\frac{(n_{1}-1)S_{X}^{2}}{\sigma_{X}^{2}}\right/
(n_{1}-1)}{\left.\frac{(n_{2}-1)S_{Y}^{2}}{\sigma_{Y}^{2}}\right/
(n_{2}-1)}, \] and we recognize \(F\) to be the ratio of independent
\(\mathsf{chisq}\) distributions, each divided by its respective
numerator \(\mathtt{df}=n_{1}-1\) and denominator
\(\mathtt{df}=n_{1}-1\) degrees of freedom. This is, indeed, the
definition of Snedecor's \(F\) distribution.
\end{proof}

\begin{rem}
For the special case of \(\sigma_{X}=\sigma_{Y}\) we have shown that
\begin{equation}
F=\frac{S_{X}^{2}}{S_{Y}^{2}}
\end{equation}
has an \(\mathsf{f}(\mathtt{df1}=n_{1}-1,\,\mathtt{df2}=n_{2}-1)\)
sampling distribution. This will be important in Chapters
\ref{sec-9} onward.
\end{rem}

\section{Simulated Sampling Distributions}
\label{sec-8-5}

Some comparisons are meaningful, but their sampling distribution is
not quite so tidy to describe analytically. What do we do then?

As it turns out, we do not need to know the exact analytical form of
the sampling distribution; sometimes it is enough to approximate it
with a simulated distribution. In this section we will show you
how. Note that \(\mathsf{R}\) is particularly well suited to compute
simulated sampling distributions, much more so than, say, SPSS or SAS.

\subsection{The Interquartile Range}
\label{sec-8-5-1}

\begin{verbatim}
iqrs <- replicate(100, IQR(rnorm(100)))
\end{verbatim}

We can look at the mean of the simulated values
\begin{verbatim}
mean(iqrs)    # close to 1
\end{verbatim}

\begin{verbatim}
: [1] 1.331184
\end{verbatim}

and we can see the standard deviation
\begin{verbatim}
sd(iqrs)
\end{verbatim}

\begin{verbatim}
: [1] 0.1683535
\end{verbatim}

Now let's take a look at a plot of the simulated values

\begin{verbatim}
hist(iqrs, breaks = 20)
\end{verbatim}

\begin{figure}[ht!]
\centering
\includegraphics[width=0.9\textwidth]{fig/sampdist-simulated-IQR.ps}
\caption[Plot of simulated IQRs]{\label{fig-simulated-IQR}\small A plot of simulated IQRs.}
\end{figure}

\subsection{The Median Absolute Deviation}
\label{sec-8-5-2}

\begin{verbatim}
mads <- replicate(100, mad(rnorm(100)))
\end{verbatim}

We can look at the mean of the simulated values

\begin{verbatim}
mean(mads)    # close to 1.349
\end{verbatim}

\begin{verbatim}
: [1] 1.007404
\end{verbatim}

and we can see the standard deviation

\begin{verbatim}
sd(mads)
\end{verbatim}

\begin{verbatim}
: [1] 0.1138528
\end{verbatim}

Now let's take a look at a plot of the simulated values

\begin{verbatim}
hist(mads, breaks = 20)
\end{verbatim}

\begin{figure}[ht!]
\centering
\includegraphics[width=0.9\textwidth]{fig/sampdist-simulated-MAD.ps}
\caption[Plot of simulated MADs]{\label{fig-simulated-MAD}\small A plot of simulated MADs.}
\end{figure}

\newpage{}

\section{Exercises}
\label{sec-8-6}
\setcounter{thm}{0}

\begin{xca}
Suppose that we observe a random sample \(X_{1}\), \(X_{2}\), \ldots{} ,
\(X_{n}\) of size SRS( n = 24) from a \(
\mathsf{norm}(\mathtt{mean}= SRC_R[:eval no-export]{mu[1]} 20 )\) distribution.
\begin{enumerate}
\item What is the mean of \(\overline{X}\)?
\item What is the standard deviation of \(\overline{X}\)?
\item What is the distribution of \(\overline{X}\)? (approximately)
\item Find \(\mathbb{P}(a< \overline{X} \leq b)\)
\item Find \(\mathbb{P}(\overline{X} > c)\).
\end{enumerate}
\end{xca}

\begin{xca}
\label{xca-clt123} In this exercise we will investigate how the shape of
the population distribution affects the time until the distribution of
\(\overline{X}\) is acceptably normal.
\end{xca}

\begin{xca}
Let \(X_{1}\),\ldots{}, \(X_{25}\) be a random sample from a
\(\mathsf{norm}(\mathtt{mean}=37,\,\mathtt{sd}=45)\) distribution, and
let \(\overline{X}\) be the sample mean of these \(n=25\)
observations.
\begin{enumerate}
\item How is \(\overline{X}\) distributed? 
\(\mathsf{norm}(\mathtt{mean}=37,\,\mathtt{sd}=45/\sqrt{25})\)
\item Find \(\mathbb{P}(\overline{X} > 43.1)\).
\end{enumerate}
And that's all she wrote.
\end{xca}
