* Continuous Distributions                                         :contdist:
:PROPERTIES:
:tangle: R/06-contdist.R
:CUSTOM_ID: cha-Continuous-Distributions
:END:

#+BEGIN_SRC R :exports none :eval never
#    IPSUR: Introduction to Probability and Statistics Using R
#    Copyright (C) 2014  G. Jay Kerns
#
#    Chapter: Continuous Distributions
#
#    This file is part of IPSUR.
#
#    IPSUR is free software: you can redistribute it and/or modify
#    it under the terms of the GNU General Public License as published by
#    the Free Software Foundation, either version 3 of the License, or
#    (at your option) any later version.
#
#    IPSUR is distributed in the hope that it will be useful,
#    but WITHOUT ANY WARRANTY; without even the implied warranty of
#    MERCHANTABILITY or FITNESS FOR A PARTICULAR PURPOSE.  See the
#    GNU General Public License for more details.
#
#    You should have received a copy of the GNU General Public License
#    along with IPSUR.  If not, see <http://www.gnu.org/licenses/>.
#+END_SRC

#+BEGIN_SRC R :exports none :eval no-export
# This chapter's package dependencies
library(distrEx)
library(actuar)
#+END_SRC

#+LaTeX: \noindent 
The focus of the last chapter was on random variables whose support
can be written down in a list of values (finite or countably
infinite), such as the number of successes in a sequence of Bernoulli
trials. Now we move to random variables whose support is a whole range
of values, say, an interval \((a,b)\). It is shown in later classes
that it is impossible to write all of the numbers down in a list;
there are simply too many of them.

This chapter begins with continuous random variables and the
associated PDFs and CDFs The continuous uniform distribution is
highlighted, along with the Gaussian, or normal, distribution. Some
mathematical details pave the way for a catalogue of models.

The interested reader who would like to learn more about any of the
assorted discrete distributions mentioned below should take a look at
/Continuous Univariate Distributions, Volumes 1/ and /2/ by Johnson
/et al/ \cite{Johnson1994,Johnson1995}.

*What do I want them to know?*
- how to choose a reasonable continuous model under a variety of
  physical circumstances
- basic correspondence between continuous versus discrete random
  variables
- the general tools of the trade for manipulation of continuous random
  variables, integration, /etc/.
- some details on a couple of continuous models, and exposure to a
  bunch of other ones
- how to make new continuous random variables from old ones

** Continuous Random Variables
:PROPERTIES:
:CUSTOM_ID: sec-continuous-random-variables
:END:

*** Probability Density Functions
:PROPERTIES:
:CUSTOM_ID: sub-probability-density-functions
:END:

Continuous random variables have supports that look like
\begin{equation}
S_{X}=[a,b]\mbox{ or }(a,b),
\end{equation}
or unions of intervals of the above form. Examples of random variables
that are often taken to be continuous are:

- the height or weight of an individual,
- other physical measurements such as the length or size of an object,
  and
- durations of time (usually).

Every continuous random variable \(X\) has a /probability density
function/ (PDF) denoted \(f_{X}\) associated with it[fn:fn-patho] that
satisfies three basic properties:
1. \(f_{X}(x)>0\) for \(x\in S_{X}\),
2. \(\int_{x\in S_{X}}f_{X}(x)\,\mathrm{d} x=1\), and
3. <<enu-contrvcond3>> \(\mathbb{P}(X\in A)=\int_{x\in
   A}f_{X}(x)\:\mathrm{d} x\), for an event \(A\subset S_{X}\).

[fn:fn-patho] Not true. There are pathological random variables with
no density function. (This is one of the crazy things that can happen
in the world of measure theory). But in this book we will not get even
close to these anomalous beasts, and regardless it can be proved that
the CDF always exists.

#+BEGIN_rem
We can say the following about continuous random variables:
- Usually, the set \(A\) in [[enu-contrvcond3]] takes the form of an
  interval, for example, \(A=[c,d]\), in which case
  \begin{equation}
  \mathbb{P}(X\in A)=\int_{c}^{d}f_{X}(x)\:\mathrm{d} x.
  \end{equation}
- It follows that the probability that \(X\) falls in a given interval
  is simply the /area under the curve/ of \(f_{X}\) over the interval.
- Since the area of a line \(x=c\) in the plane is zero,
  \(\mathbb{P}(X=c)=0\) for any value \(c\). In other words, the
  chance that \(X\) equals a particular value \(c\) is zero, and this
  is true for any number \(c\). Moreover, when \(a<b\) all of the
  following probabilities are the same:
  \begin{equation}
  \mathbb{P}(a\leq X\leq b)=\mathbb{P}(a<X\leq b)=\mathbb{P}(a\leq X<b)=\mathbb{P}(a<X<b).
  \end{equation}
- The PDF \(f_{X}\) can sometimes be greater than 1. This is in
  contrast to the discrete case; every nonzero value of a PMF is a
  probability which is restricted to lie in the interval \([0,1]\).
#+END_rem

We met the cumulative distribution function, \(F_{X}\), in Chapter
[[#cha-Discrete-Distributions]]. Recall that it is defined by
\(F_{X}(t)=\mathbb{P}(X\leq t)\), for \(-\infty<t<\infty\). While in
the discrete case the CDF is unwieldy, in the continuous case the CDF
has a relatively convenient form:
\begin{equation}
F_{X}(t)=\mathbb{P}(X\leq t)=\int_{-\infty}^{t}f_{X}(x)\:\mathrm{d} x,\quad -\infty < t < \infty.
\end{equation}

#+BEGIN_rem
For any continuous CDF \(F_{X}\) the following are true.
- \(F_{X}\) is nondecreasing , that is, \(t_{1}\leq t_{2}\) implies
  \(F_{X}(t_{1})\leq F_{X}(t_{2})\).
- \(F_{X}\) is continuous (see Appendix
  [[#sec-Differential-and-Integral]]). Note the distinction from the
  discrete case: CDFs of discrete random variables are not continuous,
  they are only right continuous.
- \(\lim_{t\to-\infty}F_{X}(t)=0\) and
  \(\lim_{t\to\infty}F_{X}(t)=1\).
#+END_rem

There is a handy relationship between the CDF and PDF in the
continuous case. Consider the derivative of \(F_{X}\):
\begin{equation}
F'_{X}(t)=\frac{\mathrm{d}}{\mathrm{d} t}F_{X}(t)=\frac{\mathrm{d}}{\mathrm{d} t}\,\int_{-\infty}^{t}f_{X}(x)\,\mathrm{d} x=f_{X}(t),
\end{equation}
the last equality being true by the Fundamental Theorem of Calculus,
part (2) (see Appendix [[#sec-Differential-and-Integral]]). In short,
\((F_{X})'=f_{X}\) in the continuous case[fn:fn-disc].

[fn:fn-disc] In the discrete case, \(f_{X}(x)=F_{X}(x)-\lim_{t\to
x^{-}}F_{X}(t)\).

*** Expectation of Continuous Random Variables
:PROPERTIES:
:CUSTOM_ID: sub-Expectation-of-Continuous
:END:

For a continuous random variable \(X\) the expected value of \(g(X)\)
is
\begin{equation}
\mathbb{E} g(X)=\int_{x\in S}g(x)f_{X}(x)\:\mathrm{d} x,
\end{equation}
provided the (potentially improper) integral \(\int_{S}|g(x)|\,
f(x)\mathrm{d} x\) is convergent. One important example is the mean
\(\mu\), also known as \(\mathbb{E} X\):
\begin{equation}
\mu=\mathbb{E} X=\int_{x\in S}xf_{X}(x)\:\mathrm{d} x,
\end{equation}
provided \(\int_{S}|x|f(x)\mathrm{d} x\) is finite. Also there is the variance
\begin{equation}
\sigma^{2}=\mathbb{E}(X-\mu)^{2}=\int_{x\in S}(x-\mu)^{2}f_{X}(x)\,\mathrm{d} x,
\end{equation}
which can be computed with the alternate formula
\(\sigma^{2}=\mathbb{E} X^{2}-(\mathbb{E} X)^{2}\). In addition, there
is the standard deviation \(\sigma=\sqrt{\sigma^{2}}\). The moment
generating function is given by
\begin{equation}
M_{X}(t)=\mathbb{E}\:\mathrm{e}^{tX}=\int_{-\infty}^{\infty}\mathrm{e}^{tx}f_{X}(x)\:\mathrm{d} x,
\end{equation}
provided the integral exists (is finite) for all \(t\) in a
neighborhood of \(t=0\).

# +BEGIN_exampletoo
<<exa-cont-pdf3x2>> Let the continuous random variable \(X\) have PDF
\[ f_{X}(x)=3x^{2},\quad 0\leq x\leq 1. \] We will see later that
\(f_{X}\) belongs to the /Beta/ family of distributions. It is easy to
see that \(\int_{-\infty}^{\infty}f(x)\mathrm{d} x=1\).
\begin{align*}
\int_{-\infty}^{\infty}f_{X}(x)\mathrm{d} x & =\int_{0}^{1}3x^{2}\:\mathrm{d} x\\
 & =\left.x^{3}\right|_{x=0}^{1}\\
 & =1^{3}-0^{3}\\
 & =1.
\end{align*}
This being said, we may find \(\mathbb{P}(0.14\leq X<0.71)\).
\begin{align*}
\mathbb{P}(0.14\leq X<0.71) & =\int_{0.14}^{0.71}3x^{2}\mathrm{d} x,\\
 & =\left.x^{3}\right|_{x=0.14}^{0.71}\\
 & =0.71^{3}-0.14^{3}\\
 & \approx0.355167.
\end{align*}
We can find the mean and variance in an identical manner.
\begin{align*}
\mu=\int_{-\infty}^{\infty}xf_{X}(x)\mathrm{d} x & =\int_{0}^{1}x\cdot3x^{2}\:\mathrm{d} x,\\
 & =\frac{3}{4}x^{4}|_{x=0}^{1},\\
 & =\frac{3}{4}.
\end{align*}
It would perhaps be best to calculate the variance with the shortcut
formula \(\sigma^{2}=\mathbb{E} X^{2}-\mu^{2}\):
\begin{align*}
\mathbb{E} X^{2}=\int_{-\infty}^{\infty}x^{2}f_{X}(x)\mathrm{d} x & =\int_{0}^{1}x^{2}\cdot3x^{2}\:\mathrm{d} x\\
 & =\left.\frac{3}{5}x^{5}\right|_{x=0}^{1}\\
 & =3/5.
\end{align*}
which gives \(\sigma^{2}=3/5-(3/4)^{2}=3/80\).
# +END_exampletoo


# +BEGIN_exampletoo
<<exa-cont-pdf-3x4>> We will try one with unbounded support to brush
up on improper integration. Let the random variable \(X\) have PDF \[
f_{X}(x)=\frac{3}{x^{4}},\quad x>1.  \] We can show that
\(\int_{-\infty}^{\infty}f(x)\mathrm{d} x=1\):
\begin{align*}
\int_{-\infty}^{\infty}f_{X}(x)\mathrm{d} x & =\int_{1}^{\infty}\frac{3}{x^{4}}\:\mathrm{d} x,\\
 & =\lim_{t\to\infty}\int_{1}^{t}\frac{3}{x^{4}}\:\mathrm{d} x,\\
 & =\lim_{t\to\infty}\ \left.3\,\frac{1}{-3}x^{-3}\right|_{x=1}^{t},\\
 & =-\left(\lim_{t\to\infty}\frac{1}{t^{3}}-1\right),\\
 & =1.
\end{align*}
We calculate \(\mathbb{P}(3.4\leq X<7.1)\):
\begin{align*}
\mathbb{P}(3.4\leq X<7.1) & =\int_{3.4}^{7.1}3x^{-4}\mathrm{d} x,\\
 & =\left.3\,\frac{1}{-3}x^{-3}\right|_{x=3.4}^{7.1},\\
 & =-1(7.1^{-3}-3.4^{-3}),\\
 & \approx0.0226487123.
\end{align*}
We locate the mean and variance just like before.
\begin{align*}
\mu=\int_{-\infty}^{\infty}xf_{X}(x)\mathrm{d} x & =\int_{1}^{\infty}x\cdot\frac{3}{x^{4}}\:\mathrm{d} x,\\
 & =\left.3\,\frac{1}{-2}x^{-2}\right|_{x=1}^{\infty},\\
 & =-\frac{3}{2}\left(\lim_{t\to\infty}\frac{1}{t^{2}}-1\right),\\
 & =\frac{3}{2}.
\end{align*}
Again we use the shortcut \(\sigma^{2}=\mathbb{E} X^{2}-\mu^{2}\):
\begin{align*}
\mathbb{E} X^{2}=\int_{-\infty}^{\infty}x^{2}f_{X}(x)\mathrm{d} x & =\int_{1}^{\infty}x^{2}\cdot\frac{3}{x^{4}}\:\mathrm{d} x,\\
 & =\left.3\:\frac{1}{-1}x^{-1}\right|_{x=1}^{\infty},\\
 & =-3\left(\lim_{t\to\infty}\frac{1}{t^{2}}-1\right),\\
 & =3,
\end{align*}
which closes the example with \(\sigma^{2}=3-(3/2)^{2}=3/4\).
# +END_exampletoo

**** How to do it with \(\mathsf{R}\)

There exist utilities to calculate probabilities and expectations for
general continuous random variables, but it is better to find a
built-in model, if possible. Sometimes it is not possible. We show how
to do it the long way, and the =distr= @@latex:\index{R packages@\textsf{R}
packages!distr@\texttt{distr}}@@ package \cite{distr} way.

# +BEGIN_exampletoo

Let \(X\) have PDF \(f(x)=3x^{2}\), \(0<x<1\) and find
\(\mathbb{P}(0.14\leq X\leq0.71)\). (We will ignore that \(X\) is a
beta random variable for the sake of argument.)

#+BEGIN_SRC R :exports both :results output pp 
f <- function(x) 3*x^2
integrate(f, lower = 0.14, upper = 0.71)
#+END_SRC

#+RESULTS:
: 0.355167 with absolute error < 3.9e-15

Compare this to the answer we found in Example [[exa-cont-pdf3x2]]. We could
integrate the function \(x \cdot f(x)=\) =3*x^3= from zero to one to
get the mean, and use the shortcut \(\sigma^{2}=\mathbb{E}
X^{2}-\left(\mathbb{E} X\right)^{2}\) for the variance.

# +END_exampletoo


# +BEGIN_exampletoo

Let \(X\) have PDF \(f(x)=3/x^{4}\), \(x>1\). We may integrate the
function \(g(x) = x \cdot f(x)=\) =3/x^3= from zero to infinity to get
the mean of \(X\).

#+BEGIN_SRC R :exports both :results output pp 
g <- function(x) 3/x^3
integrate(g, lower = 1, upper = Inf)
#+END_SRC

#+RESULTS:
: 1.5 with absolute error < 1.7e-14

Compare this to the answer we got in Example [[exa-cont-pdf-3x4]]. Use =-Inf=
for \(-\infty\).

# +END_exampletoo


# +BEGIN_exampletoo

Let us redo Example [[exa-cont-pdf3x2]] with the =distr= package
\cite{distr}. The method is similar to that encountered in Section
[[#sub-disc-rv-how-r]] in Chapter [[#cha-Discrete-Distributions]]. We define an
absolutely continuous random variable:

#+BEGIN_SRC R :exports both :results output pp
f <- function(x) 3*x^2
X <- AbscontDistribution(d = f, low1 = 0, up1 = 1)
p(X)(0.71) - p(X)(0.14)
#+END_SRC

#+RESULTS:
: [1] 0.355167

Compare this to the answer we found earlier. Now let us try
expectation with the =distrEx= package \cite{distrEx}:
#+BEGIN_SRC R :exports both :results output pp 
E(X); var(X); 3/80
#+END_SRC

#+RESULTS:
: [1] 0.7496337
: [1] 0.03768305
: [1] 0.0375

Compare these answers to the ones we found in Example [[exa-cont-pdf3x2]]. Why
are they different? Because the =distrEx= package resorts to numerical
methods when it encounters a model it does not recognize. This means
that the answers we get for calculations may not exactly match the
theoretical values. Be careful.
# +END_exampletoo

** The Continuous Uniform Distribution
:PROPERTIES:
:CUSTOM_ID: sec-The-Continuous-Uniform
:END:

A random variable \(X\) with the continuous uniform distribution on
the interval \((a,b)\) has PDF
\begin{equation}
f_{X}(x)=\frac{1}{b-a}, \quad a < x < b.
\end{equation}
The associated \(\mathsf{R}\) function is
\(\mathsf{dunif}(\mathtt{min}=a,\,\mathtt{max}=b)\). We write
\(X\sim\mathsf{unif}(\mathtt{min}=a,\,\mathtt{max}=b)\). Due to the
particularly simple form of this PDF we can also write down explicitly
a formula for the CDF \(F_{X}\): @@latex:\begin{equation} \label{eq-unif-cdf} F_{X}(t) = \begin{cases} 0, & t < 0,\\ \frac{t-a}{b-a}, & a\leq t < b,\\ 1, & t \geq b. \end{cases} \end{equation}@@

The continuous uniform distribution is the continuous analogue of the
discrete uniform distribution; it is used to model experiments whose
outcome is an interval of numbers that are "equally likely" in the
sense that any two intervals of equal length in the support have the
same probability associated with them.

# +BEGIN_exampletoo

Choose a number in \( [0,1] \) at random, and let \(X\) be the number
chosen. Then \(X\sim\mathsf{unif}(\mathtt{min}=0,\,\mathtt{max}=1)\).
The mean of \(X\sim\mathsf{unif}(\mathtt{min}=a,\,\mathtt{max}=b)\) is
relatively simple to calculate:
\begin{align*}
\mu=\mathbb{E} X & =\int_{-\infty}^{\infty}x\, f_{X}(x)\,\mathrm{d} x,\\
 & =\int_{a}^{b}x\ \frac{1}{b-a}\ \mathrm{d} x,\\
 & =\left.\frac{1}{b-a}\ \frac{x^{2}}{2}\ \right|_{x=a}^{b},\\
 & =\frac{1}{b-a}\ \frac{b^{2}-a^{2}}{2},\\
 & =\frac{b+a}{2},
\end{align*}
using the popular formula for the difference of squares. The variance
is left to Exercise [[xca-variance-dunif]].
# +END_exampletoo

** The Normal Distribution
:PROPERTIES:
:CUSTOM_ID: sec-The-Normal-Distribution
:END:

We say that \(X\) has a /normal distribution/ if it has PDF
\begin{equation}
f_{X}(x)=\frac{1}{\sigma\sqrt{2\pi}}\exp \left\{ \frac{-(x-\mu)^{2}}{2\sigma^{2}} \right\},\quad -\infty < x < \infty.
\end{equation}
We write
\(X\sim\mathsf{norm}(\mathtt{mean}=\mu,\,\mathtt{sd}=\sigma)\), and
the associated \(\mathsf{R}\) function is =dnorm(x, mean = 0, sd =
1)=.

The familiar bell-shaped curve, the normal distribution is also known
as the /Gaussian distribution/ because the German mathematician
C. F. Gauss largely contributed to its mathematical development. This
distribution is by far the most important distribution, continuous or
discrete. The normal model appears in the theory of all sorts of
natural phenomena, from to the way particles of smoke dissipate in a
closed room, to the journey of a bottle in the ocean to the white
noise of cosmic background radiation.

When \(\mu=0\) and \(\sigma=1\) we say that the random variable has a
/standard normal/ distribution and we typically write
\(Z\sim\mathsf{norm}(\mathtt{mean}=0,\,\mathtt{sd}=1)\). The lowercase
Greek letter phi (\(\phi\)) is used to denote the standard normal PDF
and the capital Greek letter phi \(\Phi\) is used to denote the
standard normal CDF: for \(-\infty<z<\infty\),
\begin{equation}
\phi(z)=\frac{1}{\sqrt{2\pi}}\,\mathrm{e}^{-z^{2}/2}\mbox{ and }\Phi(t)=\int_{-\infty}^{t}\phi(z)\,\mathrm{d} z.
\end{equation}

#+BEGIN_prop
If \(X\sim\mathsf{norm}(\mathtt{mean}=\mu,\,\mathtt{sd}=\sigma)\) then
\begin{equation}
Z=\frac{X-\mu}{\sigma}\sim\mathsf{norm}(\mathtt{mean}=0,\,\mathtt{sd}=1).
\end{equation}
#+END_prop

The MGF of \(Z\sim\mathsf{norm}(\mathtt{mean}=0,\,\mathtt{sd}=1)\) is
relatively easy to derive:
\begin{eqnarray*}
M_{Z}(t) & = & \int_{-\infty}^{\infty}\mathrm{e}^{tz}\frac{1}{\sqrt{2\pi}}\mathrm{e}^{-z^{2}/2}\mathrm{d} z,\\
 & = & \int_{-\infty}^{\infty}\frac{1}{\sqrt{2\pi}}\exp \{ -\frac{1}{2}\left(z^{2}+2tz+t^{2}\right)+\frac{t^{2}}{2} \} \mathrm{d} z,\\
 & = & \mathrm{e}^{t^{2}/2}\left(\int_{-\infty}^{\infty}\frac{1}{\sqrt{2\pi}}\mathrm{e}^{-[z-(-t)]^{2}/2}\mathrm{d} z\right),
\end{eqnarray*}
and the quantity in the parentheses is the total area under a \(\mathsf{norm}(\mathtt{mean}=-t,\,\mathtt{sd}=1)\) density, which is one. Therefore,
\begin{equation}
M_{Z}(t)=\mathrm{e}^{t^{2}/2},\quad -\infty < t < \infty.
\end{equation}

# +BEGIN_exampletoo

The MGF of
\(X\sim\mathsf{norm}(\mathtt{mean}=\mu,\,\mathtt{sd}=\sigma)\) is then
not difficult either because \[ Z=\frac{X-\mu}{\sigma},\mbox{ or
rewriting, }X=\sigma Z+\mu.  \] Therefore \[
M_{X}(t)=\mathbb{E}\mathrm{e}^{tX}=\mathbb{E}\mathrm{e}^{t(\sigma
Z+\mu)}=\mathbb{E}\mathrm{e}^{\sigma
tZ}\mathrm{e}^{t\mu}=\mathrm{e}^{t\mu}M_{Z}(\sigma t), \] and we know
that \(M_{Z}(t)=\mathrm{e}^{t^{2}/2}\), thus substituting we get \[
M_{X}(t)=\mathrm{e}^{t\mu}\mathrm{e}^{(\sigma t)^{2}/2}=\exp\left\{
\mu t+\sigma^{2}t^{2}/2\right\} , \] for \(-\infty<t<\infty\).
# +END_exampletoo


#+BEGIN_fact
The same argument above shows that if \(X\) has MGF \(M_{X}(t)\) then
the MGF of \(Y=a+bX\) is
\begin{equation}
M_{Y}(t)=\mathrm{e}^{ta}M_{X}(bt).
\end{equation}
#+END_fact

# +BEGIN_exampletoo

The 68-95-99.7 Rule. We saw in Section [[#sub-Measures-of-Spread]] that when an
empirical distribution is approximately bell shaped there are specific
proportions of the observations which fall at varying distances from
the (sample) mean. We can see where these come from -- and obtain more
precise proportions -- with the following:
# +END_exampletoo


#+BEGIN_SRC R :exports both :results output pp 
pnorm(1:3) - pnorm(-(1:3))
#+END_SRC

#+RESULTS:
: [1] 0.6826895 0.9544997 0.9973002

# +BEGIN_exampletoo
<<exa-iq-model>> Let the random experiment consist of a person taking
an IQ test, and let \(X\) be the score on the test. The scores on such
a test are typically standardized to have a mean of 100 and a standard
deviation of 15, and IQ tests have (approximately and notoriously) a
bell-shaped distribution. What is \(\mathbb{P}(85\leq X\leq115)\)?

/Solution/: this one is easy because the limits 85 and 115 fall
exactly one standard deviation (below and above, respectively) from
the mean of 100. The answer is therefore approximately 68%.
# +END_exampletoo

*** Normal Quantiles and the Quantile Function
:PROPERTIES:
:CUSTOM_ID: sub-Normal-Quantiles-QF
:END:

Until now we have been given two values and our task has been to find
the area under the PDF between those values. In this section, we go in
reverse: we are given an area, and we would like to find the value(s)
that correspond to that area.

# +BEGIN_exampletoo
<<exa-iq-quantile-state-problem>> Assuming the IQ model of Example
[[exa-iq-model]], what is the lowest possible IQ score that a person can
have and still be in the top 1% of all IQ scores?  /Solution/: If a
person is in the top 1%, then that means that 99% of the people have
lower IQ scores. So, in other words, we are looking for a value \(x\)
such that \(F(x)=\mathbb{P}(X\leq x)\) satisfies \(F(x)=0.99\), or yet
another way to say it is that we would like to solve the equation
\(F(x)-0.99=0\). For the sake of argument, let us see how to do this
the long way. We define the function \(g(x)=F(x)-0.99\), and then look
for the root of \(g\) with the =uniroot= function. It uses numerical
procedures to find the root so we need to give it an interval of \(x\)
values in which to search for the root. We can get an educated guess
from the Empirical Rule [[fac-Empirical-Rule]]; the root should be
somewhere between two and three standard deviations (15 each) above
the mean (which is 100).
#+BEGIN_SRC R :exports both :results output pp 
g <- function(x) pnorm(x, mean = 100, sd = 15) - 0.99
uniroot(g, interval = c(130, 145))
#+END_SRC

#+RESULTS:
#+BEGIN_example
$root
[1] 134.8952

$f.root
[1] -4.873083e-09

$iter
[1] 6

$estim.prec
[1] 6.103516e-05
#+END_example

#+BEGIN_SRC R :exports none :results silent
temp <- round(uniroot(g, interval = c(130, 145))$root, 4)
#+END_SRC

The answer is shown in =$root= which is approximately SRC_R[:eval no-export]{temp}
134.8952, that is, a person with this IQ score or higher falls in the
top 1% of all IQ scores.
# +END_exampletoo


The discussion in Example [[exa-iq-quantile-state-problem]] was centered on the search for a
value \(x\) that solved an equation \(F(x)=p\), for some given
probability \(p\), or in mathematical parlance, the search for
\(F^{-1}\), the inverse of the CDF of \(X\), evaluated at \(p\). This
is so important that it merits a definition all its own.

#+BEGIN_defn
The /quantile function/ [fn:fn-quantile] of a random variable \(X\) is
the inverse of its cumulative distribution function:
\begin{equation}
Q_{X}(p)=\min\left\{ x:\ F_{X}(x)\geq p\right\} ,\quad 0 < p <1.
\end{equation}
#+END_defn

[fn:fn-quantile] The precise definition of the quantile function is
\(Q_{X}(p)=\inf \{ x:\ F_{X}(x)\geq p \}\), so at least it is well
defined (though perhaps infinite) for the values \(p=0\) and \(p=1\).

#+BEGIN_rem
Here are some properties of quantile functions:
1. The quantile function is defined and finite for all \(0<p<1\).
2. \(Q_{X}\) is left-continuous (see Appendix
   [[#sec-Differential-and-Integral]]). For discrete random variables
   it is a step function, and for continuous random variables it is a
   continuous function.
3. In the continuous case the graph of \(Q_{X}\) may be obtained by
   reflecting the graph of \(F_{X}\) about the line \(y=x\). In the
   discrete case, before reflecting one should: 1) connect the dots to
   get rid of the jumps -- this will make the graph look like a set of
   stairs, 2) erase the horizontal lines so that only vertical lines
   remain, and finally 3) swap the open circles with the solid
   dots. Please see Figure [[fig-binom-plot-distr]] for a comparison.
4. The two limits \[ \lim_{p\to0^{+}}Q_{X}(p)\quad \mbox{and}\quad
   \lim_{p\to1^{-}}Q_{X}(p) \] always exist, but may be infinite (that
   is, sometimes \(\lim_{p\to0}Q(p)=-\infty\) and/or
   \(\lim_{p\to1}Q(p)=\infty\)).

#+END_rem

As the reader might expect, the standard normal distribution is a very
special case and has its own special notation.

#+BEGIN_defn
For \(0<\alpha<1\), the symbol \(z_{\alpha}\) denotes the unique
solution of the equation \( \mathbb{P} ( Z > z_{\alpha}) = \alpha\),
where \(Z \sim \mathsf{norm}(\mathtt{mean} = 0,\,\mathtt{sd} =
1)\). It can be calculated in one of two equivalent ways:
\(\mathtt{qnorm(} 1 - \alpha \mathtt{)} \) and \(\mathtt{qnorm(}
\alpha \mathtt{, lower.tail = FALSE)} \).
#+END_defn

There are a few other very important special cases which we will
encounter in later chapters.

**** How to do it with \(\mathsf{R}\)

Quantile functions are defined for all of the base distributions with
the =q= prefix to the distribution name, except for the ECDF whose
quantile function is exactly the \( Q_{x}(p) = \mathsf{quantile}(x,
\mathtt{probs} = p, \mathtt{type} = 1) \) function.

# +BEGIN_exampletoo

Back to Example [[exa-iq-quantile-state-problem]], we are looking for \(Q_{X}(0.99)\), where
\(X\sim\mathsf{norm}(\mathtt{mean}=100,\,\mathtt{sd}=15)\). It could
not be easier to do with \(\mathsf{R}\).

#+BEGIN_SRC R :exports both :results output pp 
qnorm(0.99, mean = 100, sd = 15)
#+END_SRC

#+RESULTS:
: [1] 134.8952

Compare this answer to the one obtained earlier with =uniroot=.
# +END_exampletoo


# +BEGIN_exampletoo

Find the values \(z_{0.025}\), \(z_{0.01}\), and \(z_{0.005}\) (these
will play an important role from Chapter [[#cha-Estimation]] onward).
# +END_exampletoo


#+BEGIN_SRC R :exports both :results output pp 
qnorm(c(0.025, 0.01, 0.005), lower.tail = FALSE)
#+END_SRC

#+RESULTS:
: [1] 1.959964 2.326348 2.575829

Note the =lower.tail= argument. We would get the same answer with
: qnorm(c(0.975, 0.99, 0.995))

** Functions of Continuous Random Variables
:PROPERTIES:
:CUSTOM_ID: sec-Functions-of-Continuous
:END:

The goal of this section is to determine the distribution of
\(U=g(X)\) based on the distribution of \(X\). In the discrete case
all we needed to do was back substitute for \(x=g^{-1}(u)\) in the PMF
of \(X\) (sometimes accumulating probability mass along the way). In
the continuous case, however, we need more sophisticated tools. Now
would be a good time to review Appendix [[#sec-Differential-and-Integral]].

*** The PDF Method

#+BEGIN_prop
<<pro-func-cont-rvs-pdf-formula>> Let \(X\) have PDF \(f_{X}\) and let
\(g\) be a function which is one-to-one with a differentiable inverse
\(g^{-1}\). Then the PDF of \(U=g(X)\) is given by
\begin{equation}
\label{eq-univ-trans-pdf-long}
f_{U}(u)=f_{X}\left[g^{-1}(u)\right]\ \left|\frac{\mathrm{d}}{\mathrm{d} u}g^{-1}(u)\right|.
\end{equation}
#+END_prop

#+BEGIN_rem
The formula in Equation \eqref{eq-univ-trans-pdf-long} is nice, but does not
really make any sense. It is better to write in the intuitive form
\begin{equation}
\label{eq-univ-trans-pdf-short}
f_{U}(u)=f_{X}(x)\left|\frac{\mathrm{d} x}{\mathrm{d} u}\right|.
\end{equation}
#+END_rem

# +BEGIN_exampletoo
<<exa-lnorm-transformation>> Let
\(X\sim\mathsf{norm}(\mathtt{mean}=\mu,\,\mathtt{sd}=\sigma)\), and
let \(Y=\mathrm{e}^{X}\). What is the PDF of \(Y\)?  *Solution:*
Notice first that \(\mathrm{e}^{x}>0\) for any \(x\), so the support
of \(Y\) is \((0,\infty)\). Since the transformation is monotone, we
can solve \(y=\mathrm{e}^{x}\) for \(x\) to get \(x=\ln\, y\), giving
\(\mathrm{d} x/\mathrm{d} y=1/y\). Therefore, for any \(y>0\), \[
f_{Y}(y)=f_{X}(\ln
y)\cdot\left|\frac{1}{y}\right|=\frac{1}{\sigma\sqrt{2\pi}}\exp\left\{
\frac{(\ln y-\mu)^{2}}{2\sigma^{2}}\right\} \cdot\frac{1}{y}, \] where
we have dropped the absolute value bars since \(y>0\). The random
variable \(Y\) is said to have a /lognormal distribution/; see Section
[[#sec-Other-Continuous-Distributions]].
# +END_exampletoo


# +BEGIN_exampletoo
<<exa-lin-trans-norm>> Suppose
\(X\sim\mathsf{norm}(\mathtt{mean}=0,\,\mathtt{sd}=1)\) and let
\(Y=4-3X\). What is the PDF of \(Y\)?
# +END_exampletoo


The support of \(X\) is \((-\infty,\infty)\), and as \(x\) goes from
\(-\infty\) to \(\infty\), the quantity \(y=4-3x\) also traverses
\((-\infty,\infty)\). Solving for \(x\) in the equation \(y=4-3x\)
yields \(x=-(y-4)/3\) giving \(\mathrm{d} x/\mathrm{d} y=-1/3\). And
since \[ f_{X}(x)=\frac{1}{\sqrt{2\pi}}\mathrm{e}^{-x^{2}/2}, \quad
-\infty < x < \infty , \] we have
\begin{eqnarray*}
f_{Y}(y) & = & f_{X}\left(\frac{y-4}{3}\right)\cdot\left|-\frac{1}{3}\right|,\quad -\infty < y < \infty,\\
 & = & \frac{1}{3\sqrt{2\pi}}\mathrm{e}^{-(y-4)^{2}/2\cdot3^{2}},\quad -\infty < y < \infty.
\end{eqnarray*}
We recognize the PDF of \(Y\) to be that of a
\(\mathsf{norm}(\mathtt{mean}=4,\,\mathtt{sd}=3)\)
distribution. Indeed, we may use an identical argument as the above to
prove the following fact:

#+BEGIN_fact
<<fac-lin-trans-norm-is-norm>> If
\(X\sim\mathsf{norm}(\mathtt{mean}=\mu,\,\mathtt{sd}=\sigma)\) and if
\(Y=a+bX\) for constants \(a\) and \(b\), with \(b\neq0\), then
\(Y\sim\mathsf{norm}(\mathtt{mean}=a+b\mu,\,\mathtt{sd}=|b|\sigma)\).
#+END_fact

Note that it is sometimes easier to /postpone/ solving for the inverse
transformation \(x=x(u)\). Instead, leave the transformation in the
form \(u=u(x)\) and calculate the derivative of the /original/
transformation
\begin{equation}
\mathrm{d} u/\mathrm{d} x=g'(x).
\end{equation}
Once this is known, we can get the PDF of \(U\) with
\begin{equation}
f_{U}(u)=f_{X}(x)\left|\frac{1}{\mathrm{d} u/\mathrm{d} x}\right|.
\end{equation}
In many cases there are cancellations and the work is shorter. Of course, it is not always true that
\begin{equation}
\label{eq-univ-jacob-recip}
\frac{\mathrm{d} x}{\mathrm{d} u}=\frac{1}{\mathrm{d} u/\mathrm{d} x},
\end{equation}
but for the well-behaved examples in this book the trick works just fine.

#+BEGIN_rem
In the case that \(g\) is not monotone we cannot apply Proposition
[[pro-func-cont-rvs-pdf-formula]] directly. However, hope is not
lost. Rather, we break the support of \(X\) into pieces such that
\(g\) is monotone on each one. We apply Proposition
[[pro-func-cont-rvs-pdf-formula]] on each piece, and finish up by
adding the results together.
#+END_rem

*** The CDF method

We know from Section [[#sec-continuous-random-variables]] that \(f_{X}=F_{X}'\)
in the continuous case. Starting from the equation
\(F_{Y}(y)=\mathbb{P}(Y\leq y)\), we may substitute \(g(X)\) for
\(Y\), then solve for \(X\) to obtain \(\mathbb{P}[X\leq g^{-1}(y)]\),
which is just another way to write
\(F_{X}[g^{-1}(y)]\). Differentiating this last quantity with respect
to \(y\) will yield the PDF of \(Y\).

# +BEGIN_exampletoo

Suppose \(X\sim\mathsf{unif}(\mathtt{min}=0,\,\mathtt{max}=1)\) and
suppose that we let \(Y=-\ln\, X\). What is the PDF of \(Y\)?

The support set of \(X\) is \((0,1),\) and \(y\) traverses
\((0,\infty)\) as \(x\) ranges from \(0\) to \(1\), so the support set
of \(Y\) is \(S_{Y}=(0,\infty)\). For any \(y>0\), we consider \[
F_{Y}(y)=\mathbb{P}(Y\leq y)=\mathbb{P}(-\ln\, X\leq
y)=\mathbb{P}(X\geq\mathrm{e}^{-y})=1-\mathbb{P}(X<\mathrm{e}^{-y}),
\] where the next to last equality follows because the exponential
function is /monotone/ (this point will be revisited later). Now since
\(X\) is continuous the two probabilities
\(\mathbb{P}(X<\mathrm{e}^{-y})\) and
\(\mathbb{P}(X\leq\mathrm{e}^{-y})\) are equal; thus \[ 1-\mathbb{P}(X
<
\mathrm{e}^{-y})=1-\mathbb{P}(X\leq\mathrm{e}^{-y})=1-F_{X}(\mathrm{e}^{-y}).
\] Now recalling that the CDF of a
\(\mathsf{unif}(\mathtt{min}=0,\,\mathtt{max}=1)\) random variable
satisfies \(F(u)=u\) (see Equation \eqref{eq-unif-cdf}), we can say \[
F_{Y}(y)=1-F_{X}(\mathrm{e}^{-y})=1-\mathrm{e}^{-y},\quad \mbox{for
}y>0.  \] We have consequently found the formula for the CDF of \(Y\);
to obtain the PDF \(f_{Y}\) we need only differentiate \(F_{Y}\): \[
f_{Y}(y)=\frac{\mathrm{d}}{\mathrm{d}
y}\left(1-\mathrm{e}^{-y}\right)=0-\mathrm{e}^{-y}(-1), \] or
\(f_{Y}(y)=\mathrm{e}^{-y}\) for \(y>0\). This turns out to be a
member of the exponential family of distributions, see Section
[[#sec-Other-Continuous-Distributions]].
# +END_exampletoo


# +BEGIN_exampletoo

*The Probability Integral Transform*. Given a continuous random
variable \(X\) with strictly increasing CDF \(F_{X}\), let the random
variable \(Y\) be defined by \(Y=F_{X}(X)\). Then the distribution of
\(Y\) is \(\mathsf{unif}(\mathtt{min}=0,\,\mathtt{max}=1)\).
# +END_exampletoo


#+BEGIN_proof
We employ the CDF method. First note that the support of \(Y\) is
\((0,1)\). Then for any \(0<y<1\), \[ F_{Y}(y)=\mathbb{P}(Y\leq
y)=\mathbb{P}(F_{X}(X)\leq y).  \] Now since \(F_{X}\) is strictly
increasing, it has a well defined inverse function
\(F_{X}^{-1}\). Therefore, \[ \mathbb{P}(F_{X}(X)\leq
y)=\mathbb{P}(X\leq F_{X}^{-1}(y))=F_{X}[F_{X}^{-1}(y)]=y.  \]
Summarizing, we have seen that \(F_{Y}(y)=y\), \(0<y<1\). But this is
exactly the CDF of a
\(\mathsf{unif}(\mathtt{min}=0,\,\mathtt{max}=1)\) random variable.
#+END_proof

#+BEGIN_fact
The Probability Integral Transform is true for all continuous random
variables with continuous CDFs, not just for those with strictly
increasing CDFs (but the proof is more complicated). The transform is
*not* true for discrete random variables, or for continuous random
variables having a discrete component (that is, with jumps in their
CDF).
#+END_fact

# +BEGIN_exampletoo
<<exa-distn-of-z-squared>> Let
\(Z\sim\mathsf{norm}(\mathtt{mean}=0,\,\mathtt{sd}=1)\) and let
\(U=Z^{2}\). What is the PDF of \(U\)?  Notice first that
\(Z^{2}\geq0\), and thus the support of \(U\) is \([0,\infty)\). And
for any \(u\geq0\), \[ F_{U}(u)=\mathbb{P}(U\leq
u)=\mathbb{P}(Z^{2}\leq u).  \] But \(Z^{2}\leq u\) occurs if and only
if \(-\sqrt{u}\leq Z\leq\sqrt{u}\). The last probability above is
simply the area under the standard normal PDF from \(-\sqrt{u}\) to
\(\sqrt{u}\), and since \(\phi\) is symmetric about 0, we have \[
\mathbb{P}(Z^{2}\leq u)=2\mathbb{P}(0\leq
Z\leq\sqrt{u})=2\left[F_{Z}(\sqrt{u})-F_{Z}(0)\right]=2\Phi(\sqrt{u})-1,
\] because \(\Phi(0)=1/2\). To find the PDF of \(U\) we differentiate
the CDF recalling that \(\Phi'= \phi\).  \[
f_{U}(u)=\left(2\Phi(\sqrt{u})-1\right)'=2\phi(\sqrt{u})\cdot\frac{1}{2\sqrt{u}}=u^{-1/2}\phi(\sqrt{u}).
\] Substituting, \[ f_{U}(u) =
u^{-1/2}\frac{1}{\sqrt{2\pi}}\,\mathrm{e}^{-(\sqrt{u})^{2}/2}=(2\pi
u)^{-1/2}\mathrm{e}^{-u/2},\quad u > 0.  \] This is what we will later
call a /chi-square distribution with 1 degree of freedom/. See Section
[[#sec-Other-Continuous-Distributions]].
# +END_exampletoo

**** How to do it with \(\mathsf{R}\)

The =distr= package \cite{distr} has functionality to investigate
transformations of univariate distributions. There are exact results
for ordinary transformations of the standard distributions, and
=distr= takes advantage of these in many cases. For instance, the
=distr= package can handle the transformation in Example [[exa-lin-trans-norm]] quite nicely:

#+BEGIN_SRC R :exports both :results output pp 
X <- Norm(mean = 0, sd = 1)
Y <- 4 - 3*X
Y
#+END_SRC

#+RESULTS:
: Distribution Object of Class: Norm
:  mean: 4
:  sd: 3

So =distr= "knows" that a linear transformation of a normal random
variable is again normal, and it even knows what the correct =mean=
and =sd= should be. But it is impossible for =distr= to know
everything, and it is not long before we venture outside of the
transformations that =distr= recognizes. Let us try Example
[[exa-lnorm-transformation]]:

#+BEGIN_SRC R :exports both :results output pp 
Y <- exp(X)
Y
#+END_SRC

#+RESULTS:
: Distribution Object of Class: AbscontDistribution

The result is an object of class =AbscontDistribution=, which is one
of the classes that =distr= uses to denote general distributions that
it does not recognize (it turns out that \(Z\) has a /lognormal/
distribution; see Section [[#sec-Other-Continuous-Distributions]]). A
simplified description of the process that =distr= undergoes when it
encounters a transformation \(Y=g(X)\) that it does not recognize is
1. Randomly generate many, many copies \(X_{1}\), \(X_{2}\), ...,
   \(X_{n}\) from the distribution of \(X\),
1. Compute \(Y_{1}=g(X_{1})\), \(Y_{2}=g(X_{2})\), ...,
   \(Y_{n}=g(X_{n})\) and store them for use.
1. Calculate the PDF, CDF, quantiles, and random variates using the
   simulated values of \(Y\).
As long as the transformation is sufficiently nice, such as a linear
transformation, the exponential, absolute value, /etc./, the =d-p-q=
functions are calculated analytically based on the =d-p-q= functions
associated with \(X\). But if we try a crazy transformation then we
are greeted by a warning:

#+BEGIN_SRC R :exports both :results output pp 
W <- sin(exp(X) + 27)
W
#+END_SRC

#+RESULTS:
: Distribution Object of Class: AbscontDistribution

The warning confirms that the =d-p-q= functions are not calculated
analytically, but are instead based on the randomly simulated values
of \(Y\). /We must be careful to remember this./ The nature of random
simulation means that we can get different answers to the same
question: watch what happens when we compute \(\mathbb{P}(W\leq0.5)\)
using the \(W\) above, then define \(W\) again, and compute the
(supposedly) same \(\mathbb{P}(W\leq0.5)\) a few moments later.

#+BEGIN_SRC R :exports both :results output pp 
p(W)(0.5)
W <- sin(exp(X) + 27)
p(W)(0.5)
#+END_SRC

#+RESULTS:
: [1] 0.5793242
: [1] 0.5793242

The answers are not the same! Furthermore, if we were to repeat the
process we would get yet another answer for \(\mathbb{P}(W\leq0.5)\).

The answers were close, though. And the underlying randomly generated
\(X\)'s were not the same so it should hardly be a surprise that the
calculated \(W\)'s were not the same, either. This serves as a warning
(in concert with the one that =distr= provides) that we should be
careful to remember that complicated transformations computed by
\(\mathsf{R}\) are only approximate and may fluctuate slightly due to
the nature of the way the estimates are calculated.

** Other Continuous Distributions
:PROPERTIES:
:CUSTOM_ID: sec-Other-Continuous-Distributions
:END:

*** Waiting Time Distributions
:PROPERTIES:
:CUSTOM_ID: sub-Waiting-Time-Distributions
:END:

In some experiments, the random variable being measured is the time
until a certain event occurs. For example, a quality control
specialist may be testing a manufactured product to see how long it
takes until it fails. An efficiency expert may be recording the
customer traffic at a retail store to streamline scheduling of staff.

**** The Exponential Distribution
:PROPERTIES:
:CUSTOM_ID: sub-The-Exponential-Distribution
:END:

We say that \(X\) has an /exponential distribution/ and write
\(X\sim\mathsf{exp}(\mathtt{rate}=\lambda)\).
\begin{equation}
f_{X}(x)=\lambda\mathrm{e}^{-\lambda x},\quad x>0
\end{equation}
The associated \(\mathsf{R}\) functions are =dexp(x, rate = 1)=,
=pexp=, =qexp=, and =rexp=, which give the PDF, CDF, quantile
function, and simulate random variates, respectively.

The parameter \(\lambda\) measures the rate of arrivals (to be
described later) and must be positive. The CDF is given by the formula
\begin{equation}
F_{X}(t)=1-\mathrm{e}^{-\lambda t},\quad t>0.
\end{equation}
The mean is \(\mu=1/\lambda\) and the variance is
\(\sigma^{2}=1/\lambda^{2}\).

The exponential distribution is closely related to the Poisson
distribution. If customers arrive at a store according to a Poisson
process with rate \(\lambda\) and if \(Y\) counts the number of
customers that arrive in the time interval \([0,t)\), then we saw in
Section [[#sec-other-discrete-distributions]] that \( Y \sim
\mathsf{pois}(\mathtt{lambda}=\lambda t). \) Now consider a different
question: let us start our clock at time 0 and stop the clock when the
first customer arrives. Let \(X\) be the length of this random time
interval. Then \(X\sim\mathsf{exp}(\mathtt{rate}=\lambda)\). Observe
the following string of equalities:
\begin{align*}
\mathbb{P}(X>t) & =\mathbb{P}(\mbox{first arrival after time \emph{t}}),\\
 & =\mathbb{P}(\mbox{no events in [0,\emph{t})}),\\
 & =\mathbb{P}(Y=0),\\
 & =\mathrm{e}^{-\lambda t},
\end{align*}
where the last line is the PMF of \(Y\) evaluated at \(y=0\). In other
words, \(\mathbb{P}(X\leq t)=1-\mathrm{e}^{-\lambda t}\), which is
exactly the CDF of an \(\mathsf{exp}(\mathtt{rate}=\lambda)\)
distribution.

The exponential distribution is said to be /memoryless/ because
exponential random variables "forget" how old they are at every
instant. That is, the probability that we must wait an additional five
hours for a customer to arrive, given that we have already waited
seven hours, is exactly the probability that we needed to wait five
hours for a customer in the first place. In mathematical symbols, for
any \(s,\, t>0\),
\begin{equation}
\mathbb{P}(X>s+t\,|\, X>t)=\mathbb{P}(X>s).
\end{equation}
See Exercise [[xca-prove-the-memoryless]].

**** The Gamma Distribution
:PROPERTIES:
:CUSTOM_ID: sub-The-Gamma-Distribution
:END:

This is a generalization of the exponential distribution. We say that
\(X\) has a gamma distribution and write
\(X\sim\mathsf{gamma}(\mathtt{shape}=\alpha,\,\mathtt{rate}=\lambda)\). It
has PDF
\begin{equation}
f_{X}(x)=\frac{\lambda^{\alpha}}{\Gamma(\alpha)}\: x^{\alpha-1}\mathrm{e}^{-\lambda x},\quad x>0.
\end{equation}

The associated \(\mathsf{R}\) functions are =dgamma(x, shape, rate =
1)=, =pgamma=, =qgamma=, and =rgamma=, which give the PDF, CDF,
quantile function, and simulate random variates, respectively. If
\(\alpha=1\) then \(X\sim\mathsf{exp}(\mathtt{rate}=\lambda)\). The
mean is \(\mu=\alpha/\lambda\) and the variance is
\(\sigma^{2}=\alpha/\lambda^{2}\).

To motivate the gamma distribution recall that if \(X\) measures the
length of time until the first event occurs in a Poisson process with
rate \(\lambda\) then \(X\sim\mathsf{exp}(\mathtt{rate}=\lambda)\). If
we let \(Y\) measure the length of time until the
\(\alpha^{\mathrm{th}}\) event occurs then
\(Y\sim\mathsf{gamma}(\mathtt{shape}=\alpha,\,\mathtt{rate}=\lambda)\). When
\(\alpha\) is an integer this distribution is also known as the
/Erlang/ distribution.

# +BEGIN_exampletoo

At a car wash, two customers arrive per hour on the average. We decide
to measure how long it takes until the third customer arrives. If
\(Y\) denotes this random time then
\(Y\sim\mathsf{gamma}(\mathtt{shape}=3,\,\mathtt{rate}=1/2)\).
# +END_exampletoo

*** The Chi square, Student's \(t\), and Snedecor's \(F\) Distributions
:PROPERTIES:
:CUSTOM_ID: sub-The-Chi-Square-t-F
:END:

**** The Chi square Distribution
:PROPERTIES:
:CUSTOM_ID: sub-The-Chi-Square
:END:

A random variable \(X\) with PDF
\begin{equation}
f_{X}(x)=\frac{1}{\Gamma(p/2)2^{p/2}}x^{p/2-1}\mathrm{e}^{-x/2},\quad x>0,
\end{equation}
is said to have a /chi-square distribution/ with \(p\) /degrees of
freedom/. We write \(X\sim\mathsf{chisq}(\mathtt{df}=p)\). The
associated \(\mathsf{R}\) functions are =dchisq(x, df)=, =pchisq=,
=qchisq=, and =rchisq=, which give the PDF, CDF, quantile function,
and simulate random variates, respectively. See Figure
[[fig-chisq-dist-vary-df]]. In an obvious notation we may define
\(\chi_{\alpha}^{2}(p)\) as the number on the \(x\)-axis such that
there is exactly \(\alpha\) area under the
\(\mathsf{chisq}(\mathtt{df}=p)\) curve to its right.

The code to produce Figure [[fig-chisq-dist-vary-df]] is

#+NAME: chisq-dist-vary-df
#+BEGIN_SRC R :exports both :results graphics :file fig/contdist-chisq-dist-vary-df.ps
curve(dchisq(x, df = 3), from = 0, to = 20, ylab = "y")
ind <- c(4, 5, 10, 15)
for (i in ind) curve(dchisq(x, df = i), 0, 20, add = TRUE)
#+END_SRC

#+NAME: fig-chisq-dist-vary-df
#+CAPTION[Chi square distribution for various degrees of freedom]: \small The chi square distribution for various degrees of freedom.
#+ATTR_LaTeX: :width 0.9\textwidth :placement [ht!]
#+RESULTS: chisq-dist-vary-df
[[file:fig/contdist-chisq-dist-vary-df.ps]]

#+BEGIN_rem
Here are some useful things to know about the chi-square distribution.
1. If \(Z\sim\mathtt{norm}(\mathtt{mean}=0,\,\mathtt{sd}=1)\), then
   \(Z^{2}\sim\mathsf{chisq}(\mathtt{df}=1)\). We saw this in Example
   [[exa-distn-of-z-squared]], and the fact is important when it
   comes time to find the distribution of the sample variance,
   \(S^{2}\). See Theorem [[thm-Xbar-andS]] in Section
   [[#sub-Samp-Var-Dist]].
2. The chi-square distribution is supported on the positive
   \(x\)-axis, with a right-skewed distribution.
3. The \(\mathsf{chisq}(\mathtt{df}=p)\) distribution is the same as a
   \(\mathsf{gamma}(\mathtt{shape}=p/2,\,\mathtt{rate}=1/2)\)
   distribution.
4. The MGF of \(X\sim\mathsf{chisq}(\mathtt{df}=p)\) is
   \begin{equation}
   \label{eq-mgf-chisq}
   M_{X}(t)=\left(1-2t\right)^{-p},\quad t < 1/2.
   \end{equation}
#+END_rem

**** Student's \(t\) distribution
:PROPERTIES:
:CUSTOM_ID: sub-Student's-t-distribution
:END:

A random variable \(X\) with PDF
\begin{equation}
f_{X}(x) = \frac{\Gamma\left[ (r+1)/2\right] }{\sqrt{r\pi}\,\Gamma(r/2)}\left( 1 + \frac{x^{2}}{r} \right)^{-(r+1)/2},\quad -\infty < x < \infty
\end{equation}
is said to have /Student's/ \(t\) distribution with \(r\) /degrees of
freedom/, and we write \(X\sim\mathsf{t}(\mathtt{df}=r)\). The
associated \(\mathsf{R}\) functions are =dt=,=pt=, =qt=, and =rt=,
which give the PDF, CDF, quantile function, and simulate random
variates, respectively. See Section [[#sec-sampling-from-normal-dist]].

**** Snedecor's \(F\) distribution
:PROPERTIES:
:CUSTOM_ID: sub-snedecor-F-distribution
:END:

A random variable \(X\) with PDF
\begin{equation}
f_{X}(x)=\frac{\Gamma[(m+n)/2]}{\Gamma(m/2)\Gamma(n/2)}\left(\frac{m}{n}\right)^{m/2}x^{m/2-1}\left(1+\frac{m}{n}x\right)^{-(m+n)/2},\quad x>0.
\end{equation}
is said to have an \(F\) distribution with \((m,n)\) degrees of
freedom. We write
\(X\sim\mathsf{f}(\mathtt{df1}=m,\,\mathtt{df2}=n)\). The associated
\(\mathsf{R}\) functions are =df(x, df1, df2)=, =pf=, =qf=, and =rf=,
which give the PDF, CDF, quantile function, and simulate random
variates, respectively. We define \(F_{\alpha}(m,n)\) as the number on
the \(x\)-axis such that there is exactly \(\alpha\) area under the
\(\mathsf{f}(\mathtt{df1}=m,\,\mathtt{df2}=n)\) curve to its right.

#+BEGIN_rem
Here are some notes about the \(F\) distribution.
1. If \(X\sim\mathsf{f}(\mathtt{df1}=m,\,\mathtt{df2}=n)\) and
   \(Y=1/X\), then
   \(Y\sim\mathsf{f}(\mathtt{df1}=n,\,\mathtt{df2}=m)\). Historically,
   this fact was especially convenient. In the old days, statisticians
   used printed tables for their statistical calculations. Since the
   \(F\) tables were symmetric in \(m\) and \(n\), it meant that
   publishers could cut the size of their printed tables in half. It
   plays less of a role today now that personal computers are
   widespread.
1. If \(X\sim\mathsf{t}(\mathtt{df}=r)\), then
   \(X^{2}\sim\mathsf{f}(\mathtt{df1}=1,\,\mathtt{df2}=r)\). We will
   see this again in Section [[#sub-slr-overall-F-statistic]].
#+END_rem

*** Other Popular Distributions
:PROPERTIES:
:CUSTOM_ID: sub-Other-Popular-Distributions
:END:

**** The Cauchy Distribution
:PROPERTIES:
:CUSTOM_ID: sub-The-Cauchy-Distribution
:END:

This is a special case of the Student's \(t\) distribution. It has PDF
\begin{equation}
f_{X}(x) = \frac{1}{\beta\pi} \left[ 1+\left( \frac{x-m}{\beta} \right)^{2} \right]^{-1},\quad -\infty < x < \infty.
\end{equation}
We write \(X \sim \mathsf{cauchy}(\mathtt{location} =
m,\,\mathtt{scale} = \beta)\). The associated \(\mathsf{R}\) function
is =dcauchy(x, location = 0, scale = 1)=.

It is easy to see that a \(\mathsf{cauchy}(\mathtt{location} =
0,\,\mathtt{scale} = 1)\) distribution is the same as a
\(\mathsf{t}(\mathtt{df}=1)\) distribution. The \(\mathsf{cauchy}\)
distribution looks like a \(\mathsf{norm}\) distribution but with very
heavy tails. The mean (and variance) do not exist, that is, they are
infinite. The median is represented by the \(\mathtt{location}\)
parameter, and the \(\mathtt{scale}\) parameter influences the spread
of the distribution about its median.

**** The Beta Distribution
:PROPERTIES:
:CUSTOM_ID: sub-The-Beta-Distribution
:END:

This is a generalization of the continuous uniform distribution.
\begin{equation}
f_{X}(x)=\frac{\Gamma(\alpha+\beta)}{\Gamma(\alpha)\Gamma(\beta)}\, x^{\alpha-1}(1-x)^{\beta-1},\quad 0 < x < 1.
\end{equation}
We write
\(X\sim\mathsf{beta}(\mathtt{shape1}=\alpha,\,\mathtt{shape2}=\beta)\). The
associated \(\mathsf{R}\) function is =dbeta(x, shape1, shape2)=. The
mean and variance are
\begin{equation} 
\mu=\frac{\alpha}{\alpha+\beta}\mbox{ and }\sigma^{2}=\frac{\alpha\beta}{\left(\alpha+\beta\right)^{2}\left(\alpha+\beta+1\right)}.
\end{equation}
See Example [[exa-cont-pdf3x2]]. This distribution comes up a lot in Bayesian
statistics because it is a good model for one's prior beliefs about a
population proportion \(p\), \(0\leq p\leq1\).

**** The Logistic Distribution
:PROPERTIES:
:CUSTOM_ID: sub-The-Logistic-Distribution
:END:

\begin{equation}
f_{X}(x)=\frac{1}{\sigma}\exp\left(-\frac{x-\mu}{\sigma}\right)\left[1+\exp\left(-\frac{x-\mu}{\sigma}\right)\right]^{-2},\quad -\infty < x < \infty.
\end{equation}
We write
\(X\sim\mathsf{logis}(\mathtt{location}=\mu,\,\mathtt{scale}=\sigma)\). The
associated \(\mathsf{R}\) function is =dlogis(x, location = 0, scale =
1)=. The logistic distribution comes up in differential equations as a
model for population growth under certain assumptions. The mean is
\(\mu\) and the variance is \(\pi^{2}\sigma^{2}/3\).

**** The Lognormal Distribution
:PROPERTIES:
:CUSTOM_ID: sub-The-Lognormal-Distribution
:END:

This is a distribution derived from the normal distribution (hence the
name). If
\(U\sim\mathtt{norm}(\mathtt{mean}=\mu,\,\mathtt{sd}=\sigma)\), then
\( X = \mathrm{e}^{U} \) has PDF
\begin{equation}
f_{X}(x)=\frac{1}{\sigma x\sqrt{2\pi}}\exp\left[\frac{-(\ln x-\mu)^{2}}{2\sigma^{2}}\right], \quad 0 < x < \infty.
\end{equation}
We write
\(X\sim\mathsf{lnorm}(\mathtt{meanlog}=\mu,\,\mathtt{sdlog}=\sigma)\). The
associated \(\mathsf{R}\) function is =dlnorm(x, meanlog = 0, sdlog =
1)=. Notice that the support is concentrated on the positive \(x\)
axis; the distribution is right-skewed with a heavy tail. See Example
[[exa-lnorm-transformation]].

**** The Weibull Distribution
:PROPERTIES:
:CUSTOM_ID: sub-The-Weibull-Distribution
:END:

This has PDF
\begin{equation}
f_{X}(x)=\frac{\alpha}{\beta}\left(\frac{x}{\beta}\right)^{\alpha-1}\exp\left(\frac{x}{\beta}\right)^{\alpha},\quad x>0.
\end{equation}
We write
\(X\sim\mathsf{weibull}(\mathtt{shape}=\alpha,\,\mathtt{scale}=\beta)\). The
associated \(\mathsf{R}\) function is =dweibull(x, shape, scale = 1)=.

**** How to do it with \(\mathsf{R}\)

There is some support of moments and moment generating functions for
some continuous probability distributions included in the =actuar=
package \cite{actuar}. The convention is =m= in front of the
distribution name for raw moments, and =mgf= in front of the
distribution name for the moment generating function. At the time of
this writing, the following distributions are supported: gamma,
inverse Gaussian, (non-central) chi-squared, exponential, and uniform.

# +BEGIN_exampletoo

Calculate the first four raw moments for
\(X\sim\mathsf{gamma}(\mathtt{shape}=13,\,\mathtt{rate}=1)\) and plot
the moment generating function.

We load the =actuar= package and use the functions =mgamma= and
=mgfgamma=:
#+BEGIN_SRC R :exports both :results output pp 
mgamma(1:4, shape = 13, rate = 1)
#+END_SRC

#+RESULTS:
: [1]    13   182  2730 43680

For the plot we can use the function in the following form:

#+NAME: gamma-mgf
#+BEGIN_SRC R :exports both :results graphics :file fig/contdist-gamma-mgf.ps
plot(function(x){mgfgamma(x, shape = 13, rate = 1)}, 
     from=-0.1, to=0.1, ylab = "gamma mgf")
#+END_SRC

#+NAME: fig-gamma-mgf
#+CAPTION[Plot of the \textsf{gamma}(=shape= = 13, =rate= = 1) MGF]: \small A plot of the \textsf{gamma}(=shape= = 13, =rate= = 1) MGF.
#+ATTR_LaTeX: :width 0.9\textwidth :placement [ht!]
#+RESULTS: gamma-mgf
[[file:fig/contdist-gamma-mgf.ps]]

# +END_exampletoo


#+LaTeX: \newpage{}

** Exercises
#+LaTeX: \setcounter{thm}{0}

#+BEGIN_xca
Find the constant \(C\) so that the given function is a valid PDF of a random variable \(X\).
1. \( f(x) = Cx^{n},\quad 0 < x <1 \).
1. \( f(x) = Cx\mathrm{e}^{-x},\quad 0 < x < \infty\).
1. \( f(x) = \mathrm{e}^{-(x - C)}, \quad 7 < x < \infty.\)
1. \( f(x) = Cx^{3}(1 - x)^{2},\quad 0 < x < 1.\)
1. \( f(x) = C(1 + x^{2}/4)^{-1}, \quad -\infty < x < \infty.\)
#+END_xca

#+BEGIN_xca
For the following random experiments, decide what the distribution of
\(X\) should be. In nearly every case, there are additional
assumptions that should be made for the distribution to apply;
identify those assumptions (which may or may not strictly hold in
practice).
1. We throw a dart at a dart board. Let \(X\) denote the squared
   linear distance from the bulls-eye to the where the dart landed.
1. We randomly choose a textbook from the shelf at the bookstore and
   let \(P\) denote the proportion of the total pages of the book
   devoted to exercises.
1. We measure the time it takes for the water to completely drain out
   of the kitchen sink.
1. We randomly sample strangers at the grocery store and ask them how
   long it will take them to drive home.
#+END_xca

#+BEGIN_xca
If \(Z\) is \(\mathsf{norm}(\mathtt{mean} = 0,\,\mathtt{sd} = 1)\), find 
1. \(\mathbb{P}(Z > 2.64)\)
1. \(\mathbb{P}(0 \leq Z < 0.87)\)
1. \(\mathbb{P}(|Z| > 1.39)\) (Hint: draw a picture!)
#+END_xca

#+BEGIN_xca
<<xca-variance-dunif>> Calculate the variance of
\(X\sim\mathsf{unif}(\mathtt{min}=a,\,\mathtt{max}=b)\). /Hint:/ First
calculate \(\mathbb{E} X^{2}\).
#+END_xca

#+BEGIN_xca
<<xca-prove-the-memoryless>> Prove the memoryless property for
exponential random variables. That is, for \(X \sim
\mathsf{exp}(\mathtt{rate} = \lambda)\) show that for any \(s,t > 0\),
\[ \mathbb{P}(X > s + t\,|\, X > t) = \mathbb{P}(X > s).  \]
#+END_xca
