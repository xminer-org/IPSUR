* Probability                                                          :prob:
:PROPERTIES:
:tangle: R/04-prob.R
:CUSTOM_ID: cha-Probability
:END:

#+BEGIN_SRC R :exports none :eval never
#    IPSUR: Introduction to Probability and Statistics Using R
#    Copyright (C) 2014  G. Jay Kerns
#
#    Chapter: Probability
#
#    This file is part of IPSUR.
#
#    IPSUR is free software: you can redistribute it and/or modify
#    it under the terms of the GNU General Public License as published by
#    the Free Software Foundation, either version 3 of the License, or
#    (at your option) any later version.
#
#    IPSUR is distributed in the hope that it will be useful,
#    but WITHOUT ANY WARRANTY; without even the implied warranty of
#    MERCHANTABILITY or FITNESS FOR A PARTICULAR PURPOSE.  See the
#    GNU General Public License for more details.
#
#    You should have received a copy of the GNU General Public License
#    along with IPSUR.  If not, see <http://www.gnu.org/licenses/>.
#+END_SRC

#+BEGIN_SRC R :exports none :eval no-export
# This chapter's package dependencies
library(diagram)
library(prob)
library(RcmdrPlugin.IPSUR)
#+END_SRC

#+LaTeX: \noindent 
In this chapter we define the basic terminology associated with
probability and derive some of its properties. We discuss three
interpretations of probability. We discuss conditional probability and
independent events, along with Bayes' Theorem. We finish the chapter
with an introduction to random variables, which paves the way for the
next two chapters.

In this book we distinguish between two types of experiments:
/deterministic/ and /random/. A /deterministic/ experiment is one
whose outcome may be predicted with certainty beforehand, such as
combining Hydrogen and Oxygen, or adding two numbers such as
\(2+3\). A /random/ experiment is one whose outcome is determined by
chance. We posit that the outcome of a random experiment may not be
predicted with certainty beforehand, even in principle. Examples of
random experiments include tossing a coin, rolling a die, and throwing
a dart on a board, how many red lights you encounter on the drive
home, how many ants traverse a certain patch of sidewalk over a short
period, /etc/.

 *What do I want them to know?*
- that there are multiple interpretations of probability, and the
  methods used depend somewhat on the philosophy chosen
- nuts and bolts of basic probability jargon: sample spaces, events,
  probability functions, /etc/.
- how to count
- conditional probability and its relationship with independence
- Bayes' Rule and how it relates to the subjective view of probability
- what we mean by 'random variables', and where they come from

#+NAME: diagram
#+BEGIN_SRC R :exports results :results graphics :file fig/prob-diagram.ps
require(diagram)
par(mex = 0.2, cex = 0.5)
openplotmat(frame.plot=TRUE)
straightarrow(from = c(0.46,0.74), to = c(0.53,0.71), arr.pos = 1)
straightarrow(from = c(0.3,0.65), to = c(0.3,0.51), arr.pos = 1)
textellipse(mid = c(0.74,0.55), box.col = grey(0.95), 
  radx = 0.24, rady = 0.22, 
  lab = c(expression(bold(underline(DETERMINISTIC))), 
          expression(2*H[2]+O[2] %->% H[2]*O), "3 + 4 = 7"), cex = 2 )
textrect(mid = c(0.3, 0.75), radx = 0.15, rady = 0.1, 
  lab = c(expression(bold(Experiments))), cex = 2 )
textellipse(mid = c(0.29,0.25), box.col = grey(0.95), 
  radx = 0.27, rady = 0.22, lab = c(expression(bold(underline(RANDOM))), 
  "toss coin, roll die", "count ants on sidewalk", "measure rainfall" ), 
  cex = 2 )
#+END_SRC

#+NAME: fig-diagram
#+CAPTION[Two types of experiments]: \small Two types of experiments.
#+ATTR_LaTeX: :width 0.9\textwidth :placement [ht!]
#+RESULTS: diagram
[[file:fig/prob-diagram.ps]]

** Sample Spaces
:PROPERTIES:
:CUSTOM_ID: sec-Sample-Spaces
:END:

For a random experiment \(E\), the set of all possible outcomes of
\(E\) is called the /sample space/ @@latex:\index{sample space}@@ and is denoted
by the letter \(S\). For a coin-toss experiment, \(S\) would be the
results "Head" and "Tail", which we may represent by \( S = \{H,T
\} \). Formally, the performance of a random experiment is the
unpredictable selection of an outcome in \(S\).

*** How to do it with \(\mathsf{R}\)

Most of the probability work in this book is done with the =prob=
package \cite{prob}. A sample space is (usually) represented by a
/data frame/, that is, a rectangular collection of variables (see
Section [[#sub-Multivariate-Data]]). Each row of the data frame corresponds to
an outcome of the experiment. The data frame choice is convenient both
for its simplicity and its compatibility with the \(\mathsf{R}\)
Commander. Data frames alone are, however, not sufficient to describe
some of the more interesting probabilistic applications we will study
later; to handle those we will need to consider a more general /list/
data structure. See Section [[#sub-howto-ps-objects]] for details.

# +BEGIN_exampletoo

Consider the random experiment of dropping a Styrofoam cup onto the
floor from a height of four feet. The cup hits the ground and
eventually comes to rest. It could land upside down, right side up, or
it could land on its side. We represent these possible outcomes of the
random experiment by the following.

#+BEGIN_SRC R :exports both :results output pp  
S <- data.frame(lands = c("down","up","side"))
S
#+END_SRC

#+RESULTS:
:   lands
: 1  down
: 2    up
: 3  side

The sample space =S= contains the column =lands= which stores the
outcomes =down=, =up=, and =side=.

# +END_exampletoo


Some sample spaces are so common that convenience wrappers were
written to set them up with minimal effort. The underlying machinery
that does the work includes the =expand.grid= function in the =base=
package \cite{base}, =combn= in the =combinat= package
\cite{combinat}, and =permsn= in the =prob= package
\cite{prob}[fn:fn-seasoned].

[fn:fn-seasoned] The seasoned \(\mathsf{R}\) user can get the job done
without the convenience wrappers. I encourage the beginner to use them
to get started, but I also recommend that introductory students wean
themselves as soon as possible. The wrappers were designed for ease
and intuitive use, not for speed or efficiency.

Consider the random experiment of tossing a coin. The outcomes are
\(H\) and \(T\). We can set up the sample space quickly with the
=tosscoin= function:

#+BEGIN_SRC R :exports both :results output pp  
tosscoin(1)
#+END_SRC

#+RESULTS:
:   toss1
: 1     H
: 2     T

The number =1= tells =tosscoin= that we only want to toss the coin
once. We could toss it three times:

#+BEGIN_SRC R :exports both :results output pp  
tosscoin(3)
#+END_SRC

#+RESULTS:
:   toss1 toss2 toss3
: 1     H     H     H
: 2     T     H     H
: 3     H     T     H
: 4     T     T     H
: 5     H     H     T
: 6     T     H     T
: 7     H     T     T
: 8     T     T     T

Alternatively we could roll a fair die: 

#+BEGIN_SRC R :exports both :results output pp  
rolldie(1) 
#+END_SRC

#+RESULTS:
:   X1
: 1  1
: 2  2
: 3  3
: 4  4
: 5  5
: 6  6

The =rolldie= function defaults to a 6-sided die, but we can specify
others with the =nsides= argument. The command =rolldie(3, nsides =
4)= would be used to roll a 4-sided die three times.

Perhaps we would like to draw one card from a standard set of playing
cards (it is a long data frame):

#+BEGIN_SRC R :exports both :results output pp
head(cards()) 
#+END_SRC

#+RESULTS:
:   rank suit
: 1    2 Club
: 2    3 Club
: 3    4 Club
: 4    5 Club
: 5    6 Club
: 6    7 Club

The =cards= function that we just used has optional arguments =jokers=
(if you would like Jokers to be in the deck) and =makespace= which we
will discuss later. There is also a =roulette= function which returns
the sample space associated with one spin on a roulette wheel. There
are EU and USA versions available. Interested readers may contribute
any other game or sample spaces that may be of general interest.

*** Sampling from Urns
:PROPERTIES:
:CUSTOM_ID: sub-sampling-from-urns
:END:

This is perhaps the most fundamental type of random experiment. We
have an urn that contains a bunch of distinguishable objects (balls)
inside. We shake up the urn, reach inside, grab a ball, and take a
look. That's all.

But there are all sorts of variations on this theme. Maybe we would
like to grab more than one ball -- say, two balls. What are all of the
possible outcomes of the experiment now? It depends on how we
sample. We could select a ball, take a look, put it back, and sample
again. Another way would be to select a ball, take a look -- but do
not put it back -- and sample again (equivalently, just reach in and
grab two balls). There are certainly more possible outcomes of the
experiment in the former case than in the latter. In the first
(second) case we say that sampling is done /with (without)
replacement/.

There is more. Suppose we do not actually keep track of which ball
came first. All we observe are the two balls, and we have no idea
about the order in which they were selected. We call this /unordered
sampling/ (in contrast to /ordered/) because the order of the
selections does not matter with respect to what we observe. We might
as well have selected the balls and put them in a bag before looking.

Note that this one general class of random experiments contains as a
special case all of the common elementary random experiments. Tossing
a coin twice is equivalent to selecting two balls labeled \(H\) and
\(T\) from an urn, with replacement. The die-roll experiment is
equivalent to selecting a ball from an urn with six elements, labeled
1 through 6.

**** How to do it with \(\mathsf{R}\) 

The =prob= package \cite{prob} accomplishes sampling from urns with
the =urnsamples= @@latex:\index{urnsamples@\texttt{urnsamples}}@@
function, which has arguments =x=, =size=, =replace=, and
=ordered=. The argument =x= represents the urn from which sampling is
to be done. The =size= argument tells how large the sample will
be. The =ordered= and =replace= arguments are logical and specify how
sampling will be performed. We will discuss each in turn.

# +BEGIN_exampletoo
<<exa-sample-urn-two-from-three>> Let our urn simply contain three
balls, labeled 1, 2, and 3, respectively. We are going to take a
sample of size 2 from the urn.

# +END_exampletoo

**** Ordered, With Replacement

If sampling is with replacement, then we can get any outcome 1, 2, or
3 on any draw. Further, by "ordered" we mean that we shall keep
track of the order of the draws that we observe. We can accomplish
this in \(\mathsf{R}\) with

#+BEGIN_SRC R :exports both :results output pp  
urnsamples(1:3, size = 2, replace = TRUE, ordered = TRUE)
#+END_SRC 

#+RESULTS:
#+BEGIN_example
  X1 X2
1  1  1
2  2  1
3  3  1
4  1  2
5  2  2
6  3  2
7  1  3
8  2  3
9  3  3
#+END_example

Notice that rows 2 and 4 are identical, save for the order in which
the numbers are shown. Further, note that every possible pair of the
numbers 1 through 3 are listed. This experiment is equivalent to
rolling a 3-sided die twice, which we could have accomplished with
=rolldie(2, nsides = 3)=.

**** Ordered, Without Replacement

Here sampling is without replacement, so we may not observe the same
number twice in any row. Order is still important, however, so we
expect to see the outcomes =1,2= and =2,1= somewhere in our data
frame.

#+BEGIN_SRC R :exports both :results output pp   
urnsamples(1:3, size = 2, replace = FALSE, ordered = TRUE)
#+END_SRC

#+RESULTS:
:   X1 X2
: 1  1  2
: 2  2  1
: 3  1  3
: 4  3  1
: 5  2  3
: 6  3  2

This is just as we expected. Notice that there are less rows in this
answer due to the more restrictive sampling procedure. If the numbers
1, 2, and 3 represented "Fred", "Mary", and "Sue", respectively,
then this experiment would be equivalent to selecting two people of
the three to serve as president and vice-president of a company,
respectively, and the sample space shown above lists all possible ways
that this could be done.

**** Unordered, Without Replacement

Again, we may not observe the same outcome twice, but in this case, we
will only retain those outcomes which (when jumbled) would not
duplicate earlier ones.

#+BEGIN_SRC R :exports both :results output pp   
urnsamples(1:3, size = 2, replace = FALSE, ordered = FALSE) 
#+END_SRC 

#+RESULTS:
:   X1 X2
: 1  1  2
: 2  1  3
: 3  2  3

This experiment is equivalent to reaching in the urn, picking a pair,
and looking to see what they are. This is the default setting of
=urnsamples=, so we would have received the same output by simply
typing =urnsamples(1:3, 2)=.

**** Unordered, With Replacement

The last possibility is perhaps the most interesting. We replace the
balls after every draw, but we do not remember the order in which the
draws came.

#+BEGIN_SRC R :exports both :results output pp   
urnsamples(1:3, size = 2, replace = TRUE, ordered = FALSE) 
#+END_SRC

#+RESULTS:
:   X1 X2
: 1  1  1
: 2  1  2
: 3  1  3
: 4  2  2
: 5  2  3
: 6  3  3

We may interpret this experiment in a number of alternative ways. One
way is to consider this as simply putting two 3-sided dice in a cup,
shaking the cup, and looking inside -- as in a game of /Liar's Dice/,
for instance. Each row of the sample space is a potential pair we
could observe. Another way is to view each outcome as a separate
method to distribute two identical golf balls into three boxes labeled
1, 2, and 3. Regardless of the interpretation, =urnsamples= lists
every possible way that the experiment can conclude.



Note that the urn does not need to contain numbers; we could have just
as easily taken our urn to be =x = c("Red","Blue","Green")=. But,
there is an *important* point to mention before proceeding. Astute
readers will notice that in our example, the balls in the urn were
/distinguishable/ in the sense that each had a unique label to
distinguish it from the others in the urn. A natural question would
be, "What happens if your urn has indistinguishable elements, for
example, what if =x = c("Red","Red","Blue")=?" The answer is that
=urnsamples= behaves as if each ball in the urn is distinguishable,
regardless of its actual contents. We may thus imagine that while
there are two red balls in the urn, the balls are such that we can
tell them apart (in principle) by looking closely enough at the
imperfections on their surface.

In this way, when the =x= argument of =urnsamples= has repeated
elements, the resulting sample space may appear to be =ordered = TRUE=
even when, in fact, the call to the function was =urnsamples(...,
ordered = FALSE)=. Similar remarks apply for the =replace= argument.

** Events
:PROPERTIES:
:CUSTOM_ID: sec-Events
:END:

An /event/ @@latex:\index{event}@@ \(A\) is merely a collection of outcomes, or
in other words, a subset of the sample space[fn:fn-event]. After the
performance of a random experiment \(E\) we say that the event \(A\)
/occurred/ if the experiment's outcome belongs to \(A\). We say that a
bunch of events \(A_{1}\), \(A_{2}\), \(A_{3}\), ... are /mutually
exclusive/ @@latex:\index{mutually exclusive}@@ or /disjoint/ if \(A_{i}\cap
A_{j}=\emptyset\) for any distinct pair \(A_{i}\neq A_{j}\). For
instance, in the coin-toss experiment the events \( A = \{
\mbox{Heads} \}\) and \( B = \{ \mbox{Tails} \} \) would be mutually
exclusive. Now would be a good time to review the algebra of sets in
Appendix [[#sec-The-Algebra-of]].

[fn:fn-event] This naive definition works for finite or countably
infinite sample spaces, but is inadequate for sample spaces in
general. In this book, we will not address the subtleties that arise,
but will refer the interested reader to any text on advanced
probability or measure theory.

*** How to do it with \(\mathsf{R}\)

Given a data frame sample/probability space =S=, we may extract rows
using the =[]= operator:

#+BEGIN_SRC R :exports both :results output pp   
S <- tosscoin(2, makespace = TRUE) 
S[1:3, ] 
#+END_SRC

#+RESULTS:
:   toss1 toss2 probs
: 1     H     H  0.25
: 2     T     H  0.25
: 3     H     T  0.25

#+BEGIN_SRC R :exports both :results output pp   
S[c(2,4), ] 
#+END_SRC

#+RESULTS:
:   toss1 toss2 probs
: 2     T     H  0.25
: 4     T     T  0.25

and so forth. We may also extract rows that satisfy a logical
expression using the =subset= function, for instance

#+BEGIN_SRC R :exports code :results silent
S <- cards() 
#+END_SRC 

#+BEGIN_SRC R :exports both :results output pp  
subset(S, suit == "Heart") 
#+END_SRC

#+RESULTS:
#+BEGIN_example
   rank  suit
27    2 Heart
28    3 Heart
29    4 Heart
30    5 Heart
31    6 Heart
32    7 Heart
33    8 Heart
34    9 Heart
35   10 Heart
36    J Heart
37    Q Heart
38    K Heart
39    A Heart
#+END_example

#+BEGIN_SRC R :exports both :results output pp  
subset(S, rank %in% 7:9)
#+END_SRC

#+RESULTS:
#+BEGIN_example
   rank    suit
6     7    Club
7     8    Club
8     9    Club
19    7 Diamond
20    8 Diamond
21    9 Diamond
32    7   Heart
33    8   Heart
34    9   Heart
45    7   Spade
46    8   Spade
47    9   Spade
#+END_example

We could continue indefinitely. Also note that mathematical
expressions are allowed:

#+BEGIN_SRC R :exports both :results output pp   
subset(rolldie(3), X1+X2+X3 > 16) 
#+END_SRC

#+RESULTS:
:     X1 X2 X3
: 180  6  6  5
: 210  6  5  6
: 215  5  6  6
: 216  6  6  6

*** Functions for Finding Subsets

It does not take long before the subsets of interest become
complicated to specify. Yet the main idea remains: we have a
particular logical condition to apply to each row. If the row
satisfies the condition, then it should be in the subset. It should
not be in the subset otherwise. The ease with which the condition may
be coded depends of course on the question being asked. Here are a few
functions to get started.

**** The =%in%= function

The function =%in%= helps to learn whether each value of one vector
lies somewhere inside another vector.

#+BEGIN_SRC R :exports both :results output pp  
x <- 1:10 
y <- 8:12 
y %in% x
#+END_SRC

#+RESULTS:
: [1]  TRUE  TRUE  TRUE FALSE FALSE

Notice that the returned value is a vector of length 5 which tests
whether each element of =y= is in =x=, in turn.

**** The =isin= function

It is more common to want to know whether the /whole/ vector =y= is in
=x=. We can do this with the =isin= function.

#+BEGIN_SRC R :exports both :results output pp   
isin(x,y) 
#+END_SRC

#+RESULTS:
: [1] FALSE

Of course, one may ask why we did not try something like =all(y %in%
x)=, which would give a single result, =TRUE=. The reason is that the
answers are different in the case that =y= has repeated
values. Compare:

#+BEGIN_SRC R :exports code :results silent
x <- 1:10 
y <- c(3,3,7) 
#+END_SRC 

#+BEGIN_SRC R :exports both :results output pp   
all(y %in% x)
isin(x,y) 
#+END_SRC 

#+RESULTS:
: [1] TRUE
: [1] FALSE

The reason for the above is of course that =x= contains the value 3,
but =x= does not have /two/ 3's. The difference is important when
rolling multiple dice, playing cards, /etc/. Note that there is an
optional argument =ordered= which tests whether the elements of =y=
appear in =x= in the order in which they are appear in =y=. The
consequences are

#+BEGIN_SRC R :exports both :results output pp   
isin(x, c(3,4,5), ordered = TRUE) 
isin(x, c(3,5,4), ordered = TRUE) 
#+END_SRC 

#+RESULTS:
: [1] TRUE
: [1] FALSE

The connection to probability is that have a data frame sample space
and we would like to find a subset of that space. A =data.frame=
method was written for =isin= that simply applies the function to each
row of the data frame. We can see the method in action with the
following:

#+BEGIN_SRC R :exports both :results output pp  
S <- rolldie(4) 
subset(S, isin(S, c(2,2,6), ordered = TRUE)) 
#+END_SRC

#+RESULTS:
#+BEGIN_example
     X1 X2 X3 X4
188   2  2  6  1
404   2  2  6  2
620   2  2  6  3
836   2  2  6  4
1052  2  2  6  5
1088  2  2  1  6
1118  2  1  2  6
1123  1  2  2  6
1124  2  2  2  6
1125  3  2  2  6
1126  4  2  2  6
1127  5  2  2  6
1128  6  2  2  6
1130  2  3  2  6
1136  2  4  2  6
1142  2  5  2  6
1148  2  6  2  6
1160  2  2  3  6
1196  2  2  4  6
1232  2  2  5  6
1268  2  2  6  6
#+END_example

There are a few other functions written to find useful subsets,
namely, =countrep= and =isrep=. Essentially these were written to test
for (or count) a specific number of designated values in outcomes. See
the documentation for details.

*** Set Union, Intersection, and Difference

Given subsets \(A\) and \(B\), it is often useful to manipulate them
in an algebraic fashion. To this end, we have three set operations at
our disposal: union, intersection, and difference. Below is a table
that summarizes the pertinent information about these operations.

#+CAPTION[Set operations]: Basic set operations.  The first column lists the name, the second shows the typical notation, the third describes set membership, and the fourth shows how to accomplish it with R.
|--------------+-------------------+---------------------------+------------------|
| Name         | Denoted           | Defined by elements       | Code             |
|--------------+-------------------+---------------------------+------------------|
| Union        | \(A\cup B\)       | in \(A\) or \(B\) or both | =union(A,B)=     |
| Intersection | \(A\cap B\)       | in both \(A\) and \(B\)   | =intersect(A,B)= |
| Difference   | \(A\backslash B\) | in \(A\) but not in \(B\) | =setdiff(A,B)=   |
|--------------+-------------------+---------------------------+------------------|

Some examples follow. 

#+BEGIN_SRC R :exports code :results silent
S <- cards() 
A <- subset(S, suit == "Heart") 
B <- subset(S, rank %in% 7:9)
#+END_SRC 

We can now do some set algebra: 

#+BEGIN_SRC R :exports both :results output pp  
union(A,B)
#+END_SRC

#+RESULTS:
#+BEGIN_example
   rank    suit
6     7    Club
7     8    Club
8     9    Club
19    7 Diamond
20    8 Diamond
21    9 Diamond
27    2   Heart
28    3   Heart
29    4   Heart
30    5   Heart
31    6   Heart
32    7   Heart
33    8   Heart
34    9   Heart
35   10   Heart
36    J   Heart
37    Q   Heart
38    K   Heart
39    A   Heart
45    7   Spade
46    8   Spade
47    9   Spade
#+END_example

#+BEGIN_SRC R :exports both :results output pp  
intersect(A,B)
#+END_SRC 

#+RESULTS:
:    rank  suit
: 32    7 Heart
: 33    8 Heart
: 34    9 Heart

#+BEGIN_SRC R :exports both :results output pp  
setdiff(A,B)
#+END_SRC

#+RESULTS:
#+BEGIN_example
   rank  suit
27    2 Heart
28    3 Heart
29    4 Heart
30    5 Heart
31    6 Heart
35   10 Heart
36    J Heart
37    Q Heart
38    K Heart
39    A Heart
#+END_example

#+BEGIN_SRC R :exports both :results output pp  
setdiff(B,A) 
#+END_SRC 

#+RESULTS:
#+BEGIN_example
   rank    suit
6     7    Club
7     8    Club
8     9    Club
19    7 Diamond
20    8 Diamond
21    9 Diamond
45    7   Spade
46    8   Spade
47    9   Spade
#+END_example

Notice that =setdiff= is not symmetric. Further, note that we can
calculate the /complement/ of a set \(A\), denoted \(A^{c}\) and
defined to be the elements of \(S\) that are not in \(A\) simply with
=setdiff(S,A)=. There have been methods written for =intersect=,
=setdiff=, =subset=, and =union= in the case that the input objects
are of class =ps=. See Section [[#sub-howto-ps-objects]].

#+BEGIN_note

When the =prob= package \cite{prob} loads you will notice a message:
=The following object(s) are masked from package:base: intersect,
setdiff=. The reason for this message is that there already exist
methods for the functions =intersect=, =setdiff=, =subset=, and
=union= in the =base= package which ships with
\(\mathsf{R}\). However, these methods were designed for when the
arguments are vectors of the same mode. Since we are manipulating
sample spaces which are data frames and lists, it was necessary to
write methods to handle those cases as well. When the =prob= package
is loaded, \(\mathsf{R}\) recognizes that there are multiple versions
of the same function in the search path and acts to shield the new
definitions from the existing ones. But there is no cause for alarm,
thankfully, because the =prob= functions have been carefully defined
to match the usual =base= package definition in the case that the
arguments are vectors.

#+END_note

** Model Assignment
:PROPERTIES:
:CUSTOM_ID: sec-Interpreting-Probabilities
:END:

Let us take a look at the coin-toss experiment more closely. What do
we mean when we say "the probability of Heads" or write
\(\mathbb{P}(\mbox{Heads})\)? Given a coin and an itchy thumb, how do
we go about finding what \(\mathbb{P}(\mbox{Heads})\) should be?

*** The Measure Theory Approach

This approach states that the way to handle
\(\mathbb{P}(\mbox{Heads})\) is to define a mathematical function,
called a /probability measure/, on the sample space. Probability
measures satisfy certain axioms (to be introduced later) and have
special mathematical properties, so not just any mathematical function
will do. But in any given physical circumstance there are typically
all sorts of probability measures from which to choose, and it is left
to the experimenter to make a reasonable choice -- one usually based
on considerations of objectivity. For the tossing coin example, a
valid probability measure assigns probability \(p\) to the event \( \{
\mbox{Heads} \} \), where \(p\) is some number \(0\leq p\leq1\). An
experimenter that wishes to incorporate the symmetry of the coin would
choose \(p=1/2\) to balance the likelihood of \( \{\mbox{Heads} \} \)
and \( \{ \mbox{Tails} \} \).

Once the probability measure is chosen (or determined), there is not
much left to do. All assignments of probability are made by the
probability function, and the experimenter needs only to plug the
event \(\{ \mbox{Heads} \}\) into to the probability function to find
\(\mathbb{P}(\mbox{Heads})\). In this way, the probability of an event
is simply a calculated value, nothing more, nothing less. Of course
this is not the whole story; there are many theorems and consequences
associated with this approach that will keep us occupied for the
remainder of this book. The approach is called /measure theory/
because the measure (probability) of a set (event) is associated with
how big it is (how likely it is to occur).

The measure theory approach is well suited for situations where there
is symmetry to the experiment, such as flipping a balanced coin or
spinning an arrow around a circle with well-defined pie slices. It is
also handy because of its mathematical simplicity, elegance, and
flexibility. There are literally volumes of information that one can
prove about probability measures, and the cold rules of mathematics
allow us to analyze intricate probabilistic problems with vigor.

The large degree of flexibility is also a disadvantage, however. When
symmetry fails it is not always obvious what an "objective" choice
of probability measure should be; for instance, what probability
should we assign to \( \{ \mbox{Heads} \} \) if we spin the coin
rather than flip it? (It is not \(1/2\).) Furthermore, the
mathematical rules are restrictive when we wish to incorporate
subjective knowledge into the model, knowledge which changes over time
and depends on the experimenter, such as personal knowledge about the
properties of the specific coin being flipped, or of the person doing
the flipping.

The mathematician who revolutionized this way to do probability theory
was Andrey Kolmogorov, who published a landmark monograph in 1933. See
[[http://www-history.mcs.st-andrews.ac.uk/Mathematicians/Kolmogorov.html][here]] for more information.

*** Relative Frequency Approach

This approach states that the way to determine
\(\mathbb{P}(\mbox{Heads})\) is to flip the coin repeatedly, in
exactly the same way each time. Keep a tally of the number of flips
and the number of Heads observed. Then a good approximation to
\(\mathbb{P}(\mbox{Heads})\) will be
\begin{equation} 
\mathbb{P}(\mbox{Heads})\approx\frac{\mbox{number of observed Heads}}{\mbox{total number of flips}}.
\end{equation}
The mathematical underpinning of this approach is the celebrated *Law
of Large Numbers* which may be loosely described as follows. Let \(E\)
be a random experiment in which the event \(A\) either does or does
not occur. Perform the experiment repeatedly, in an identical manner,
in such a way that the successive experiments do not influence each
other. After each experiment, keep a running tally of whether or not
the event \(A\) occurred. Let \(S_{n}\) count the number of times that
\(A\) occurred in the \(n\) experiments. Then the law of large numbers
says that
\begin{equation}
\frac{S_{n}}{n}\to\mathbb{P}(A)\mbox{ as }n\to\infty.
\end{equation}
As the reasoning goes, to learn about the probability of an event
\(A\) we need only repeat the random experiment to get a reasonable
estimate of the probability's value, and if we are not satisfied with
our estimate then we may simply repeat the experiment more times all
the while confident that with more and more experiments our estimate
will stabilize to the true value.

The frequentist approach is good because it is relatively light on
assumptions and does not worry about symmetry or claims of objectivity
like the measure-theoretic approach does. It is perfect for the
spinning coin experiment. One drawback to the method is that one can
never know the exact value of a probability, only a long-run
approximation. It also does not work well with experiments that can
not be repeated indefinitely, say, the probability that it will rain
today, the chances that you get will get an A in your Statistics
class, or the probability that the world is destroyed by nuclear war.

This approach was espoused by Richard von Mises in the early twentieth
century, and some of his main ideas were incorporated into the measure
theory approach. See [[http://www-history.mcs.st-andrews.ac.uk/Biographies/Mises.html][here]] for more.

*** The Subjective Approach

The subjective approach interprets probability as the experimenter's
/degree of belief/ that the event will occur. The estimate of the
probability of an event is based on the totality of the individual's
knowledge at the time. As new information becomes available, the
estimate is modified accordingly to best reflect his/her current
knowledge. The method by which the probabilities are updated is
commonly done with Bayes' Rule, discussed in Section [[#sec-Bayes-Rule]].

So for the coin toss example, a person may have
\(\mathbb{P}(\mbox{Heads})=1/2\) in the absence of additional
information. But perhaps the observer knows additional information
about the coin or the thrower that would shift the probability in a
certain direction. For instance, parlor magicians may be trained to be
quite skilled at tossing coins, and some are so skilled that they may
toss a fair coin and get nothing but Heads, indefinitely. I have
/seen/ this. It was similarly claimed in /Bringing Down the House/
\cite{Mezrich2003} that MIT students were accomplished enough with
cards to be able to cut a deck to the same location, every single
time. In such cases, one clearly should use the additional information
to assign \(\mathbb{P}(\mbox{Heads})\) away from the symmetry value of
\(1/2\).

This approach works well in situations that cannot be repeated
indefinitely, for example, to assign your probability that you will
get an A in this class, the chances of a devastating nuclear war, or
the likelihood that a cure for the common cold will be discovered.

The roots of subjective probability reach back a long time. See [[http://en.wikipedia.org/wiki/Subjective_probability][here]]
for a short discussion and links to references about the subjective
approach.

*** Equally Likely Model (ELM)

We have seen several approaches to the assignment of a probability
model to a given random experiment and they are very different in
their underlying interpretation. But they all cross paths when it
comes to the equally likely model which assigns equal probability to
all elementary outcomes of the experiment.

The ELM appears in the measure theory approach when the experiment
boasts symmetry of some kind. If symmetry guarantees that all outcomes
have equal "size", and if outcomes with equal "size" should get
the same probability, then the ELM is a logical objective choice for
the experimenter. Consider the balanced 6-sided die, the fair coin, or
the dart board with equal-sized wedges.

The ELM appears in the subjective approach when the experimenter
resorts to indifference or ignorance with respect to his/her knowledge
of the outcome of the experiment. If the experimenter has no prior
knowledge to suggest that (s)he prefer Heads over Tails, then it is
reasonable for the him/her to assign equal subjective probability to
both possible outcomes.

The ELM appears in the relative frequency approach as a fascinating
fact of Nature: when we flip balanced coins over and over again, we
observe that the proportion of times that the coin comes up Heads
tends to \(1/2\). Of course if we assume that the measure theory
applies then we can prove that the sample proportion must tend to 1/2
as expected, but that is putting the cart before the horse, in a
manner of speaking.

The ELM is only available when there are finitely many elements in the
sample space.

**** How to do it with \(\mathsf{R}\)

In the =prob= package \cite{prob}, a probability space is an object of
outcomes =S= and a vector of probabilities (called =probs=) with
entries that correspond to each outcome in =S=. When =S= is a data
frame, we may simply add a column called =probs= to =S= and we will be
finished; the probability space will simply be a data frame which we
may call =S=. In the case that S is a list, we may combine the
=outcomes= and =probs= into a larger list, =space=; it will have two
components: =outcomes= and =probs=. The only requirements we need are
for the entries of =probs= to be nonnegative and =sum(probs)= to be
one.

To accomplish this in \(\mathsf{R}\), we may use the =probspace=
function. The general syntax is =probspace(x, probs)=, where =x= is a
sample space of outcomes and =probs= is a vector (of the same length
as the number of outcomes in =x=). The specific choice of =probs=
depends on the context of the problem, and some examples follow to
demonstrate some of the more common choices.

# +BEGIN_exampletoo

The Equally Likely Model asserts that every outcome of the sample
space has the same probability, thus, if a sample space has \(n\)
outcomes, then =probs= would be a vector of length \(n\) with
identical entries \(1/n\). The quickest way to generate =probs= is
with the =rep= function. We will start with the experiment of rolling
a die, so that \(n=6\). We will construct the sample space, generate
the =probs= vector, and put them together with =probspace=.

#+BEGIN_SRC R :exports both :results output pp   
outcomes <- rolldie(1) 
p <- rep(1/6, times = 6) 
probspace(outcomes, probs = p) 
#+END_SRC

#+RESULTS:
:   X1     probs
: 1  1 0.1666667
: 2  2 0.1666667
: 3  3 0.1666667
: 4  4 0.1666667
: 5  5 0.1666667
: 6  6 0.1666667

The =probspace= function is designed to save us some time in the most
common situations. For example, due to the especial simplicity of the
sample space in this case, we could have achieved the same result with
only (note the name change for the first column)

#+BEGIN_SRC R :exports both :results output pp   
probspace(1:6, probs = p) 
#+END_SRC 

#+RESULTS:
:   x     probs
: 1 1 0.1666667
: 2 2 0.1666667
: 3 3 0.1666667
: 4 4 0.1666667
: 5 5 0.1666667
: 6 6 0.1666667

Further, since the equally likely model plays such a fundamental role
in the study of probability the =probspace= function will assume that
the equally model is desired if no =probs= are specified. Thus, we get
the same answer with only

#+BEGIN_SRC R :exports both :results output pp   
probspace(1:6) 
#+END_SRC 

#+RESULTS:
:   x     probs
: 1 1 0.1666667
: 2 2 0.1666667
: 3 3 0.1666667
: 4 4 0.1666667
: 5 5 0.1666667
: 6 6 0.1666667

And finally, since rolling dice is such a common experiment in
probability classes, the =rolldie= function has an additional logical
argument =makespace= that will add a column of equally likely =probs=
to the generated sample space:

#+BEGIN_SRC R :exports both :results output pp   
rolldie(1, makespace = TRUE)
#+END_SRC

#+RESULTS:
:   X1     probs
: 1  1 0.1666667
: 2  2 0.1666667
: 3  3 0.1666667
: 4  4 0.1666667
: 5  5 0.1666667
: 6  6 0.1666667

#+LaTeX: \noindent 
or just =rolldie(1, TRUE)=. Many of the other sample space functions
(=tosscoin=, =cards=, =roulette=, /etc/.) have similar =makespace=
arguments. Check the documentation for details.

# +END_exampletoo


One sample space function that does NOT have a =makespace= option is
the =urnsamples= function. This was intentional. The reason is that
under the varied sampling assumptions the outcomes in the respective
sample spaces are NOT, in general, equally likely. It is important for
the user to carefully consider the experiment to decide whether or not
the outcomes are equally likely and then use =probspace= to assign the
model.

# +BEGIN_exampletoo
<<exa-unbalanced-coin>> *An unbalanced coin.* While the =makespace=
argument to =tosscoin= is useful to represent the tossing of a /fair/
coin, it is not always appropriate. For example, suppose our coin is
not perfectly balanced, for instance, maybe the \(H\) side is somewhat
heavier such that the chances of a \(H\) appearing in a single toss is
0.70 instead of 0.5. We may set up the probability space with

#+BEGIN_SRC R :exports both :results output pp   
probspace(tosscoin(1), probs = c(0.70, 0.30)) 
#+END_SRC 

#+RESULTS:
:   toss1 probs
: 1     H   0.7
: 2     T   0.3

The same procedure can be used to represent an unbalanced die,
roulette wheel, /etc/.

# +END_exampletoo

*** Words of Warning

It should be mentioned that while the splendour of \(\mathsf{R}\) is
uncontested, it, like everything else, has limits both with respect to
the sample/probability spaces it can manage and with respect to the
finite accuracy of the representation of most numbers (see the
\(\mathsf{R}\) FAQ 7.31). When playing around with probability, one
may be tempted to set up a probability space for tossing 100 coins or
rolling 50 dice in an attempt to answer some scintillating
question. (Bear in mind: rolling a die just 9 times has a sample space
with over /10 million/ outcomes.)

Alas, even if there were enough RAM to barely hold the sample space
(and there were enough time to wait for it to be generated), the
infinitesimal probabilities that are associated with /so many/
outcomes make it difficult for the underlying machinery to handle
reliably. In some cases, special algorithms need to be called just to
give something that holds asymptotically. User beware.

** Properties of Probability
:PROPERTIES:
:CUSTOM_ID: sec-Properties-of-Probability
:END:

*** Probability Functions
:PROPERTIES:
:CUSTOM_ID: sub-Probability-Functions
:END:

A /probability function/ is a rule that associates with each event
\(A\) of the sample space a single number \(\mathbb{P}(A)=p\), called
the /probability of/ \(A\). Any probability function \(\mathbb{P}\)
satisfies the following three Kolmogorov Axioms:

#+BEGIN_ax
<<ax-prob-nonnegative>> \(\mathbb{P}(A)\geq0\) for any event \(A\subset S\).
#+END_ax

#+BEGIN_ax
<<ax-total-mass-one>> \(\mathbb{P}(S)=1\).
#+END_ax

#+BEGIN_ax
<<ax-countable-additivity>> If the events \(A_{1}\), \(A_{2}\),
\(A_{3}\)... are disjoint then
\begin{equation}
\mathbb{P}\left(\bigcup_{i=1}^{n}A_{i}\right)=\sum_{i=1}^{n}\mathbb{P}(A_{i})\mbox{ for every }n,
\end{equation}
and furthermore,
\begin{equation}
\mathbb{P}\left(\bigcup_{i=1}^{\infty}A_{i}\right)=\sum_{i=1}^{\infty}\mathbb{P}(A_{i}).
\end{equation}
#+END_ax

The intuition behind the axioms goes like this: first, the probability
of an event should never be negative. Second, since the sample space
contains all possible outcomes, its probability should be one, or
100%. The last axiom may look intimidating but it simply means that in
a sequence of disjoint events (in other words, sets that do not
overlap), the total probability (measure) should equal the sum of its
parts. For example, the chance of rolling a 1 or a 2 on a die should
be the chance of rolling a 1 plus the chance of rolling a 2.

*** Properties

For any events \(A\) and \(B\),

1. <<enu-prop-prob-complement>> \(\mathbb{P}(A^{c})=1-\mathbb{P}(A)\).
   #+BEGIN_proof
   Since \(A\cup A^{c}=S\) and \(A\cap A^{c}=\emptyset\), we have
   \[
   1=\mathbb{P}(S)=\mathbb{P}(A\cup A^{c})=\mathbb{P}(A)+\mathbb{P}(A^{c}).
   \]
   #+END_proof
1. \(\mathbb{P}(\emptyset)=0\).
   #+BEGIN_proof
   Note that \(\emptyset=S^{c}\), and use Property 1.
   #+END_proof
1. If \(A\subset B\) , then \(\mathbb{P}(A)\leq\mathbb{P}(B)\).
   #+BEGIN_proof
   Write \(B=A\cup\left(B\cap A^{c}\right)\), and notice that \(A\cap\left(B\cap A^{c}\right)=\emptyset\); thus
   \[
   \mathbb{P}(B)=\mathbb{P}(A\cup\left(B\cap A^{c}\right))=\mathbb{P}(A)+\mathbb{P}\left(B\cap A^{c}\right)\geq\mathbb{P}(A),
   \]
   since \(\mathbb{P}\left(B\cap A^{c}\right)\ge0\). 
   #+END_proof
1. \(0\leq\mathbb{P}(A)\leq1\).
   #+BEGIN_proof
   The left inequality is immediate from Axiom [[ax-prob-nonnegative]], and the second inequality follows from Property 3 since \(A\subset S\).
   #+END_proof

1. *The General Addition Rule.*
   \begin{equation}
   \label{eq-general-addition-rule-1}
   \mathbb{P}(A\cup B)=\mathbb{P}(A)+\mathbb{P}(B)-\mathbb{P}(A\cap B).
   \end{equation}
   More generally, for events \(A_{1}\), \(A_{2}\), \(A_{3}\),..., \(A_{n}\),
   \begin{equation}
   \mathbb{P}\left(\bigcup_{i=1}^{n}A_{i}\right)=\sum_{i=1}^{n}\mathbb{P}(A_{i})-\sum_{i=1}^{n-1}\sum_{j=i+1}^{n}\mathbb{P}(A_{i}\cap A_{j})+\cdots+(-1)^{n-1}\mathbb{P}\left(\bigcap_{i=1}^{n}A_{i}\right)
   \end{equation}
1. *The Theorem of Total Probability.* Let \(B_{1}\), \(B_{2}\), ...,
   \(B_{n}\) be mutually exclusive and exhaustive. Then
   \begin{equation}
   \label{eq-theorem-total-probability}
   \mathbb{P}(A)=\mathbb{P}(A\cap B_{1})+\mathbb{P}(A\cap B_{2})+\cdots+\mathbb{P}(A\cap B_{n}).
   \end{equation}

*** Assigning Probabilities

A model of particular interest is the /equally likely model/. The idea
is to divide the sample space \(S\) into a finite collection of
elementary events \( \{ a_{1},\ a_{2}, \ldots, a_{N} \} \) that are
equally likely in the sense that each \(a_{i}\) has equal chances of
occurring. The probability function associated with this model must
satisfy \(\mathbb{P}(S)=1\), by Axiom 2. On the other hand, it must
also satisfy \[ \mathbb{P}(S)=\mathbb{P}( \{ a_{1},\
a_{2},\ldots,a_{N} \} )=\mathbb{P}(a_{1}\cup a_{2}\cup\cdots\cup
a_{N})=\sum_{i=1}^{N}\mathbb{P}(a_{i}), \] by Axiom 3. Since
\(\mathbb{P}(a_{i})\) is the same for all \(i\), each one necessarily
equals \(1/N\).

For an event \(A\subset S\), we write \(A\) as a collection of
elementary outcomes: if \( A = \{ a_{i_{1}}, a_{i_{2}}, \ldots,
a_{i_{k}} \} \) then \(A\) has \(k\) elements and
\begin{align*}
\mathbb{P}(A) & =\mathbb{P}(a_{i_{1}})+\mathbb{P}(a_{i_{2}})+\cdots+\mathbb{P}(a_{i_{k}}),\\
 & =\frac{1}{N}+\frac{1}{N}+\cdots+\frac{1}{N},\\
 & =\frac{k}{N}=\frac{\#(A)}{\#(S)}.
\end{align*}
In other words, under the equally likely model, the probability of an
event \(A\) is determined by the number of elementary events that
\(A\) contains.

# +BEGIN_exampletoo

Consider the random experiment \(E\) of tossing a coin. Then the
sample space is \(S=\{H,T\}\), and under the equally likely model,
these two outcomes have \(\mathbb{P}(H)=\mathbb{P}(T)=1/2\). This
model is taken when it is reasonable to assume that the coin is fair.
# +END_exampletoo


# +BEGIN_exampletoo

Suppose the experiment \(E\) consists of tossing a fair coin
twice. The sample space may be represented by \(S=\{HH,\, HT,\, TH,\,
TT\}\). Given that the coin is fair and that the coin is tossed in an
independent and identical manner, it is reasonable to apply the
equally likely model.

What is \(\mathbb{P}(\mbox{at least 1 Head})\)? Looking at the sample
space we see the elements \(HH\), \(HT\), and \(TH\) have at least one
Head; thus, \(\mathbb{P}(\mbox{at least 1 Head})=3/4\).

What is \(\mathbb{P}(\mbox{no Heads})\)? Notice that the event \(\{
\mbox{no Heads} \} = \{ \mbox{at least one Head} \} ^{c}\), which by
Property [[enu-prop-prob-complement]] means \(\mathbb{P}(\mbox{no
Heads})=1-\mathbb{P}(\mbox{at least one head})=1-3/4=1/4\). It is
obvious in this simple example that the only outcome with no Heads is
\(TT\), however, this complementation trick can be handy in more
complicated problems.
# +END_exampletoo


# +BEGIN_exampletoo
<<exa-three-child-family>> Imagine a three child family, each child
being either Boy (\(B\)) or Girl (\(G\)). An example sequence of
siblings would be \(BGB\). The sample space may be written \[ S =
\left\{ BBB,\ BGB,\ GBB,\ GGB,\ BBG,\ BGG,\ GBG,\ GGG,\ \right\}.\]

Note that for many reasons (for instance, it turns out that girls are
slightly more likely to be born than boys), this sample space is /not/
equally likely. For the sake of argument, however, we will assume that
the elementary outcomes each have probability \(1/8\).

What is \(\mathbb{P}(\mbox{exactly 2 Boys})\)? Inspecting the sample
space reveals three outcomes with exactly two boys: \( \{ BBG,\,
BGB,\, GBB \} \).  Therefore \(\mathbb{P}(\mbox{exactly 2 Boys}) =
3/8\).

What is \(\mathbb{P}(\mbox{at most 2 Boys})\)? One way to solve the
problem would be to count the outcomes that have 2 or less Boys, but a
quicker way would be to recognize that the only way that the event
\(\{ \mbox{at most 2 Boys} \}\) does /not/ occur is the event \(\{
\mbox{all Boys} \}\).

Thus \[ \mathbb{P}(\mbox{at most 2 Boys}) = 1 - \mathbb{P}(BBB) = 1 -
1/8 = 7/8. \]

# +END_exampletoo


# +BEGIN_exampletoo

Consider the experiment of rolling a six-sided die, and let the
outcome be the face showing up when the die comes to rest. Then \( S =
\{ 1,\,2,\,3,\,4,\,5,\,6 \} \). It is usually reasonable to suppose
that the die is fair, so that the six outcomes are equally likely.
# +END_exampletoo


# +BEGIN_exampletoo

Consider a standard deck of 52 cards. These are usually labeled with
the four /suits/: Clubs, Diamonds, Hearts, and Spades, and the 13
/ranks/: 2, 3, 4, ..., 10, Jack (J), Queen (Q), King (K), and Ace
(A). Depending on the game played, the Ace may be ranked below 2 or
above King.

Let the random experiment \(E\) consist of drawing exactly one card
from a well-shuffled deck, and let the outcome be the face of the
card. Define the events \( A = \{ \mbox{draw an Ace} \} \) and \( B =
\{ \mbox{draw a Club} \} \). Bear in mind: we are only drawing one
card.

Immediately we have \(\mathbb{P}(A) = 4/52\) since there are four Aces
in the deck; similarly, there are \(13\) Clubs which implies
\(\mathbb{P}(B) = 13/52\).

What is \(\mathbb{P}(A\cap B)\)? We realize that there is only one
card of the 52 which is an Ace and a Club at the same time, namely,
the Ace of Clubs. Therefore \(\mathbb{P}(A\cap B)=1/52\).

To find \(\mathbb{P}(A\cup B)\) we may use the above with the General
Addition Rule to get
\begin{eqnarray*}
\mathbb{P}(A\cup B) & = & \mathbb{P}(A) + \mathbb{P}(B) - \mathbb{P}(A \cap B),\\
 & = & 4/52 + 13/52 - 1/52,\\
 & = & 16/52.
\end{eqnarray*}

# +END_exampletoo


# +BEGIN_exampletoo

Staying with the deck of cards, let another random experiment be the
selection of a five card stud poker hand, where "five card stud"
means that we draw exactly five cards from the deck without
replacement, no more, and no less. It turns out that the sample space
\(S\) is so large and complicated that we will be obliged to settle
for the trivial description \( S = \{ \mbox{all possible 5 card hands}
\} \) for the time being. We will have a more precise description
later.

What is \(\mathbb{P}(\mbox{Royal Flush})\), or in other words,
\(\mathbb{P}(\mbox{A, K, Q, J, 10 all in the same suit})\)?

It should be clear that there are only four possible royal
flushes. Thus, if we could only count the number of outcomes in \(S\)
then we could simply divide four by that number and we would have our
answer under the equally likely model. This is the subject of Section
[[#sec-Methods-of-Counting]].

# +END_exampletoo

**** How to do it with \(\mathsf{R}\)

Probabilities are calculated in the =prob= package \cite{prob} with
the =prob= function.

Consider the experiment of drawing a card from a standard deck of
playing cards. Let's denote the probability space associated with the
experiment as =S=, and let the subsets =A= and =B= be defined by the
following:

#+BEGIN_SRC R :exports code :results silent
S <- cards(makespace = TRUE) 
A <- subset(S, suit == "Heart") 
B <- subset(S, rank %in% 7:9)
#+END_SRC 

Now it is easy to calculate 

#+BEGIN_SRC R :exports both :results output pp   
Prob(A) 
#+END_SRC 

#+RESULTS:
: [1] 0.25

Note that we can get the same answer with 

#+BEGIN_SRC R :exports both :results output pp   
Prob(S, suit == "Heart") 
#+END_SRC 

#+RESULTS:
: [1] 0.25

We also find =Prob(B) = 0.23= (listed here approximately, but
12/52 actually). Internally, the =prob= function operates by summing
the =probs= column of its argument. It will find subsets on-the-fly if
desired.

We have as yet glossed over the details. More specifically, =prob= has
three arguments: =x=, which is a probability space (or a subset of
one), =event=, which is a logical expression used to define a subset,
and =given=, which is described in Section [[#sec-Conditional-Probability]].

/WARNING/. The =event= argument is used to define a subset of =x=,
that is, the only outcomes used in the probability calculation will be
those that are elements of =x= and satisfy =event= simultaneously. In
other words, =Prob(x, event)= calculates

: Prob(intersect(x, subset(x, event)))

Consequently, =x= should be the entire probability space in the case
that =event= is non-null.

** Counting Methods
:PROPERTIES:
:CUSTOM_ID: sec-Methods-of-Counting
:END:

The equally-likely model is a convenient and popular way to analyze
random experiments. And when the equally likely model applies, finding
the probability of an event \(A\) amounts to nothing more than
counting the number of outcomes that \(A\) contains (together with the
number of events in \(S\)). Hence, to be a master of probability one
must be skilled at counting outcomes in events of all kinds.

#+ATTR_LATEX: :options [\textbf{The Multiplication Principle}]
#+BEGIN_prop
Suppose that an experiment is composed of two successive
steps. Further suppose that the first step may be performed in
\(n_{1}\) distinct ways while the second step may be performed in
\(n_{2}\) distinct ways. Then the experiment may be performed in
\(n_{1}n_{2}\) distinct ways.

More generally, if the experiment is composed of \(k\) successive
steps which may be performed in \(n_{1}\), \(n_{2}\), ..., \(n_{k}\)
distinct ways, respectively, then the experiment may be performed in
\(n_{1} n_{2} \cdots n_{k}\) distinct ways.
#+END_prop

# +BEGIN_exampletoo

We would like to order a pizza. It will be sure to have cheese (and
marinara sauce), but we may elect to add one or more of the following
five (5) available toppings: \[ \mbox{pepperoni, sausage, anchovies,
olives, and green peppers.}  \] How many distinct pizzas are possible?

There are many ways to approach the problem, but the quickest avenue
employs the Multiplication Principle directly. We will separate the
action of ordering the pizza into a series of stages. At the first
stage, we will decide whether or not to include pepperoni on the pizza
(two possibilities). At the next stage, we will decide whether or not
to include sausage on the pizza (again, two possibilities). We will
continue in this fashion until at last we will decide whether or not
to include green peppers on the pizza.

At each stage we will have had two options, or ways, to select a pizza
to be made. The Multiplication Principle says that we should multiply
the 2's to find the total number of possible pizzas: \(2 \cdot 2 \cdot
2 \cdot 2 \cdot 2 = 2^{5} = 32\).

# +END_exampletoo


# +BEGIN_exampletoo

We would like to buy a desktop computer to study statistics. We go to
a website to build our computer our way. Given a line of products we
have many options to customize our computer. In particular, there are
2 choices for a processor, 3 different operating systems, 4 levels of
memory, 4 hard drives of differing sizes, and 10 choices for a
monitor. How many possible types of computer must the company be
prepared to build? *Answer:* \(2 \cdot 3 \cdot 4 \cdot 4 \cdot 10 = 960\)
# +END_exampletoo

*** Ordered Samples

Imagine a bag with \(n\) distinguishable balls inside. Now shake up
the bag and select \(k\) balls at random. How many possible sequences
might we observe?

#+BEGIN_prop
The number of ways in which one may select an ordered sample of \(k\)
subjects from a population that has \(n\) distinguishable members is
- \(n^{k}\) if sampling is done with replacement,
- \(n(n-1)(n-2)\cdots(n-k+1)\) if sampling is done without
  replacement.
#+END_prop

Recall from calculus the notation for /factorials/: 
\begin{eqnarray*}
1! & = & 1,\\
2! & = & 2 \cdot 1 = 2,\\
3! & = & 3 \cdot 2 \cdot 1 = 6,\\
 & \vdots\\
n! & = & n(n - 1)(n - 2) \cdots 3 \cdot 2 \cdot 1.
\end{eqnarray*}

#+BEGIN_fact
The number of permutations of \(n\) elements is \(n!\).
#+END_fact

# +BEGIN_exampletoo

Take a coin and flip it 7 times. How many sequences of Heads and Tails
are possible? *Answer:* \(2^{7}=128\).
# +END_exampletoo


# +BEGIN_exampletoo

In a class of 20 students, we randomly select a class president, a
class vice-president, and a treasurer. How many ways can this be
done? *Answer:* \(20\cdot19\cdot18=6840\).
# +END_exampletoo


# +BEGIN_exampletoo

We rent five movies to watch over the span of two nights. We wish to
watch 3 movies on the first night. How many distinct sequences of 3
movies could we possibly watch? *Answer:* \(5\cdot4\cdot3=60\).
# +END_exampletoo

*** Unordered Samples

#+BEGIN_prop
The number of ways in which one may select an unordered sample of
\(k\) subjects from a population that has \(n\) distinguishable
members is
- \((n-1+k)!/[(n-1)!k!]\) if sampling is done with replacement,
- \(n!/[k!(n-k)!]\) if sampling is done without replacement.
#+END_prop

The quantity \(n!/[k!(n-k)!]\) is called a /binomial coefficient/ and
plays a special role in mathematics; it is denoted  
\begin{equation}
\label{eq-binomial-coefficient}
{n \choose k}=\frac{n!}{k!(n-k)!}
\end{equation}
and is read "\(n\) choose \(k\)".

# +BEGIN_exampletoo

You rent five movies to watch over the span of two nights, but only
wish to watch 3 movies the first night. Your friend, Fred, wishes to
borrow some movies to watch at his house on the first night. You owe
Fred a favor, and allow him to select 2 movies from the set of 5. How
many choices does Fred have? *Answer:* \({5 \choose 2}=10\).
# +END_exampletoo


# +BEGIN_exampletoo

Place 3 six-sided dice into a cup. Next, shake the cup well and pour
out the dice. How many distinct rolls are possible? *Answer:*
\((6-1+3)!/[(6-1)!3!]={8 \choose 5}=56\).
# +END_exampletoo

**** How to do it with \(\mathsf{R}\)

The factorial \(n!\) is computed with the command =factorial(n)= and
the binomial coefficient \({n \choose k}\) with the command
=choose(n,k)=.

The sample spaces we have computed so far have been relatively small,
and we can visually study them without much trouble. However, it is
/very/ easy to generate sample spaces that are prohibitively
large. And while \(\mathsf{R}\) is wonderful and powerful and does
almost everything except wash windows, even \(\mathsf{R}\) has limits
of which we should be mindful.

But we often do not need to actually generate the sample space; it
suffices to count the number of outcomes. The =nsamp= function will
calculate the number of rows in a sample space made by =urnsamples=
without actually devoting the memory resources necessary to generate
the space. The arguments are =n=, the number of (distinguishable)
objects in the urn, =k=, the sample size, and =replace=, =ordered=, as
above.

#+NAME: tab-Sampling-k-from-n
#+CAPTION[Sampling \(k\) from \(n\) objects with =urnsamples=]: Sampling \(k\) from \(n\) objects with =urnsamples=.
|                   | =ordered = TRUE=    | =ordered = FALSE=           |
|-------------------+---------------------+-----------------------------|
| =replace = TRUE=  | \(n^{k}\)           | \((n-1+k)! / [(n-1)!k!]\)   |
| =replace = FALSE= | \( n! / (n-k)! \)   | \( {n \choose k} \)         |
|-------------------+---------------------+-----------------------------|


# +BEGIN_exampletoo

We will compute the number of outcomes for each of the four
=urnsamples= examples that we saw in Example
[[exa-sample-urn-two-from-three]]. Recall that we took a sample of size two from an
urn with three distinguishable elements.
# +END_exampletoo


#+BEGIN_SRC R :exports both :results output pp   
nsamp(n=3, k=2, replace = TRUE, ordered = TRUE) 
nsamp(n=3, k=2, replace = FALSE, ordered = TRUE) 
nsamp(n=3, k=2, replace = FALSE, ordered = FALSE) 
nsamp(n=3, k=2, replace = TRUE, ordered = FALSE) 
#+END_SRC 

#+RESULTS:
: [1] 9
: [1] 6
: [1] 3
: [1] 6

Compare these answers with the length of the data frames generated above.

**** The Multiplication Principle

A benefit of =nsamp= is that it is /vectorized/ so that entering
vectors instead of numbers for =n=, =k=, =replace=, and =ordered=
results in a vector of corresponding answers. This becomes
particularly convenient for combinatorics problems.

# +BEGIN_exampletoo

There are 11 artists who each submit a portfolio containing 7
paintings for competition in an art exhibition. Unfortunately, the
gallery director only has space in the winners' section to accommodate
12 paintings in a row equally spread over three consecutive walls. The
director decides to give the first, second, and third place winners
each a wall to display the work of their choice. The walls boast 31
separate lighting options apiece. How many displays are possible?

*Answer:* The judges will pick 3 (ranked) winners out of 11 (with =rep
= FALSE=, =ord = TRUE=). Each artist will select 4 of his/her
paintings from 7 for display in a row (=rep = FALSE=, =ord = TRUE=),
and lastly, each of the 3 walls has 31 lighting possibilities (=rep =
TRUE=, =ord = TRUE=). These three numbers can be calculated quickly
with

#+BEGIN_SRC R :exports code :results silent
n <- c(11,7,31) 
k <- c(3,4,3) 
r <- c(FALSE,FALSE,TRUE) 
#+END_SRC 

#+BEGIN_SRC R :exports code :results silent
x <- nsamp(n, k, rep = r, ord = TRUE) 
#+END_SRC 

(Notice that =ordered= is always =TRUE=; =nsamp= will recycle
=ordered= and =replace= to the appropriate length.) By the
Multiplication Principle, the number of ways to complete the
experiment is the product of the entries of =x=:

#+BEGIN_SRC R :exports both :results output pp   
prod(x) 
#+END_SRC 

#+RESULTS:
: [1] 24774195600

Compare this with the some other ways to compute the same thing: 

#+BEGIN_SRC R :exports both :results output pp   
(11*10*9)*(7*6*5*4)*31^3 
#+END_SRC

#+RESULTS:
: [1] 24774195600

or alternatively 

#+BEGIN_SRC R :exports both :results output pp   
prod(9:11)*prod(4:7)*31^3 
#+END_SRC 

#+RESULTS:
: [1] 24774195600

or even 

#+BEGIN_SRC R :exports both :results output pp   
prod(factorial(c(11,7))/factorial(c(8,3)))*31^3 
#+END_SRC 

#+RESULTS:
: [1] 24774195600

# +END_exampletoo


As one can guess, in many of the standard counting problems there
aren't substantial savings in the amount of typing; it is about the
same using =nsamp= versus =factorial= and =choose=. But the virtue of
=nsamp= lies in its collecting the relevant counting formulas in a
one-stop shop. Ultimately, it is up to the user to choose the method
that works best for him/herself.

# +BEGIN_exampletoo

*The Birthday Problem.* Suppose that there are \(n\) people together
in a room. Each person announces the date of his/her birthday in
turn. The question is: what is the probability of at least one match?
If we let the event \(A\) represent \[ A = \{ \mbox{there is at least
one match}\}, \] then we are looking for \(\mathbb{P}(A)\), but as we
soon will see, it will be more convenient for us to calculate
\(\mathbb{P}(A^{c})\).

For starters we will ignore leap years and assume that there are only
365 days in a year. Second, we will assume that births are equally
distributed over the course of a year (which is not true due to all
sorts of complications such as hospital delivery schedules). See [[http://en.wikipedia.org/wiki/Birthday_problem][here]]
for more.

Let us next think about the sample space. There are 365 possibilities
for the first person's birthday, 365 possibilities for the second, and
so forth. The total number of possible birthday sequences is therefore
\(\#(S)=365^{n}\).

Now we will use the complementation trick we saw in Example [[exa-three-child-family]]. We realize that the only situation in which \(A\) does
/not/ occur is if there are /no/ matches among all people in the room,
that is, only when everybody's birthday is different, so \[
\mathbb{P}(A)=1-\mathbb{P}(A^{c})=1-\frac{\#(A^{c})}{\#(S)}, \] since
the outcomes are equally likely. Let us then suppose that there are no
matches. The first person has one of 365 possible birthdays. The
second person must not match the first, thus, the second person has
only 364 available birthdays from which to choose. Similarly, the
third person has only 363 possible birthdays, and so forth, until we
reach the \(n^{\mathrm{th}}\) person, who has only \(365-n+1\)
remaining possible days for a birthday. By the Multiplication
Principle, we have \(\#(A^{c})=365\cdot364\cdots(365-n+1)\), and
\begin{equation}
\mathbb{P}(A)=1-\frac{365\cdot364\cdots(365-n+1)}{365^{n}}=1-\frac{364}{365}\cdot\frac{363}{365}\cdots\frac{(365-n+1)}{365}.
\end{equation}
As a surprising consequence, consider this: how many people does it
take to be in the room so that the probability of at least one match
is at least 0.50? Clearly, if there is only \(n=1\) person in the room
then the probability of a match is zero, and when there are \(n=366\)
people in the room there is a 100% chance of a match (recall that we
are ignoring leap years). So how many people does it take so that
there is an equal chance of a match and no match?

When I have asked this question to students, the usual response is
"somewhere around \(n=180\) people" in the room. The reasoning seems
to be that in order to get a 50% chance of a match, there should be
50% of the available days to be occupied. The number of students in a
typical classroom is 25, so as a companion question I ask students to
estimate the probability of a match when there are \(n=25\) students
in the room. Common estimates are a 1%, or 0.5%, or even 0.1% chance
of a match. After they have given their estimates, we go around the
room and each student announces their birthday. More often than not,
we observe a match in the class, to the students' disbelief.

Students are usually surprised to hear that, using the formula above,
one needs only \(n=23\) students to have a greater than 50% chance of
at least one match. Figure [[fig-birthday]] shows a graph of the birthday
probabilities:
# +END_exampletoo

#+NAME: birthday
#+BEGIN_SRC R :exports results :results graphics :file fig/prob-birthday.ps
g <- Vectorize(pbirthday.ipsur)
plot(1:50, g(1:50), xlab = "Number of people in room", ylab = "Prob(at least one match)")
remove(g)
#+END_SRC

#+NAME: fig-birthday
#+CAPTION[The birthday problem]: \small The birthday problem. The horizontal line is at \(p=0.50\) and the vertical line is at \(n=23\).
#+ATTR_LaTeX: :width 0.9\textwidth :placement [ht!]
#+RESULTS: birthday
[[file:fig/prob-birthday.ps]]

**** How to do it with \(\mathsf{R}\)

We can make the plot in Figure [[fig-birthday]] with the following
sequence of commands.

#+BEGIN_SRC R :exports code :eval never
library(RcmdrPlugin.IPSUR)
g <- Vectorize(pbirthday.ipsur)
plot(1:50, g(1:50), xlab = "Number of people in room", 
  ylab = "Prob(at least one match)" )
abline(h = 0.5)
abline(v = 23, lty = 2)
remove(g)
#+END_SRC

There is a =Birthday problem= item in the =Probability= menu of
=RcmdrPlugin.IPSUR=. In the base \(\mathsf{R}\) version, one can
compute approximate probabilities for the more general case of
probabilities other than 1/2, for differing total number of days in
the year, and even for more than two matches.

** Conditional Probability
:PROPERTIES:
:CUSTOM_ID: sec-Conditional-Probability
:END:

Consider a full deck of 52 standard playing cards. Now select two
cards from the deck, in succession. Let \( A = \{ \mbox{first card
drawn is an Ace} \} \) and \( B = \{ \mbox{second card drawn is an
Ace} \} \). Since there are four Aces in the deck, it is natural to
assign \( \mathbb{P}(A) = 4/52 \). Suppose we look at the first
card. What now is the probability of \(B\)? Of course, the answer
depends on the value of the first card. If the first card is an Ace,
then the probability that the second also is an Ace should be \( 3/51
\), but if the first card is not an Ace, then the probability that the
second is an Ace should be \( 4/51 \). As notation for these two
situations we write 
\[ \mathbb{P}(B|A)=3/51,\quad
\mathbb{P}(B|A^{c})=4/51.  
\]

#+BEGIN_defn
The conditional probability of \(B\) given \(A\), denoted
\(\mathbb{P}(B|A)\), is defined by
\begin{equation}
\mathbb{P}(B|A)=\frac{\mathbb{P}(A\cap B)}{\mathbb{P}(A)},\quad \mbox{if }\mathbb{P}(A)>0.
\end{equation}
We will not be discussing a conditional probability of \(B\) given
\(A\) when \(\mathbb{P}(A)=0\), even though this theory exists, is
well developed, and forms the foundation for the study of stochastic
processes[fn-condprob].

[fn:fn-condprob] Conditional probability in this case is defined by
means of /conditional expectation/, a topic that is well beyond the
scope of this text. The interested reader should consult an advanced
text on probability theory, such as Billingsley, Resnick, or Ash
Dooleans-Dade.
#+END_defn

# +BEGIN_exampletoo

Toss a coin twice. The sample space is given by \(S=\{ HH,\ HT,\ TH,\
TT \} \). Let \(A= \{ \mbox{a head occurs} \} \) and \(B= \{ \mbox{a
head and tail occur} \} \). It should be clear that
\(\mathbb{P}(A)=3/4\), \(\mathbb{P}(B)=2/4\), and \(\mathbb{P}(A\cap
B)=2/4\). What now are the probabilities \(\mathbb{P}(A|B)\) and
\(\mathbb{P}(B|A)\)?  

\[ \mathbb{P}(A|B)=\frac{\mathbb{P}(A\cap
B)}{\mathbb{P}(B)}=\frac{2/4}{2/4}=1, \] 

in other words, once we know that a Head and Tail occur, we may be
certain that a Head occurs. Next

\[ 
\mathbb{P}(B|A)=\frac{\mathbb{P}(A\cap
B)}{\mathbb{P}(A)}=\frac{2/4}{3/4}=\frac{2}{3}, 
\] 

which means that given the information that a Head has occurred, we no
longer need to account for the outcome \(TT\), and the remaining three
outcomes are equally likely with exactly two outcomes lying in the set
\(B\).

# +END_exampletoo


# +BEGIN_exampletoo
<<exa-Toss-a-six-sided-die-twice>> Toss a six-sided die twice. The
sample space consists of all ordered pairs \((i,j)\) of the numbers
\(1,2,\ldots,6\), that is, \( S = \{ (1,1),\ (1,2),\ldots,(6,6) \}
\). We know from Section [[#sec-Methods-of-Counting]] that \( \# (S) =
6^{2} = 36 \). Let \( A = \{ \mbox{outcomes match} \} \) and \( B = \{
\mbox{sum of outcomes at least 8} \} \). The sample space may be
represented by a matrix:

#+NAME: twodiceAB
#+BEGIN_SRC R :exports both :results graphics :file fig/prob-twodiceAB.ps
A <- rolldie(2)
B <- subset(A, X1==X2)
C <- subset(A, X1+X2 > 7)
B$lab <- rep("X", dim(B)[1])
C$lab <- rep("O", dim(C)[1])
p <- ggplot(rbind(B, C), aes(x=X1, y=X2, label=lab))
p + geom_text(size = 15) + xlab("First roll") + ylab("Second roll")
#+END_SRC

#+NAME: fig-twodiceAB
#+CAPTION[Rolling two dice]: \small Rolling two dice. The outcomes in A are marked with X, the outcomes in B are marked with O.
#+ATTR_LaTeX: :width 0.9\textwidth :placement [ht!]
#+RESULTS: twodiceAB
[[file:fig/prob-twodiceAB.ps]]

The outcomes lying in the event \(A\) are marked with the symbol
"X", the outcomes falling in \(B\) are marked with "O", and the
outcomes in \(A\cap B\) are those where the letters overlap. Now it is
clear that \(\mathbb{P}(A)=6/36\), \(\mathbb{P}(B)=15/36\), and
\(\mathbb{P}(A\cap B)=3/36\).  Finally, 

\[
\mathbb{P}(A|B)=\frac{3/36}{15/36}=\frac{1}{5},\quad
\mathbb{P}(B|A)=\frac{3/36}{6/36}=\frac{1}{2}.  
\] 

Again, we see that given the knowledge that \(B\) occurred (the 15
outcomes in the upper right triangle), there are 3 of the 15 that fall
into the set \(A\), thus the probability is \(3/15\). Similarly, given
that \(A\) occurred (we are on the diagonal), there are 3 out of 6
outcomes that also fall in \(B\), thus, the probability of \(B\) given
\(A\) is 1/2.

# +END_exampletoo

*** How to do it with \(\mathsf{R}\)

Continuing with Example [[exa-Toss-a-six-sided-die-twice]], the first thing to do is set
up the probability space with the =rolldie= function.

#+BEGIN_SRC R :exports both :results output pp  
S <- rolldie(2, makespace = TRUE)  # assumes ELM
head(S)                            #  first few rows
#+END_SRC

#+RESULTS:
:   X1 X2      probs
: 1  1  1 0.02777778
: 2  2  1 0.02777778
: 3  3  1 0.02777778
: 4  4  1 0.02777778
: 5  5  1 0.02777778
: 6  6  1 0.02777778

Next we define the events

#+BEGIN_SRC R :exports code :results silent
A <- subset(S, X1 == X2)
B <- subset(S, X1 + X2 >= 8)
#+END_SRC

And now we are ready to calculate probabilities. To do conditional
probability, we use the =given= argument of the =prob= function:

#+BEGIN_SRC R :exports both :results output pp  
Prob(A, given = B)
Prob(B, given = A)
#+END_SRC

#+RESULTS:
: [1] 0.2
: [1] 0.5

Note that we do not actually need to define the events \(A\) and \(B\)
separately as long as we reference the original probability space
\(S\) as the first argument of the =prob= calculation:

#+BEGIN_SRC R :exports both :results output pp  
Prob(S, X1==X2, given = (X1 + X2 >= 8) )
Prob(S, X1+X2 >= 8, given = (X1==X2) )
#+END_SRC

#+RESULTS:
: [1] 0.2
: [1] 0.5

*** Properties and Rules

The following theorem establishes that conditional probabilities
behave just like regular probabilities when the conditioned event is
fixed.

#+BEGIN_thm
For any fixed event \(A\) with \(\mathbb{P}(A)>0\),
1. \( \mathbb{P} (B|A)\geq 0 \), for all events \( B \subset S\),
1. \( \mathbb{P} (S|A) = 1 \), and
1. If \(B_{1}\), \(B_{2}\), \(B_{3}\),... are disjoint events, then
  \begin{equation}
  \mathbb{P}\left(\left.\bigcup_{k=1}^{\infty}B_{k}\:\right|A\right)=\sum_{k=1}^{\infty}\mathbb{P}(B_{k}|A).
  \end{equation}
#+END_thm
In other words, \(\mathbb{P}(\cdot|A)\) is a legitimate probability
function. With this fact in mind, the following properties are
immediate:

#+BEGIN_prop
For any events \(A\), \(B\), and \(C\) with \(\mathbb{P}(A)>0\),
1. \( \mathbb{P} ( B^{c} | A ) = 1 - \mathbb{P} (B|A).\)
2. If \(B\subset C\) then \(\mathbb{P}(B|A)\leq\mathbb{P}(C|A)\).
3. \( \mathbb{P} [ ( B\cup C ) | A ] = \mathbb{P} (B|A) +
   \mathbb{P}(C|A) - \mathbb{P} [ (B \cap C|A) ].\)
4. *The Multiplication Rule.* For any two events \(A\) and \(B\),
   \begin{equation}
   \label{eq-multiplication-rule-short}
   \mathbb{P}(A\cap B)=\mathbb{P}(A)\mathbb{P}(B|A).
   \end{equation}
   And more generally, for events \(A_{1}\), \(A_{2}\), \(A_{3}\),...,
   \(A_{n}\),
   \begin{equation}
   \label{eq-multiplication-rule-long}
   \mathbb{P}(A_{1}\cap A_{2}\cap\cdots\cap A_{n})=\mathbb{P}(A_{1})\mathbb{P}(A_{2}|A_{1})\cdots\mathbb{P}(A_{n}|A_{1}\cap A_{2}\cap\cdots\cap A_{n-1}).
   \end{equation}
#+END_prop
The Multiplication Rule is very important because it allows us to find
probabilities in random experiments that have a sequential structure,
as the next example shows.

# +BEGIN_exampletoo
<<exa-two-cards-both-aces>> At the beginning of the section we drew
two cards from a standard playing deck. Now we may answer our original
question, what is \(\mathbb{P}(\mbox{both Aces})\)?  \[
\mathbb{P}(\mbox{both Aces})=\mathbb{P}(A\cap
B)=\mathbb{P}(A)\mathbb{P}(B|A)=\frac{4}{52}\cdot\frac{3}{51}\approx0.00452.
\]
# +END_exampletoo

**** How to do it with \(\mathsf{R}\)
:PROPERTIES:
:CUSTOM_ID: sub-howto-ps-objects
:END:

Continuing Example [[exa-two-cards-both-aces]], we set up the probability
space by way of a three step process. First we employ the =cards=
function to get a data frame =L= with two columns: =rank= and
=suit=. Both columns are stored internally as factors with 13 and 4
levels, respectively.

Next we sample two cards randomly from the =L= data frame by way of
the =urnsamples= function. It returns a list =M= which contains all
possible pairs of rows from =L= (there are =choose(52,2)= of
them). The sample space for this experiment is exactly the list =M=.

At long last we associate a probability model with the sample
space. This is right down the =probspace= function's alley. It assumes
the equally likely model by default. We call this result =N= which is
an object of class =ps= -- short for "probability space".

But do not be intimidated. The object =N= is nothing more than a list
with two elements: =outcomes= and =probs=. The =outcomes= element is
itself just another list, with =choose(52,2)= entries, each one a data
frame with two rows which correspond to the pair of cards chosen. The
=probs= element is just a vector with =choose(52,2)= entries all the
same: =1/choose(52,2)=.

Putting all of this together we do 

#+BEGIN_SRC R :exports code :results output pp :cache yes
L <- cards()
M <- urnsamples(L, size = 2)
N <- probspace(M)
N[[1]][[1]];  N$probs[1]
#+END_SRC

#+RESULTS[8abfbbaacd3af28ad892eff896d2918972f4d9bd]:
:   rank suit
: 1    2 Club
: 2    3 Club
: [1] 0.0007541478

Now that we have the probability space =N= we are ready to do some
probability. We use the =prob= function, just like before. The only
trick is to specify the event of interest correctly, and recall that
we were interested in \(\mathbb{P}(\mbox{both Aces})\). But if the
cards are both Aces then the =rank= of both cards should be =A=, which
sounds like a job for the =all= function:

#+BEGIN_SRC R :exports both :results output pp
Prob(N, all(rank == "A"))
#+END_SRC

#+RESULTS:
: [1] 0.004524887

Note that this value matches what we found in Example
[[exa-two-cards-both-aces]], above. We could calculate all sorts of
probabilities at this point; we are limited only by the complexity of
the event's computer representation.


# +BEGIN_exampletoo
<<exa-urn-7-red-3-green>> Consider an urn with 10 balls inside, 7 of
which are red and 3 of which are green. Select 3 balls successively
from the urn. Let \( A = \{ 1^{\mathrm{st}} \mbox{ ball is red} \} \),
\( B = \{ 2^{\mathrm{nd}} \mbox{ ball is red} \} \), and \( C = \{
3^{\mathrm{rd}} \mbox{ ball is red} \} \). Then \[
\mathbb{P}(\mbox{all 3 balls are red})=\mathbb{P}(A\cap B\cap
C)=\frac{7}{10}\cdot\frac{6}{9}\cdot\frac{5}{8}\approx 0.2917.  \]
# +END_exampletoo

**** How to do it with \(\mathsf{R}\)

Example [[exa-urn-7-red-3-green]] is similar to Example
[[exa-two-cards-both-aces]], but it is even easier. We need to set up
an urn (vector =L=) to hold the balls, we sample from =L= to get the
sample space (data frame =M=), and we associate a probability vector
(column =probs=) with the outcomes (rows of =M=) of the sample
space. The final result is a probability space (an ordinary data frame
=N=).

It is easier for us this time because our urn is a vector instead of a
=cards()= data frame. Before there were two dimensions of information
associated with the outcomes (rank and suit) but presently we have
only one dimension (color).

#+BEGIN_SRC R :exports code :results silent
L <- rep(c("red","green"), times = c(7,3))
M <- urnsamples(L, size = 3, replace = FALSE, ordered = TRUE)
N <- probspace(M)
#+END_SRC

Now let us think about how to set up the event \(\{ \mbox{all 3 balls
are red}\} \). Rows of =N= that satisfy this condition have
\(\mathtt{X1=="red"\ \&\ X2=="red"\ \&\ X3=="red"}\), but there must
be an easier way. Indeed, there is. The =isrep= function (short for
"is repeated") in the =prob= package was written for this
purpose. The command =isrep(N,"red",3)= will test each row of =N= to
see whether the value =red= appears =3= times. The result is exactly
what we need to define an event with the =prob= function. Observe

#+BEGIN_SRC R :exports both :results output pp  
Prob(N, isrep(N, "red", 3))
#+END_SRC

#+RESULTS:
: [1] 0.2916667

Note that this answer matches what we found in Example
[[exa-urn-7-red-3-green]]. Now let us try some other probability
questions. What is the probability of getting two =red='s?

#+BEGIN_SRC R :exports both :results output pp  
Prob(N, isrep(N, "red", 2))
#+END_SRC

#+RESULTS:
: [1] 0.525

Note that the exact value is \(21/40\); we will learn a quick way to
compute this in Section [[#sec-other-discrete-distributions]]. What is the
probability of observing =red=, then =green=, then =red=?

#+BEGIN_SRC R :exports both :results output pp  
Prob(N, isin(N, c("red","green","red"), ordered = TRUE))
#+END_SRC

#+RESULTS:
: [1] 0.175

Note that the exact value is \(7/40\) (do it with the Multiplication
Rule). What is the probability of observing =red=, =green=, and =red=,
in no particular order?

#+BEGIN_SRC R :exports both :results output pp  
Prob(N, isin(N, c("red","green","red")))
#+END_SRC

#+RESULTS:
: [1] 0.525

We already knew this. It is the probability of observing two =red='s,
above.

# +BEGIN_exampletoo

Consider two urns, the first with 5 red balls and 3 green balls, and
the second with 2 red balls and 6 green balls. Your friend randomly
selects one ball from the first urn and transfers it to the second
urn, without disclosing the color of the ball. You select one ball
from the second urn. What is the probability that the selected ball is
red? Let \( A = \{ \mbox{transferred ball is red} \} \) and \( B = \{
\mbox{selected ball is red} \} \). Write
\begin{align*}
B & =S\cap B\\
 & =(A\cup A^{c})\cap B\\
 & =(A\cap B)\cup(A^{c}\cap B)
\end{align*}
and notice that \(A\cap B\) and \(A^{c}\cap B\) are disjoint. Therefore
\begin{align*}
\mathbb{P}(B) & =\mathbb{P}(A\cap B)+\mathbb{P}(A^{c}\cap B)\\
 & =\mathbb{P}(A)\mathbb{P}(B|A)+\mathbb{P}(A^{c})\mathbb{P}(B|A^{c})\\
 & =\frac{5}{8}\cdot\frac{3}{9}+\frac{3}{8}\cdot\frac{2}{9}\\
 & =\frac{21}{72}\ 
\end{align*}
(which is 7/24 in lowest terms).

# +END_exampletoo


# +BEGIN_exampletoo

We saw the =RcmdrTestDrive= data set in Chapter [[#cha-introduction-to-R]]
in which a two-way table of the smoking status versus the gender was

#+BEGIN_SRC R :exports both :results output pp
library(RcmdrPlugin.IPSUR)
data(RcmdrTestDrive)  
.Table <- xtabs( ~ smoking + gender, data = RcmdrTestDrive)
addmargins(.Table) # Table with marginal distributions
#+END_SRC

#+RESULTS:
:            gender
: smoking     Female Male Sum
:   Nonsmoker     61   75 136
:   Smoker         9   23  32
:   Sum           70   98 168

If one person were selected at random from the data set, then we see
from the two-way table that \(\mathbb{P}(\mbox{Female})=70/168\) and
\(\mathbb{P}(\mbox{Smoker})=32/168\). Now suppose that one of the
subjects quits smoking, but we do not know the person's gender. If we
now select one nonsmoker at random, what would be
\(\mathbb{P}(\mbox{Female})\)? This example is just like the last
example, but with different labels. Let \( A = \{ \mbox{the quitter is
a female} \} \) and \( B = \{ \mbox{selected nonsmoker is a female} \}
\). Write

\begin{align*}
B & =S\cap B\\
 & =(A\cup A^{c})\cap B\\
 & =(A\cap B)\cup(A^{c}\cap B)
\end{align*}
and notice that \(A\cap B\) and \(A^{c}\cap B\) are disjoint. Therefore
\begin{align*}
\mathbb{P}(B) & =\mathbb{P}(A\cap B)+\mathbb{P}(A^{c}\cap B),\\
 & =\mathbb{P}(A)\mathbb{P}(B|A)+\mathbb{P}(A^{c})\mathbb{P}(B|A^{c}),\\
 & =\frac{9}{32}\cdot\frac{62}{137}+\frac{23}{32}\cdot\frac{76}{137},\\
 & =\frac{2306}{4384},
\end{align*}
(which is 1153/2192 in lowest terms).

# +END_exampletoo

Using the same reasoning, we can return to the example from the
beginning of the section and show that 
\[ \mathbb{P}(\{ \mbox{second
card is an Ace} \} )=4/52.  
\]
 
** Independent Events
:PROPERTIES:
:CUSTOM_ID: sec-Independent-Events
:END:

Toss a coin twice. The sample space is \(S= \{ HH,\ HT,\ TH,\ TT \}
\). We know that \(\mathbb{P}(1^{\mathrm{st}}\mbox{ toss is }H)=2/4\),
\(\mathbb{P}(2^{\mathrm{nd}}\mbox{ toss is }H)=2/4\), and
\(\mathbb{P}(\mbox{both }H)=1/4\). Then

\begin{align*} 
\mathbb{P}(2^{\mathrm{nd}}\mbox{ toss is }H\ \vert \ 1^{\mathrm{st}}\mbox{ toss is }H) & =\frac{\mathbb{P}(\mbox{both }H)}{\mathbb{P}(1^{\mathrm{st}}\mbox{ toss is }H)}, \\
 & =\frac{1/4}{2/4},\\
 & =\mathbb{P}(2^{\mathrm{nd}}\mbox{ toss is }H).
\end{align*}

Intuitively, this means that the information that the first toss is
\(H\) has no bearing on the probability that the second toss is
\(H\). The coin does not remember the result of the first toss.

#+BEGIN_defn
Events \(A\) and \(B\) are said to be /independent/ if 
\begin{equation}
\mathbb{P}(A\cap B)=\mathbb{P}(A)\mathbb{P}(B).
\end{equation}
Otherwise, the events are said to be /dependent/. 
#+END_defn

The connection with the above example stems from the following. We
know from Section [[#sec-Conditional-Probability]] that when
\(\mathbb{P}(B)>0\) we may write
\begin{equation}
\mathbb{P}(A|B)=\frac{\mathbb{P}(A\cap B)}{\mathbb{P}(B)}.
\end{equation}

In the case that \(A\) and \(B\) are independent, the numerator of the
fraction factors so that \(\mathbb{P}(B)\) cancels with the result:
\begin{equation}
\mathbb{P}(A|B)=\mathbb{P}(A)\mbox{ when \(A\), \(B\) are independent.}
\end{equation}

The interpretation in the case of independence is that the information
that the event \(B\) occurred does not influence the probability of
the event \(A\) occurring. Similarly,
\(\mathbb{P}(B|A)=\mathbb{P}(B)\), and so the occurrence of the event
\(A\) likewise does not affect the probability of event \(B\). It may
seem more natural to define \(A\) and \(B\) to be independent when
\(\mathbb{P}(A|B)=\mathbb{P}(A)\); however, the conditional
probability \(\mathbb{P}(A|B)\) is only defined when
\(\mathbb{P}(B)>0\). Our definition is not limited by this
restriction. It can be shown that when \(\mathbb{P}(A),\
\mathbb{P}(B)>0\) the two notions of independence are equivalent.

#+BEGIN_prop
If the events \(A\) and \(B\) are independent then
- \(A\) and \(B^{c}\) are independent,
- \(A^{c}\) and \(B\) are independent,
- \(A^{c}\) and \(B^{c}\) are independent.
#+END_prop

#+BEGIN_proof
Suppose that \(A\) and \(B\) are independent. We will show the second
one; the others are similar. We need to show that \[
\mathbb{P}(A^{c}\cap B)=\mathbb{P}(A^{c})\mathbb{P}(B).  \] To this
end, note that the Multiplication Rule, Equation
\eqref{eq-multiplication-rule-short} implies
\begin{eqnarray*}
\mathbb{P}(A^{c}\cap B) & = & \mathbb{P}(B)\mathbb{P}(A^{c}|B),\\
 & = & \mathbb{P}(B)[1-\mathbb{P}(A|B)],\\
 & = & \mathbb{P}(B)\mathbb{P}(A^{c}).
\end{eqnarray*}
#+END_proof

#+BEGIN_defn
The events \(A\), \(B\), and \(C\) are /mutually independent/ if the
following four conditions are met:
\begin{eqnarray*}
\mathbb{P}(A\cap B) & = & \mathbb{P}(A)\mathbb{P}(B),\\
\mathbb{P}(A\cap C) & = & \mathbb{P}(A)\mathbb{P}(C),\\
\mathbb{P}(B\cap C) & = & \mathbb{P}(B)\mathbb{P}(C),
\end{eqnarray*}
and
\[
\mathbb{P}(A\cap B\cap C)=\mathbb{P}(A)\mathbb{P}(B)\mathbb{P}(C).
\]
If only the first three conditions hold then \(A\), \(B\), and \(C\)
are said to be independent /pairwise/. Note that pairwise independence
is not the same as mutual independence when the number of events is
larger than two.
#+END_defn

We can now deduce the pattern for \(n\) events, \(n>3\). The events
will be mutually independent only if they satisfy the product equality
pairwise, then in groups of three, in groups of four, and so forth, up
to all \(n\) events at once. For \(n\) events, there will be
\(2^{n}-n-1\) equations that must be satisfied (see Exercise [[xca-numb-cond-indep]]). Although these requirements for a set of events to
be mutually independent may seem stringent, the good news is that for
most of the situations considered in this book the conditions will all
be met (or at least we will suppose that they are).

# +BEGIN_exampletoo
<<exa-toss-ten-coins>> Toss ten coins. What is the probability of
observing at least one Head? Answer: Let \(A_{i}= \{ \mbox{the
}i^{\mathrm{th}}\mbox{ coin shows }H \} ,\
i=1,2,\ldots,10\). Supposing that we toss the coins in such a way that
they do not interfere with each other, this is one of the situations
where all of the \(A_{i}\) may be considered mutually independent due
to the nature of the tossing. Of course, the only way that there will
not be at least one Head showing is if all tosses are
Tails. Therefore,
\begin{align*}
\mathbb{P}(\mbox{at least one }H) & =1-\mathbb{P}(\mbox{all }T),\\
 & =1-\mathbb{P}(A_{1}^{c}\cap A_{2}^{c}\cap\cdots\cap A_{10}^{c}),\\
 & =1-\mathbb{P}(A_{1}^{c})\mathbb{P}(A_{2}^{c})\cdots\mathbb{P}(A_{10}^{c}),\\
 & =1-\left(\frac{1}{2}\right)^{10},
\end{align*}
which is approximately \(0.9990234\).

# +END_exampletoo

*** How to do it with \(\mathsf{R}\)

# +BEGIN_exampletoo

Toss ten coins. What is the probability of observing at least one
Head?

#+BEGIN_SRC R :exports both :results output pp  
S <- tosscoin(10, makespace = TRUE)
A <- subset(S, isrep(S, vals = "T", nrep = 10))
1 - Prob(A)
#+END_SRC

#+RESULTS:
: [1] 0.9990234

Compare this answer to what we got in Example [[exa-toss-ten-coins]].

# +END_exampletoo

*** Independent, Repeated Experiments

Generalizing from above it is common to repeat a certain experiment
multiple times under identical conditions and in an independent
manner. We have seen many examples of this already: tossing a coin
repeatedly, rolling a die or dice, /etc/.

The =iidspace= function was designed specifically for this
situation. It has three arguments: =x=, which is a vector of outcomes,
=ntrials=, which is an integer telling how many times to repeat the
experiment, and =probs= to specify the probabilities of the outcomes
of =x= in a single trial.

# +BEGIN_exampletoo

*An unbalanced coin* (continued, see Example [[exa-unbalanced-coin]]). It was
easy enough to set up the probability space for one unbalanced toss,
however, the situation becomes more complicated when there are many
tosses involved. Clearly, the outcome \(HHH\) should not have the same
probability as \(TTT\), which should again not have the same
probability as \(HTH\). At the same time, there is symmetry in the
experiment in that the coin does not remember the face it shows from
toss to toss, and it is easy enough to toss the coin in a similar way
repeatedly.

We may represent tossing our unbalanced coin three times with the following: 

#+BEGIN_SRC R :exports both :results output pp  
iidspace(c("H","T"), ntrials = 3, probs = c(0.7, 0.3)) 
#+END_SRC 

#+RESULTS:
:   X1 X2 X3 probs
: 1  H  H  H 0.343
: 2  T  H  H 0.147
: 3  H  T  H 0.147
: 4  T  T  H 0.063
: 5  H  H  T 0.147
: 6  T  H  T 0.063
: 7  H  T  T 0.063
: 8  T  T  T 0.027

As expected, the outcome \(HHH\) has the largest probability, while
\(TTT\) has the smallest. (Since the trials are independent,
\(\mathbb{P}(HHH)=0.7^{3}\) and \(\mathbb{P}(TTT)=0.3^{3}\), /etc/.)
Note that the result of the function call is a probability space, not
a sample space (which we could construct already with the =tosscoin=
or =urnsamples= functions). The same procedure could be used to model
an unbalanced die or any other experiment that may be represented with
a vector of possible outcomes.

# +END_exampletoo


Note that =iidspace= will assume =x= has equally likely outcomes if no
=probs= argument is specified. Also note that the argument =x= is a
/vector/, not a data frame. Something like 
: iidspace(tosscoin(1),...)
would give an error.

** Bayes' Rule
:PROPERTIES:
:CUSTOM_ID: sec-Bayes-Rule
:END:

We mentioned the subjective view of probability in Section
[[#sec-Interpreting-Probabilities]]. In this section we introduce a rule that
allows us to update our probabilities when new information becomes
available.

#+ATTR_LATEX: :options [\textbf{Bayes' Rule}]
#+BEGIN_thm
Let \(B_{1}\), \(B_{2}\), ..., \(B_{n}\) be mutually exclusive and
exhaustive and let \(A\) be an event with \(\mathbb{P}(A)>0\). Then
\begin{equation}
\label{eq-bayes-rule}
\mathbb{P}(B_{k}|A)=\frac{\mathbb{P}(B_{k})\mathbb{P}(A|B_{k})}{\sum_{i=1}^{n}\mathbb{P}(B_{i})\mathbb{P}(A|B_{i})},\quad k=1,2,\ldots,n.
\end{equation}
#+END_thm

#+BEGIN_proof
The proof follows from looking at \(\mathbb{P}(B_{k}\cap A)\) in two
different ways. For simplicity, suppose that \(P(B_{k})>0\) for all
\(k\). Then \[ \mathbb{P}(A)\mathbb{P}(B_{k}|A)=\mathbb{P}(B_{k}\cap
A)=\mathbb{P}(B_{k})\mathbb{P}(A|B_{k}).  \] Since \(\mathbb{P}(A)>0\)
we may divide through to obtain \[
\mathbb{P}(B_{k}|A)=\frac{\mathbb{P}(B_{k})\mathbb{P}(A|B_{k})}{\mathbb{P}(A)}.
\] Now remembering that \(\{ B_{k} \}\) is a partition, the Theorem of
Total Probability (Equation \eqref{eq-theorem-total-probability} )
gives the denominator of the last expression to be \[
\mathbb{P}(A)=\sum_{k=1}^{n}\mathbb{P}(B_{k}\cap
A)=\sum_{k=1}^{n}\mathbb{P}(B_{k})\mathbb{P}(A|B_{k}).  \]
#+END_proof

What does it mean? Usually in applications we are given (or know) /a
priori/ probabilities \(\mathbb{P}(B_{k})\). We go out and collect
some data, which we represent by the event \(A\). We want to know: how
do we *update* \(\mathbb{P}(B_{k})\) to \(\mathbb{P}(B_{k}|A)\)? The
answer: Bayes' Rule.

# +BEGIN_exampletoo
<<exa-misfiling-assistants>> *Misfiling Assistants.* In this problem,
there are three assistants working at a company: Moe, Larry, and
Curly. Their primary job duty is to file paperwork in the filing
cabinet when papers become available. The three assistants have
different work schedules:

#+CAPTION[Misfiling assistants: workload]: Misfiling assistants: workload.
|          | Moe      | Larry    | Curly    |
|----------+----------+----------+----------|
| Workload | \(60\%\) | \(30\%\) | \(10\%\) |

That is, Moe works 60% of the time, Larry works 30% of the time, and
Curly does the remaining 10%, and they file documents at approximately
the same speed. Suppose a person were to select one of the documents
from the cabinet at random. Let \(M\) be the event \[ M= \{ \mbox{Moe
filed the document} \} \] and let \(L\) and \(C\) be the events that
Larry and Curly, respectively, filed the document. What are these
events' respective probabilities? In the absence of additional
information, reasonable prior probabilities would just be

#+CAPTION[Misfiling assistants: prior]: Misfiling assistants: prior.
|                   | Moe | Larry | Curly |
|-------------------+-----+-------+-------|
| Prior Probability | 0.6 |   0.3 |   0.1 |

Now, the boss comes in one day, opens up the file cabinet, and selects
a file at random. The boss discovers that the file has been
misplaced. The boss is so angry at the mistake that (s)he threatens to
fire the one who erred. The question is: who misplaced the file?

The boss decides to use probability to decide, and walks straight to
the workload schedule. (S)he reasons that, since the three employees
work at the same speed, the probability that a randomly selected file
would have been filed by each one would be proportional to his
workload. The boss notifies *Moe* that he has until the end of the day
to empty his desk.

But Moe argues in his defense that the boss has ignored additional
information. Moe's likelihood of having misfiled a document is smaller
than Larry's and Curly's, since he is a diligent worker who pays close
attention to his work. Moe admits that he works longer than the
others, but he doesn't make as many mistakes as they do. Thus, Moe
recommends that -- before making a decision -- the boss should update
the probability (initially based on workload alone) to incorporate the
likelihood of having observed a misfiled document.

And, as it turns out, the boss has information about Moe, Larry, and
Curly's filing accuracy in the past (due to historical performance
evaluations). The performance information may be represented by the
following table:

#+CAPTION[Misfiling assistants: misfile rate]: Misfiling assistants: misfile rate.
|              |   Moe | Larry | Curly |
|--------------+-------+-------+-------|
| Misfile Rate | 0.003 | 0.007 | 0.010 |

In other words, on the average, Moe misfiles 0.3% of the documents he
is supposed to file. Notice that Moe was correct: he is the most
accurate filer, followed by Larry, and lastly Curly. If the boss were
to make a decision based only on the worker's overall accuracy,
then *Curly* should get the axe. But Curly hears this and interjects
that he only works a short period during the day, and consequently
makes mistakes only very rarely; there is only the tiniest chance that
he misfiled this particular document.

The boss would like to use this updated information to update the
probabilities for the three assistants, that is, (s)he wants to use
the additional likelihood that the document was misfiled to update
his/her beliefs about the likely culprit. Let \(A\) be the event that
a document is misfiled. What the boss would like to know are the three
probabilities
\[
\mathbb{P}(M|A),\mbox{ }\mathbb{P}(L|A),\mbox{ and }\mathbb{P}(C|A).
\]
We will show the calculation for \(\mathbb{P}(M|A)\), the other two
cases being similar. We use Bayes' Rule in the form
\[
\mathbb{P}(M|A)=\frac{\mathbb{P}(M\cap A)}{\mathbb{P}(A)}.
\]
Let's try to find \(\mathbb{P}(M\cap A)\), which is just
\(\mathbb{P}(M)\cdot\mathbb{P}(A|M)\) by the Multiplication Rule. We
already know \(\mathbb{P}(M)=0.6\) and \(\mathbb{P}(A|M)\) is nothing
more than Moe's misfile rate, given above to be
\(\mathbb{P}(A|M)=0.003\). Thus, we compute
\[
\mathbb{P}(M\cap A)=(0.6)(0.003)=0.0018.
\]
Using the same procedure we may calculate
\[
\mathbb{P}(L \cap A)=0.0021\mbox{ and }\mathbb{P}(C \cap A)=0.0010.
\]

Now let's find the denominator, \(\mathbb{P}(A)\). The key here is the
notion that if a file is misplaced, then either Moe or Larry or Curly
must have filed it; there is no one else around to do the
misfiling. Further, these possibilities are mutually exclusive. We may
use the Theorem of Total Probability
\eqref{eq-theorem-total-probability} to write \[
\mathbb{P}(A)=\mathbb{P}(A\cap M)+\mathbb{P}(A\cap L)+\mathbb{P}(A\cap
C).  \] Luckily, we have computed these above. Thus \[
\mathbb{P}(A)=0.0018+0.0021+0.0010=0.0049.  \] Therefore, Bayes' Rule
yields \[ \mathbb{P}(M|A)=\frac{0.0018}{0.0049}\approx0.37.  \] This
last quantity is called the posterior probability that Moe misfiled
the document, since it incorporates the observed data that a randomly
selected file was misplaced (which is governed by the misfile
rate). We can use the same argument to calculate

#+CAPTION[Misfiling assistants: posterior]: Misfiling assistants: posterior.
|                       | Moe           | Larry         | Curly         |
|-----------------------+---------------+---------------+---------------|
| Posterior Probability | \(\approx0.37\) | \(\approx0.43\) | \(\approx0.20\) |

The conclusion: *Larry* gets the axe. What is happening is an
intricate interplay between the time on the job and the misfile
rate. It is not obvious who the winner (or in this case, loser) will
be, and the statistician needs to consult Bayes' Rule to determine the
best course of action.
# +END_exampletoo


# +BEGIN_exampletoo
<<exa-misfiling-assistants-multiple>> Suppose the boss gets a change
of heart and does not fire anybody. But the next day (s)he randomly
selects another file and again finds it to be misplaced. To decide
whom to fire now, the boss would use the same procedure, with one
small change. (S)he would not use the prior probabilities 60%, 30%,
and 10%; those are old news. Instead, she would replace the prior
probabilities with the posterior probabilities just calculated. After
the math she will have new posterior probabilities, updated even more
from the day before.

In this way, probabilities found by Bayes' rule are always on the
cutting edge, always updated with respect to the best information
available at the time.
# +END_exampletoo

*** How to do it with \(\mathsf{R}\)

There are not any special functions for Bayes' Rule in the =prob=
package \cite{prob}, but problems like the ones above are easy enough
to do by hand.

# +BEGIN_exampletoo

*Misfiling assistants* (continued from Example [[exa-misfiling-assistants]]). We store the prior probabilities and the likelihoods in
vectors and go to town.

#+BEGIN_SRC R :exports both :results output pp  
prior <- c(0.6, 0.3, 0.1)
like <- c(0.003, 0.007, 0.010)
post <- prior * like
post / sum(post)
#+END_SRC

#+RESULTS:
: [1] 0.3673469 0.4285714 0.2040816

# +END_exampletoo

Compare these answers with what we got in Example [[exa-misfiling-assistants]]. We would replace =prior= with =post= in a future
calculation. We could raise =like= to a power to see how the posterior
is affected by future document mistakes. (Do you see why? Think back
to Section [[#sec-Independent-Events]].)


# +BEGIN_exampletoo

Let us incorporate the posterior probability (=post=) information from
the last example and suppose that the assistants misfile seven more
documents. Using Bayes' Rule, what would the new posterior
probabilities be?

#+BEGIN_SRC R :exports both :results output pp  
newprior <- post
post <- newprior * like^7
post / sum(post)
#+END_SRC

#+RESULTS:
: [1] 0.0003355044 0.1473949328 0.8522695627

We see that the individual with the highest probability of having
misfiled all eight documents given the observed data is no longer
Larry, but Curly.
# +END_exampletoo

There are two important points. First, we did not divide =post= by the
sum of its entries until the very last step; we do not need to
calculate it, and it will save us computing time to postpone
normalization until absolutely necessary, namely, until we finally
want to interpret them as probabilities.

Second, the reader might be wondering what the boss would get if (s)he
skipped the intermediate step of calculating the posterior after only
one misfiled document. What if she started from the /original/ prior,
then observed eight misfiled documents, and calculated the posterior?
What would she get? It must be the same answer, of course.

#+BEGIN_SRC R :exports both :results output pp  
fastpost <- prior * like^8
fastpost / sum(fastpost)
#+END_SRC

#+RESULTS:
: [1] 0.0003355044 0.1473949328 0.8522695627

Compare this to what we got in Example [[exa-misfiling-assistants-multiple]].

** Random Variables
:PROPERTIES:
:CUSTOM_ID: sec-Random-Variables
:END:

We already know about experiments, sample spaces, and events. In this
section, we are interested in a /number/ that is associated with the
experiment. We conduct a random experiment \(E\) and after learning
the outcome \(\omega\) in \(S\) we calculate a number \(X\). That is,
to each outcome \(\omega\) in the sample space we associate a number
\(X(\omega)=x\).

#+BEGIN_defn
A /random variable/ \(X\) is a function \(X:S\to\mathbb{R}\) that
associates to each outcome \(\omega\in S\) exactly one number
\(X(\omega)=x\).
#+END_defn

We usually denote random variables by uppercase letters such as \(X\),
\(Y\), and \(Z\), and we denote their observed values by lowercase
letters \(x\), \(y\), and \(z\). Just as \(S\) is the set of all
possible outcomes of \(E\), we call the set of all possible values of
\(X\) the /support/ of \(X\) and denote it by \(S_{X}\).

# +BEGIN_exampletoo

Let \(E\) be the experiment of flipping a coin twice. We have seen
that the sample space is \( S = \{ HH,\ HT,\ TH,\ TT \} \). Now define
the random variable \(X = \mbox{the number of heads}\). That is, for
example, \(X(HH)=2\), while \(X(HT)=1\). We may make a table of the
possibilities:

#+NAME: tab-flip-coin-twice
#+CAPTION[Flipping a coin twice]: Flipping a coin twice.
| \(\omega\in S\) | \(HH\) | \(HT\) | \(TH\) | \(TT\) |
|-----------------+--------+--------+--------+--------|
| \(X(\omega)=x\) |      2 |      1 |      1 |      0 |

Taking a look at the second row of the table, we see that the support
of \(X\) -- the set of all numbers that \(X\) assumes -- would be \(
S_{X}= \{ 0,1,2 \} \).
# +END_exampletoo


# +BEGIN_exampletoo

Let \(E\) be the experiment of flipping a coin repeatedly until
observing a Head. The sample space would be \(S= \{ H,\ TH,\ TTH,\
TTTH,\ \ldots \} \). Now define the random variable \(Y=\mbox{the
number of Tails before the first head}\). Then the support of \(Y\)
would be \( S_{Y}= \{ 0,1,2,\ldots \} \).
# +END_exampletoo


# +BEGIN_exampletoo

Let \(E\) be the experiment of tossing a coin in the air, and define
the random variable \( Z = \mbox{the time (in seconds) until the coin
hits the ground} \). In this case, the sample space is inconvenient to
describe. Yet the support of \(Z\) would be \((0,\infty)\). Of course,
it is reasonable to suppose that the coin will return to Earth in a
short amount of time; in practice, the set \((0,\infty)\) is
admittedly too large. However, we will find that in many circumstances
it is mathematically convenient to study the extended set rather than
a restricted one.
# +END_exampletoo


There are important differences between the supports of \(X\), \(Y\),
and \(Z\). The support of \(X\) is a finite collection of elements
that can be inspected all at once. And while the support of \(Y\)
cannot be exhaustively written down, its elements can nevertheless be
listed in a naturally ordered sequence. Random variables with supports
similar to those of \(X\) and \(Y\) are called /discrete random
variables/. We study these in Chapter [[#cha-Discrete-Distributions]].

In contrast, the support of \(Z\) is a continuous interval, containing
all rational and irrational positive real numbers. For this
reason[fn:fn-reason], random variables with supports like \(Z\) are
called /continuous random variables/, to be studied in Chapter
[[#cha-Continuous-Distributions]].

[fn:fn-reason] This isn't really the reason, but it serves as an
effective litmus test at the introductory level. See Billingsley or
Resnick.

*** How to do it with \(\mathsf{R}\)

The primary vessel for this task is the =addrv= function. There are
two ways to use it, and we will describe both.

**** Supply a Defining Formula

The first method is based on the =transform= function. See
=?transform=. The idea is to write a formula defining the random
variable inside the function, and it will be added as a column to the
data frame. As an example, let us roll a 4-sided die three times, and
let us define the random variable \(U=X1-X2+X3\).

#+BEGIN_SRC R :exports code :results silent
S <- rolldie(3, nsides = 4, makespace = TRUE) 
S <- addrv(S, U = X1-X2+X3) 
#+END_SRC 

Now let's take a look at the values of \(U\). In the interest of
space, we will only reproduce the first few rows of \(S\) (there are
\(4^{3}=64\) rows in total).

#+BEGIN_SRC R :exports both :results output pp   
head(S)
#+END_SRC 

#+RESULTS:
:   X1 X2 X3 U    probs
: 1  1  1  1 1 0.015625
: 2  2  1  1 2 0.015625
: 3  3  1  1 3 0.015625
: 4  4  1  1 4 0.015625
: 5  1  2  1 0 0.015625
: 6  2  2  1 1 0.015625

We see from the \(U\) column it is operating just like it should. We
can now answer questions like

#+BEGIN_SRC R :exports both :results output pp   
Prob(S, U > 6) 
#+END_SRC 

#+RESULTS:
: [1] 0.015625

**** Supply a Function

Sometimes we have a function laying around that we would like to apply
to some of the outcome variables, but it is unfortunately tedious to
write out the formula defining what the new variable would be. The
=addrv= function has an argument =FUN= specifically for this case. Its
value should be a legitimate function from \(\mathsf{R}\), such as
=sum=, =mean=, =median=, and so forth. Or, you can define your own
function. Continuing the previous example, let's define
\(V=\max(X1,X2,X3)\) and \(W=X1+X2+X3\).

#+BEGIN_SRC R :exports both :results output pp  
S <- addrv(S, FUN = max, invars = c("X1","X2","X3"), name = "V") 
S <- addrv(S, FUN = sum, invars = c("X1","X2","X3"), name = "W") 
head(S) 
#+END_SRC 

#+RESULTS:
:   X1 X2 X3 U V W    probs
: 1  1  1  1 1 1 3 0.015625
: 2  2  1  1 2 2 4 0.015625
: 3  3  1  1 3 3 5 0.015625
: 4  4  1  1 4 4 6 0.015625
: 5  1  2  1 0 2 4 0.015625
: 6  2  2  1 1 2 5 0.015625

Notice that =addrv= has an =invars= argument to specify exactly to
which columns one would like to apply the function =FUN=. If no input
variables are specified, then =addrv= will apply =FUN= to all
non-=probs= columns. Further, =addrv= has an optional argument =name=
to give the new variable; this can be useful when adding several
random variables to a probability space (as above). If not specified,
the default name is =X=.

*** Marginal Distributions

As we can see above, often after adding a random variable \(V\) to a
probability space one will find that \(V\) has values that are
repeated, so that it becomes difficult to understand what the ultimate
behavior of \(V\) actually is. We can use the =marginal= function to
aggregate the rows of the sample space by values of \(V\), all the
while accumulating the probability associated with \(V\)'s distinct
values. Continuing our example from above, suppose we would like to
focus entirely on the values and probabilities of
\(V=\max(X1,X2,X3)\).

#+BEGIN_SRC R :exports both :results output pp   
marginal(S, vars = "V") 
#+END_SRC

#+RESULTS:
:   V    probs
: 1 1 0.015625
: 2 2 0.109375
: 3 3 0.296875
: 4 4 0.578125

We could save the probability space of \(V\) in a data frame and study
it further, if we wish. As a final remark, we can calculate the
marginal distributions of multiple variables desired using the =vars=
argument. For example, suppose we would like to examine the joint
distribution of \(V\) and \(W\).

#+BEGIN_SRC R :exports both :results output pp   
marginal(S, vars = c("V", "W")) 
#+END_SRC 

#+RESULTS:
#+BEGIN_example
   V  W    probs
1  1  3 0.015625
2  2  4 0.046875
3  2  5 0.046875
4  3  5 0.046875
5  2  6 0.015625
6  3  6 0.093750
7  4  6 0.046875
8  3  7 0.093750
9  4  7 0.093750
10 3  8 0.046875
11 4  8 0.140625
12 3  9 0.015625
13 4  9 0.140625
14 4 10 0.093750
15 4 11 0.046875
16 4 12 0.015625
#+END_example

Note that the default value of =vars= is the names of all columns
except =probs=. This can be useful if there are duplicated rows in the
probability space.

#+LaTeX: \newpage{}

** Exercises

#+LaTeX: \setcounter{thm}{0}

#+BEGIN_xca
<<xca-numb-cond-indep>> Prove the assertion given in the text: the
number of conditions that the events \(A_{1}\), \(A_{2}\), ...,
\(A_{n}\) must satisfy in order to be mutually independent is
\(2^{n} - n - 1\). (/Hint/: think about Pascal's triangle.)
#+END_xca
