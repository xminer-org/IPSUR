* Discrete Distributions                                           :discdist:
:PROPERTIES:
:tangle: R/05-discdist.R
:CUSTOM_ID: cha-Discrete-Distributions
:END:

#+BEGIN_SRC R :exports none :eval never
#    IPSUR: Introduction to Probability and Statistics Using R
#    Copyright (C) 2014  G. Jay Kerns
#
#    Chapter: Discrete Distributions
#
#    This file is part of IPSUR.
#
#    IPSUR is free software: you can redistribute it and/or modify
#    it under the terms of the GNU General Public License as published by
#    the Free Software Foundation, either version 3 of the License, or
#    (at your option) any later version.
#
#    IPSUR is distributed in the hope that it will be useful,
#    but WITHOUT ANY WARRANTY; without even the implied warranty of
#    MERCHANTABILITY or FITNESS FOR A PARTICULAR PURPOSE.  See the
#    GNU General Public License for more details.
#
#    You should have received a copy of the GNU General Public License
#    along with IPSUR.  If not, see <http://www.gnu.org/licenses/>.
#+END_SRC

#+BEGIN_SRC R :exports none :eval no-export
# This chapter's package dependencies
library(distrEx)
#+END_SRC

#+LaTeX: \noindent 
In this chapter we introduce discrete random variables, those who take
values in a finite or countably infinite support set. We discuss
probability mass functions and some special expectations, namely, the
mean, variance and standard deviation. Some of the more important
discrete distributions are explored in detail, and the more general
concept of expectation is defined, which paves the way for moment
generating functions.

We give special attention to the empirical distribution since it plays
such a fundamental role with respect to resampling and Chapter
[[#cha-resampling-methods]]; it will also be needed in Section
[[#sub-Kolmogorov-Smirnov-Goodness-of-Fit-Test]] where we discuss the Kolmogorov-Smirnov
test. Following this is a section in which we introduce a catalogue of
discrete random variables that can be used to model experiments.

There are some comments on simulation, and we mention transformations
of random variables in the discrete case. The interested reader who
would like to learn more about any of the assorted discrete
distributions mentioned here should take a look at /Univariate
Discrete Distributions/ by Johnson /et al/\cite{Johnson1993}.

*What do I want them to know?*
- how to choose a reasonable discrete model under a variety of
  physical circumstances
- item the notion of mathematical expectation, how to calculate it,
  and basic properties- moment generating functions (yes, I want them
  to hear about those)
- the general tools of the trade for manipulation of continuous random
  variables, integration, /etc/.
- some details on a couple of discrete models, and exposure to a bunch
  of other ones
- how to make new discrete random variables from old ones

** Discrete Random Variables
:PROPERTIES:
:CUSTOM_ID: sec-discrete-random-variables
:END:

*** Probability Mass Functions
:PROPERTIES:
:CUSTOM_ID: sub-probability-mass-functions
:END:

Discrete random variables are characterized by their supports which
take the form
\begin{equation}
S_{X}=\{u_{1},u_{2},\ldots,u_{k}\}\mbox{ or }S_{X}=\{u_{1},u_{2},u_{3}\ldots\}.
\end{equation}

Every discrete random variable \(X\) has associated with it a
probability mass function (PMF) \(f_{X}:S_{X}\to[0,1]\) defined by
\begin{equation}
f_{X}(x)=\mathbb{P}(X=x),\quad x\in S_{X}.
\end{equation}

Since values of the PMF represent probabilities, we know from Chapter
[[#cha-Probability]] that PMFs enjoy certain properties. In particular, all
PMFs satisfy
1. \(f_{X}(x)>0\) for \(x\in S\),
2. \(\sum_{x\in S}f_{X}(x)=1\), and
3. \(\mathbb{P}(X\in A)=\sum_{x\in A}f_{X}(x)\), for any event
   \(A\subset S\).

# +BEGIN_exampletoo
<<exa-Toss-a-coin>> Toss a coin 3 times. The sample space would be \[
S=\{ HHH,\ HTH,\ THH,\ TTH,\ HHT,\ HTT,\ THT,\ TTT\}.  \]

Now let \(X\) be the number of Heads observed. Then \(X\) has support
\(S_{X}=\{ 0,1,2,3\} \). Assuming that the coin is fair and was tossed
in exactly the same way each time, it is not unreasonable to suppose
that the outcomes in the sample space are all equally likely.

What is the PMF of \(X\)? Notice that \(X\) is zero exactly when the
outcome \(TTT\) occurs, and this event has probability
\(1/8\). Therefore, \(f_{X}(0)=1/8\), and the same reasoning shows
that \(f_{X}(3)=1/8\). Exactly three outcomes result in \(X=1\), thus,
\(f_{X}(1)=3/8\) and \(f_{X}(3)\) holds the remaining \(3/8\)
probability (the total is 1). We can represent the PMF with a table:

#+NAME: tab-pmf-flip-coin-three
#+CAPTION[Flipping a coin three times]: Flipping a coin three times: the PMF.
| \(x\in S_{X}\)               |   0 |   1 |   2 |   3 | Total |
|------------------------------+-----+-----+-----+-----+-------|
| \(f_{X}(x)=\mathbb{P}(X=x)\) | 1/8 | 3/8 | 3/8 | 1/8 |     1 |

# +END_exampletoo

*** Mean, Variance, and Standard Deviation
:PROPERTIES:
:CUSTOM_ID: sub-mean-variance-sd
:END:

There are numbers associated with PMFs. One important example is the
mean \(\mu\), also known as \(\mathbb{E} X\) (which we will discuss
later):
\begin{equation}
\mu=\mathbb{E} X=\sum_{x\in S}xf_{X}(x),
\end{equation}
provided the (potentially infinite) series \(\sum|x|f_{X}(x)\) is convergent. Another important number is the variance:
\begin{equation}
\sigma^{2}=\sum_{x\in S}(x-\mu)^{2}f_{X}(x),
\end{equation}
which can be computed (see Exercise [[xca-variance-shortcut]]) with the
alternate formula \(\sigma^{2}=\sum
x{}^{2}f_{X}(x)-\mu^{2}\). Directly defined from the variance is the
standard deviation \(\sigma=\sqrt{\sigma^{2}}\).
 
# +BEGIN_exampletoo
<<exa-disc-pmf-mean>> We will calculate the mean of \(X\) in Example
[[exa-Toss-a-coin]].  \[ \mu = \sum_{x = 0}^{3}xf_{X}(x) = 0 \cdot
\frac{1}{8} + 1 \cdot \frac{3}{8} + 2 \cdot
\frac{3}{8}+3\cdot\frac{1}{8} = 1.5. \] We interpret \(\mu = 1.5\) by
reasoning that if we were to repeat the random experiment many times,
independently each time, observe many corresponding outcomes of the
random variable \(X\), and take the sample mean of the observations,
then the calculated value would fall close to 1.5. The approximation
would get better as we observe more and more values of \(X\) (another
form of the Law of Large Numbers; see Section
[[#sec-Interpreting-Probabilities]]). Another way it is commonly stated is
that \(X\) is 1.5 "on the average" or "in the long run".
# +END_exampletoo


#+BEGIN_rem
Note that although we say \(X\) is 3.5 on the average, we must keep in
mind that our \(X\) never actually equals 3.5 (in fact, it is
impossible for \(X\) to equal 3.5).
#+END_rem

Related to the probability mass function \(f_{X}(x)=\mathbb{P}(X=x)\)
is another important function called the /cumulative distribution
function/ (CDF), \(F_{X}\). It is defined by the formula
\begin{equation}
F_{X}(t)=\mathbb{P}(X\leq t),\quad -\infty < t < \infty.
\end{equation}

We know that all PMFs satisfy certain properties, and a similar
statement may be made for CDFs. In particular, any CDF \(F_{X}\)
satisfies
- \(F_{X}\) is nondecreasing (\(t_{1}\leq t_{2}\) implies
  \(F_{X}(t_{1})\leq F_{X}(t_{2})\)).
- \(F_{X}\) is right-continuous (\(\lim_{t\to
  a^{+}}F_{X}(t)=F_{X}(a)\) for all \(a\in\mathbb{R}\)).
- \(\lim_{t\to-\infty}F_{X}(t)=0\) and
  \(\lim_{t\to\infty}F_{X}(t)=1\).
We say that \(X\) has the distribution \(F_{X}\) and we write \(X\sim
F_{X}\). In an abuse of notation we will also write \(X\sim f_{X}\)
and for the named distributions the PMF or CDF will be identified by
the family name instead of the defining formula.

**** How to do it with \(\mathsf{R}\)
:PROPERTIES:
:CUSTOM_ID: sub-disc-rv-how-r
:END:

The mean and variance of a discrete random variable is easy to compute
at the console. Let's return to Example [[exa-disc-pmf-mean]]. We will start
by defining a vector =x= containing the support of \(X\), and a vector
=f= to contain the values of \(f_{X}\) at the respective outcomes in
=x=:

#+BEGIN_SRC R :exports code :results silent
x <- c(0,1,2,3)
f <- c(1/8, 3/8, 3/8, 1/8)
#+END_SRC

To calculate the mean \(\mu\), we need to multiply the corresponding
values of =x= and =f= and add them. This is easily accomplished in
\(\mathsf{R}\) since operations on vectors are performed
/element-wise/ (see Section [[#sub-Functions-and-Expressions]]):

#+BEGIN_SRC R :exports both :results output pp  
mu <- sum(x * f)
mu
#+END_SRC

#+RESULTS:
: [1] 1.5

To compute the variance \(\sigma^{2}\), we subtract the value of =mu=
from each entry in =x=, square the answers, multiply by =f=,and
=sum=. The standard deviation \(\sigma\) is simply the square root of
\(\sigma^{2}\).

#+BEGIN_SRC R :exports both :results output pp  
sigma2 <- sum((x-mu)^2 * f)
sigma2
#+END_SRC

#+RESULTS:
: [1] 0.75

#+BEGIN_SRC R :exports both :results output pp  
sigma <- sqrt(sigma2)
sigma
#+END_SRC

#+RESULTS:
: [1] 0.8660254

Finally, we may find the values of the CDF \(F_{X}\) on the support by
accumulating the probabilities in \(f_{X}\) with the =cumsum=
function.

#+BEGIN_SRC R :exports both :results output pp  
F <- cumsum(f)
F
#+END_SRC

#+RESULTS:
: [1] 0.125 0.500 0.875 1.000

As easy as this is, it is even easier to do with the =distrEx= package
\cite{distrEx}. We define a random variable =X= as an object, then
compute things from the object such as mean, variance, and standard
deviation with the functions =E=, =var=, and =sd=:

#+BEGIN_SRC R :exports both :results output pp  
X <- DiscreteDistribution(supp = 0:3, prob = c(1,3,3,1)/8)
E(X); var(X); sd(X)
#+END_SRC

#+RESULTS:
: [1] 1.5
: [1] 0.75
: [1] 0.8660254

** The Discrete Uniform Distribution
:PROPERTIES:
:CUSTOM_ID: sec-disc-uniform-dist
:END:

We have seen the basic building blocks of discrete distributions and
we now study particular models that statisticians often encounter in
the field. Perhaps the most fundamental of all is the /discrete
uniform/ distribution.

A random variable \(X\) with the discrete uniform distribution on the
integers \(1,2,\ldots,m\) has PMF
\begin{equation}
f_{X}(x)=\frac{1}{m},\quad x=1,2,\ldots,m.
\end{equation}

We write \(X\sim\mathsf{disunif}(m)\). A random experiment where this
distribution occurs is the choice of an integer at random between 1
and 100, inclusive. Let \(X\) be the number chosen. Then
\(X\sim\mathsf{disunif}(m=100)\) and
\[
\mathbb{P}(X=x)=\frac{1}{100},\quad x=1,\ldots,100.
\]
We find a direct formula for the mean of \(X\sim\mathsf{disunif}(m)\):
\begin{equation}
\mu = \sum_{x = 1}^{m}xf_{X}(x) = \sum_{x = 1}^{m}x \cdot \frac{1}{m} = \frac{1}{m}(1 + 2 + \cdots + m) = \frac{m + 1}{2},
\end{equation}
where we have used the famous identity \(1 + 2 + \cdots + m = m(m +
1)/2\). That is, if we repeatedly choose integers at random from 1 to
\(m\) then, on the average, we expect to get \((m+1)/2\). To get the
variance we first calculate \[ \sum_{x = 1}^{m} x^{2} f_{X}(x) =
\frac{1}{m} \sum_{x = 1}^{m} x^{2} = \frac{1}{m}\frac{m(m + 1)(2m +
1)}{6} = \frac{(m + 1)(2m + 1)}{6}, \] and finally,
\begin{equation}
\sigma^{2} = \sum_{x = 1}^{m} x^{2} f_{X}(x) - \mu^{2} = \frac{(m + 1)(2m + 1)}{6} - \left(\frac{m + 1}{2}\right)^{2} = \cdots = \frac{m^{2} - 1}{12}.
\end{equation}

# +BEGIN_exampletoo

Roll a die and  let \(X\) be the upward face showing.  Then \(m = 6\),
\(\mu = 7/2 = 3.5\), and \(\sigma^{2} = (6^{2} - 1)/12 = 35/12\).
# +END_exampletoo

*** How to do it with \(\mathsf{R}\)

*** From the console:
One can choose an integer at random with the =sample= function. The
general syntax to simulate a discrete uniform random variable is
=sample(x, size, replace = TRUE)=.

The argument =x= identifies the numbers from which to randomly
sample. If =x= is a number, then sampling is done from 1 to =x=. The
argument =size= tells how big the sample size should be, and =replace=
tells whether or not numbers should be replaced in the urn after
having been sampled. The default option is =replace = FALSE= but for
discrete uniforms the sampled values should be replaced. Some examples
follow.

*** Examples
- To roll a fair die 3000 times, do =sample(6, size = 3000, replace = TRUE)=.
- To choose 27 random numbers from 30 to 70, do =sample(30:70, size = 27, replace = TRUE)=.
- To flip a fair coin 1000 times, do =sample(c("H","T"), size = 1000, replace = TRUE)=.

*** With the \(\mathsf{R}\) Commander:

Follow the sequence 
1. =Probability= \(\triangleright\) 
2. =Discrete Distributions= \(\triangleright\) 
3. =Discrete Uniform distribution= \(\triangleright\) 
4. =Simulate Discrete uniform variates...=.

Suppose we would like to roll a fair die 3000 times. In the =Number of
samples= field we enter =1=. Next, we describe what interval of
integers to be sampled. Since there are six faces numbered 1 through
6, we set =from = 1=, we set =to = 6=, and set =by = 1= (to indicate
that we travel from 1 to 6 in increments of 1 unit). We will generate
a list of 3000 numbers selected from among 1, 2, ..., 6, and we store
the results of the simulation. For the time being, we select =New Data
set=. Click =OK=.

Since we are defining a new data set, the \(\mathsf{R}\) Commander
requests a name for the data set. The default name is =Simset1=,
although in principle you could name it whatever you like (according
to \(\mathsf{R}\)'s rules for object names). We wish to have a list
that is 3000 long, so we set =Sample Size = 3000= and click =OK=.

In the \(\mathsf{R}\) Console window, the \(\mathsf{R}\) Commander
should tell you that =Simset1= has been initialized, and it should
also alert you that =There was 1 discrete uniform variate sample
stored in Simset 1.=. To take a look at the rolls of the die, we click
=View data set= and a window opens.

The default name for the variable is =disunif.sim1=.

** The Binomial Distribution
:PROPERTIES:
:CUSTOM_ID: sec-binom-dist
:END:

The binomial distribution is based on a /Bernoulli trial/, which is a
random experiment in which there are only two possible outcomes:
success (\(S\)) and failure (\(F\)). We conduct the Bernoulli trial
and let @@latex:\begin{equation} X = \begin{cases} 1 & \mbox{if the outcome is $S$},\\ 0 & \mbox{if the outcome is $F$}. \end{cases} \end{equation}@@
If the probability of success is \(p\) then the probability of failure
must be \(1-p=q\) and the PMF of \(X\) is
\begin{equation}
f_{X}(x)=p^{x}(1-p)^{1-x},\quad x=0,1.
\end{equation}
It is easy to calculate \(\mu=\mathbb{E} X=p\) and \(\mathbb{E}
X^{2}=p\) so that \(\sigma^{2}=p-p^{2}=p(1-p)\).

*** The Binomial Model
:PROPERTIES:
:CUSTOM_ID: sub-The-Binomial-Model
:END:

The Binomial model has three defining properties:
- Bernoulli trials are conducted \(n\) times,
- the trials are independent,
- the probability of success \(p\) does not change between trials.
If \(X\) counts the number of successes in the \(n\) independent
trials, then the PMF of \(X\) is 
\begin{equation}
f_{X}(x)={n \choose x}p^{x}(1-p)^{n-x},\quad x=0,1,2,\ldots,n.
\end{equation}

We say that \(X\) has a /binomial distribution/ and we write
\(X\sim\mathsf{binom}(\mathtt{size}=n,\,\mathtt{prob}=p)\). It is
clear that \(f_{X}(x)\geq0\) for all \(x\) in the support because the
value is the product of nonnegative numbers. We next check that \(\sum
f(x)=1\): \[ \sum_{x = 0}^{n}{n \choose x} p^{x} (1 - p)^{n - x} =
[p + (1 - p)]^{n} = 1^{n} = 1.  \] We next find the mean:
\begin{alignat*}{1}
\mu= & \sum_{x=0}^{n}x\,{n \choose x}p^{x}(1-p)^{n-x},\\
= & \sum_{x=1}^{n}x\,\frac{n!}{x!(n-x)!}p^{x}q^{n-x},\\
= & n\cdot p\sum_{x=1}^{n}\frac{(n-1)!}{(x-1)!(n-x)!}p^{x-1}q^{n-x},\\
= & np\,\sum_{x-1=0}^{n-1}{n-1 \choose x-1}p^{(x-1)}(1-p)^{(n-1)-(x-1)},\\
= & np.
\end{alignat*}
A similar argument shows that \(\mathbb{E} X(X - 1) = n(n - 1)p^{2}\) (see
Exercise [[xca-binom-factorial-expectation]]). Therefore
\begin{alignat*}{1}
\sigma^{2}= & \mathbb{E} X(X-1)+\mathbb{E} X-[\mathbb{E} X]^{2},\\
= & n(n-1)p^{2}+np-(np)^{2},\\
= & n^{2}p^{2}-np^{2}+np-n^{2}p^{2},\\
= & np-np^{2}=np(1-p).
\end{alignat*}

# +BEGIN_exampletoo

A four-child family. Each child may be either a boy (\(B\)) or a girl
(\(G\)). For simplicity we suppose that
\(\mathbb{P}(B)=\mathbb{P}(G)=1/2\) and that the genders of the
children are determined independently. If we let \(X\) count the
number of \(B\)'s, then
\(X\sim\mathsf{binom}(\mathtt{size}=4,\,\mathtt{prob}=1/2)\). Further,
\(\mathbb{P}(X=2)\) is
\[
f_{X}(2)={4 \choose 2}(1/2)^{2}(1/2)^{2}=\frac{6}{2^{4}}.
\]
The mean number of boys is \(4(1/2)=2\) and the variance of \(X\) is
\(4(1/2)(1/2)=1\).
# +END_exampletoo

**** How to do it with \(\mathsf{R}\)

The corresponding \(\mathsf{R}\) function for the PMF and CDF are
=dbinom= and =pbinom=, respectively. We demonstrate their use in the
following examples.

# +BEGIN_exampletoo

We can calculate it in \(\mathsf{R}\) Commander under the =Binomial
Distribution= menu with the =Binomial probabilities= menu item.

#+BEGIN_SRC R :exports results :results output pp
A <- data.frame(Pr=dbinom(0:4, size = 4, prob = 0.5))
rownames(A) <- 0:4 
A
#+END_SRC

#+RESULTS:
:       Pr
: 0 0.0625
: 1 0.2500
: 2 0.3750
: 3 0.2500
: 4 0.0625

# +END_exampletoo

We know that the
\(\mathsf{binom}(\mathtt{size}=4,\,\mathtt{prob}=1/2)\) distribution
is supported on the integers 0, 1, 2, 3, and 4; thus the table is
complete. We can read off the answer to be \(\mathbb{P}(X=2)=0.3750\).

# +BEGIN_exampletoo

Roll 12 dice simultaneously, and let \(X\) denote the number of 6's
that appear. We wish to find the probability of getting seven, eight,
or nine 6's. If we let \(S=\{ \mbox{get a 6 on one roll} \} \), then
\(\mathbb{P}(S)=1/6\) and the rolls constitute Bernoulli trials; thus
\(X\sim\mathsf{binom}(\mathtt{size}=12,\ \mathtt{prob}=1/6)\) and our
task is to find \(\mathbb{P}(7\leq X\leq9)\). This is just
\[ 
\mathbb{P}(7\leq X\leq9)=\sum_{x=7}^{9}{12 \choose x}(1/6)^{x}(5/6)^{12-x}.
\]
Again, one method to solve this problem would be to generate a
probability mass table and add up the relevant rows. However, an
alternative method is to notice that \(\mathbb{P}(7\leq
X\leq9)=\mathbb{P}(X\leq9)-\mathbb{P}(X\leq6)=F_{X}(9)-F_{X}(6)\), so
we could get the same answer by using the =Binomial tail
probabilities...= menu in the \(\mathsf{R}\) Commander or the
following from the command line:

#+BEGIN_SRC R :exports both :results output pp  
pbinom(9, size=12, prob=1/6) - pbinom(6, size=12, prob=1/6)
diff(pbinom(c(6,9), size = 12, prob = 1/6))  # same thing
#+END_SRC

#+RESULTS:
: [1] 0.001291758
: [1] 0.001291758

# +END_exampletoo


# +BEGIN_exampletoo
<<exa-toss-coin-3-withR>> Toss a coin three times and let \(X\) be the
number of Heads observed. We know from before that
\(X\sim\mathsf{binom}(\mathtt{size}=3,\,\mathtt{prob}=1/2)\) which
implies the following PMF:

#+NAME: tab-flip-coin-thrice
#+CAPTION[Flipping a coin thrice: PMF]: Flipping a coin three times: the PMF.
| \(x=\mbox{num. of Heads}\)   |   0 |   1 |   2 |   3 | Total |
|------------------------------+-----+-----+-----+-----+-------|
| \(f(x) = \mathbb{P}(X = x)\) | 1/8 | 3/8 | 3/8 | 1/8 |     1 |

Our next goal is to write down the CDF of \(X\) explicitly. The first
case is easy: it is impossible for \(X\) to be negative, so if \(x<0\)
then we should have \(\mathbb{P}(X\leq x)=0\). Now choose a value
\(x\) satisfying \(0\leq x<1\), say, \(x=0.3\). The only way that
\(X\leq x\) could happen would be if \(X=0\), therefore,
\(\mathbb{P}(X\leq x)\) should equal \(\mathbb{P}(X=0)\), and the same
is true for any \(0\leq x<1\). Similarly, for any \(1\leq x<2\), say,
\(x=1.73\), the event \(\{ X\leq x \}\) is exactly the event \(\{
X=0\mbox{ or }X=1 \}\). Consequently, \(\mathbb{P}(X\leq x)\) should
equal \(\mathbb{P}(X=0\mbox{ or
}X=1)=\mathbb{P}(X=0)+\mathbb{P}(X=1)\). Continuing in this fashion,
we may figure out the values of \(F_{X}(x)\) for all possible inputs
\(-\infty<x<\infty\), and we may summarize our observations with the
following piecewise defined function: @@latex:\[ F_{X}(x)=\mathbb{P}(X\leq x) = \begin{cases} 0, & x < 0,\\ \frac{1}{8}, & 0\leq x < 1,\\ \frac{1}{8} + \frac{3}{8} = \frac{4}{8}, & 1\leq x < 2,\\ \frac{4}{8} + \frac{3}{8} = \frac{7}{8}, & 2\leq x < 3,\\ 1, & x \geq 3. \end{cases} \]@@
In particular, the CDF of \(X\) is defined for the entire real line,
\(\mathbb{R}\). The CDF is right continuous and nondecreasing. A graph
of the \(\mathsf{binom}(\mathtt{size}=3,\,\mathtt{prob}=1/2)\) CDF is
shown in Figure [[fig-binom-cdf-base]].
# +END_exampletoo

#+NAME: binom-cdf-base
#+BEGIN_SRC R :exports results :results graphics :file fig/discdist-binom-cdf-base.ps
plot(0, xlim = c(-1.2, 4.2), ylim = c(-0.04, 1.04), type = "n", xlab = "number of successes", ylab = "cumulative probability")
abline(h = c(0,1), lty = 2, col = "grey")
lines(stepfun(0:3, pbinom(-1:3, size = 3, prob = 0.5)), verticals = FALSE, do.p = FALSE)
points(0:3, pbinom(0:3, size = 3, prob = 0.5), pch = 16, cex = 1.2)
points(0:3, pbinom(-1:2, size = 3, prob = 0.5), pch = 1, cex = 1.2)
#+END_SRC

#+NAME: fig-binom-cdf-base
#+CAPTION[Graph of the \(\mathsf{binom}(\mathtt{size}=3,\,\mathtt{prob}=1/2)\) CDF]: \small A graph of the \(\mathsf{binom}(\mathtt{size}=3,\,\mathtt{prob}=1/2)\) CDF. 
#+ATTR_LaTeX: :width 0.9\textwidth :placement [ht!]
#+RESULTS: binom-cdf-base
[[file:fig/discdist-binom-cdf-base.ps]]

# +BEGIN_exampletoo

Another way to do Example [[exa-toss-coin-3-withR]] is with the =distr=
family of packages \cite{distr}. They use an object oriented approach
to random variables, that is, a random variable is stored in an object
=X=, and then questions about the random variable translate to
functions on and involving =X=. Random variables with distributions
from the =base= package\cite{base} are specified by capitalizing the
name of the distribution.

#+BEGIN_SRC R :exports both :results output pp  
X <- Binom(size = 3, prob = 1/2)
X
#+END_SRC

#+RESULTS:
: Distribution Object of Class: Binom
:  size: 3
:  prob: 0.5

The analogue of the =dbinom= function for =X= is the =d(X)= function,
and the analogue of the =pbinom= function is the =p(X)=
function. Compare the following:

#+BEGIN_SRC R :exports both :results output pp  
d(X)(1)   # pmf of X evaluated at x = 1
p(X)(2)   # cdf of X evaluated at x = 2
#+END_SRC

#+RESULTS:
: [1] 0.375
: [1] 0.875

# +END_exampletoo


Random variables defined via the =distr= package \cite{distr} may be
/plotted/, which will return graphs of the PMF, CDF, and quantile
function (introduced in Section [[#sub-Normal-Quantiles-QF]]). See
Figure [[fig-binom-plot-distr]] for an example.

#+NAME: binom-plot-distr
#+BEGIN_SRC R :exports results :results graphics :file fig/discdist-binom-plot-distr.ps
X <- Binom(size = 3, prob = 1/2)
plot(X, cex = 0.2)
#+END_SRC

#+NAME: fig-binom-plot-distr
#+CAPTION[The \textsf{binom}(=size= = 3, =prob= = 0.5) distribution from the =distr= package]: \small The \textsf{binom}(=size= = 3, =prob= = 0.5) distribution from the =distr= package.
#+ATTR_LaTeX: :width 0.9\textwidth :placement [ht!]
#+RESULTS: binom-plot-distr
[[file:fig/discdist-binom-plot-distr.ps]]


#+CAPTION[Correspondence between =stats= and =distr=]: Correspondence between =stats= and =distr=. We are given \(X\sim\mathsf{dbinom}(\mathtt{size}=n,\,\mathtt{prob}=p)\).  For the =distr= package we must first set \(\mathtt{X\ <-\ Binom(size=}n\mathtt{,\ prob=}p\mathtt{)}\).
| How to do:              | with =stats= (default)               | with =distr=         |
|-------------------------+--------------------------------------+----------------------|
| PMF: \(\mathbb{P}(X=x)\)       | \(\mathtt{dbinom(x,size=n,prob=p)}\) | \(\mathtt{d(X)(x)}\) |
| CDF:  \(\mathbb{P}(X\leq x)\)  | \(\mathtt{pbinom(x,size=n,prob=p)}\) | \(\mathtt{p(X)(x)}\) |
| Simulate \(k\) variates | \(\mathtt{rbinom(k,size=n,prob=p)}\) | \(\mathtt{r(X)(k)}\) |
|-------------------------+--------------------------------------+----------------------|

** Expectation and Moment Generating Functions
:PROPERTIES:
:CUSTOM_ID: sec-expectation-and-mgfs
:END:

*** The Expectation Operator
:PROPERTIES:
:CUSTOM_ID: sub-expectation-operator
:END:

We next generalize some of the concepts from Section
[[#sub-mean-variance-sd]]. There we saw that every[fn:fn-converge] PMF has
two important numbers associated with it:

\begin{equation}
\mu = \sum_{x \in S}x f_{X}(x),\quad \sigma^{2} = \sum_{x \in S}(x - \mu)^{2} f_{X}(x).
\end{equation}

Intuitively, for repeated observations of \(X\) we would expect the
sample mean to closely approximate \(\mu\) as the sample size
increases without bound. For this reason we call \(\mu\) the /expected
value/ of \(X\) and we write \(\mu=\mathbb{E} X\), where
\(\mathbb{E}\) is an /expectation operator/.

[fn:fn-converge] Not every, only those PMFs for which the (potentially infinite) series converges.

#+BEGIN_defn
More generally, given a function \(g\) we define the /expected value
of/ \(g(X)\) by
\begin{equation}
\mathbb{E}\, g(X)=\sum_{x\in S}g(x)f_{X}(x),
\end{equation}
provided the (potentially infinite) series \(\sum_{x} \vert g(x) \vert f(x)\) is
convergent. We say that \(\mathbb{E} g(X)\) /exists/.
#+END_defn

In this notation the variance is \(\sigma^{2} = \mathbb{E}(X -
\mu)^{2}\) and we prove the identity
\begin{equation}
\mathbb{E}(X - \mu)^{2} = \mathbb{E} X^{2} - (\mathbb{E} X)^{2}
\end{equation}
in Exercise [[xca-variance-shortcut]]. Intuitively, for repeated observations
of \(X\) we would expect the sample mean of the \(g(X)\) values to
closely approximate \(\mathbb{E}\, g(X)\) as the sample size increases
without bound.

Let us take the analogy further. If we expect \(g(X)\) to be close to
\(\mathbb{E} g(X)\) on the average, where would we expect \(3g(X)\) to
be on the average? It could only be \(3\mathbb{E} g(X)\). The
following theorem makes this idea precise.

#+BEGIN_prop
<<pro-expectation-properties>> For any functions \(g\) and \(h\), any
random variable \(X\), and any constant \(c\):
1. \(\mathbb{E}\: c=c\),
2. \(\mathbb{E}[c\cdot g(X)]=c\mathbb{E} g(X)\)
3. \(\mathbb{E}[g(X)+h(X)]=\mathbb{E} g(X)+\mathbb{E} h(X)\), provided
   \(\mathbb{E} g(X)\) and \(\mathbb{E} h(X)\) exist.
#+END_prop

#+BEGIN_proof
Go directly from the definition. For example, \[ \mathbb{E}[c \cdot
g(X)] = \sum_{x \in S} c \cdot g(x) f_{X}(x) = c \cdot \sum_{x \in S}
g(x) f_{X}(x) = c \mathbb{E} g(X).  \]
#+END_proof

*** Moment Generating Functions
:PROPERTIES:
:CUSTOM_ID: sub-MGFs
:END:

#+BEGIN_defn
Given a random variable \(X\), its /moment generating function/
(abbreviated MGF) is defined by the formula
\begin{equation}
M_{X}(t)=\mathbb{E}\mathrm{e}^{tX}=\sum_{x\in S}\mathrm{e}^{tx}f_{X}(x),
\end{equation}
provided the (potentially infinite) series is convergent for all \(t\)
in a neighborhood of zero (that is, for all \(-\epsilon < t <
\epsilon\), for some \(\epsilon > 0\)).
#+END_defn

Note that for any MGF \(M_{X}\),
\begin{equation}
M_{X}(0) = \mathbb{E} \mathrm{e}^{0 \cdot X} = \mathbb{E} 1 = 1.
\end{equation}
We will calculate the MGF for the two distributions introduced above.

# +BEGIN_exampletoo

Find the MGF for \(X\sim\mathsf{disunif}(m)\). Since \(f(x) = 1/m\),
the MGF takes the form \[ M(t) = \sum_{x = 1}^{m} \mathrm{e}^{tx}
\frac{1}{m} = \frac{1}{m}(\mathrm{e}^{t} + \mathrm{e}^{2t} + \cdots +
\mathrm{e}^{mt}),\quad \mbox{for any $t$.}  \]

# +END_exampletoo


# +BEGIN_exampletoo
Find the MGF for
\(X\sim\mathsf{binom}(\mathtt{size}=n,\,\mathtt{prob}=p)\).
# +END_exampletoo

\begin{alignat*}{1}
M_{X}(t)= & \sum_{x=0}^{n}\mathrm{e}^{tx}\,{n \choose x}\, p^{x}(1-p)^{n-x},\\
= & \sum_{x=0}^{n}{n \choose x}\,(p\mathrm{e}^{t})^{x}q^{n-x},\\
= & (p\mathrm{e}^{t}+q)^{n},\quad \mbox{for any $t$.}
\end{alignat*}

**** Applications

We will discuss three applications of moment generating functions in
this book. The first is the fact that an MGF may be used to accurately
identify the probability distribution that generated it, which rests
on the following:

#+BEGIN_thm
<<thm-mgf-unique>> The moment generating function, if it exists in a
neighborhood of zero, determines a probability distribution
/uniquely/.
#+END_thm

#+BEGIN_proof
Unfortunately, the proof of such a theorem is beyond the scope of a
text like this one. Interested readers could consult Billingsley
\cite{Billingsley1995}.
#+END_proof


We will see an example of Theorem [[thm-mgf-unique]] in action.

# +BEGIN_exampletoo

Suppose we encounter a random variable which has MGF
\[
M_{X}(t)=(0.3+0.7\mathrm{e}^{t})^{13}.
\]
Then \(X\sim\mathsf{binom}(\mathtt{size}=13,\,\mathtt{prob}=0.7)\).
# +END_exampletoo

An MGF is also known as a "Laplace Transform" and is manipulated in
that context in many branches of science and engineering.

**** Why is it called a Moment Generating Function?

This brings us to the second powerful application of MGFs. Many of the
models we study have a simple MGF, indeed, which permits us to
determine the mean, variance, and even higher moments very
quickly. Let us see why. We already know that
\begin{alignat*}{1}
M(t)= & \sum_{x\in S}\mathrm{e}^{tx}f(x).
\end{alignat*}
Take the derivative with respect to \(t\) to get
\begin{equation}
M'(t)=\frac{\mathrm{d}}{\mathrm{d} t}\left(\sum_{x\in S}\mathrm{e}^{tx}f(x)\right)=\sum_{x\in S}\ \frac{\mathrm{d}}{\mathrm{d} t}\left(\mathrm{e}^{tx}f(x)\right)=\sum_{x\in S}x\mathrm{e}^{tx}f(x),
\end{equation}
and so if we plug in zero for \(t\) we see
\begin{equation}
M'(0)=\sum_{x\in S}x\mathrm{e}^{0}f(x)=\sum_{x\in S}xf(x)=\mu=\mathbb{E} X.
\end{equation}
Similarly, \(M''(t) = \sum x^{2} \mathrm{e}^{tx} f(x) \) so that
\(M''(0) = \mathbb{E} X^{2}\). And in general, we can
see[fn:fn-details] that
\begin{equation}
M_{X}^{(r)}(0)=\mathbb{E} X^{r}=\mbox{\(r^{\mathrm{th}}\) moment of \(X\) about the origin.}
\end{equation}

These are also known as /raw moments/ and are sometimes denoted
\(\mu_{r}'\). In addition to these are the so called /central moments/
\(\mu_{r}\) defined by
\begin{equation}
\mu_{r}=\mathbb{E}(X-\mu)^{r},\quad r=1,2,\ldots
\end{equation}

[fn:fn-details] We are glossing over some significant mathematical
details in our derivation. Suffice it to say that when the MGF exists
in a neighborhood of \(t=0\), the exchange of differentiation and
summation is valid in that neighborhood, and our remarks hold true.

# +BEGIN_exampletoo

Let \(X \sim \mathsf{binom}(\mathtt{size} = n,\,\mathtt{prob} = p)
\mbox{ with $M(t) = (q + p \mathrm{e}^{t})^{n}$}\).

We calculated the mean and variance of a binomial random variable in
Section [[#sec-binom-dist]] by means of the binomial series. But
look how quickly we find the mean and variance with the moment
generating function.
\begin{alignat*}{1}
M'(t)= & n(q+p\mathrm{e}^{t})^{n-1}p\mathrm{e}^{t}\left|_{t=0}\right.,\\
= & n\cdot1^{n-1}p,\\
= & np.
\end{alignat*}
And
\begin{alignat*}{1}
M''(0)= & n(n-1)[q+p\mathrm{e}^{t}]^{n-2}(p\mathrm{e}^{t})^{2}+n[q+p\mathrm{e}^{t}]^{n-1}p\mathrm{e}^{t}\left|_{t=0}\right.,\\
\mathbb{E} X^{2}= & n(n-1)p^{2}+np.
\end{alignat*}
Therefore
\begin{alignat*}{1}
\sigma^{2}= & \mathbb{E} X^{2}-(\mathbb{E} X)^{2},\\
= & n(n-1)p^{2}+np-n^{2}p^{2},\\
= & np-np^{2}=npq.
\end{alignat*}
See how much easier that was?
# +END_exampletoo

#+BEGIN_rem
We learned in this section that \(M^{(r)}(0) = \mathbb{E} X^{r}\). We
remember from Calculus II that certain functions \(f\) can be
represented by a Taylor series expansion about a point \(a\), which
takes the form
\begin{equation}
f(x)=\sum_{r=0}^{\infty}\frac{f^{(r)}(a)}{r!}(x-a)^{r},\quad \mbox{for all \(|x-a| < R\),}
\end{equation}
where \(R\) is called the /radius of convergence/ of the series (see
Appendix [[#sec-Sequences-and-Series]]). We combine the two to say that if an
MGF exists for all \(t\) in the interval \((-\epsilon,\epsilon)\),
then we can write
\begin{equation}
M_{X}(t)=\sum_{r=0}^{\infty}\frac{\mathbb{E} X^{r}}{r!}t^{r},\quad \mbox{for all $|t|<\epsilon$.}
\end{equation}
#+END_rem

**** How to do it with \(\mathsf{R}\)

The =distrEx= package \cite{distrEx} provides an expectation operator
=E= which can be used on random variables that have been defined in
the ordinary =distr= sense:

#+BEGIN_SRC R :exports both :results output pp  
X <- Binom(size = 3, prob = 0.45)
E(X)
E(3*X + 4)
#+END_SRC

#+RESULTS:
: [1] 1.35
: [1] 8.05

For discrete random variables with finite support, the expectation is
simply computed with direct summation. In the case that the random
variable has infinite support and the function is crazy, then the
expectation is not computed directly, rather, it is estimated by first
generating a random sample from the underlying model and next
computing a sample mean of the function of interest.

There are methods for other population parameters:

#+BEGIN_SRC R :exports both :results output pp  
var(X)
sd(X)
#+END_SRC

#+RESULTS:
: [1] 0.7425
: [1] 0.8616844

There are even methods for =IQR=, =mad=, =skewness=, and =kurtosis=.

** The Empirical Distribution
:PROPERTIES:
:CUSTOM_ID: sec-empirical-distribution
:END:

Do an experiment \(n\) times and observe \(n\) values \(x_{1}\),
\(x_{2}\), ..., \(x_{n}\) of a random variable \(X\). For simplicity
in most of the discussion that follows it will be convenient to
imagine that the observed values are distinct, but the remarks are
valid even when the observed values are repeated.

#+BEGIN_defn
The /empirical cumulative distribution function/ \(F_{n}\) (written
ECDF) @@latex:\index{Empirical distribution}@@ is the probability distribution
that places probability mass \(1/n\) on each of the values \(x_{1}\),
\(x_{2}\), ..., \(x_{n}\). The empirical PMF takes the form
\begin{equation} 
f_{X}(x)=\frac{1}{n},\quad x\in \{ x_{1},x_{2},...,x_{n} \}.
\end{equation}
If the value \(x_{i}\) is repeated \(k\) times, the mass at \(x_{i}\) is accumulated to \(k/n\).
#+END_defn

The mean of the empirical distribution is
\begin{equation}
\mu=\sum_{x\in S}xf_{X}(x)=\sum_{i=1}^{n}x_{i}\cdot\frac{1}{n}
\end{equation}
and we recognize this last quantity to be the sample mean, \(\overline{x}\). The variance of the empirical distribution is
\begin{equation}
\sigma^{2}=\sum_{x\in S}(x-\mu)^{2}f_{X}(x)=\sum_{i=1}^{n}(x_{i}-\overline{x})^{2}\cdot\frac{1}{n}
\end{equation}
and this last quantity looks very close to what we already know to be the sample variance.
\begin{equation}
s^{2}=\frac{1}{n-1}\sum_{i=1}^{n}(x_{i}-\overline{x})^{2}.
\end{equation}
The /empirical quantile function/ is the inverse of the ECDF. See
Section [[#sub-Normal-Quantiles-QF]].

*** How to do it with \(\mathsf{R}\)

The empirical distribution is not directly available as a distribution
in the same way that the other base probability distributions are, but
there are plenty of resources available for the determined
investigator.  Given a data vector of observed values =x=, we can see
the empirical CDF with the =ecdf= @@latex:\index{ecdf@\texttt{ecdf}}@@ function:

#+BEGIN_SRC R :exports both :results output pp  
x <- c(4, 7, 9, 11, 12)
ecdf(x)
#+END_SRC

#+RESULTS:
: Empirical CDF 
: Call: ecdf(x)
:  x[1:5] =      4,      7,      9,     11,     12

The above shows that the returned value of =ecdf(x)= is not a /number/
but rather a /function/. The ECDF is not usually used by itself in
this form. More commonly it is used as an intermediate step in a more
complicated calculation, for instance, in hypothesis testing (see
Chapter [[#cha-Hypothesis-Testing]]) or resampling (see Chapter
[[#cha-resampling-methods]]). It is nevertheless instructive to see
what the =ecdf= looks like, and there is a special plot method for
=ecdf= objects.

#+NAME: empirical-CDF
#+BEGIN_SRC R :exports both :results graphics :file fig/discdist-empirical-CDF.ps
plot(ecdf(x))
#+END_SRC

#+NAME: fig-empirical-CDF
#+CAPTION[The empirical CDF]: \small The empirical CDF. 
#+ATTR_LaTeX: :width 0.9\textwidth :placement [ht!]
#+RESULTS: empirical-CDF
[[file:fig/discdist-empirical-CDF.ps]]

See Figure [[fig-empirical-CDF]]. The graph is of a right-continuous function
with jumps exactly at the locations stored in =x=. There are no
repeated values in =x= so all of the jumps are equal to \(1/5=0.2\).

The empirical PDF is not usually of particular interest in itself, but
if we really wanted we could define a function to serve as the
empirical PDF:

#+BEGIN_SRC R :exports both :results output pp  
epdf <- function(x) function(t){sum(x %in% t)/length(x)}
x <- c(0,0,1)
epdf(x)(0)       # should be 2/3
#+END_SRC

#+RESULTS:
: [1] 0.6666667

To simulate from the empirical distribution supported on the vector
=x=, we use the =sample= @@latex:\index{sample@\texttt{sample}}@@ function.

#+BEGIN_SRC R :exports both :results output pp  
x <- c(0,0,1)
sample(x, size = 7, replace = TRUE)
#+END_SRC

#+RESULTS:
: [1] 1 0 0 0 0 0 0

We can get the empirical quantile function in \(\mathsf{R}\) with
=quantile(x, probs = p, type = 1)=; see Section [[#sub-Normal-Quantiles-QF]].

As we hinted above, the empirical distribution is significant more
because of how and where it appears in more sophisticated
applications. We will explore some of these in later chapters -- see,
for instance, Chapter [[#cha-resampling-methods]].

** Other Discrete Distributions
:PROPERTIES:
:CUSTOM_ID: sec-other-discrete-distributions
:END:

The binomial and discrete uniform distributions are popular, and
rightly so; they are simple and form the foundation for many other
more complicated distributions. But the particular uniform and
binomial models only apply to a limited range of problems. In this
section we introduce situations for which we need more than what the
uniform and binomial offer.

*** Dependent Bernoulli Trials
:PROPERTIES:
:CUSTOM_ID: sec-non-bernoulli-trials
:END:

**** The Hypergeometric Distribution
:PROPERTIES:
:CUSTOM_ID: sub-hypergeometric-dist
:END:

Consider an urn with 7 white balls and 5 black balls. Let our random
experiment be to randomly select 4 balls, without replacement, from
the urn. Then the probability of observing 3 white balls (and thus 1
black ball) would be
\begin{equation}
\mathbb{P}(3W,1B)=\frac{{7 \choose 3}{5 \choose 1}}{{12 \choose 4}}.
\end{equation}
More generally, we sample without replacement \(K\) times from an urn
with \(M\) white balls and \(N\) black balls. Let \(X\) be the number
of white balls in the sample. The PMF of \(X\) is
\begin{equation}
f_{X}(x)=\frac{{M \choose x}{N \choose K-x}}{{M+N \choose K}}.
\end{equation}
We say that \(X\) has a /hypergeometric distribution/ and write
\(X\sim\mathsf{hyper}(\mathtt{m}=M,\,\mathtt{n}=N,\,\mathtt{k}=K)\).

The support set for the hypergeometric distribution is a little bit
tricky. It is tempting to say that \(x\) should go from 0 (no white
balls in the sample) to \(K\) (no black balls in the sample), but that
does not work if \(K>M\), because it is impossible to have more white
balls in the sample than there were white balls originally in the
urn. We have the same trouble if \(K>N\). The good news is that the
majority of examples we study have \(K\leq M\) and \(K\leq N\) and we
will happily take the support to be \(x=0,\ 1,\ \ldots,\ K\).

It is shown in Exercise [[xca-hyper-mean-variance]] that
\begin{equation}
\mu=K\frac{M}{M+N},\quad \sigma^{2}=K\frac{MN}{(M+N)^{2}}\frac{M+N-K}{M+N-1}.
\end{equation}

The associated \(\mathsf{R}\) functions for the PMF and CDF are
=dhyper(x, m, n, k)= and =phyper=, respectively. There are two more
functions: =qhyper=, which we will discuss in Section
[[#sub-Normal-Quantiles-QF]], and =rhyper=, discussed below.

# +BEGIN_exampletoo

Suppose in a certain shipment of 250 Pentium processors there are 17
defective processors. A quality control consultant randomly collects 5
processors for inspection to determine whether or not they are
defective. Let \(X\) denote the number of defectives in the sample.

Find the probability of exactly 3 defectives in the sample, that is,
find \(\mathbb{P}(X=3)\).  /Solution:/ We know that
\(X\sim\mathsf{hyper}(\mathtt{m}=17,\,\mathtt{n}=233,\,\mathtt{k}=5)\). So
the required probability is just \[ f_{X}(3)=\frac{{17 \choose 3}{233
\choose 2}}{{250 \choose 5}}.  \] To calculate it in \(\mathsf{R}\) we
just type

#+BEGIN_SRC R :exports both :results output pp  
dhyper(3, m = 17, n = 233, k = 5)
#+END_SRC

#+RESULTS:
: [1] 0.002351153

To find it with the \(\mathsf{R}\) Commander we go 
1. =Probability= \(\triangleright\) 
2. =Discrete Distributions= \(\triangleright\) 
3. =Hypergeometric distribution= \(\triangleright\) 
4. =Hypergeometric probabilities...=. 

We fill in the parameters \(m = 17\), \(n = 233\), and \(k =
5\). Click =OK=, and the following table is shown in the window.

#+BEGIN_SRC R :exports both :results output pp  
A <- data.frame(Pr=dhyper(0:4, m = 17, n = 233, k = 5))
rownames(A) <- 0:4 
A
#+END_SRC

#+RESULTS:
:             Pr
: 0 7.011261e-01
: 1 2.602433e-01
: 2 3.620776e-02
: 3 2.351153e-03
: 4 7.093997e-05

We wanted \(\mathbb{P}(X=3)\), and this is found from the table to be
approximately 0.0024. The value is rounded to the fourth decimal
place.

We know from our above discussion that the sample space should be
\(x=0,1,2,3,4,5\), yet, in the table the probabilities are only
displayed for \(x = 1,2,3,\) and 4. What is happening? As it turns
out, the \(\mathsf{R}\) Commander will only display probabilities that
are 0.00005 or greater. Since \(x=5\) is not shown, it suggests that
the outcome has a tiny probability. To find its exact value we use the
=dhyper= function:

#+BEGIN_SRC R :exports both :results output pp  
dhyper(5, m = 17, n = 233, k = 5)
#+END_SRC

#+RESULTS:
: [1] 7.916049e-07

In other words, \(\mathbb{P}(X=5)\approx0.0000007916049\), a small
number indeed.

Find the probability that there are at most 2 defectives in the
sample, that is, compute \(\mathbb{P}(X\leq2)\).  /Solution:/ Since
\(\mathbb{P}(X\leq2)=\mathbb{P}(X=0,1,2)\), one way to do this would
be to add the 0, 1, and 2 entries in the above table. this gives
\(0.7011+0.2602+0.0362=0.9975\). Our answer should be correct up to
the accuracy of 4 decimal places. However, a more precise method is
provided by the \(\mathsf{R}\) Commander. Under the =Hypergeometric
distribution= menu we select =Hypergeometric tail
probabilities...=. We fill in the parameters \(m\), \(n\), and \(k\)
as before, but in the =Variable value(s)= dialog box we enter the
value 2. We notice that the =Lower tail= option is checked, and we
leave that alone. Click =OK=.

#+BEGIN_SRC R :exports both :results output pp  
phyper(2, m = 17, n = 233, k = 5)
#+END_SRC

#+RESULTS:
: [1] 0.9975771

And thus \(\mathbb{P}(X\leq2)\approx 0.9975771\). We have confirmed
that the above answer was correct up to four decimal places.

Find \(\mathbb{P}(X>1)\). The table did not give us the explicit
probability \(\mathbb{P}(X=5)\), so we can not use the table to give
us this probability. We need to use another method. Since
\(\mathbb{P}(X>1)=1-\mathbb{P}(X\leq1)=1-F_{X}(1)\), we can find the
probability with =Hypergeometric tail probabilities...=. We enter 1
for =Variable Value(s)=, we enter the parameters as before, and in
this case we choose the =Upper tail= option. This results in the
following output.

#+BEGIN_SRC R :exports both :results output pp  
phyper(1, m = 17, n = 233, k = 5, lower.tail = FALSE)
#+END_SRC

#+RESULTS:
: [1] 0.03863065

In general, the =Upper tail= option of a tail probabilities dialog
computes \(\mathbb{P}(X > x)\) for all given =Variable Value(s)=
\(x\).  

Generate \(100,000\) observations of the random variable
\(X\).  We can randomly simulate as many observations of \(X\) as we want in \(\mathsf{R}\) Commander. Simply choose =Simulate hypergeometric variates...= in the =Hypergeometric distribution= dialog. 

In the =Number of samples= dialog, type 1. Enter the parameters as
above. Under the =Store Values= section, make sure =New Data set= is
selected. Click =OK=.

A new dialog should open, with the default name =Simset1=.  We could
change this if we like, according to the rules for \(\mathsf{R}\)
object names. In the sample size box, enter 100000. Click =OK=.

In the Console Window, \(\mathsf{R}\) Commander should issue an alert
that =Simset1= has been initialized, and in a few seconds, it should
also state that 100,000 hypergeometric variates were stored in
=hyper.sim1=. We can view the sample by clicking the =View Data Set=
button on the \(\mathsf{R}\) Commander interface.

We know from our formulas that \(\mu=K\cdot
M/(M+N)=5*17/250=0.34\). We can check our formulas using the fact that
with repeated observations of \(X\) we would expect about 0.34
defectives on the average. To see how our sample reflects the true
mean, we can compute the sample mean

:  Rcmdr> mean(Simset2$hyper.sim1, na.rm=TRUE)
:  [1] 0.340344

:  Rcmdr> sd(Simset2$hyper.sim1, na.rm=TRUE)
:  [1] 0.5584982
:  ...

We see that when given many independent observations of \(X\), the
sample mean is very close to the true mean \(\mu\). We can repeat the
same idea and use the sample standard deviation to estimate the true
standard deviation of \(X\). From the output above our estimate is
0.5584982, and from our formulas we get
\[
\sigma^{2}=K\frac{MN}{(M+N)^{2}}\frac{M+N-K}{M+N-1}\approx0.3117896,
\]
with \(\sigma=\sqrt{\sigma^{2}}\approx0.5583811944\). Our estimate was
pretty close.

From the console we can generate random hypergeometric variates with
the =rhyper= function, as demonstrated below.

#+BEGIN_SRC R :exports both :results output pp  
rhyper(10, m = 17, n = 233, k = 5)
#+END_SRC

#+RESULTS:
:  [1] 0 0 0 2 0 0 0 0 0 0

# +END_exampletoo

**** Sampling With and Without Replacement
:PROPERTIES:
:CUSTOM_ID: sub-Sampling-With-and
:END:

Suppose that we have a large urn with, say, \(M\) white balls and
\(N\) black balls. We take a sample of size \(n\) from the urn, and
let \(X\) count the number of white balls in the sample. If we sample

- without replacement, :: then
     \(X\sim\mathsf{hyper}(\mathtt{m=}M,\,\mathtt{n}=N,\,\mathtt{k}=n)\)
     and has mean and variance
     \begin{alignat*}{1}
     \mu= & n\frac{M}{M+N},\\
     \sigma^{2}= & n\frac{MN}{(M+N)^{2}}\frac{M+N-n}{M+N-1},\\
     = & n\frac{M}{M+N}\left(1-\frac{M}{M+N}\right)\frac{M+N-n}{M+N-1}.
     \end{alignat*}
     On the other hand, if we sample
- with replacement, :: then
     \(X\sim\mathsf{binom}(\mathtt{size}=n,\,\mathtt{prob}=M/(M+N))\)
     with mean and variance
     \begin{alignat*}{1}
     \mu= & n\frac{M}{M+N},\\
     \sigma^{2}= & n\frac{M}{M+N}\left(1-\frac{M}{M+N}\right).
     \end{alignat*}

We see that both sampling procedures have the same mean, and the
method with the larger variance is the "with replacement"
scheme. The factor by which the variances differ,
\begin{equation}
\frac{M+N-n}{M+N-1},
\end{equation}
is called a /finite population correction/. For a fixed sample size
\(n\), as \(M,N\to\infty\) it is clear that the correction goes to 1,
that is, for infinite populations the sampling schemes are essentially
the same with respect to mean and variance.

*** Waiting Time Distributions
:PROPERTIES:
:CUSTOM_ID: sec-Waiting-Time-Distributions
:END:

Another important class of problems is associated with the amount of
time it takes for a specified event of interest to occur. For example,
we could flip a coin repeatedly until we observe Heads. We could toss
a piece of paper repeatedly until we make it in the trash can.

**** The Geometric Distribution
:PROPERTIES:
:CUSTOM_ID: sub-The-Geometric-Distribution
:END:

Suppose that we conduct Bernoulli trials repeatedly, noting the
successes and failures. Let \(X\) be the number of failures before a
success. If \(\mathbb{P}(S)=p\) then \(X\) has PMF
\begin{equation}
f_{X}(x)=p(1-p)^{x},\quad x=0,1,2,\ldots
\end{equation}
(Why?) We say that \(X\) has a /Geometric distribution/ and we write
\(X\sim\mathsf{geom}(\mathtt{prob}=p)\). The associated \(\mathsf{R}\)
functions are =dgeom(x, prob)=, =pgeom=, =qgeom=, and =rhyper=, which
give the PMF, CDF, quantile function, and simulate random variates,
respectively.

Again it is clear that \(f(x)\geq0\) and we check that \(\sum f(x)=1\)
(see Equation \eqref{eq-geom-series} in Appendix [[#sec-Sequences-and-Series]]):
\begin{alignat*}{1}
\sum_{x=0}^{\infty}p(1-p)^{x}= & p\sum_{x=0}^{\infty}q^{x}=p\,\frac{1}{1-q}=1.
\end{alignat*}
We will find in the next section that the mean and variance are
\begin{equation}
\mu=\frac{1-p}{p}=\frac{q}{p}\mbox{ and }\sigma^{2}=\frac{q}{p^{2}}.
\end{equation}

# +BEGIN_exampletoo

The Pittsburgh Steelers place kicker, Jeff Reed, made 81.2% of his
attempted field goals in his career up to 2006. Assuming that his
successive field goal attempts are approximately Bernoulli trials,
find the probability that Jeff misses at least 5 field goals before
his first successful goal.

/Solution/: If \(X=\) the number of missed goals until Jeff's first
success, then \(X\sim\mathsf{geom}(\mathtt{prob}=0.812)\) and we want
\(\mathbb{P}(X\geq5)=\mathbb{P}(X>4)\). We can find this in
\(\mathsf{R}\) with

#+BEGIN_SRC R :exports both :results output pp  
pgeom(4, prob = 0.812, lower.tail = FALSE)
#+END_SRC

#+RESULTS:
: [1] 0.0002348493

# +END_exampletoo


#+BEGIN_note
Some books use a slightly different definition of the geometric
distribution. They consider Bernoulli trials and let \(Y\) count
instead the number of trials until a success, so that \(Y\) has PMF
\begin{equation}
f_{Y}(y)=p(1-p)^{y-1},\quad y=1,2,3,\ldots
\end{equation}
When they say "geometric distribution", this is what they mean. It
is not hard to see that the two definitions are related. In fact, if
\(X\) denotes our geometric and \(Y\) theirs, then
\(Y=X+1\). Consequently, they have \(\mu_{Y}=\mu_{X}+1\) and
\(\sigma_{Y}^{2}=\sigma_{X}^{2}\).
#+END_note

**** The Negative Binomial Distribution
:PROPERTIES:
:CUSTOM_ID: sub-The-Negative-Binomial
:END:

We may generalize the problem and consider the case where we wait for
/more/ than one success. Suppose that we conduct Bernoulli trials
repeatedly, noting the respective successes and failures. Let \(X\)
count the number of failures before \(r\) successes. If
\(\mathbb{P}(S)=p\) then \(X\) has PMF
\begin{equation}
f_{X}(x)={r+x-1 \choose r-1}\, p^{r}(1-p)^{x},\quad x=0,1,2,\ldots
\end{equation}

We say that \(X\) has a /Negative Binomial distribution/ and write
\(X\sim\mathsf{nbinom}(\mathtt{size}=r,\,\mathtt{prob}=p)\). The
associated \(\mathsf{R}\) functions are =dnbinom(x, size, prob)=,
=pnbinom=, =qnbinom=, and =rnbinom=, which give the PMF, CDF, quantile
function, and simulate random variates, respectively.

As usual it should be clear that \(f_{X}(x)\geq 0\) and the fact that
\(\sum f_{X}(x)=1\) follows from a generalization of the geometric
series by means of a Maclaurin's series expansion:
\begin{alignat}{1}
\frac{1}{1-t}= & \sum_{k=0}^{\infty}t^{k},\quad \mbox{for \(-1 < t < 1\)},\mbox{ and}\\
\frac{1}{(1-t)^{r}}= & \sum_{k=0}^{\infty}{r+k-1 \choose r-1}\, t^{k},\quad \mbox{for \(-1 < t < 1\)}.
\end{alignat}
Therefore
\begin{equation}
\sum_{x=0}^{\infty}f_{X}(x)=p^{r}\sum_{x=0}^{\infty}{r+x-1 \choose r-1}\, q^{x}=p^{r}(1-q)^{-r}=1,
\end{equation}
since \(|q|=|1-p|<1\). 

# +BEGIN_exampletoo

We flip a coin repeatedly and let \(X\) count the number of Tails
until we get seven Heads. What is \(\mathbb{P}(X=5)?\) /Solution/: We
know that
\(X\sim\mathsf{nbinom}(\mathtt{size}=7,\,\mathtt{prob}=1/2)\).  
\[
\mathbb{P}(X=5)=f_{X}(5)={7+5-1 \choose 7-1}(1/2)^{7}(1/2)^{5}={11
\choose 6}2^{-12} 
\]
and we can get this in \(\mathsf{R}\) with

#+BEGIN_SRC R :exports both :results output pp  
dnbinom(5, size = 7, prob = 0.5)
#+END_SRC

#+RESULTS:
: [1] 0.112793

Let us next compute the MGF of
\(X\sim\mathsf{nbinom}(\mathtt{size}=r,\,\mathtt{prob}=p)\).
\begin{alignat*}{1}
M_{X}(t)= & \sum_{x=0}^{\infty}\mathrm{e}^{tx}\ {r+x-1 \choose r-1}p^{r}q^{x}\\
= & p^{r}\sum_{x=0}^{\infty}{r+x-1 \choose r-1}[q\mathrm{e}^{t}]^{x}\\
= & p^{r}(1-qe^{t})^{-r},\quad \mbox{provided $|q\mathrm{e}^{t}|<1$,}
\end{alignat*}
and so
\begin{equation}
M_{X}(t)=\left(\frac{p}{1-q\mathrm{e}^{t}}\right)^{r},\quad \mbox{for $q\mathrm{e}^{t}<1$}.
\end{equation}
We see that \(q\mathrm{e}^{t}<1\) when \(t<-\ln(1-p)\).

Let \(X\sim\mathsf{nbinom}(\mathtt{size}=r,\mathtt{prob}=p)\mbox{ with
$M(t)=p^{r}(1-q\mathrm{e}^{t})^{-r}$}\). We proclaimed above the
values of the mean and variance. Now we are equipped with the tools to
find these directly.
\begin{alignat*}{1}
M'(t)= & p^{r}(-r)(1-q\mathrm{e}^{t})^{-r-1}(-q\mathrm{e}^{t}),\\
= & rq\mathrm{e}^{t}p^{r}(1-q\mathrm{e}^{t})^{-r-1},\\
= & \frac{rq\mathrm{e}^{t}}{1-q\mathrm{e}^{t}}M(t),\mbox{ and so }\\
M'(0)= & \frac{rq}{1-q}\cdot1=\frac{rq}{p}.
\end{alignat*}
Thus \(\mu=rq/p\). We next find \(\mathbb{E} X^{2}\).
\begin{alignat*}{1}
M''(0)= & \left.\frac{rq\mathrm{e}^{t}(1-q\mathrm{e}^{t})-rq\mathrm{e}^{t}(-q\mathrm{e}^{t})}{(1-q\mathrm{e}^{t})^{2}}M(t)+\frac{rq\mathrm{e}^{t}}{1-q\mathrm{e}^{t}}M'(t)\right|_{t=0},\\
= & \frac{rqp+rq^{2}}{p^{2}}\cdot1+\frac{rq}{p}\left(\frac{rq}{p}\right),\\
= & \frac{rq}{p^{2}}+\left(\frac{rq}{p}\right)^{2}.
\end{alignat*}
Finally we may say \( \sigma^{2} = M''(0) - [M'(0)]^{2} = rq/p^{2}. \)
# +END_exampletoo

# +BEGIN_exampletoo

A random variable has MGF
\[
M_{X}(t)=\left(\frac{0.19}{1-0.81\mathrm{e}^{t}}\right)^{31}.
\]
Then \(X\sim\mathsf{nbinom}(\mathtt{size}=31,\,\mathtt{prob}=0.19)\).
# +END_exampletoo


#+BEGIN_note
As with the Geometric distribution, some books use a slightly
different definition of the Negative Binomial distribution. They
consider Bernoulli trials and let \(Y\) be the number of trials until
\(r\) successes, so that \(Y\) has PMF
\begin{equation}
f_{Y}(y)={y-1 \choose r-1}p^{r}(1-p)^{y-r},\quad y=r,r+1,r+2,\ldots
\end{equation}
It is again not hard to see that if \(X\) denotes our Negative
Binomial and \(Y\) theirs, then \(Y=X+r\). Consequently, they have
\(\mu_{Y}=\mu_{X}+r\) and \(\sigma_{Y}^{2}=\sigma_{X}^{2}\).
#+END_note

*** Arrival Processes
:PROPERTIES:
:CUSTOM_ID: sec-Arrival-Processes
:END:

**** The Poisson Distribution
:PROPERTIES:
:CUSTOM_ID: sub-The-Poisson-Distribution
:END:

This is a distribution associated with "rare events", for reasons
which will become clear in a moment. The events might be:
- traffic accidents,
- typing errors, or
- customers arriving in a bank.

Let \(\lambda\) be the average number of events in the time interval
\([0,1]\). Let the random variable \(X\) count the number of events
occurring in the interval. Then under certain reasonable conditions it
can be shown that
\begin{equation}
f_{X}(x)=\mathbb{P}(X=x)=\mathrm{e}^{-\lambda}\frac{\lambda^{x}}{x!},\quad x=0,1,2,\ldots
\end{equation}
We use the notation
\(X\sim\mathsf{pois}(\mathtt{lambda}=\lambda)\). The associated
\(\mathsf{R}\) functions are =dpois(x, lambda)=, =ppois=, =qpois=, and
=rpois=, which give the PMF, CDF, quantile function, and simulate
random variates, respectively.

**** What are the reasonable conditions?

Divide \([0,1]\) into subintervals of length \(1/n\). A /Poisson
process/ @@latex:\index{Poisson process}@@ satisfies the following conditions:
- the probability of an event occurring in a particular subinterval is
  \(\approx\lambda/n\).
- the probability of two or more events occurring in any subinterval
  is \(\approx 0\).
- occurrences in disjoint subintervals are independent.

#+BEGIN_rem
<<rem-poisson-process>> If \(X\) counts the number of events in the
interval \([0,t]\) and \(\lambda\) is the average number that occur in
unit time, then \(X\sim\mathsf{pois}(\mathtt{lambda}=\lambda t)\),
that is,
\begin{equation}
\mathbb{P}(X=x)=\mathrm{e}^{-\lambda t}\frac{(\lambda t)^{x}}{x!},\quad x=0,1,2,3\ldots
\end{equation}
#+END_rem

# +BEGIN_exampletoo
On the average, five cars arrive at a particular car wash every
hour. Let \(X\) count the number of cars that arrive from 10AM to
11AM. Then \(X\sim\mathsf{pois}(\mathtt{lambda}=5)\). Also,
\(\mu=\sigma^{2}=5\). What is the probability that no car arrives
during this period?  /Solution/: The probability that no car arrives
is \[
\mathbb{P}(X=0)=\mathrm{e}^{-5}\frac{5^{0}}{0!}=\mathrm{e}^{-5}\approx0.0067.
\]
# +END_exampletoo


# +BEGIN_exampletoo

Suppose the car wash above is in operation from 8AM to 6PM, and we let
\(Y\) be the number of customers that appear in this period. Since
this period covers a total of 10 hours, from Remark [[rem-poisson-process]] we
get that \(Y\sim\mathsf{pois}(\mathtt{lambda}=5\ast10=50)\). What is
the probability that there are between 48 and 50 customers, inclusive?
/Solution/: We want \(\mathbb{P}(48\leq
Y\leq50)=\mathbb{P}(X\leq50)-\mathbb{P}(X\leq47)\).

#+BEGIN_SRC R :exports both :results output pp  
diff(ppois(c(47, 50), lambda = 50))
#+END_SRC

#+RESULTS:
: [1] 0.1678485

# +END_exampletoo

** Functions of Discrete Random Variables
:PROPERTIES:
:CUSTOM_ID: sec-functions-discrete-rvs
:END:

We have built a large catalogue of discrete distributions, but the
tools of this section will give us the ability to consider infinitely
many more. Given a random variable \(X\) and a given function \(h\),
we may consider \(Y=h(X)\). Since the values of \(X\) are determined
by chance, so are the values of \(Y\). The question is, what is the
PMF of the random variable \(Y\)? The answer, of course, depends on
\(h\). In the case that \(h\) is one-to-one (see Appendix
[[#sec-Differential-and-Integral]]), the solution can be found by simple
substitution.

# +BEGIN_exampletoo

Let \(X\sim\mathsf{nbinom}(\mathtt{size}=r,\,\mathtt{prob}=p)\). We
saw in [[#sec-other-discrete-distributions]] that \(X\) represents the number
of failures until \(r\) successes in a sequence of Bernoulli
trials. Suppose now that instead we were interested in counting the
number of trials (successes and failures) until the
\(r^{\mathrm{th}}\) success occurs, which we will denote by \(Y\). In
a given performance of the experiment, the number of failures (\(X\))
and the number of successes (\(r\)) together will comprise the total
number of trials (\(Y\)), or in other words, \(X+r=Y\). We may let
\(h\) be defined by \(h(x)=x+r\) so that \(Y=h(X)\), and we notice
that \(h\) is linear and hence one-to-one. Finally, \(X\) takes values
\(0,\ 1,\ 2,\ldots\) implying that the support of \(Y\) would be \(\{
r,\ r+1,\ r+2,\ldots \}\). Solving for \(X\) we get
\(X=Y-r\). Examining the PMF of \(X\)
\begin{equation}
f_{X}(x)={r+x-1 \choose r-1}\, p^{r}(1-p)^{x},
\end{equation}
we can substitute \( x = y - r \) to get
\begin{eqnarray*}
f_{Y}(y) & = & f_{X}(y-r),\\
 & = & {r+(y-r)-1 \choose r-1}\, p^{r}(1-p)^{y-r},\\
 & = & {y-1 \choose r-1}\, p^{r}(1-p)^{y-r},\quad y=r,\, r+1,\ldots
\end{eqnarray*}
# +END_exampletoo

Even when the function \(h\) is not one-to-one, we may still find the
PMF of \(Y\) simply by accumulating, for each \(y\), the probability
of all the \(x\)'s that are mapped to that \(y\).
#+BEGIN_prop
Let \(X\) be a discrete random variable with PMF \(f_{X}\) supported
on the set \(S_{X}\). Let \(Y=h(X)\) for some function \(h\). Then
\(Y\) has PMF \(f_{Y}\) defined by
\begin{equation}
f_{Y}(y)=\sum_{\{x\in S_{X}|\, h(x)=y\}}f_{X}(x)
\end{equation}
#+END_prop

# +BEGIN_exampletoo

Let \(X\sim\mathsf{binom}(\mathtt{size}=4,\,\mathtt{prob}=1/2)\), and let \(Y=(X-1)^{2}\). Consider the following table:

#+NAME: tab-disc-transf
#+CAPTION[Transform discrete random variable]: Transforming a discrete random variable.
| x               |    0 |   1 |    2 |   3 |    4 |
|-----------------+------+-----+------+-----+------|
| \(f_{X}(x)\)    | 1/16 | 1/4 | 6/16 | 1/4 | 1/16 |
|-----------------+------+-----+------+-----+------|
| \(y=(x-1)^{2}\) |    1 |   0 |    1 |   4 |    9 |

From this we see that \(Y\) has support \(S_{Y}=\{0,1,4,9\}\). We also
see that \(h(x)=(x-1)^{2}\) is not one-to-one on the support of \(X\),
because both \(x=0\) and \(x=2\) are mapped by \(h\) to
\(y=1\). Nevertheless, we see that \(Y=0\) only when \(X=1\), which
has probability \(1/4\); therefore, \(f_{Y}(0)\) should equal
\(1/4\). A similar approach works for \(y=4\) and \(y=9\). And \(Y=1\)
exactly when \(X=0\) or \(X=2\), which has total probability
\(7/16\). In summary, the PMF of \(Y\) may be written:

#+NAME: tab-disc-transf-pmf
#+CAPTION[Transforming discrete random variable: PMF]: Transforming a discrete random variable, its PMF.
| y            |   0 |    1 |   4 |    9 |
|--------------+-----+------+-----+------|
| \(f_{Y}(y)\) | 1/4 | 7/16 | 1/4 | 1/16 |

There is not a special name for the distribution of \(Y\), it is just
an example of what to do when the transformation of a random variable
is not one-to-one. The method is the same for more complicated
problems.
# +END_exampletoo


#+BEGIN_prop
If \(X\) is a random variable with \(\mathbb{E} X=\mu\) and \(\mbox{Var}(X)=\sigma^{2}\), then the mean and variance of \(Y=mX+b\) is
\begin{equation}
\mu_{Y}=m\mu+b,\quad \sigma_{Y}^{2}=m^{2}\sigma^{2},\quad \sigma_{Y}=|m|\sigma.
\end{equation}
#+END_prop

#+LaTeX: \newpage{}

** Exercises
#+LaTeX: \setcounter{thm}{0}

#+BEGIN_xca
A recent national study showed that approximately 44.7% of college
students have used Wikipedia as a source in at least one of their term
papers. Let \(X\) equal the number of students in a random sample of
size \(n=31\) who have used Wikipedia as a source.
- How is \(X\) distributed? 
- Sketch the probability mass function (roughly).
- Sketch the cumulative distribution function (roughly).
- Find the probability that \(X\) is equal to 17.
- Find the probability that \(X\) is at most 13.
- Find the probability that \(X\) is bigger than 11.
- Find the probability that \(X\) is at least 15.
- Find the probability that \(X\) is between 16 and 19, inclusive.
- Give the mean of \(X\), denoted \(\mathbb{E} X\).
- Give the variance of \(X\).
- Give the standard deviation of \(X\).
- Find \(\mathbb{E}(4X + 51.324)\).
#+END_xca

#+BEGIN_xca
For the following situations, decide what the distribution of \(X\)
should be. In nearly every case, there are additional assumptions that
should be made for the distribution to apply; identify those
assumptions (which may or may not hold in practice.)
- We shoot basketballs at a basketball hoop, and count the number of
  shots until we make a goal. Let \(X\) denote the number of missed
  shots. On a normal day we would typically make about 37% of the
  shots.
- In a local lottery in which a three digit number is selected
  randomly, let \(X\) be the number selected.
- We drop a Styrofoam cup to the floor twenty times, each time
  recording whether the cup comes to rest perfectly right side up, or
  not. Let \(X\) be the number of times the cup lands perfectly right
  side up.
- We toss a piece of trash at the garbage can from across the room. If
  we miss the trash can, we retrieve the trash and try again,
  continuing to toss until we make the shot. Let \(X\) denote the
  number of missed shots.
- Working for the border patrol, we inspect shipping cargo as when it
  enters the harbor looking for contraband. A certain ship comes to
  port with 557 cargo containers. Standard practice is to select 10
  containers randomly and inspect each one very carefully, classifying
  it as either having contraband or not. Let \(X\) count the number of
  containers that illegally contain contraband.
- At the same time every year, some migratory birds land in a bush
  outside for a short rest. On a certain day, we look outside and let
  \(X\) denote the number of birds in the bush.
- We count the number of rain drops that fall in a circular area on a
  sidewalk during a ten minute period of a thunder storm.
- We count the number of moth eggs on our window screen.
- We count the number of blades of grass in a one square foot patch of
  land.
- We count the number of pats on a baby's back until (s)he burps.
#+END_xca

#+BEGIN_xca
<<xca-variance-shortcut>> Show that \(\mathbb{E}(X - \mu)^{2} =
\mathbb{E} X^{2} - \mu^{2}\). /Hint/: expand the quantity \((X -
\mu)^{2}\) and distribute the expectation over the resulting terms.
#+END_xca

#+BEGIN_xca
<<xca-binom-factorial-expectation>> If
\(X\sim\mathsf{binom}(\mathtt{size}=n,\,\mathtt{prob}=p)\) show that
\(\mathbb{E} X(X - 1) = n(n - 1)p^{2}\).
#+END_xca

#+BEGIN_xca
<<xca-hyper-mean-variance>> Calculate the mean and variance of the
hypergeometric distribution. Show that
\begin{equation}
\mu=K\frac{M}{M + N},\quad \sigma^{2} = K\frac{MN}{(M + N)^{2}}\frac{M + N - K}{M + N - 1}.
\end{equation}
#+END_xca
