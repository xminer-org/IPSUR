* Estimation                                                       :estimate:
:PROPERTIES:
:tangle: R/09-estimate.R
:CUSTOM_ID: cha-Estimation
:END:

#+BEGIN_SRC R :exports none :eval never
#    IPSUR: Introduction to Probability and Statistics Using R
#    Copyright (C) 2014  G. Jay Kerns
#
#    Chapter: Estimation
#
#    This file is part of IPSUR.
#
#    IPSUR is free software: you can redistribute it and/or modify
#    it under the terms of the GNU General Public License as published by
#    the Free Software Foundation, either version 3 of the License, or
#    (at your option) any later version.
#
#    IPSUR is distributed in the hope that it will be useful,
#    but WITHOUT ANY WARRANTY; without even the implied warranty of
#    MERCHANTABILITY or FITNESS FOR A PARTICULAR PURPOSE.  See the
#    GNU General Public License for more details.
#
#    You should have received a copy of the GNU General Public License
#    along with IPSUR.  If not, see <http://www.gnu.org/licenses/>.
#+END_SRC

#+BEGIN_SRC R :exports none :eval no-export
# This chapter's package dependencies
library(TeachingDemos)
library(Hmisc)
library(RcmdrPlugin.IPSUR)
library(reshape)
#+END_SRC

#+LaTeX: \noindent 
We will discuss two branches of estimation procedures: point
estimation and interval estimation. We briefly discuss point
estimation first and then spend the rest of the chapter on interval
estimation.

We find an estimator with the methods of Section
[[#sec-Point-Estimation-1]]. We make some assumptions about the
underlying population distribution and use what we know from Chapter
[[#cha-Sampling-Distributions]] about sampling
distributions both to study how the estimator will perform, and to
find intervals of confidence for underlying parameters associated with
the population distribution. Once we have confidence intervals we can
do inference in the form of hypothesis tests in the next chapter.

*What do I want them to know?*
- how to look at a problem, identify a reasonable model, and estimate
  a parameter associated with the model
- about maximum likelihood, and in particular, how to
   - eyeball a likelihood to get a maximum
   - use calculus to find an MLE for one-parameter families
- about properties of the estimators they find, such as bias, minimum
  variance, MSE
- point versus interval estimation, and how to find and interpret
  confidence intervals for basic experimental designs
- the concept of margin of error and its relationship to sample size

** Point Estimation
:PROPERTIES:
:CUSTOM_ID: sec-Point-Estimation-1
:END:

The following example is how I was introduced to maximum likelihood.

# +BEGIN_exampletoo
<<exa-how-many-fish>> Suppose we have a small pond in our backyard,
and in the pond there live some fish. We would like to know how many
fish live in the pond. How can we estimate this? One procedure
developed by researchers is the capture-recapture method. Here is how
it works.

We will fish from the pond and suppose that we capture \(M=7\)
fish. On each caught fish we attach an unobtrusive tag to the fish's
tail, and release it back into the water.

Next, we wait a few days for the fish to remix and become accustomed
to their new tag. Then we go fishing again. On the second trip some of
the fish we catch may be tagged; some may not be. Let \(X\) denote the
number of caught fish which are tagged[fn:fn-samefish], and suppose
for the sake of argument that we catch \(K=4\) fish and we find that 3
of them are tagged.

[fn:fn-samefish] It is theoretically possible that we could catch the
same tagged fish more than once, which would inflate our count of
tagged fish. To avoid this difficulty, suppose that on the second trip
we use a tank on the boat to hold the caught fish until data
collection is completed.

Now let \(F\) denote the (unknown) total number of fish in the
pond. We know that \(F\geq7\), because we tagged that many on the
first trip. In fact, if we let \(N\) denote the number of untagged
fish in the pond, then \(F=M+N\). Using our earlier language, we have
sampled \(K=4\) times, without replacement, from an urn (pond) which
has \(M=7\) white balls (tagged fish) and \(N=F-M\) black balls
(untagged fish), and we have observed \(x=3\) of them to be white
(tagged). What is the probability of this?

Looking back to Section [[#sec-other-discrete-distributions]], we see that the
random variable \(X\) has a
\(\mathsf{hyper}(\mathtt{m}=M,\,\mathtt{n}=F-M,\,\mathtt{k}=K)\)
distribution. Therefore, for an observed value \(X=x\) the probability
would be \[ \mathbb{P}(X=x)=\frac{{M \choose x}{F-M \choose K-x}}{{F
\choose K}}.  \] First we notice that \(F\) must be at least 7. Could
\(F\) be equal to seven? If \(F=7\) then all of the fish would have
been tagged on the first run, and there would be no untagged fish in
the pond, thus, \(\mathbb{P}(\mbox{3 successes in 4 trials})=0\).
What about \(F=8\); what would be the probability of observing \(X=3\)
tagged fish?  \[ \mathbb{P}(\mbox{3 successes in 4 trials})=\frac{{7
\choose 3}{1 \choose 1}}{{8 \choose 4}}=\frac{35}{70}=0.5.  \]
Similarly, if \(F=9\) then the probability of observing \(X=3\) tagged
fish would be \[ \mathbb{P}(\mbox{3 successes in 4 trials})=\frac{{7
\choose 3}{2 \choose 1}}{{9 \choose 4}}=\frac{70}{126}\approx0.556.
\] We can see already that the observed data \(X=3\) is more likely
when \(F=9\) than it is when \(F=8\). And here lies the genius of Sir
Ronald Aylmer Fisher: he asks, "What is the value of \(F\) which has
the highest likelihood?" In other words, for all of the different
possible values of \(F\), which one makes the above probability the
biggest? We can answer this question with a plot of
\(\mathbb{P}(X=x)\) versus \(F\). See Figure
[[fig-capture-recapture]].
# +END_exampletoo

#+NAME: capture-recapture
#+BEGIN_SRC R :exports results :results graphics :file fig/estimate-capture-recapture.ps
heights = rep(0, 16)
for (j in 7:15) heights[j] <- dhyper(3, m = 7, n = j - 7, k = 4)
plot(6:15, heights[6:15], pch = 16, cex = 1.5, xlab = "number of fish in pond", ylab = "Likelihood")
abline(h = 0)
lines(6:15, heights[6:15], type = "h", lwd = 2, lty = 3)
text(9, heights[9]/6, bquote(hat(F)==.(9)), cex = 2, pos = 4)
lines(9, heights[9], type = "h", lwd = 2)
points(9, 0, pch = 4, lwd = 3, cex = 2)
#+END_SRC

#+NAME: fig-capture-recapture
#+CAPTION[Capture-recapture experiment]: \small A plot of maximum likelihood for the capture-recapture experiment.
#+ATTR_LaTeX: :width 0.9\textwidth :placement [ht!]
#+RESULTS: capture-recapture
[[file:fig/estimate-capture-recapture.ps]]


# +BEGIN_exampletoo
<<exa-bass-bluegill>> In the last example we were only concerned with
how many fish were in the pond, but now, we will ask a different
question. Suppose it is known that there are only two species of fish
in the pond: smallmouth bass (/Micropterus dolomieu/) and bluegill
(/Lepomis macrochirus/); perhaps we built the pond some years ago and
stocked it with only these two species. We would like to estimate the
proportion of fish in the pond which are bass.

Let \(p=\mbox{the proportion of bass}\). Without any other
information, it is conceivable for \(p\) to be any value in the
interval \([0,1]\), but for the sake of argument we will suppose that
\(p\) falls strictly between zero and one. How can we learn about the
true value of \(p\)? Go fishing! As before, we will use
catch-and-release, but unlike before, we will not tag the fish. We
will simply note the species of any caught fish before returning it to
the pond.

Suppose we catch \(n\) fish. Let @@latex:\[ X_{i} = \begin{cases} 1, &
\mbox{if the \(i\mathrm{th}\) fish is a bass,}\\ 0, & \mbox{if the
\(i\mathrm{th}\) fish is a bluegill.} \end{cases} \]@@ Since we are
returning the fish to the pond once caught, we may think of this as a
sampling scheme with replacement where the proportion of bass \(p\)
does not change. Given that we allow the fish sufficient time to "mix"
once returned, it is not completely unreasonable to model our fishing
experiment as a sequence of Bernoulli trials, so that the \(X_{i}\)'s
would be IID
\(\mathsf{binom(\mathtt{size}}=1,\,\mathtt{prob}=p)\). Under those
assumptions we would have
\begin{eqnarray*}
\mathbb{P}(X_{1}=x_{1},\, X_{2}=x_{2},\,\ldots,\, X_{n}=x_{n}) & = & \mathbb{P}(X_{1}=x_{1})\,\mathbb{P}(X_{2}=x_{2})\,\cdots\mathbb{P}(X_{n}=x_{n}),\\
 & = & p^{x_{1}}(1-p)^{x_{1}}\, p^{x_{2}}(1-p)^{x_{2}}\cdots\, p^{x_{n}}(1-p)^{x_{n}},\\
 & = & p^{\sum x_{i}}(1-p)^{n-\sum x_{i}}.
\end{eqnarray*}
That is, \[ \mathbb{P}(X_{1}=x_{1},\, X_{2}=x_{2},\,\ldots,\,
X_{n}=x_{n})=p^{\sum x_{i}}(1-p)^{n-\sum x_{i}}.  \] This last
quantity is a function of \(p\), called the /likelihood function/
\(L(p)\): \[ L(p)=p^{\sum x_{i}}(1-p)^{n-\sum x_{i}}.  \] A graph of
\(L\) for values of \(\sum x_{i}=3,\ 4\), and 5 when \(n=7\) is shown
in Figure [[fig-fishing-part-two]].

#+NAME: fishing-part-two
#+BEGIN_SRC R :exports results :results graphics :file fig/estimate-fishing-part-two.ps
curve(x^5*(1-x)^2, 0, 1, xlab = "p", ylab = "L(p)")
curve(x^4*(1-x)^3, 0, 1, add = TRUE)
curve(x^3*(1-x)^4, 0, 1, add = TRUE)
#+END_SRC

#+NAME: fig-fishing-part-two
#+CAPTION[Assorted likelihood functions for fishing, part two]: \small Assorted likelihood functions for fishing, part two.   Three graphs are shown of \(L\) when \(\sum x_{i}\) equals 3, 4, and 5, respectively, from left to right. We pick an \(L\) that matches the observed data and then maximize \(L\) as a function of \(p\). If \(\sum x_{i}=4\), then the maximum appears to occur somewhere around \(p \approx 0.6\).
#+ATTR_LaTeX: :width 0.9\textwidth :placement [ht!]
#+RESULTS: fishing-part-two
[[file:fig/estimate-fishing-part-two.ps]]


We want the value of \(p\) which has the highest likelihood, that is,
we again wish to maximize the likelihood. We know from calculus (see
Appendix [[#sec-Differential-and-Integral]]) to differentiate \(L\)
and set \(L'=0\) to find a maximum.  \[ L'(p)=\left(\sum
x_{i}\right)p^{\sum x_{i}-1}(1-p)^{n-\sum x_{i}}+p^{\sum
x_{i}}\left(n-\sum x_{i}\right)(1-p)^{n-\sum x_{i}-1}(-1).  \] The
derivative vanishes (\(L'=0\)) when
\begin{eqnarray*}
\left(\sum x_{i}\right)p^{\sum x_{i}-1}(1-p)^{n-\sum x_{i}} & = & p^{\sum x_{i}}\left(n-\sum x_{i}\right)(1-p)^{n-\sum x_{i}-1},\\
\sum x_{i}(1-p) & = & \left(n-\sum x_{i}\right)p,\\
\sum x_{i}-p\sum x_{i} & = & np-p\sum x_{i},\\
\frac{1}{n}\sum_{i=1}^{n}x_{i} & = & p.
\end{eqnarray*}
This "best" \(p\), the one which maximizes the likelihood, is called
the maximum likelihood estimator (MLE) of \(p\) and is denoted
\(\hat{p}\). That is,
\begin{equation} 
\hat{p}=\frac{\sum_{i=1}^{n}x_{i}}{n}=\overline{x}.
\end{equation}

#+BEGIN_rem
Strictly speaking we have only shown that the derivative equals zero
at \(\hat{p}\), so it is theoretically possible that the critical
value \(\hat{p}=\overline{x}\) is located at a minimum[fn:fn-sdt]
instead of a maximum! We should be thorough and check that \(L'>0\)
when \(p<\overline{x}\) and \(L'<0\) when \(p>\overline{x}\). Then by
the First Derivative Test (Theorem [[thm-First-Derivative-Test]]) we could
be certain that \(\hat{p}=\overline{x}\) is indeed a maximum
likelihood estimator, and not a minimum likelihood estimator.
#+END_rem

The result is shown in Figure [[fig-species-mle]].
# +END_exampletoo

[fn:fn-sdt] We can tell from the graph that our value of \(\hat{p}\)
is a maximum instead of a minimum so we do not really need to worry
for this example. Other examples are not so easy, however, and we
should be careful to be cognizant of this extra step.

#+NAME: species-mle
#+BEGIN_SRC R :exports results :results graphics :file fig/estimate-species-mle.ps
dat <- rbinom(27, size = 1, prob = 0.3)
like <- function(x){
r <- 1
for (k in 1:27){ r <- r*dbinom(dat[k], size = 1, prob = x)}
return(r)
}
curve(like, from = 0, to = 1, xlab = "parameter space", ylab = "Likelihood", lwd = 3, col = "blue")
abline(h = 0, lwd = 1, lty = 3, col = "grey")
mle <- mean(dat)
mleobj <- like(mle)
lines(mle, mleobj, type = "h", lwd = 2, lty = 3, col = "red")
points(mle, 0, pch = 4, lwd = 2, cex = 2, col = "red")
text(mle, mleobj/6, substitute(hat(theta)==a, list(a=round(mle, 4))), cex = 2, pos = 4)
#+END_SRC

#+NAME: fig-species-mle
#+CAPTION[Species maximum likelihood]: \small Species maximum likelihood.
#+ATTR_LaTeX: :width 0.9\textwidth :placement [ht!]
#+RESULTS: species-mle
[[file:fig/estimate-species-mle.ps]]

In general, we have a family of PDFs \(f(x|\theta)\) indexed by a
parameter \(\theta\) in some parameter space \(\Theta\). We want to
learn about \(\theta\). We take a \(SRS(n)\):
\begin{equation}
X_{1},\, X_{2},\,\ldots,X_{n}\mbox{ which are IID \( f(x| \theta ) \).}
\end{equation}

#+BEGIN_defn
Given the observed data \(x_{1}\), \(x_{2}\),..., \(x_{n}\), the
/likelihood function/ \(L\) is defined by \[
L(\theta)=\prod_{i=1}^{n}f(x_{i}|\theta),\quad \theta\in\Theta.  \]
#+END_defn

The next step is to maximize \(L\). The method we will use in this
book is to find the derivative \(L'\) and solve the equation
\(L'(\theta)=0\). Call a solution \(\hat{\theta}\). We will check that
\(L\) is maximized at \(\hat{\theta}\) using the First Derivative Test
or the Second Derivative Test \(\left(L''(\hat{\theta})<0\right)\).

#+BEGIN_defn
A value \(\theta\) that maximizes \(L\) is called a /maximum
likelihood estimator/ (MLE) and is denoted \(\hat{\theta}\). It is a
function of the sample, \(\hat{\theta}=\hat{\theta}\left(X_{1},\,
X_{2},\,\ldots,X_{n}\right)\), and is called a /point estimator/ of
\(\theta\).
#+END_defn

#+BEGIN_rem
Some comments about maximum likelihood estimators:
- Often it is easier to maximize the /log-likelihood/ \(l(\theta)=\ln
  L(\theta)\) instead of the likelihood \(L\). Since the logarithmic
  function \(y=\ln x\) is a monotone transformation, the solutions to
  both problems are the same.
- MLEs do not always exist (for instance, sometimes the likelihood has
  a vertical asymptote), and even when they do exist, they are not
  always unique (imagine a function with a bunch of humps of equal
  height). For any given problem, there could be zero, one, or any
  number of values of \(\theta\) for which \(L(\theta)\) is a maximum.
- The problems we encounter in this book are all very nice with
  likelihood functions that have closed form representations and which
  are optimized by some calculus acrobatics. In practice, however,
  likelihood functions are sometimes nasty in which case we are
  obliged to use numerical methods to find maxima (if there are any).
- MLEs are just one of _many_ possible estimators. One of the more
  popular alternatives are the /method of moments estimators/; see
  Casella and Berger \cite{Casella2002} for more.
#+END_rem

Notice, in Example [[exa-bass-bluegill]] we had \(X_{i}\) IID
\(\mathsf{binom}(\mathtt{size}=1,\,\mathtt{prob}=p)\), and we saw that
the MLE was \(\hat{p}=\overline{X}\). But further
\begin{eqnarray*}
\mathbb{E}\overline{X} & = & \mathbb{E}\frac{X_{1}+X_{2}+\cdots+X_{n}}{n},\\
 & = & \frac{1}{n}\left(\mathbb{E} X_{1}+\mathbb{E} X_{2}+\cdots+\mathbb{E} X_{n}\right),\\
 & = & \frac{1}{n}\left(np\right),\\
 & = & p,
\end{eqnarray*}
which is exactly the same as the parameter which we estimated. More
concisely, \(\mathbb{E}\hat{p}=p\), that is, on the average, the
estimator is exactly right.

#+BEGIN_defn
Let \(s(X_{1},X_{2},\ldots,X_{n})\) be a statistic which estimates
\(\theta\). If \[ \mathbb{E} s(X_{1},X_{2},\ldots,X_{n})=\theta, \]
then the statistic \(s(X_{1},X_{2},\ldots,X_{n})\) is said to be an
/unbiased estimator/ of \(\theta\). Otherwise, it is /biased/.
#+END_defn

# +BEGIN_exampletoo
<<exa-normal-MLE-both>> Let \(X_{1}\), \(X_{2}\), ... , \(X_{n}\) be
an \(SRS(n)\) from a
\(\mathsf{norm}(\mathtt{mean}=\mu,\,\mathtt{sd}=\sigma)\)
distribution. It can be shown (in Exercise [[xca-norm-mu-sig-MLE]]) that
if \(\mbox{$\theta$}=(\mu,\sigma^{2})\) then the MLE of \(\theta\) is
\begin{equation}
\hat{\theta}=(\hat{\mu},\hat{\sigma}^{2}),
\end{equation}
where \(\hat{\mu}=\overline{X}\) and
\begin{equation}
\hat{\sigma^{2}}=\frac{1}{n}\sum_{i=1}^{n}\left(X_{i}-\overline{X}\right)^{2}=\frac{n-1}{n}S^{2}.
\end{equation}
We of course know from [[pro-mean-sd-xbar]] that \(\hat{\mu}\) is
unbiased. What about \(\hat{\sigma^{2}}\)? Let us check:
\begin{eqnarray*}
\mathbb{E}\,\hat{\sigma^{2}} & = & \mathbb{E}\,\frac{n-1}{n}S^{2}\\
 & = & \mathbb{E}\left(\frac{\sigma^{2}}{n}\frac{(n-1)S^{2}}{\sigma^{2}}\right)\\
 & = & \frac{\sigma^{2}}{n}\mathbb{E}\ \mathsf{chisq}(\mathtt{df}=n-1)\\
 & = & \frac{\sigma^{2}}{n}(n-1),
\end{eqnarray*}
from which we may conclude two things:
- \(\hat{\sigma^{2}}\) is a biased estimator of \(\sigma^{2}\), and
- \(S^{2}=n\hat{\sigma^{2}}/(n-1)\) is an unbiased estimator of
  \(\sigma^{2}\).

# +END_exampletoo


One of the most common questions in an introductory statistics class
is, "Why do we divide by \(n-1\) when we compute the sample variance?
Why do we not divide by \(n\)?" We see now that division by \(n\)
amounts to the use of a /biased/ estimator for \(\sigma^{2}\), that
is, if we divided by \(n\) then on the average we would
/underestimate/ the true value of \(\sigma^{2}\). We use \(n-1\) so
that, on the average, our estimator of \(\sigma^{2}\) will be exactly
right.

*** How to do it with \(\mathsf{R}\)

\(\mathsf{R}\) can be used to find maximum likelihood estimators in a
lot of diverse settings. We will discuss only the most basic here and
will leave the rest to more sophisticated texts.

For one parameter estimation problems we may use the =optimize=
function to find MLEs. The arguments are the function to be maximized
(the likelihood function), the range over which the optimization is to
take place, and optionally any other arguments to be passed to the
likelihood if needed.

Let us see how to do Example [[exa-bass-bluegill]]. Recall that our
likelihood function was given by
\begin{equation}
L(p)=p^{\sum x_{i}}(1-p)^{n-\sum x_{i}}.
\end{equation}
Notice that the likelihood is just a product of
\(\mathsf{binom}(\mathtt{size}=1,\,\mathtt{prob}=p)\) PMFs. We first
give some sample data (in the vector =datavals=), next we define the
likelihood function =L=, and finally we =optimize= =L= over the range
=c(0,1)=.

#+BEGIN_SRC R :exports both :results output pp 
x <- mtcars$am
L <- function(p,x) prod(dbinom(x, size = 1, prob = p))
optimize(L, interval = c(0,1), x = x, maximum = TRUE)
#+END_SRC

#+RESULTS:
: $maximum
: [1] 0.4062458
: 
: $objective
: [1] 4.099989e-10

#+BEGIN_SRC R :exports none :results silent
A <- optimize(L, interval = c(0,1), x = x, maximum = TRUE)
amax <- A$maximum; aobj <- A$objective
#+END_SRC

Note that the =optimize= function by default minimizes the function
=L=, so we have to set =maximum = TRUE= to get an MLE. The returned
value of =$maximum= gives an approximate value of the MLE to be
SRC_R[:eval no-export]{round(amax, 3)} 0.406 and =objective= gives =L= evaluated at the
MLE which is approximately SRC_R[:eval no-export]{round(aobj, 3)} 0.

We previously remarked that it is usually more numerically convenient
to maximize the log-likelihood (or minimize the negative
log-likelihood), and we can just as easily do this with
\(\mathsf{R}\). We just need to calculate the log-likelihood
beforehand which (for this example) is \[ -l(p)=-\sum x_{i}\ln\,
p-\left(n-\sum x_{i}\right)\ln(1-p).  \]

It is done in \(\mathsf{R}\) with

#+BEGIN_SRC R :exports both :results output pp 
minuslogL <- function(p,x){
                -sum(dbinom(x, size = 1, prob = p, log = TRUE))
             }
optimize(minuslogL, interval = c(0,1), x = x)
#+END_SRC

#+RESULTS:
: $minimum
: [1] 0.4062525
: 
: $objective
: [1] 21.61487

Note that we did not need =maximum = TRUE= because we minimized the
negative log-likelihood. The answer for the MLE is essentially the
same as before, but the =$objective= value was different, of course.

For multiparameter problems we may use a similar approach by way of
the =mle= function in the =stats4= package \cite{stats4}.

# +BEGIN_exampletoo


*Plant Growth.* We will investigate the =weight= variable of the
=PlantGrowth= data. We will suppose that the weights constitute a
random observations \(X_{1}\), \(X_{2}\), ... , \(X_{n}\) that are IID
\(\mathsf{norm}(\mathtt{mean}=\mu,\,\mathtt{sd}=\sigma)\) which is not
unreasonable based on a histogram and other exploratory measures. We
will find the MLE of \(\theta=(\mu,\sigma^{2})\). We claimed in
Example [[exa-normal-MLE-both]] that
\(\hat{\theta}=(\hat{\mu},\hat{\sigma}^{2})\) had the form given
above. Let us check whether this is plausible numerically. The
negative log-likelihood function is

#+BEGIN_SRC R :exports code :results silent
minuslogL <- function(mu, sigma2){
  -sum(dnorm(x, mean = mu, sd = sqrt(sigma2), log = TRUE))
}
#+END_SRC

Note that we omitted the data as an argument to the log-likelihood
function; the only arguments were the parameters over which the
maximization is to take place. Now we will simulate some data and find
the MLE. The optimization algorithm requires starting values
(intelligent guesses) for the parameters. We choose values close to
the sample mean and variance (which turn out to be approximately 5 and
0.5, respectively) to illustrate the procedure.

#+BEGIN_SRC R :exports both :results output pp 
x <- PlantGrowth$weight
MaxLikeEst <- mle(minuslogL, start = list(mu = 5, sigma2 = 0.5))
summary(MaxLikeEst)
#+END_SRC

#+RESULTS:
#+BEGIN_example
Maximum likelihood estimation

Call:
mle(minuslogl = minuslogL, start = list(mu = 5, sigma2 = 0.5))

Coefficients:
        Estimate Std. Error
mu     5.0729848  0.1258666
sigma2 0.4752721  0.1227108

-2 log L: 62.82084
#+END_example

The outputted MLEs are shown above, and =mle= even gives us estimates
for the standard errors of \(\hat{\mu}\) and \(\hat{\sigma}^{2}\)
(which were obtained by inverting the numerical Hessian matrix at the
optima; see Appendix [[#sec-Multivariable-Calculus]]). Let us check how close
the numerical MLEs came to the theoretical MLEs:

#+BEGIN_SRC R :exports both :results output pp 
mean(x); var(x)*29/30; sd(x)/sqrt(30)
#+END_SRC

#+RESULTS:
: [1] 5.073
: [1] 0.475281
: [1] 0.1280195

The numerical MLEs were very close to the theoretical MLEs. We already
knew that the standard error of \(\hat{\mu}=\overline{X}\) is
\(\sigma/\sqrt{n}\), and the numerical estimate of this was very close
too.

# +END_exampletoo


There is functionality in the =distrTest= package \cite{distrTest} to
calculate theoretical MLEs; we will skip examples of these for the
time being.

** Confidence Intervals for Means
:PROPERTIES:
:CUSTOM_ID: sec-Confidence-Intervals-for-Means
:END:

We are given \(X_{1}\), \(X_{2}\), ..., \(X_{n}\) that are an
\(SRS(n)\) from a
\(\mathsf{norm}(\mathtt{mean}=\mu,\,\mathtt{sd}=\sigma)\)
distribution, where \(\mu\) is unknown. We know that we may estimate
\(\mu\) with \(\overline{X}\), and we have seen that this estimator is
the MLE. But how good is our estimate? We know that
\begin{equation} 
\frac{\overline{X}-\mu}{\sigma/\sqrt{n}}\sim\mathsf{norm}(\mathtt{mean}=0,\,\mathtt{sd}=1).
\end{equation}
For a big probability \(1-\alpha\), for instance, 95%, we can
calculate the quantile \(z_{\alpha/2}\). Then
\begin{equation}
\mathbb{P}\left(-z_{\alpha/2}\leq\frac{\overline{X}-\mu}{\sigma/\sqrt{n}}\leq z_{\alpha/2}\right)=1-\alpha.
\end{equation}
But now consider the following string of equivalent inequalities:
\[
-z_{\alpha/2}\leq\frac{\overline{X}-\mu}{\sigma/\sqrt{n}}\leq z_{\alpha/2},
\]
\[
-z_{\alpha/2}\left(\frac{\sigma}{\sqrt{n}}\right)\leq\overline{X}-\mu\leq z_{\alpha/2}\left(\frac{\sigma}{\sqrt{n}}\right),
\]
\[
-\overline{X} - z_{\alpha/2}\left(\frac{\sigma}{\sqrt{n}}\right)\leq - \mu \leq - \overline{X} + z_{\alpha/2} \left( \frac{\sigma}{\sqrt{n}} \right),
\]
\[
\overline{X} - z_{\alpha/2} \left( \frac{\sigma}{\sqrt{n}} \right) \leq \mu \leq \overline{X} + z_{\alpha/2} \left( \frac{\sigma}{\sqrt{n}} \right).
\]
That is, 
\begin{equation}
\mathbb{P}\left(\overline{X}-z_{\alpha/2}\frac{\sigma}{\sqrt{n}}\leq\mu\leq\overline{X}+z_{\alpha/2}\frac{\sigma}{\sqrt{n}}\right)=1-\alpha.
\end{equation}

#+BEGIN_defn
The interval
\begin{equation}
\left[\overline{X}-z_{\alpha/2}\frac{\sigma}{\sqrt{n}},\ \overline{X}+z_{\alpha/2}\frac{\sigma}{\sqrt{n}}\right]
\end{equation}
is a \(100(1-\alpha)\%\) /confidence interval for/ \(\mu\). The quantity \(1-\alpha\) is called the /confidence coefficient/.
#+END_defn

#+BEGIN_rem
The interval is also sometimes written more compactly as
\begin{equation}
\label{eq-z-interval}
\overline{X}\pm z_{\alpha/2}\frac{\sigma}{\sqrt{n}}.
\end{equation}
#+END_rem

The interpretation of confidence intervals is tricky and often
mistaken by novices. When I am teaching the concept "live" during
class, I usually ask the students to imagine that my piece of chalk
represents the "unknown" parameter, and I lay it down on the desk in
front of me. Once the chalk has been lain, it is /fixed/; it does not
move. Our goal is to estimate the parameter. For the estimator I pick
up a sheet of loose paper lying nearby. The estimation procedure is to
randomly drop the piece of paper from above, and observe where it
lands. If the piece of paper covers the piece of chalk, then we are
successful -- our estimator covers the parameter. If it falls off to
one side or the other, then we are unsuccessful; our interval fails to
cover the parameter.

Then I ask them: suppose we were to repeat this procedure hundreds,
thousands, millions of times. Suppose we kept track of how many times
we covered and how many times we did not. What percentage of the time
would we be successful?

In the demonstration, the parameter corresponds to the chalk, the
sheet of paper corresponds to the confidence interval, and the random
experiment corresponds to dropping the sheet of paper. The percentage
of the time that we are successful /exactly/ corresponds to the
/confidence coefficient/. That is, if we use a 95% confidence
interval, then we can say that, in the long run, approximately 95% of
our intervals will cover the true parameter (which is fixed, but
unknown).

See Figure [[fig-ci-examp]], which is a graphical display of these ideas.

#+NAME: ci-examp
#+BEGIN_SRC R :exports both :results graphics :file fig/hypoth-ci-examp.ps
ci.examp()
#+END_SRC

#+NAME: fig-ci-examp
#+CAPTION[Simulated confidence intervals]: \small The graph was generated by the =ci.examp= function from the =TeachingDemos= package. Fifty (50) samples of size twenty five (25) were generated from a \( \mathsf{norm}(\mathtt{mean}=100,\,\mathtt{sd}=10) \) distribution, and each sample was used to find a 95% confidence interval for the population mean using Equation \eqref{eq-z-interval}. The 50 confidence intervals are represented above by horizontal lines, and the respective sample means are denoted by vertical slashes. Confidence intervals that "cover" the true mean value of 100 are plotted in black; those that fail to cover are plotted in a lighter color. In the plot we see that only one (1) of the simulated intervals out of the 50 failed to cover \(\mu=100\), which is a success rate of 98%. If the number of generated samples were to increase from 50 to 500 to 50000, ..., then we would expect our success rate to approach the exact value of 95%.
#+ATTR_LaTeX: :width 0.9\textwidth :placement [ht!]
#+RESULTS: ci-examp
[[file:fig/hypoth-ci-examp.ps]]

Under the above framework, we can reason that an "interval" with a
/larger/ confidence coefficient corresponds to a /wider/ sheet of
paper. Furthermore, the width of the confidence interval (sheet of
paper) should be /somehow/ related to the amount of information
contained in the random sample, \(X_{1}\), \(X_{2}\), ...,
\(X_{n}\). The following remarks makes these notions precise.

#+BEGIN_rem
For a fixed confidence coefficient \(1-\alpha\), if \(n\) increases,
then the confidence interval gets /SHORTER/.

For a fixed sample size \(n\), if \(1-\alpha\) increases, then the
confidence interval gets /WIDER/.
#+END_rem

# +BEGIN_exampletoo
<<exa-plant-one-samp-z-int>> *Results from an Experiment on Plant
Growth.* The =PlantGrowth= data frame gives the results of an
experiment to measure plant yield (as measured by the weight of the
plant). We would like to a 95% confidence interval for the mean weight
of the plants. Suppose that we know from prior research that the true
population standard deviation of the plant weights is \(0.7\) g.

The parameter of interest is \(\mu\), which represents the true mean
weight of the population of all plants of the particular species in
the study. We will first take a look at a stem-and-leaf display of the
data:

# +END_exampletoo


#+BEGIN_SRC R :exports both :results output pp 
with(PlantGrowth, stem.leaf(weight))
#+END_SRC

#+RESULTS:
#+BEGIN_example
1 | 2: represents 1.2
 leaf unit: 0.1
            n: 30
   1     f | 5
         s | 
   2    3. | 8
   4    4* | 11
   5     t | 3
   8     f | 455
  10     s | 66
  13    4. | 889
  (4)   5* | 1111
  13     t | 2233
   9     f | 555
         s | 
   6    5. | 88
   4    6* | 011
   1     t | 3
#+END_example

The data appear to be approximately normal with no extreme values. The
data come from a designed experiment, so it is reasonable to suppose
that the observations constitute a simple random sample of
weights[fn:fn-srs]. We know the population standard deviation
\(\sigma=0.70\) from prior research. We are going to use the
one-sample \(z\)-interval.

[fn:fn-srs] Actually we will see later that there is reason to believe
that the observations are simple random samples from three distinct
populations. See Section \ref{sec-Analysis-of-Variance}.

#+BEGIN_SRC R :exports both :results output pp 
dim(PlantGrowth)   # sample size is first entry
#+END_SRC

#+RESULTS:
: [1] 30  2

#+BEGIN_SRC R :exports both :results output pp 
with(PlantGrowth, mean(weight))
#+END_SRC

#+RESULTS:
: [1] 5.073

#+BEGIN_SRC R :exports both :results output pp 
qnorm(0.975)
#+END_SRC

#+RESULTS:
: [1] 1.959964

We find the sample mean of the data to be \(\overline{x}=5.073\) and
\(z_{\alpha/2}=z_{0.025}\approx1.96\). Our interval is therefore \[
\overline{x}\pm
z_{\alpha/2}\frac{\sigma}{\sqrt{n}}=5.073\pm1.96\cdot\frac{0.70}{\sqrt{30}},
\] which comes out to approximately \([4.823,\,5.323]\). In
conclusion, we are 95% confident that the true mean weight \(\mu\) of
all plants of this species lies somewhere between 4.823 g and 5.323 g,
that is, we are 95% confident that the interval \([4.823,\,5.323]\)
covers \(\mu\).

# +BEGIN_exampletoo

Give some data with \(X_{1}\), \(X_{2}\), ..., \(X_{n}\) an \(SRS(n)\)
from a \(\mathsf{norm}(\mathtt{mean}=\mu,\,\mathtt{sd}=\sigma)\)
distribution. Maybe small sample?
# +END_exampletoo


1. What is the parameter of interest? in the context of the problem.
2. Give a point estimate for \(\mu\).
3. What are the assumptions being made in the problem? Do they meet
   the conditions of the interval?
4. Calculate the interval.
5. Draw the conclusion.

#+BEGIN_rem
What if \(\sigma\) is unknown? We instead use the interval
\begin{equation}
\overline{X}\pm z_{\alpha/2}\frac{S}{\sqrt{n}},
\end{equation}
where \(S\) is the sample standard deviation.
- If \(n\) is large, then \(\overline{X}\) will have an approximately
  normal distribution regardless of the underlying population (by the
  CLT) and \(S\) will be very close to the parameter \(\sigma\) (by
  the SLLN); thus the above interval will have approximately
  \(100(1-\alpha)\%\) confidence of covering \(\mu\).
- If \(n\) is small, then
   - If the underlying population is normal then we may replace
     \(z_{\alpha/2}\) with \(t_{\alpha/2}(\mathtt{df}=n-1)\). The
     resulting \(100(1-\alpha)\%\) confidence interval is
     \begin{equation}
     \label{eq-one-samp-t-int}
     \overline{X}\pm t_{\alpha/2}(\mathtt{df}=n-1)\frac{S}{\sqrt{n}}.
     \end{equation}
   - If the underlying population is not normal, but approximately
     normal, then we may use the \(t\) interval, Equation
     \eqref{eq-one-samp-t-int}. The interval will have approximately
     \(100(1-\alpha)\%\) confidence of covering \(\mu\). However, if
     the population is highly skewed or the data have outliers, then
     we should ask a professional statistician for advice.
#+END_rem

The author learned of a handy acronym from AP Statistics Exam graders
that summarizes the important parts of confidence interval estimation,
which is PANIC: /P/-arameter, /A/-ssumptions, /N/-ame, /I/-nterval,
and /C/-onclusion.
- Parameter: :: identify the parameter of interest with the proper
                symbols. Write down what the parameter means in the
                context of the problem.
- Assumptions: :: list any assumptions made in the experiment. If
                  there are any other assumptions needed or that were
                  not checked, state what they are and why they are
                  important.
- Name: :: choose a statistical procedure from your bag of tricks
           based on the answers to the previous two parts. The
           assumptions of the procedure you choose should match those
           of the problem; if they do not match then either pick a
           different procedure or openly admit that the results may
           not be reliable. Write down any underlying formulas used.
- Interval: :: calculate the interval from the sample data. This can
               be done by hand but will more often be done with the
               aid of a computer. Regardless of the method, all
               calculations or code should be shown so that the entire
               process is repeatable by a subsequent reader.
- Conclusion: :: state the final results, using language in the
                 context of the problem. Include the appropriate
                 interpretation of the interval, making reference to
                 the confidence coefficient.

#+BEGIN_rem
All of the above intervals for \(\mu\) were two-sided, but there are
also one-sided intervals for \(\mu\). They look like
\begin{equation}
\left[\overline{X}-z_{\alpha}\frac{\sigma}{\sqrt{n}},\ \infty\right)\quad \mbox{or}\quad \left(-\infty,\ \overline{X}+z_{\alpha}\frac{\sigma}{\sqrt{n}}\right]
\end{equation}
and satisfy
\begin{equation}
\mathbb{P}\left(\overline{X}-z_{\alpha}\frac{\sigma}{\sqrt{n}}\leq\mu\right)=1-\alpha\quad \mbox{and}\quad \mathbb{P}\left(\overline{X}+z_{\alpha}\frac{\sigma}{\sqrt{n}}\geq\mu\right)=1-\alpha.
\end{equation}
#+END_rem


# +BEGIN_exampletoo

Small sample, some data with \(X_{1}\), \(X_{2}\), ..., \(X_{n}\) an
\(SRS(n)\) from a
\(\mathsf{norm}(\mathtt{mean}=\mu,\,\mathtt{sd}=\sigma)\)
distribution.  PANIC
# +END_exampletoo
 
*** How to do it with \(\mathsf{R}\)
We can do Example [[exa-plant-one-samp-z-int]] with the following code.

#+BEGIN_SRC R :exports both :results output pp 
temp <- with(PlantGrowth, z.test(weight, stdev = 0.7))
temp
#+END_SRC

#+RESULTS:
#+BEGIN_example

	One Sample z-test

data:  weight
z = 39.6942, n = 30.000, Std. Dev. = 0.700, Std. Dev. of the sample mean = 0.128,
p-value < 2.2e-16
alternative hypothesis: true mean is not equal to 0
95 percent confidence interval:
 4.822513 5.323487
sample estimates:
mean of weight 
         5.073
#+END_example

The confidence interval bounds are shown in the sixth line down of the
output (please disregard all of the additional output information for
now -- we will use it in Chapter [[#cha-Hypothesis-Testing]]). We can make the
plot for Figure [[fig-plant-z-int-plot]] with

#+BEGIN_SRC R :exports code :eval never
plot(temp, "Conf")
#+END_SRC

** Confidence Intervals for Differences of Means
:PROPERTIES:
:CUSTOM_ID: sec-Conf-Interv-for-Diff-Means
:END:

Let \(X_{1}\), \(X_{2}\), ..., \(X_{n}\) be a \(SRS(n)\) from a
\(\mathsf{norm}(\mathtt{mean}=\mu_{X},\,\mathtt{sd}=\sigma_{X})\)
distribution and let \(Y_{1}\), \(Y_{2}\), ..., \(Y_{m}\) be a
\(SRS(m)\) from a
\(\mathsf{norm}(\mathtt{mean}=\mu_{Y},\,\mathtt{sd}=\sigma_{Y})\)
distribution. Further, assume that the \(X_{1}\), \(X_{2}\), ...,
\(X_{n}\) sample is independent of the \(Y_{1}\), \(Y_{2}\), ...,
\(Y_{m}\) sample.

Suppose that \(\sigma_{X}\) and \(\sigma_{Y}\) are known. We would
like a confidence interval for \(\mu_{X}-\mu_{Y}\). We know that
\begin{equation}
\overline{X}-\overline{Y}\sim\mathsf{norm}\left(\mathtt{mean}=\mu_{X}-\mu_{Y},\,\mathtt{sd}=\sqrt{\frac{\sigma_{X}^{2}}{n}+\frac{\sigma_{Y}^{2}}{m}}\right).
\end{equation}
Therefore, a \( 100(1-\alpha)\% \) confidence interval for
\(\mu_{X}-\mu_{Y}\) is given by
\begin{equation}
\label{eq-two-samp-mean-CI}
\left(\overline{X}-\overline{Y}\right)\pm z_{\alpha/2}\sqrt{\frac{\sigma_{X}^{2}}{n}+\frac{\sigma_{Y}^{2}}{m}}.
\end{equation}
Unfortunately, most of the time the values of \(\sigma_{X}\) and
\(\sigma_{Y}\) are unknown. This leads us to the following:
- If both sample sizes are large, then we may appeal to the CLT/SLLN
  (see Section [[#sec-Central-Limit-Theorem]]) and substitute \(S_{X}^{2}\) and
  \(S_{Y}^{2}\) for \(\sigma_{X}^{2}\) and \(\sigma_{Y}^{2}\) in the
  interval \eqref{eq-two-samp-mean-CI}. The resulting confidence interval will
  have approximately \(100(1-\alpha)\%\) confidence.
- If one or more of the sample sizes is small then we are in trouble,
  unless -the underlying populations are both normal and
  \(\sigma_{X}=\sigma_{Y}\). In this case (setting
  \(\sigma=\sigma_{X}=\sigma_{Y}\)),
  \begin{equation}
  \overline{X}-\overline{Y}\sim\mathsf{norm}\left(\mathtt{mean}=\mu_{X}-\mu_{Y},\,\mathtt{sd}=\sigma\sqrt{\frac{1}{n}+\frac{1}{m}}\right).
  \end{equation} 
  Now let
  \begin{equation}
  U=\frac{n-1}{\sigma^{2}}S_{X}^{2}+\frac{m-1}{\sigma^{2}}S_{Y}^{2}.
  \end{equation}
  Then by Exercise [[xca-sum-indep-chisq]] we know that
  \(U\sim\mathsf{chisq}(\mathtt{df}=n+m-2)\) and it is not a large
  leap to believe that \(U\) is independent of
  \(\overline{X}-\overline{Y}\); thus
  \begin{equation}
  T = \frac{Z}{\sqrt{\left.U\right/ (n+m-2)}}\sim\mathsf{t}(\mathtt{df}=n+m-2).
  \end{equation}
  But
  \begin{align*}
  T & =\frac{\frac{\overline{X}-\overline{Y}-(\mu_{X}-\mu_{Y})}{\sigma\sqrt{\frac{1}{n}+\frac{1}{m}}}}{\sqrt{\left.\frac{n-1}{\sigma^{2}}S_{X}^{2}+\frac{m-1}{\sigma^{2}}S_{Y}^{2}\right/ (n+m-2)}},\\
  & =\frac{\overline{X}-\overline{Y}-(\mu_{X}-\mu_{Y})}{\sqrt{\left(\frac{1}{n}+\frac{1}{m}\right)\left(\frac{(n-1)S_{X}^{2}+(m-1)S_{Y}^{2}}{n+m-2}\right)}},\\
  & \sim\mathsf{t}(\mathtt{df}=n+m-2).
  \end{align*}
  Therefore a \(100(1-\alpha)\%\) confidence interval for
  \(\mu_{X}-\mu_{Y}\) is given by
  \begin{equation}
  \left(\overline{X}-\overline{Y}\right)\pm t_{\alpha/2}(\mathtt{df}=n+m-2)\, S_{p}\sqrt{\frac{1}{n}+\frac{1}{m}},
  \end{equation}
  where
  \begin{equation}
  S_{p}=\sqrt{\frac{(n-1)S_{X}^{2}+(m-1)S_{Y}^{2}}{n+m-2}}
  \end{equation}
  is called the "pooled" estimator of \(\sigma\).
    - If one of the samples is small, and both underlying populations
      are normal, but \(\sigma_{X}\neq\sigma_{Y}\), then we may use a
      Welch (or Satterthwaite) approximation to the degrees of
      freedom. See Welch \cite{Welch1947}, Satterthwaite
      \cite{Satterthwaite1946}, or Neter /et al/ \cite{Neter1996}. The
      idea is to use an interval of the form
      \begin{equation}
      \left(\overline{X}-\overline{Y}\right)\pm\mathsf{t}_{\alpha/2}(\mathtt{df}=r)\,\sqrt{\frac{S_{X}^{2}}{n}+\frac{S_{Y}^{2}}{m}},
      \end{equation}
      where the degrees of freedom \(r\) is chosen so that the
      interval has nice statistical properties. It turns out that a
      good choice for \(r\) is given by
      \begin{equation}
      r=\frac{\left(S_{X}^{2}/n+S_{Y}^{2}/m\right)^{2}}{\frac{1}{n-1}\left(S_{X}^{2}/n\right)^{2}+\frac{1}{m-1}\left(S_{Y}^{2}/m\right)^{2}},
      \end{equation}
      where we understand that \(r\) is rounded down to the nearest
      integer. The resulting interval has approximately
      \(100(1-\alpha)\%\) confidence.

*** How to do it with \(\mathsf{R}\)

The basic function is =t.test= which has a =var.equal= argument that
may be set to =TRUE= or =FALSE=. The confidence interval is shown as
part of the output, although there is a lot of additional information
that is not needed until Chapter [[#cha-Hypothesis-Testing]].

There is not any specific functionality to handle the \(z\)-interval
for small samples, but if the samples are large then =t.test= with
=var.equal = FALSE= will be essentially the same thing. The standard
deviations are never (?) known in advance anyway so it does not really
matter in practice.

** Confidence Intervals for Proportions
:PROPERTIES:
:CUSTOM_ID: sec-Confidence-Intervals-Proportions
:END:

We would like to know \(p\) which is the "proportion of
successes". For instance, \(p\) could be:
- the proportion of U.S. citizens that support Obama,
- the proportion of smokers among adults age 18 or over,
- the proportion of people worldwide infected by the H1N1 virus.

We are given an \(SRS(n)\) \(X_{1}\), \(X_{2}\), ..., \(X_{n}\)
distributed
\(\mathsf{binom}(\mathtt{size}=1,\,\mathtt{prob}=p)\). Recall from
Section [[#sec-binom-dist]] that the common mean of these variables
is \(\mathbb{E} X=p\) and the variance is
\(\mathbb{E}(X-p)^{2}=p(1-p)\). If we let \(Y=\sum X_{i}\), then from
Section [[#sec-binom-dist]] we know that
\(Y\sim\mathsf{binom}(\mathtt{size}=n,\,\mathtt{prob}=p)\) and that \[
\overline{X}=\frac{Y}{n}\mbox{ has }\mathbb{E}\overline{X}=p\mbox{ and
}\mathrm{Var}(\overline{X})=\frac{p(1-p)}{n}.  \] Thus if \(n\) is
large (here is the CLT) then an approximate \(100(1-\alpha)\%\)
confidence interval for \(p\) would be given by
\begin{equation}
\label{eq-ci-p-no-good}
\overline{X}\pm z_{\alpha/2}\sqrt{\frac{p(1-p)}{n}}.
\end{equation}
OOPS...! Equation \eqref{eq-ci-p-no-good} is of no use to us because the
\underbar{unknown} parameter \(p\) is in the formula! (If we knew what
\(p\) was to plug in the formula then we would not need a confidence
interval in the first place.) There are two solutions to this problem.
1. Replace \(p\) with \(\hat{p}=\overline{X}\). Then an approximate
   \(100(1-\alpha)\%\) confidence interval for \(p\) is given by
   \begin{equation}
   \hat{p}\pm z_{\alpha/2}\sqrt{\frac{\hat{p}(1-\hat{p})}{n}}.
   \end{equation}
   This approach is called the /Wald interval/ and is also known as
   the /asymptotic interval/ because it appeals to the CLT for large
   sample sizes.
2. Go back to first principles. Note that \[
   -z_{\alpha/2}\leq\frac{Y/n-p}{\sqrt{p(1-p)/n}}\leq z_{\alpha/2} \]
   exactly when the function \(f\) defined by \[
   f(p)=\left(Y/n-p\right)^{2}-z_{\alpha/2}^{2}\frac{p(1-p)}{n} \]
   satisfies \(f(p)\leq0\). But \(f\) is quadratic in \(p\) so its
   graph is a parabola; it has two roots, and these roots form the
   limits of the confidence interval. We can find them with the
   quadratic formula (see Exercise [[xca-CI-quad-form]]):
   \begin{equation}
   \left.\left[\left(\hat{p}+\frac{z_{\alpha/2}^{2}}{2n}\right)\pm z_{\alpha/2}\sqrt{\frac{\hat{p}(1-\hat{p})}{n}+\frac{z_{\alpha/2}^{2}}{(2n)^{2}}}\right]\right/ \left(1+\frac{z_{\alpha/2}^{2}}{n}\right)
   \end{equation}
   This approach is called the /score interval/ because it is based on
   the inversion of the "Score test". See Chapter
   [[#cha-Categorical-Data-Analysis]]. It is also known as the /Wilson
   interval/; see Agresti \cite{Agresti2002}.


For two proportions \(p_{1}\) and \(p_{2}\), we may collect
independent \(\mathsf{binom}(\mathtt{size}=1,\,\mathtt{prob}=p)\)
samples of size \(n_{1}\) and \(n_{2}\), respectively. Let \(Y_{1}\)
and \(Y_{2}\) denote the number of successes in the respective
samples.  We know that \[
\frac{Y_{1}}{n_{1}}\approx\mathsf{norm}\left(\mathtt{mean}=p_{1},\,\mathtt{sd}=\sqrt{\frac{p_{1}(1-p_{1})}{n_{1}}}\right)
\] and \[
\frac{Y_{2}}{n_{2}}\approx\mathsf{norm}\left(\mathtt{mean}=p_{2},\,\mathtt{sd}=\sqrt{\frac{p_{2}(1-p_{2})}{n_{2}}}\right)
\] so it stands to reason that an approximate \(100(1-\alpha)\%\)
confidence interval for \(p_{1}-p_{2}\) is given by
\begin{equation}
\left(\hat{p}_{1}-\hat{p}_{2}\right)\pm z_{\alpha/2}\sqrt{\frac{\hat{p}_{1}(1-\hat{p}_{1})}{n_{1}}+\frac{\hat{p}_{2}(1-\hat{p}_{2})}{n_{2}}},
\end{equation}
where \(\hat{p}_{1}=Y_{1}/n_{1}\) and \(\hat{p}_{2}=Y_{2}/n_{2}\).

#+BEGIN_rem
When estimating a single proportion, one-sided intervals are sometimes
needed. They take the form
\begin{equation}
\left[0,\ \hat{p}+z_{\alpha/2}\sqrt{\frac{\hat{p}(1-\hat{p})}{n}}\right]
\end{equation}
or
\begin{equation}
\left[\hat{p}-z_{\alpha/2}\sqrt{\frac{\hat{p}(1-\hat{p})}{n}},\ 1\right]
\end{equation}
or in other words, we know in advance that the true proportion is
restricted to the interval \([0,1]\), so we can truncate our
confidence interval to those values on either side.
#+END_rem

*** How to do it with \(\mathsf{R}\)

#+BEGIN_SRC R :exports both :results output pp 
binconf(x = 7, n = 25, method = "asymptotic")
#+END_SRC

#+RESULTS:
:  PointEst     Lower     Upper
:      0.28 0.1039957 0.4560043

#+BEGIN_SRC R :exports both :results output pp 
binconf(x = 7, n = 25, method = "wilson")
#+END_SRC

#+RESULTS:
:  PointEst     Lower     Upper
:      0.28 0.1428385 0.4757661

The default value of the =method= argument is =wilson=.  An alternate way is 
#+BEGIN_SRC R :exports none :results silent
data(RcmdrTestDrive)
#+END_SRC

#+BEGIN_SRC R :exports both :results output pp 
tab <- xtabs(~gender, data = RcmdrTestDrive)
prop.test(rbind(tab), conf.level = 0.95, correct = FALSE)
#+END_SRC

#+RESULTS:
#+BEGIN_example

	1-sample proportions test without continuity correction

data:  rbind(tab), null probability 0.5
X-squared = 4.6667, df = 1, p-value = 0.03075
alternative hypothesis: true p is not equal to 0.5
95 percent confidence interval:
 0.3447942 0.4922649
sample estimates:
        p 
0.4166667
#+END_example

#+BEGIN_SRC R :exports code :results silent
A <- as.data.frame(Titanic)
B <- with(A, untable(A, Freq))
#+END_SRC

** Confidence Intervals for Variances
:PROPERTIES:
:CUSTOM_ID: sec-Confidence-Intervals-for-Variances
:END:

I am thinking one and two sample problems here.

*** How to do it with \(\mathsf{R}\)

I am thinking about =sigma.test= in the =TeachingDemos= package
\cite{TeachingDemos} and =var.test= in base \(\mathsf{R}\) \cite{base}
here.

** Fitting Distributions
:PROPERTIES:
:CUSTOM_ID: sec-Fitting-Distributions
:END:

*** How to do it with \(\mathsf{R}\)

I am thinking about =fitdistr= from the =MASS= package \cite{MASS}.

** Sample Size and Margin of Error
:PROPERTIES:
:CUSTOM_ID: sec-Sample-Size-and-MOE
:END:

Sections [[#sec-Confidence-Intervals-for-Means]] through [[#sec-Confidence-Intervals-for-Variances]] all began the same way: we were given the sample size
\(n\) and the confidence coefficient \(1-\alpha\), and our task was to
find a margin of error \(E\) so that \[ \hat{\theta}\pm E\mbox{ is a
}100(1-\alpha)\%\mbox{ confidence interval for }\theta.  \]

Some examples we saw were:
- \(E=z_{\alpha/2}\sigma/\sqrt{n}\), in the one-sample \(z\)-interval,
- \(E=t_{\alpha/2}(\mathtt{df}=n+m-2)S_{p}\sqrt{n^{-1}+m^{-1}}\), in
  the two-sample pooled \(t\)-interval.

We already know (we can see in the formulas above) that \(E\)
decreases as \(n\) increases. Now we would like to use this
information to our advantage: suppose that we have a fixed margin of
error \(E,\) say \(E=3\), and we want a \(100(1-\alpha)\%\) confidence
interval for \(\mu\). The question is: how big does \(n\) have to be?

For the case of a population mean the answer is easy: we set up an
equation and solve for \(n\).

# +BEGIN_exampletoo

Given a situation, given \(\sigma\), given \(E\), we would like to
know how big \(n\) has to be to ensure that \(\overline{X}\pm5\) is a
95% confidence interval for \(\mu\).
# +END_exampletoo


#+BEGIN_rem
Always round up any decimal values of \(n\), no matter how small the
decimal is. Another name for \(E\) is the "maximum error of the
estimate".
#+END_rem

For proportions, recall that the asymptotic formula to estimate \(p\)
was \[ \hat{p}\pm z_{\alpha/2}\sqrt{\frac{\hat{p}(1-\hat{p})}{n}}.  \]
Reasoning as above we would want
\begin{align}
\label{eq-samp-size-prop-ME}
E & =z_{\alpha/2}\sqrt{\frac{\hat{p}(1-\hat{p})}{n}},\mbox{ or}\\
n & =z_{\alpha/2}^{2}\frac{\hat{p}(1-\hat{p})}{E^{2}}.
\end{align}
OOPS! Recall that \(\hat{p}=Y/n\), which would put the variable \(n\)
on both sides of Equation \eqref{eq-samp-size-prop-ME}. Again, there are two
solutions to the problem.

1. If we have a good idea of what \(p\) is, say \(p^{\ast}\) then we
   can plug it in to get
   \begin{equation}
   n=z_{\alpha/2}^{2}\frac{p^{\ast}(1-p^{\ast})}{E^{2}}.
   \end{equation}
2. Even if we have no idea what \(p\) is, we do know from calculus
   that \(p(1-p)\leq1/4\) because the function \(f(x)=x(1-x)\) is
   quadratic (so its graph is a parabola which opens downward) with
   maximum value attained at \(x=1/2\). Therefore, regardless of our
   choice for \(p^{\ast}\) the sample size must satisfy
   \begin{equation}
   n=z_{\alpha/2}^{2}\frac{p^{\ast}(1-p^{\ast})}{E^{2}}\leq\frac{z_{\alpha/2}^{2}}{4E^{2}}.
   \end{equation}
   The quantity \(z_{\alpha/2}^{2}/4E^{2}\) is large enough to
   guarantee \(100(1-\alpha)\%\) confidence.

# +BEGIN_exampletoo

Proportion example.
# +END_exampletoo

#+BEGIN_rem
For very small populations sometimes the value of \(n\) obtained from
the formula is too big. In this case we should use the hypergeometric
distribution for a sampling model rather than the binomial model. With
this modification the formulas change to the following: if \(N\)
denotes the population size then let
\begin{equation}
m=z_{\alpha/2}^{2}\frac{p^{\ast}(1-p^{\ast})}{E^{2}}
\end{equation}
and the sample size needed to ensure \(100(1-\alpha)\%\) confidence is
achieved is
\begin{equation}
n=\frac{m}{1+\frac{m-1}{N}}.
\end{equation}
If we do not have a good value for the estimate \(p^{\ast}\) then we may use \(p^{\ast}=1/2\).
#+END_rem

*** How to do it with \(\mathsf{R}\)
I am thinking about =power.t.test=, =power.prop.test=,
=power.anova.test=, and I am also thinking about =replicate=.

** Other Topics
:PROPERTIES:
:CUSTOM_ID: sec-Other-Topics
:END:

Mention =mle= from the =stats4= package \cite{stats4}.

#+LaTeX: \newpage{}

** Exercises
#+LaTeX: \setcounter{thm}{0}

#+BEGIN_xca
<<xca-norm-mu-sig-MLE>> Let \(X_{1}\), \(X_{2}\), ..., \(X_{n}\) be an
\(SRS(n)\) from a \(\mathsf{norm}(\mathtt{mean} = \mu, \, \mathtt{sd}
= \sigma)\) distribution. Find a two-dimensional MLE for
\(\theta=(\mu,\sigma)\).
#+END_xca

#+BEGIN_xca
<<xca-CI-quad-form>> Find the upper and lower limits for the
confidence interval procedure by finding the roots of \(f\) defined by
\[ f(p)=\left(Y/n-p\right)^{2}-z_{\alpha/2}^{2}\frac{p(1-p)}{n}.  \]
You are going to need the quadratic formula.
#+END_xca
