\chapter{An Introduction to R}
\label{sec-2}

\textbf{Highlights:}
\begin{itemize}
\item Where to find \(\mathsf{R}\) to install on a computer.
\item Basic operations (arithmetic, entering data, vectors) at the command
prompt.
\item How and where to find help when they get in trouble.
\end{itemize}

\section{Downloading and Installing \(\mathsf{R}\)}
\label{sec-2-1}

The instructions for obtaining \(\mathsf{R}\) largely depend on the
user's hardware and operating system. The \(\mathsf{R}\) Project has
written \href{http://cran.r-project.org/manuals.html}{The \(\mathsf{R}\) Manual} with
complete, precise instructions about what to do, together with all
sorts of additional information. The following is just a primer to get
a person started.

\subsection{Installing \(\mathsf{R}\)}
\label{sec-2-1-1}

Visit one of the links below to download the latest version of \(\mathsf{R}\) 
for your operating system:

\begin{description}
\item[{Microsoft Windows:}] \url{http://cran.r-project.org/bin/windows/base/}
\item[{MacOS:}] \url{http://cran.r-project.org/bin/macosx/}
\item[{Linux:}] \url{http://cran.r-project.org/bin/linux/}
\end{description}

A great tool for working with R is RStudio, which is convenient, popular, and free.
Please read \url{https://en.wikipedia.org/wiki/RStudio}.
You can download and install RStudio from its official website \url{https://www.rstudio.com}.

\subsection{Installing and Loading Add-on Packages}
\label{sec-2-1-2}

There are \emph{base} packages (which come with \(\mathsf{R}\)
automatically), and \emph{contributed} packages (which must be downloaded
for installation). For example, on the version of \(\mathsf{R}\) being
used for this document the default base packages loaded at startup are

\begin{Verbatim}
getOption("defaultPackages")
\end{Verbatim}

\begin{verbatim}
: [1] "datasets"  "utils"     "grDevices" "graphics"  "stats"     "methods"
\end{verbatim}

The base packages are maintained by a select group of volunteers,
called \(\mathsf{R}\) Core. In addition to the base packages, there
are literally thousands of additional contributed packages written by
individuals all over the world. These are stored worldwide on mirrors
of the Comprehensive \(\mathsf{R}\) Archive Network, or \texttt{CRAN} for
short. Given an active Internet connection, anybody is free to
download and install these packages and even inspect the source code.

To install a package named \texttt{foo}, open up \(\mathsf{R}\) and type
\texttt{install.packages("foo")}
\index{install.packages@\texttt{install.packages}}. To
install \texttt{foo} and additionally install all of the other packages on
which \texttt{foo} depends, instead type \texttt{install.packages("foo", depends =
TRUE)}.

No matter how many packages are installed onto the system, each one
must first be loaded for use with the
\texttt{library} \index{library@\texttt{library}} function. For instance, the
\texttt{foreign} package \cite{foreign} contains all sorts of functions
needed to import data sets into \(\mathsf{R}\) from other software
such as SPSS, SAS, \emph{etc}. But none of those functions will be
available until the command \texttt{library("foreign")} is issued.

Type \texttt{library()} at the command prompt to see a list
of all available packages in your library.

\section{Basic \(\mathsf{R}\) Operations and Concepts}
\label{sec-2-3}

The \(\mathsf{R}\) developers have written an introductory document
entitled "An Introduction to \(\mathsf{R}\)". There is a sample
session included which shows what basic interaction with
\(\mathsf{R}\) looks like. I recommend that all new users of
\(\mathsf{R}\) read that document, but bear in mind that there are
concepts mentioned which will be unfamiliar to the beginner.

Please read: \url{https://cran.r-project.org/manuals.html}

\subsection{Arithmetic}
\label{sec-2-3-1}

\begin{Verbatim}
2 + 3       # add
4 * 5 / 6   # multiply and divide
7^8         # 7 to the 8th power
\end{Verbatim}

\begin{verbatim}
: [1] 5
: [1] 3.333333
: [1] 5764801
\end{verbatim}

Notice the comment character \texttt{\#}. Anything typed
after a \texttt{\#} symbol is ignored by \(\mathsf{R}\). We know that \(20/6\)
is a repeating decimal, but the above example shows only 7 digits. We
can change the number of digits displayed with
\texttt{options} \index{options@\texttt{options}}:

\begin{Verbatim}
options(digits = 16)
10/3                 # see more digits
sqrt(2)              # square root
exp(1)               # Euler's constant, e
pi       
options(digits = 7)  # back to default
\end{Verbatim}

\begin{verbatim}
: [1] 3.333333333333333
: [1] 1.414213562373095
: [1] 2.718281828459045
: [1] 3.141592653589793
\end{verbatim}

Note that it is possible to set \texttt{digits}
\index{digits@\texttt{digits}} up to 22, but setting them
over 16 is not recommended (the extra significant digits are not
necessarily reliable). Above notice the \texttt{sqrt}
\index{sqrt@\texttt{sqrt}} function for square roots and the
\texttt{exp} \index{exp@\texttt{exp}} function for powers of
\(\mathrm{e}\), Euler's number.
Also notice the number $pi$, a mathematical constant, which is often represented by the Greek letter $\pi$.

\subsection{Assignment, Object names, and Data types}
\label{sec-2-3-2}

It is often convenient to assign numbers and values to variables
(objects) to be used later. The proper way to assign values to a
variable is with the \texttt{<-} operator (with a space on either side). The
\verb~=~ symbol works too, but it is recommended by the \(\mathsf{R}\)
masters to reserve \verb~=~ for specifying arguments to functions
(discussed later). In this book we will follow their advice and use
\texttt{<-} for assignment. Once a variable is assigned, its value can be
printed by simply entering the variable name by itself.

\begin{Verbatim}
x <- 7*41/pi   # don't see the calculated value
x              # take a look
\end{Verbatim}

\begin{verbatim}
: [1] 91.35494
\end{verbatim}

When choosing a variable name you can use letters, numbers, dots
"\texttt{.}", or underscore "\texttt{\_}" characters. You cannot
use mathematical operators, and a leading dot may not be followed by a
number. Examples of valid names are: \texttt{x}, \texttt{x1}, \texttt{y.value}, and
\texttt{y\_hat}.

Objects can be of many \emph{types}, \emph{modes}, and \emph{classes}. At this level,
it is not necessary to investigate all of the intricacies of the
respective types, but there are some with which you need to become
familiar:
\begin{description}
\item[{integer:}] the values \(0\), \(\pm1\), \(\pm2\), \ldots{};
\item[{double:}] real numbers (rational and irrational);
\item[{character:}] elements that are wrapped with pairs of quotes;
\item[{logical:}] includes \texttt{TRUE}, \texttt{FALSE}, and \texttt{NA} (which are reserved
words); the \texttt{NA} \index{NA@\texttt{NA}} stands
for "not available", \emph{i.e.}, a missing value.
\end{description}

You can determine an object's type with the \texttt{typeof}
\index{typeof@\texttt{typeof}} function. In addition to the above,
there is the \texttt{complex} \index{complex@\texttt{complex}}
\index{as.complex@\texttt{as.complex}} data type:

\begin{Verbatim}
sqrt(-1)              # isn't defined
sqrt(-1+0i)           # is defined
sqrt(as.complex(-1))  # same thing
(0 + 1i)^2            # should be -1
typeof((0 + 1i)^2)
\end{Verbatim}

\begin{verbatim}
: [1] NaN
: [1] 0+1i
: [1] 0+1i
: [1] -1+0i
: [1] "complex"
\end{verbatim}

Note that you can just type \texttt{(1i)\textasciicircum{}2} to get the same answer. The
\texttt{NaN} \index{NaN@\texttt{NaN}} stands for "not a number"; it is
represented internally as \texttt{double} \index{double}.

\subsection{Vectors}
\label{sec-2-3-3}

All of this time we have been manipulating vectors of length 1. Now
let us move to vectors with multiple entries.

\subsubsection{Entering data vectors}
\label{sec-2-3-3-1}

\textbf{The long way:} \index{c@\texttt{c}} If you would like to enter the
data \texttt{74,31,95,61,76,34,23,54,96} into \(\mathsf{R}\), you may create
a data vector with the \texttt{c} function (which is short for
\emph{concatenate}).

\begin{Verbatim}
x <- c(74, 31, 95, 61, 76, 34, 23, 54, 96)
x
\end{Verbatim}

\begin{verbatim}
: [1] 74 31 95 61 76 34 23 54 96
\end{verbatim}

The elements of a vector are usually coerced by \(\mathsf{R}\) to the
the most general type of any of the elements, so if you do \texttt{c(1, "2")}
then the result will be \texttt{c("1", "2")}.

\index{scan@\texttt{scan}}
\textbf{A shorter way:}
The \texttt{scan} method is
useful when the data are stored somewhere else. For instance, you may
type \texttt{x <- scan()} at the command prompt and \(\mathsf{R}\) will
display \texttt{1:} to indicate that it is waiting for the first data
value. Type a value and press \texttt{Enter}, at which point \(\mathsf{R}\)
will display \texttt{2:}, and so forth. Note that entering an empty line
stops the scan. This method is especially handy when you have a column
of values, say, stored in a text file or spreadsheet. You may copy and
paste them all at the \texttt{1:} prompt, and \(\mathsf{R}\) will store all
of the values instantly in the vector \texttt{x}.

\textbf{Repeated data; regular patterns:} the \texttt{seq} \index{seq@\texttt{seq}}
function will generate all sorts of sequences of numbers. It has the
arguments \texttt{from}, \texttt{to}, \texttt{by}, and \texttt{length.out} which can be set in
concert with one another. We will do a couple of examples to show you
how it works.

\begin{Verbatim}
seq(from = 1, to = 5)
seq(from = 2, by = -0.1, length.out = 4)
\end{Verbatim}

\begin{verbatim}
: [1] 1 2 3 4 5
: [1] 2.0 1.9 1.8 1.7
\end{verbatim}

Note that we can get the first line much quicker with the colon
operator.

\begin{Verbatim}
1:5
\end{Verbatim}

\begin{verbatim}
: [1] 1 2 3 4 5
\end{verbatim}

The vector \texttt{LETTERS} \index{LETTERS@\texttt{LETTERS}} has the 26
letters of the English alphabet in uppercase and
\texttt{letters} \index{letters@\texttt{letters}} has all of them in
lowercase.

\subsubsection{Indexing data vectors}
\label{sec-2-3-3-2}

Sometimes we do not want the whole vector, but just a piece of it. We
can access the intermediate parts with the \texttt{[]} operator. Observe
(with \texttt{x} defined above)

\begin{Verbatim}
x[1]
x[2:4]
x[c(1,3,4,8)]
x[-c(1,3,4,8)]
\end{Verbatim}

\begin{verbatim}
: [1] 74
: [1] 31 95 61
: [1] 74 95 61 54
: [1] 31 76 34 23 96
\end{verbatim}

Notice that we used the minus sign to specify those elements that we
do \emph{not} want.

\begin{Verbatim}
LETTERS[1:5]
letters[-(6:24)]
\end{Verbatim}

\begin{verbatim}
: [1] "A" "B" "C" "D" "E"
: [1] "a" "b" "c" "d" "e" "y" "z"
\end{verbatim}

\subsection{Functions and Expressions}
\label{sec-2-3-4}

A function takes arguments as input and returns an object as
output. There are functions to do all sorts of things. We show some
examples below.

\begin{Verbatim}
x <- 1:5
sum(x)
length(x)
min(x)
mean(x)      # sample mean
sd(x)        # sample standard deviation
\end{Verbatim}

\begin{verbatim}
: [1] 15
: [1] 5
: [1] 1
: [1] 3
: [1] 1.581139
\end{verbatim}

\section{Getting Help}
\label{sec-2-4}

When you are using \(\mathsf{R}\), it will not take long before you
find yourself needing help. Fortunately, \(\mathsf{R}\) has extensive
help resources and you should immediately become familiar with
them.
Begin by clicking \texttt{Help} on \texttt{RStudio}.

\index{help.start@\texttt{help.start}}
If you want to browse the manuals with point-and-click links, the best method for beginners can be started with \texttt{help.start()} from the command line.

\index{help@\texttt{help}}
If you know the
\emph{exact} name of the function you want to know more about, for
example, \texttt{mean} or \texttt{plot}, you can type \texttt{help("mean")} or more simply \texttt{?mean} \index{?@\texttt{?}} at the command line.

\index{??@\texttt{??}}
If you do not know the exact name of
the function of interest, or if the function is in a package that
has not been loaded yet, you can try \texttt{help.search("plo")} \index{help.search@\texttt{help.search}} or equivalently \texttt{??plo} at the command line.

On the help pages for a function there are sometimes "Examples"
listed at the bottom of the page, which will work if copy-pasted at
the command line (unless marked otherwise). The \texttt{example}
\index{example@\texttt{example}} function will run the code
automatically, skipping the intermediate step. For instance, we may
try \texttt{example(plot)} to see a few examples of how the \texttt{plot} function
works.

In addition, there is a mountain of information and examples on the Internet about
\(\mathsf{R}\).

\section{Other Tips}
\label{sec-2-6}

\begin{itemize}
\item
It is unnecessary to retype commands repeatedly, since \(\mathsf{R}\)
remembers what you have recently entered on the command line.
To cycle through the previous commands just push the \(\uparrow\) (up arrow) key.
More generally, the command \texttt{history()}
\index{history@\texttt{history}} will show a whole list of recently
entered commands.
\item To find out what all variables are in the current work environment,
use the commands \texttt{objects()} \index{objects@\texttt{objects}} or
\texttt{ls()} \index{ls@\texttt{ls}}. These list all available objects in
the workspace. If you wish to remove one or more variables, use
\texttt{remove(var1, var2, var3)} \index{remove@\texttt{remove}}, or more
simply use \texttt{rm(var1, var2, var3)}, and to remove all objects use
\texttt{rm(list = ls())}.
\item Once you use \(\mathsf{R}\) for awhile there may be some commands
that you wish to run automatically whenever \(\mathsf{R}\)
starts.
You can make a file
\texttt{.Rprofile} \index{.Rprofile@\texttt{.Rprofile}} to be
stored in the user's home directory, or anywhere \(\mathsf{R}\) is
invoked. This allows for multiple configurations for different
projects or users. See "Customizing the Environment" of \emph{An
Introduction to R} for more details.
\item When exiting \(\mathsf{R}\) the user is given the option to "save
the workspace". I recommend that beginners DO NOT save the
workspace when quitting. If \texttt{Yes} is selected, then all of the
objects and data currently in \(\mathsf{R}\)'s memory is saved in a
file located in the working directory called
\texttt{.RData} \index{.RData@\texttt{.RData}}. This file is then
automatically loaded the next time \(\mathsf{R}\) starts (in which
case \(\mathsf{R}\) will say \texttt{[previously saved workspace
  restored]}). This is a valuable feature for experienced users of
\(\mathsf{R}\), but I find that it causes more trouble than it saves
with beginners.
\end{itemize}
