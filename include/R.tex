* An Introduction to R                                               :introR:
:PROPERTIES:
:tangle: R/02-introR.R
:CUSTOM_ID: cha-introduction-to-R
:END:

#+BEGIN_SRC R :exports none :eval never
#    IPSUR: Introduction to Probability and Statistics Using R
#    Copyright (C) 2014 G. Jay Kerns
#
#    Chapter: An Introduction to R
#
#    This file is part of IPSUR.
#
#    IPSUR is free software: you can redistribute it and/or modify
#    it under the terms of the GNU General Public License as published by
#    the Free Software Foundation, either version 3 of the License, or
#    (at your option) any later version.
#
#    IPSUR is distributed in the hope that it will be useful,
#    but WITHOUT ANY WARRANTY; without even the implied warranty of
#    MERCHANTABILITY or FITNESS FOR A PARTICULAR PURPOSE.  See the
#    GNU General Public License for more details.
#
#    You should have received a copy of the GNU General Public License
#    along with IPSUR.  If not, see <http://www.gnu.org/licenses/>.
#+END_SRC

Every \(\mathsf{R}\) book I have ever seen has had a section/chapter
that is an introduction to \(\mathsf{R}\), and so does this one.  The
goal of this chapter is for a person to get up and running, ready for
the material that follows.  See Section [[#sec-External-Resources]] for links
to other material which the reader may find useful.

*What do I want them to know?*
- Where to find \(\mathsf{R}\) to install on a home computer, and a
  few comments to help with the usual hiccups that occur when
  installing something.
- Abbreviated remarks about the available options to interact with
  \(\mathsf{R}\).
- Basic operations (arithmetic, entering data, vectors) at the command
  prompt.
- How and where to find help when they get in trouble.
- Other little shortcuts I am usually asked when introducing
  \(\mathsf{R}\).

** Downloading and Installing \(\mathsf{R}\)
:PROPERTIES:
:CUSTOM_ID: sec-download-install-R
:END:  

The instructions for obtaining \(\mathsf{R}\) largely depend on the
user's hardware and operating system. The \(\mathsf{R}\) Project has
written an \(\mathsf{R}\) Installation and Administration manual with
complete, precise instructions about what to do, together with all
sorts of additional information. The following is just a primer to get
a person started.

*** Installing \(\mathsf{R}\) 

Visit one of the links below to download the latest version of \(\mathsf{R}\) 
for your operating system:

- Microsoft Windows: :: http://cran.r-project.org/bin/windows/base/
- MacOS: :: http://cran.r-project.org/bin/macosx/
- Linux: :: http://cran.r-project.org/bin/linux/

On Microsoft Windows, click the =R-x.y.z.exe= installer to start
installation. When it asks for "Customized startup options", specify
=Yes=. In the next window, be sure to select the SDI (single document
interface) option; this is useful later when we discuss three
dimensional plots with the =rgl= package \cite{rgl}.

**** Installing \(\mathsf{R}\) on a USB drive (Windows)

With this option you can use \(\mathsf{R}\) portably and without
administrative privileges. There is an entry in the \(\mathsf{R}\) for
Windows FAQ about this. Here is the procedure I use:
1. Download the Windows installer above and start installation as
   usual. When it asks /where/ to install, navigate to the top-level
   directory of the USB drive instead of the default =C= drive.
2. When it asks whether to modify the Windows registry, uncheck the
   box; we do NOT want to tamper with the registry.
3. After installation, change the name of the folder from =R-x.y.z= to
   just plain \(\mathsf{R}\). (Even quicker: do this in step 1.)
4. [[http://ipsur.r-forge.r-project.org/book/download/R.exe][Download this shortcut]] and move it to the top-level directory of
   the USB drive, right beside the \(\mathsf{R}\) folder, not inside
   the folder. Use the downloaded shortcut to run \(\mathsf{R}\).

Steps 3 and 4 are not required but save you the trouble of navigating
to the =R-x.y.z/bin= directory to double-click =Rgui.exe= every time
you want to run the program. It is useless to create your own shortcut
to =Rgui.exe=. Windows does not allow shortcuts to have relative
paths; they always have a drive letter associated with them. So if you
make your own shortcut and plug your USB drive into some /other/
machine that happens to assign your drive a different letter, then
your shortcut will no longer be pointing to the right place.

*** Installing and Loading Add-on Packages
:PROPERTIES:
:CUSTOM_ID: sub-installing-loading-packages
:END:

There are /base/ packages (which come with \(\mathsf{R}\)
automatically), and /contributed/ packages (which must be downloaded
for installation). For example, on the version of \(\mathsf{R}\) being
used for this document the default base packages loaded at startup are

#+BEGIN_SRC R :exports both :results output pp
getOption("defaultPackages")
#+END_SRC

#+RESULTS:
: [1] "datasets"  "utils"     "grDevices" "graphics"  "stats"     "methods"

The base packages are maintained by a select group of volunteers,
called \(\mathsf{R}\) Core. In addition to the base packages, there
are literally thousands of additional contributed packages written by
individuals all over the world. These are stored worldwide on mirrors
of the Comprehensive \(\mathsf{R}\) Archive Network, or =CRAN= for
short. Given an active Internet connection, anybody is free to
download and install these packages and even inspect the source code.

To install a package named =foo=, open up \(\mathsf{R}\) and type
=install.packages("foo")=
@@latex:\index{install.packages@\texttt{install.packages}}@@. To
install =foo= and additionally install all of the other packages on
which =foo= depends, instead type =install.packages("foo", depends =
TRUE)=.

The general command =install.packages()= will (on most operating
systems) open a window containing a huge list of available packages;
simply choose one or more to install.

No matter how many packages are installed onto the system, each one
must first be loaded for use with the
=library= @@latex:\index{library@\texttt{library}}@@ function. For instance, the
=foreign= package \cite{foreign} contains all sorts of functions
needed to import data sets into \(\mathsf{R}\) from other software
such as SPSS, SAS, /etc/. But none of those functions will be
available until the command =library("foreign")= is issued.

Type =library()= at the command prompt (described below) to see a list
of all available packages in your library.

For complete, precise information regarding installation of
\(\mathsf{R}\) and add-on packages, see the [[http://cran.r-project.org/manuals.html][R Installation and Administration manual]].

** Communicating with \(\mathsf{R}\)
:PROPERTIES:
:CUSTOM_ID: sec-Communicating-with-R
:END:

*** One line at a time

This is the most basic method and is the first one that beginners will use.
- RGui (Microsoft \(\circledR\) Windows)
- Terminal
- Emacs/ESS, XEmacs
- JGR

*** Multiple lines at a time

For longer programs (called /scripts/) there is too much code to write
all at once at the command prompt. Furthermore, for longer scripts it
is convenient to be able to only modify a certain piece of the script
and run it again in \(\mathsf{R}\). Programs called /script editors/
are specially designed to aid the communication and code writing
process. They have all sorts of helpful features including
\(\mathsf{R}\) syntax highlighting, automatic code completion,
delimiter matching, and dynamic help on the \(\mathsf{R}\) functions
as they are being written. Even more, they often have all of the text
editing features of programs like Microsoft\(\circledR\)Word. Lastly,
most script editors are fully customizable in the sense that the user
can customize the appearance of the interface to choose what colors to
display, when to display them, and how to display them.

- \(\mathsf{R}\) Editor (Windows): :: @@latex:\index{R
     Editor@\textsf{R} Editor}@@ In Microsoft\(\circledR\) Windows,
     \(\mathsf{R}\) Gui has its own built-in script editor, called
     \(\mathsf{R}\) Editor. From the console window, select =File=
     \(\triangleright\) =New Script=. A script window opens, and the
     lines of code can be written in the window. When satisfied with
     the code, the user highlights all of the commands and presses
     \textsf{Ctrl+R}. The commands are automatically run at once in
     \(\mathsf{R}\) and the output is shown. To save the script for
     later, click =File= \(\triangleright\) =Save as...= in
     \(\mathsf{R}\) Editor. The script can be reopened later with
     =File= \(\triangleright\)} =Open Script...= in =RGui=. Note that
     \(\mathsf{R}\) Editor does not have the fancy syntax highlighting
     that the others do.
- \(\mathsf{R}\) WinEdt: :: @@latex:\index{RWinEdt@\textsf{R}WinEdt}@@
     This option is coordinated with WinEdt for \LaTeX{} and has
     additional features such as code highlighting, remote sourcing,
     and a ton of other things. However, one first needs to download
     and install a shareware version of another program, WinEdt, which
     is only free for a while -- pop-up windows will eventually appear
     that ask for a registration code. \(\mathsf{R}\) WinEdt is
     nevertheless a very fine choice if you already own WinEdt or are
     planning to purchase it in the near future.
- Tinn \(\mathsf{R}\) / Sciviews K: :: @@latex:\index{Tinn R@Tinn
     \textsf{R}}\index{Sciviews K}@@ This one is completely free and
     has all of the above mentioned options and more. It is simple
     enough to use that the user can virtually begin working with it
     immediately after installation. But Tinn \(\mathsf{R}\) proper is
     only available for Microsoft\(\circledR\) Windows operating
     systems. If you are on MacOS or Linux, a comparable alternative
     is Sci-Views - Komodo Edit.
- Emacs/ESS: :: @@latex:\index{Emacs}\index{ESS}@@ Emacs is an all
                purpose text editor. It can do absolutely anything
                with respect to modifying, searching, editing, and
                manipulating, text. And if Emacs can't do it, then you
                can write a program that extends Emacs to do it. Once
                such extension is called =ESS=, which stands for
                /E/-macs /S/-peaks /S/-tatistics. With ESS a person
                can speak to \(\mathsf{R}\), do all of the tricks that
                the other script editors offer, and much, much,
                more. Please see the following for installation
                details, documentation, reference cards, and a whole
                lot more: http://ess.r-project.org.  /Fair warning/:
                if you want to try Emacs and if you grew up with
                Microsoft\(\circledR\) Windows or Macintosh, then you
                are going to need to relearn everything you thought
                you knew about computers your whole life. (Or, since
                Emacs is completely customizable, you can reconfigure
                Emacs to behave the way you want.) I have personally
                experienced this transformation and I will never go
                back.
- JGR (read "Jaguar"): :: @@latex:\index{JGR}@@ This one has the bells
     and whistles of =RGui= plus it is based on Java, so it works on
     multiple operating systems. It has its own script editor like
     \(\mathsf{R}\) Editor but with additional features such as syntax
     highlighting and code-completion. If you do not use
     Microsoft\(\circledR\) Windows (or even if you do) you definitely
     want to check out this one.
- Kate, Bluefish, /etc/ :: There are literally dozens of other text
     editors available, many of them free, and each has its own
     (dis)advantages. I only have mentioned the ones with which I have
     had substantial personal experience and have enjoyed at some
     point. Play around, and let me know what you find.

*** Graphical User Interfaces (GUIs)

By the word "GUI" I mean an interface in which the user communicates
with \(\mathsf{R}\) by way of points-and-clicks in a menu of some
sort. Again, there are many, many options and I only mention ones that
I have used and enjoyed. Some of the other more popular script editors
can be downloaded from the \(\mathsf{R}\)-Project website at
http://www.sciviews.org/_rgui/. On the left side of the screen
(under *Projects*) there are several choices available.

- \(\mathsf{R}\) Commander :: provides a point-and-click interface to
     many basic statistical tasks. It is called the "Commander"
     because every time one makes a selection from the menus, the code
     corresponding to the task is listed in the output window. One can
     take this code, copy-and-paste it to a text file, then re-run it
     again at a later time without the \(\mathsf{R}\) Commander's
     assistance. It is well suited for the introductory level. =Rcmdr=
     \cite{Rcmdr} also allows for user-contributed "Plugins" which are
     separate packages on =CRAN= that add extra functionality to the
     =Rcmdr= package. The plugins are typically named with the prefix
     =RcmdrPlugin= to make them easy to identify in the =CRAN= package
     list. One such plugin is the =RcmdrPlugin.IPSUR= package
     \cite{RcmdrPlugin.IPSUR} which accompanies this text.
- Poor Man's GUI :: @@latex:\index{Poor Man's GUI}@@ is an alternative
                    to the =Rcmdr= which is based on GTk instead of
                    Tcl/Tk. It has been a while since I used it but I
                    remember liking it very much when I did. One thing
                    that stood out was that the user could
                    drag-and-drop data sets for plots. See here for
                    more information:
                    http://wiener.math.csi.cuny.edu/pmg/.
- Rattle :: @@latex:\index{Rattle}@@ is a data mining toolkit which
            was designed to manage/analyze very large data sets, but
            it provides enough other general functionality to merit
            mention here. See \cite{rattle} for more information.
- Deducer :: @@latex:\index{Deducer}@@ is relatively new and shows
             promise from what I have seen, but I have not actually
             used it in the classroom yet.

** Basic \(\mathsf{R}\) Operations and Concepts
:PROPERTIES:
:CUSTOM_ID: sec-Basic-R-Operations
:END:

The \(\mathsf{R}\) developers have written an introductory document
entitled "An Introduction to \(\mathsf{R}\)". There is a sample
session included which shows what basic interaction with
\(\mathsf{R}\) looks like. I recommend that all new users of
\(\mathsf{R}\) read that document, but bear in mind that there are
concepts mentioned which will be unfamiliar to the beginner.

Below are some of the most basic operations that can be done with
\(\mathsf{R}\). Almost every book about \(\mathsf{R}\) begins with a
section like the one below; look around to see all sorts of things
that can be done at this most basic level.

*** Arithmetic
:PROPERTIES:
:CUSTOM_ID: sub-Arithmetic
:END:

#+BEGIN_SRC R :exports both :results output pp  
2 + 3       # add
4 * 5 / 6   # multiply and divide
7^8         # 7 to the 8th power
#+END_SRC

#+RESULTS:
: [1] 5
: [1] 3.333333
: [1] 5764801

Notice the comment character =#=. Anything typed
after a =#= symbol is ignored by \(\mathsf{R}\). We know that \(20/6\)
is a repeating decimal, but the above example shows only 7 digits. We
can change the number of digits displayed with
=options= @@latex:\index{options@\texttt{options}}@@:

#+BEGIN_SRC R :exports both :results output pp 
options(digits = 16)
10/3                 # see more digits
sqrt(2)              # square root
exp(1)               # Euler's constant, e
pi       
options(digits = 7)  # back to default
#+END_SRC

#+RESULTS:
: [1] 3.333333333333333
: [1] 1.414213562373095
: [1] 2.718281828459045
: [1] 3.141592653589793

Note that it is possible to set =digits=
@@latex:\index{digits@\texttt{digits}}@@ up to 22, but setting them
over 16 is not recommended (the extra significant digits are not
necessarily reliable). Above notice the =sqrt=
@@latex:\index{sqrt@\texttt{sqrt}}@@ function for square roots and the
=exp= @@latex:\index{exp@\texttt{exp}}@@ function for powers of
\(\mathrm{e}\), Euler's number.

*** Assignment, Object names, and Data types
:PROPERTIES:
:CUSTOM_ID: sub-Assignment-Object-names
:END:

It is often convenient to assign numbers and values to variables
(objects) to be used later. The proper way to assign values to a
variable is with the =<-= operator (with a space on either side). The
~=~ symbol works too, but it is recommended by the \(\mathsf{R}\)
masters to reserve ~=~ for specifying arguments to functions
(discussed later). In this book we will follow their advice and use
=<-= for assignment. Once a variable is assigned, its value can be
printed by simply entering the variable name by itself.

#+BEGIN_SRC R :exports both :results output pp 
x <- 7*41/pi   # don't see the calculated value
x              # take a look
#+END_SRC

#+RESULTS:
: [1] 91.35494

When choosing a variable name you can use letters, numbers, dots
"\texttt{.}", or underscore "\texttt{\_}" characters. You cannot
use mathematical operators, and a leading dot may not be followed by a
number. Examples of valid names are: =x=, =x1=, =y.value=, and
=!y_hat=. (More precisely, the set of allowable characters in object
names depends on one's particular system and locale; see An
Introduction to \(\mathsf{R}\) for more discussion on this.)

Objects can be of many /types/, /modes/, and /classes/. At this level,
it is not necessary to investigate all of the intricacies of the
respective types, but there are some with which you need to become
familiar:
- integer: :: the values \(0\), \(\pm1\), \(\pm2\), ...; these are
              represented exactly by \(\mathsf{R}\).
- double: :: real numbers (rational and irrational); these numbers are
             not represented exactly (save integers or fractions with
             a denominator that is a power of 2, see
             \cite{Venables2010}).
- character: :: elements that are wrapped with pairs of ="= or ';
- logical: :: includes =TRUE=, =FALSE=, and =NA= (which are reserved
              words); the =NA= @@latex:\index{NA@\texttt{NA}}@@ stands
              for "not available", /i.e./, a missing value.

You can determine an object's type with the =typeof=
@@latex:\index{typeof@\texttt{typeof}}@@ function. In addition to the above,
there is the =complex= @@latex:\index{complex@\texttt{complex}}
\index{as.complex@\texttt{as.complex}}@@ data type:

#+BEGIN_SRC R :exports both :results output pp 
sqrt(-1)              # isn't defined
sqrt(-1+0i)           # is defined
sqrt(as.complex(-1))  # same thing
(0 + 1i)^2            # should be -1
typeof((0 + 1i)^2)
#+END_SRC

#+RESULTS:
: [1] NaN
: [1] 0+1i
: [1] 0+1i
: [1] -1+0i
: [1] "complex"

Note that you can just type =(1i)^2= to get the same answer. The
=NaN= @@latex:\index{NaN@\texttt{NaN}}@@ stands for "not a number"; it is
represented internally as =double= @@latex:\index{double}@@.

*** Vectors
:PROPERTIES:
:CUSTOM_ID: sub-Vectors
:END:

All of this time we have been manipulating vectors of length 1. Now
let us move to vectors with multiple entries.

**** Entering data vectors

*The long way:* @@latex:\index{c@\texttt{c}}@@ If you would like to enter the
data =74,31,95,61,76,34,23,54,96= into \(\mathsf{R}\), you may create
a data vector with the =c= function (which is short for
/concatenate/).

#+BEGIN_SRC R :exports both :results output pp 
x <- c(74, 31, 95, 61, 76, 34, 23, 54, 96)
x
#+END_SRC

#+RESULTS:
: [1] 74 31 95 61 76 34 23 54 96

The elements of a vector are usually coerced by \(\mathsf{R}\) to the
the most general type of any of the elements, so if you do =c(1, "2")=
then the result will be =c("1", "2")=.

*A shorter way:* @@latex:\index{scan@\texttt{scan}}@@: The =scan= method is
useful when the data are stored somewhere else. For instance, you may
type =x <- scan()= at the command prompt and \(\mathsf{R}\) will
display =1:= to indicate that it is waiting for the first data
value. Type a value and press =Enter=, at which point \(\mathsf{R}\)
will display =2:=, and so forth. Note that entering an empty line
stops the scan. This method is especially handy when you have a column
of values, say, stored in a text file or spreadsheet. You may copy and
paste them all at the =1:= prompt, and \(\mathsf{R}\) will store all
of the values instantly in the vector =x=.

*Repeated data; regular patterns:* the =seq= @@latex:\index{seq@\texttt{seq}}@@
function will generate all sorts of sequences of numbers. It has the
arguments =from=, =to=, =by=, and =length.out= which can be set in
concert with one another. We will do a couple of examples to show you
how it works.

#+BEGIN_SRC R :exports both :results output pp 
seq(from = 1, to = 5)
seq(from = 2, by = -0.1, length.out = 4)
#+END_SRC

#+RESULTS:
: [1] 1 2 3 4 5
: [1] 2.0 1.9 1.8 1.7

Note that we can get the first line much quicker with the colon
operator.

#+BEGIN_SRC R :exports both :results output pp 
1:5
#+END_SRC

#+RESULTS:
: [1] 1 2 3 4 5

The vector =LETTERS= @@latex:\index{LETTERS@\texttt{LETTERS}}@@ has the 26
letters of the English alphabet in uppercase and
=letters= @@latex:\index{letters@\texttt{letters}}@@ has all of them in
lowercase.

**** Indexing data vectors

Sometimes we do not want the whole vector, but just a piece of it. We
can access the intermediate parts with the =[]= operator. Observe
(with =x= defined above)

#+BEGIN_SRC R :exports both :results output pp 
x[1]
x[2:4]
x[c(1,3,4,8)]
x[-c(1,3,4,8)]
#+END_SRC

#+RESULTS:
: [1] 74
: [1] 31 95 61
: [1] 74 95 61 54
: [1] 31 76 34 23 96

Notice that we used the minus sign to specify those elements that we
do /not/ want.

#+BEGIN_SRC R :exports both :results output pp 
LETTERS[1:5]
letters[-(6:24)]
#+END_SRC

#+RESULTS:
: [1] "A" "B" "C" "D" "E"
: [1] "a" "b" "c" "d" "e" "y" "z"

*** Functions and Expressions
:PROPERTIES:
:CUSTOM_ID: sub-Functions-and-Expressions
:END:

A function takes arguments as input and returns an object as
output. There are functions to do all sorts of things. We show some
examples below.

#+BEGIN_SRC R :exports both :results output pp 
x <- 1:5
sum(x)
length(x)
min(x)
mean(x)      # sample mean
sd(x)        # sample standard deviation
#+END_SRC

#+RESULTS:
: [1] 15
: [1] 5
: [1] 1
: [1] 3
: [1] 1.581139

It will not be long before the user starts to wonder how a particular
function is doing its job, and since \(\mathsf{R}\) is open-source,
anybody is free to look under the hood of a function to see how things
are calculated. For detailed instructions see the article "Accessing
the Sources" by Uwe Ligges \cite{Ligges2006}. In short:

*Type the name of the function* without any parentheses or
arguments. If you are lucky then the code for the entire function will
be printed, right there looking at you. For instance, suppose that we
would like to see how the =intersect=
@@latex:\index{intersect@\texttt{intersect}}@@ function works:

#+BEGIN_SRC R :exports both :results output pp 
intersect
#+END_SRC

#+RESULTS:
: function (x, ...) 
: UseMethod("intersect")
: <environment: namespace:prob>

*If instead* it shows =UseMethod(something)=
@@latex:\index{UseMethod@\texttt{UseMethod}}@@ then you will need to choose the
/class/ of the object to be inputted and next look at the /method/
that will be /dispatched/ to the object. For instance, typing =rev=
@@latex:\index{rev@\texttt{rev}}@@ says

#+BEGIN_SRC R :exports both :results output pp 
rev
#+END_SRC

#+RESULTS:
: function (x) 
: UseMethod("rev")
: <bytecode: 0x8bd24a8>
: <environment: namespace:base>

The output is telling us that there are multiple methods associated
with the =rev= function. To see what these are, type

#+BEGIN_SRC R :exports both :results output pp 
methods(rev)
#+END_SRC

#+RESULTS:
: [1] rev.default     rev.dendrogram* rev.likert*     rev.zoo         rev.zooreg*    
: 
:    Non-visible functions are asterisked

Now we learn that there are two different =rev(x)= functions, only one
of which being chosen at each call depending on what =x= is. There is
one for =dendrogram= objects and a =default= method for everything
else. Simply type the name to see what each method does. For example,
the =default= method can be viewed with

#+BEGIN_SRC R :exports both :results output pp 
rev.default
#+END_SRC

#+RESULTS:
: function (x) 
: if (length(x)) x[length(x):1L] else x
: <bytecode: 0xa32509c>
: <environment: namespace:base>

*Some functions are hidden* by a /namespace/ (see An Introduction to
\(\mathsf{R}\) \cite{Venables2010}), and are not visible on the first
try. For example, if we try to look at the code for =wilcox.test=
@@latex:\index{wilcox.test@\texttt{wilcox.test}}@@ (see Chapter [[#cha-Nonparametric-Statistics]]) we get the following:

#+BEGIN_SRC R :exports both :results output pp 
wilcox.test
methods(wilcox.test)
#+END_SRC

#+RESULTS:
: function (x, ...) 
: UseMethod("wilcox.test")
: <bytecode: 0x8957fd0>
: <environment: namespace:stats>
: [1] wilcox.test.default* wilcox.test.formula*
: 
:    Non-visible functions are asterisked

If we were to try =wilcox.test.default= we would get a "not found"
error, because it is hidden behind the namespace for the package
=stats= \cite{stats} (shown in the last line when we tried
=wilcox.test=). In cases like these we prefix the package name to the
front of the function name with three colons; the command
=stats:::wilcox.test.default= will show the source code, omitted here
for brevity.

*If it shows* =.Internal(something)=
@@latex:\index{.Internal@\texttt{.Internal}}@@ or =.Primitive(something)=
@@latex:\index{.Primitive@\texttt{.Primitive}}@@, then it will be necessary to
download the source code of \(\mathsf{R}\) (which is /not/ a binary
version with an =.exe= extension) and search inside the code
there. See Ligges \cite{Ligges2006} for more discussion on this. An
example is =exp=:

#+BEGIN_SRC R :exports both :results output pp 
exp
#+END_SRC

#+RESULTS:
: function (x)  .Primitive("exp")

Be warned that most of the =.Internal= functions are written in other
computer languages which the beginner may not understand, at least
initially.

** Getting Help
:PROPERTIES:
:CUSTOM_ID: sec-Getting-Help
:END:

When you are using \(\mathsf{R}\), it will not take long before you
find yourself needing help. Fortunately, \(\mathsf{R}\) has extensive
help resources and you should immediately become familiar with
them. Begin by clicking =Help= on =RGui=. The following options are
available.
- Console: :: gives useful shortcuts, for instance, =Ctrl+L=, to clear
              the \(\mathsf{R}\) console screen.
- FAQ on \(\mathsf{R}\): :: frequently asked questions concerning
     general \(\mathsf{R}\) operation.
- FAQ on \(\mathsf{R}\) for Windows: :: frequently asked questions
     about \(\mathsf{R}\), tailored to the Microsoft Windows operating
     system.
- Manuals: :: technical manuals about all features of the
              \(\mathsf{R}\) system including installation, the
              complete language definition, and add-on packages.
- \(\mathsf{R}\) functions (text)...: :: use this if you know the
     /exact/ name of the function you want to know more about, for
     example, =mean= or =plot=. Typing =mean= in the window is
     equivalent to typing =help("mean")=
     @@latex:\index{help@\texttt{help}}@@ at the command line, or more
     simply, =?mean= @@latex:\index{?@\texttt{?}}@@. Note that this
     method only works if the function of interest is contained in a
     package that is already loaded into the search path with
     =library=.
- HTML Help: :: use this to browse the manuals with point-and-click
                links. It also has a Search Engine \& Keywords for
                searching the help page titles, with point-and-click
                links for the search results. This is possibly the
                best help method for beginners. It can be started from
                the command line with the command
                =help.start()= @@latex:\index{help.start@\texttt{help.start}}@@.
- Search help ...: :: use this if you do not know the exact name of
     the function of interest, or if the function is in a package that
     has not been loaded yet. For example, you may enter =plo= and a
     text window will return listing all the help files with an alias,
     concept, or title matching `=plo=' using regular expression
     matching; it is equivalent to typing
     =help.search("plo")= @@latex:\index{help.search@\texttt{help.search}}@@ at
     the command line. The advantage is that you do not need to know
     the exact name of the function; the disadvantage is that you
     cannot point-and-click the results. Therefore, one may wish to
     use the HTML Help search engine instead. An equivalent way is
     =??plo= @@latex:\index{??@\texttt{??}}@@ at the command line.
- search.r-project.org ...: :: this will search for words in help
     lists and email archives of the \(\mathsf{R}\) Project. It can be
     very useful for finding other questions that other users have
     asked.
- Apropos ...: :: use this for more sophisticated partial name
                  matching of functions. See =?apropos=
                  @@latex:\index{apropos@\texttt{apropos}}@@ for details.

On the help pages for a function there are sometimes "Examples"
listed at the bottom of the page, which will work if copy-pasted at
the command line (unless marked otherwise). The =example=
@@latex:\index{example@\texttt{example}}@@ function will run the code
automatically, skipping the intermediate step. For instance, we may
try =example(mean)= to see a few examples of how the =mean= function
works.

*** \(\mathsf{R}\) Help Mailing Lists

There are several mailing lists associated with \(\mathsf{R}\), and
there is a huge community of people that read and answer questions
related to \(\mathsf{R}\). See [[http://www.r-project.org/mail.html][here]] for an idea of what is
available. Particularly pay attention to the bottom of the page which
lists several special interest groups (SIGs) related to
\(\mathsf{R}\).

Bear in mind that \(\mathsf{R}\) is free software, which means that it
was written by volunteers, and the people that frequent the mailing
lists are also volunteers who are not paid by customer support
fees. Consequently, if you want to use the mailing lists for free
advice then you must adhere to some basic etiquette, or else you may
not get a reply, or even worse, you may receive a reply which is a bit
less cordial than you are used to. Below are a few considerations:
1. Read the [[http://cran.r-project.org/faqs.html][FAQ]]. Note that there are different FAQs for different
   operating systems. You should read these now, even without a
   question at the moment, to learn a lot about the idiosyncrasies of
   \(\mathsf{R}\).
2. Search the archives. Even if your question is not a FAQ, there is a
   very high likelihood that your question has been asked before on
   the mailing list. If you want to know about topic =foo=, then you
   can do =RSiteSearch("foo")=
   @@latex:\index{RSiteSearch@\texttt{RSiteSearch}}@@ to search the
   mailing list archives (and the online help) for it.
3. Do a Google search and an \texttt{RSeek.org} search.

If your question is not a FAQ, has not been asked on
\(\mathsf{R}\)-help before, and does not yield to a Google (or
alternative) search, then, and only then, should you even consider
writing to \(\mathsf{R}\)-help. Below are a few additional
considerations.

- Read the [[http://www.r-project.org/posting-guide.html][posting guide]] before posting. This will save you a lot of
  trouble and pain.
- Get rid of the command prompts (=>=) from output. Readers of your
  message will take the text from your mail and copy-paste into an
  \(\mathsf{R}\) session. If you make the readers' job easier then it
  will increase the likelihood of a response.
- Questions are often related to a specific data set, and the best way
  to communicate the data is with a =dump= @@latex:\index{dump@\texttt{dump}}@@
  command. For instance, if your question involves data stored in a
  vector =x=, you can type =dump("x","")= at the command prompt and
  copy-paste the output into the body of your email message. Then the
  reader may easily copy-paste the message from your email into
  \(\mathsf{R}\) and =x= will be available to him/her.
- Sometimes the answer the question is related to the operating system
  used, the attached packages, or the exact version of \(\mathsf{R}\)
  being used. The =sessionInfo()=
  @@latex:\index{sessionInfo@\texttt{sessionInfo}}@@ command collects
  all of this information to be copy-pasted into an email (and the
  Posting Guide requests this information). See Appendix
  [[#cha-R-Session-Information]] for an example.

** External Resources
:PROPERTIES:
:CUSTOM_ID: sec-External-Resources
:END:

There is a mountain of information on the Internet about
\(\mathsf{R}\). Below are a few of the important ones.
- The \(\mathsf{R}\)- Project for Statistical Computing @@latex:\index{The
  R-Project@The \textsf{R}-Project}@@: Go [[http://www.r-project.org/][there]] first.
- The Comprehensive \(\mathsf{R}\) Archive Network
  @@latex:\index{CRAN}@@: [[http://cran.r-project.org/][That is
  where]] \(\mathsf{R}\) is stored along with thousands of contributed
  packages. There are also loads of contributed information (books,
  tutorials, /etc/.). There are mirrors all over the world with
  duplicate information.
- \(\mathsf{R}\)-Forge @@latex:\index{R-Forge@\textsf{R}-Forge}@@: [[http://r-forge.r-project.org/][This is
  another location]] where \(\mathsf{R}\) packages are stored. Here you
  can find development code which has not yet been released to =CRAN=.
- \(\mathsf{R}\)-Wiki @@latex:\index{R-Wiki@\textsf{R}-Wiki}@@: There are many
  tips, tricks, and general advice [[http://wiki.r-project.org/rwiki/doku.php][listed here]]. If you find a trick of
  your own, login and share it with the world.
- Other: the [[http://addictedtor.free.fr/graphiques/][\(\mathsf{R}\) Graph Gallery]] @@latex:\index{R Graph
  Gallery@\textsf{R} Graph Gallery}@@ and [[http://bm2.genes.nig.ac.jp/RGM2/index.php][\(\mathsf{R}\) Graphical
  Manual]] @@latex:\index{R Graphical Manual@\textsf{R} Graphical Manual}@@ have
  literally thousands of graphs to peruse. [[http://www.rseek.org][\(\mathsf{R}\) Seek]] is a
  search engine based on Google specifically tailored for
  \(\mathsf{R}\) queries.

** Other Tips

It is unnecessary to retype commands repeatedly, since \(\mathsf{R}\)
remembers what you have recently entered on the command line. On the
Microsoft\(\circledR\) Windows \(\mathsf{R}\) Gui, to cycle through
the previous commands just push the \(\uparrow\) (up arrow) key. On
Emacs/ESS the command is =M-p= (which means hold down the =Alt= button
and press "p"). More generally, the command =history()=
@@latex:\index{history@\texttt{history}}@@ will show a whole list of recently
entered commands.
- To find out what all variables are in the current work environment,
  use the commands =objects()= @@latex:\index{objects@\texttt{objects}}@@ or
  =ls()= @@latex:\index{ls@\texttt{ls}}@@. These list all available objects in
  the workspace. If you wish to remove one or more variables, use
  =remove(var1, var2, var3)= @@latex:\index{remove@\texttt{remove}}@@, or more
  simply use =rm(var1, var2, var3)=, and to remove all objects use
  =rm(list = ls())=.
- Another use of =scan= is when you have a long list of numbers
  (separated by spaces or on different lines) already typed somewhere
  else, say in a text file To enter all the data in one fell swoop,
  first highlight and copy the list of numbers to the Clipboard with
  =Edit= \(\triangleright\) =Copy= (or by right-clicking and selecting
  =Copy=). Next type the =x <- scan()= command in the \(\mathsf{R}\)
  console, and paste the numbers at the =1:= prompt with =Edit=
  \(\triangleright\) =Paste=. All of the numbers will automatically be
  entered into the vector =x=.
- The command =Ctrl+l= clears the display in the
  Microsoft\(\circledR\) Windows \(\mathsf{R}\) Gui. In Emacs/ESS,
  press =Ctrl+l= repeatedly to cycle point (the place where the cursor
  is) to the bottom, middle, and top of the display.
- Once you use \(\mathsf{R}\) for awhile there may be some commands
  that you wish to run automatically whenever \(\mathsf{R}\)
  starts. These commands may be saved in a file called =Rprofile.site=
  @@latex:\index{Rprofile.site@\texttt{Rprofile.site}}@@ which is
  usually in the =etc= folder, which lives in the \(\mathsf{R}\) home
  directory (which on Microsoft\(\circledR\) Windows usually is
  =C:\Program Files\R=). Alternatively, you can make a file
  =.Rprofile= @@latex:\index{.Rprofile@\texttt{.Rprofile}}@@ to be
  stored in the user's home directory, or anywhere \(\mathsf{R}\) is
  invoked. This allows for multiple configurations for different
  projects or users. See "Customizing the Environment" of /An
  Introduction to R/ for more details.
- When exiting \(\mathsf{R}\) the user is given the option to "save
  the workspace". I recommend that beginners DO NOT save the
  workspace when quitting. If =Yes= is selected, then all of the
  objects and data currently in \(\mathsf{R}\)'s memory is saved in a
  file located in the working directory called
  =.RData= @@latex:\index{.RData@\texttt{.RData}}@@. This file is then
  automatically loaded the next time \(\mathsf{R}\) starts (in which
  case \(\mathsf{R}\) will say =[previously saved workspace
  restored]=). This is a valuable feature for experienced users of
  \(\mathsf{R}\), but I find that it causes more trouble than it saves
  with beginners.

#+LaTeX: \newpage{}

** Exercises
#+LaTeX: \setcounter{thm}{0}
