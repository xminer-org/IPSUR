* Hypothesis Testing                                                 :hypoth:
:PROPERTIES:
:tangle: R/10-hypoth.R
:CUSTOM_ID: cha-Hypothesis-Testing
:END:

#+BEGIN_SRC R :exports none :eval never
#    IPSUR: Introduction to Probability and Statistics Using R
#    Copyright (C) 2014  G. Jay Kerns
#
#    Chapter: Hypothesis Testing
#
#    This file is part of IPSUR.
#
#    IPSUR is free software: you can redistribute it and/or modify
#    it under the terms of the GNU General Public License as published by
#    the Free Software Foundation, either version 3 of the License, or
#    (at your option) any later version.
#
#    IPSUR is distributed in the hope that it will be useful,
#    but WITHOUT ANY WARRANTY; without even the implied warranty of
#    MERCHANTABILITY or FITNESS FOR A PARTICULAR PURPOSE.  See the
#    GNU General Public License for more details.
#
#    You should have received a copy of the GNU General Public License
#    along with IPSUR.  If not, see <http://www.gnu.org/licenses/>.
#+END_SRC

#+BEGIN_SRC R :exports none :eval no-export
# This chapter's package dependencies
library(TeachingDemos)
library(HH)
#+END_SRC

*What do I want them to know?*
- basic terminology and philosophy of the Neyman-Pearson paradigm
- classical hypothesis tests for the standard one and two sample
  problems with means, variances, and proportions
- the notion of between versus within group variation and how it plays
  out with one-way ANOVA
- the concept of statistical power and its relation to sample size

** Introduction
:PROPERTIES:
:CUSTOM_ID: sec-Introduction-Hypothesis
:END:

I spent a week during the summer of 2005 at the University of Nebraska
at Lincoln grading Advanced Placement Statistics exams, and while I
was there I attended a presentation by Dr. Roxy Peck. At the end of
her talk she described an activity she had used with students to
introduce the basic concepts of hypothesis testing. I was impressed by
the activity and have used it in my own classes several times since.

#+BEGIN_quote
The instructor (with a box of cookies in hand) enters a class of
fifteen or more students and produces a brand-new, sealed deck of
ordinary playing cards. The instructor asks for a student volunteer to
break the seal, and then the instructor prominently shuffles the
deck[fn:fn-jokers] several times in front of the class, after which
time the students are asked to line up in a row. They are going to
play a game. Each student will draw a card from the top of the deck,
in turn. If the card is black, then the lucky student will get a
cookie. If the card is red, then the unlucky student will sit down
empty-handed. Let the game begin.

[fn:fn-jokers] The jokers are removed before shuffling.

The first student draws a card: red. There are jeers and outbursts,
and the student slinks off to his/her chair. (S)he is disappointed, of
course, but not really. After all, (s)he had a 50-50 chance of getting
black, and it did not happen. Oh well.

The second student draws a card: red, again. There are more jeers, and
the second student slips away. This student is also disappointed, but
again, not so much, because it is probably his/her unlucky day. On to
the next student.

The student draws: red again! There are a few wiseguys who yell (happy
to make noise, more than anything else), but there are a few other
students who are not yelling any more -- they are thinking. This is
the third red in a row, which is possible, of course, but what is
going on, here? They are not quite sure. They are now concentrating on
the next card... it is bound to be black, right?

The fourth student draws: red. Hmmm... now there are groans instead of
outbursts. A few of the students at the end of the line shrug their
shoulders and start to make their way back to their desk, complaining
that the teacher does not want to give away any cookies. There are
still some students in line though, salivating, waiting for the
inevitable black to appear.

The fifth student draws red. Now it isn't funny any more. As the
remaining students make their way back to their seats an uproar
ensues, from an entire classroom demanding cookies.
#+END_quote

Keep the preceding experiment in the back of your mind as you read the
following sections. When you have finished the entire chapter, come
back and read this introduction again. All of the mathematical jargon
that follows is connected to the above paragraphs. In the meantime, I
will get you started:

- Null hypothesis: :: it is an ordinary deck of playing cards,
     shuffled thoroughly.
- Alternative hypothesis: :: either it is a trick deck of cards, or
     the instructor did some fancy shufflework.
- Observed data: :: a sequence of draws from the deck, five reds in a row.

If it were truly an ordinary, well-shuffled deck of cards, the
probability of observing zero blacks out of a sample of size five
(without replacement) from a deck with 26 black cards and 26 red cards
would be

#+BEGIN_SRC R :exports both :results output pp 
dhyper(0, m = 26, n = 26, k = 5)
#+END_SRC

#+RESULTS:
: [1] 0.02531012

There are two very important final thoughts. First, everybody gets a
cookie in the end. Second, the students invariably (and aggressively)
attempt to get me to open up the deck and reveal the true nature of
the cards. I never do.

** Tests for Proportions
:PROPERTIES:
:CUSTOM_ID: sec-Tests-for-Proportions
:END:

# +BEGIN_exampletoo
<<exa-widget-machine>>
We have a machine that makes widgets. 
- Under normal operation, about 0.10 of the widgets produced are
  defective.
- Go out and purchase a torque converter.
- Install the torque converter, and observe \(n=100\) widgets from the
  machine.
- Let \(Y=\mbox{number of defective widgets observed}\).

If
- \(Y=0\), then the torque converter is great!
- \(Y=4\), then the torque converter seems to be helping. 
- \(Y=9\), then there is not much evidence that the torque converter helps.
- \(Y=17\), then throw away the torque converter.

Let \(p\) denote the proportion of defectives produced by the
machine. Before the installation of the torque converter \(p\) was
\(0.10\). Then we installed the torque converter. Did \(p\) change?
Did it go up or down? We use statistics to decide. Our method is to
observe data and construct a 95% confidence interval for \(p\),
\begin{equation}
\hat{p} \pm z_{\alpha/2}\sqrt{\frac{\hat{p}(1 - \hat{p})}{n}}.
\end{equation}
If the confidence interval is 
- \([0.01,\,0.05]\), then we are 95% confident that \(0.01\leq
  p \leq 0.05\), so there is evidence that the torque converter is
  helping.
- \([0.15,\,0.19]\), then we are 95% confident that \(0.15\leq
  p \leq 0.19\), so there is evidence that the torque converter is
  hurting.
- \([0.07,\,0.11]\), then there is not enough evidence to conclude
  that the torque converter is doing anything at all, positive or
  negative.

# +END_exampletoo

*** Terminology

The /null hypothesis/ \(H_{0}\) is a "nothing" hypothesis, whose
interpretation could be that nothing has changed, there is no
difference, there is nothing special taking place, /etc/. In Example
[[exa-widget-machine]] the null hypothesis would be \(H_{0}:\, p = 0.10.\)
The /alternative hypothesis/ \(H_{1}\) is the hypothesis that
something has changed, in this case, \(H_{1}:\, p \neq 0.10\). Our
goal is to statistically /test/ the hypothesis \(H_{0}:\, p = 0.10\)
versus the alternative \(H_{1}:\, p \neq 0.10\). Our procedure will
be:
1. Go out and collect some data, in particular, a simple random sample
   of observations from the machine.
2. Suppose that \(H_{0}\) is true and construct a \(100(1-\alpha)\%\)
   confidence interval for \(p\).
3. If the confidence interval does not cover \(p = 0.10\), then we
   /reject/ \(H_{0}\). Otherwise, we /fail to reject/ \(H_{0}\).

#+BEGIN_rem
Every time we make a decision it is possible to be wrong, and there
are two possible mistakes we can make. We have committed a
- Type I Error :: if we reject \(H_{0}\) when in fact \(H_{0}\) is
                  true. This would be akin to convicting an innocent
                  person for a crime (s)he did not commit.
- Type II Error :: if we fail to reject \(H_{0}\) when in fact
                   \(H_{1}\) is true. This is analogous to a guilty
                   person escaping conviction.
#+END_rem

Type I Errors are usually considered worse[fn:fn-typetwo], and we
design our statistical procedures to control the probability of making
such a mistake. We define the
\begin{equation}
\mbox{significance level of the test} = \mathbb{P}(\mbox{Type I Error}) = \alpha.
\end{equation}
We want \(\alpha\) to be small which conventionally means, say,
\(\alpha=0.05\), \(\alpha=0.01\), or \(\alpha=0.005\) (but could mean
anything, in principle).
- The /rejection region/ (also known as the /critical region/) for the
  test is the set of sample values which would result in the rejection
  of \(H_{0}\). For Example [[exa-widget-machine]], the rejection region
  would be all possible samples that result in a 95% confidence
  interval that does not cover \(p = 0.10\).
- The above example with \(H_{1}:\,p \neq 0.10\) is called a
  /two-sided/ test. Many times we are interested in a /one-sided/
  test, which would look like \(H_{1}:\,p < 0.10\) or \(H_{1}:\,p >
  0.10\).

[fn:fn-typetwo] There is no mathematical difference between the
errors, however. The bottom line is that we choose one type of error
to control with an iron fist, and we try to minimize the probability
of the other type. That being said, null hypotheses are often by
design to correspond to the "simpler" model, so it is often easier to
analyze (and thereby control) the probabilities associated with Type I
Errors.

We are ready for tests of hypotheses for one proportion.  We know from
Section BLANK that when \(H_{0}:\,p = p_{0}\) is true and \(n\) is
large,
\begin{equation}
\hat{p} \sim \mathsf{norm}(\mathtt{mean} = p_{0}, \mathtt{sd} = \sqrt{p_{0}(1 - p_{0})/n}),
\end{equation}
approximately, and the approximation gets better as the sample size gets bigger. Another way to write this is 
\begin{equation}
Z = \frac{\hat{p} - p_{0}}{sqrt{p_{0}(1 - p_{0})/n}}  \sim \mathsf{norm}(\mathtt{mean} = 0, \mathtt{sd} = 1).
\end{equation}

#+NAME: tab-ztest-one-sample-prop
#+CAPTION[Hypothesis tests, population proportion, large sample]: Hypothesis tests, population proportion, large sample.
| \(H_{0}\)         | \(H_{a}\)            | Rejection Region                  |
|-------------------+----------------------+-----------------------------------|
| \(p = p_{0}\)     | \(p > p_{0}\)        | \(z > z_{\alpha}\)                |
| \(p = p_{0}\)     | \(p < p_{0}\)        | \(z < -z_{\alpha}\)               |
| \(p = p_{0}\)     | \(p \neq p_{0}\)     | \( \vert z \vert > z_{\alpha/2}\) |

*Assumptions for a valid test:*
- A simple random sample from a Bernoulli population
- The sample size \(n\) is large
  - need at least 15 successes and at least 15 failures

# +BEGIN_exampletoo
Find
1. The null and alternative hypotheses.
2. Check your assumptions.
3. Define a critical region with an \(\alpha=0.05\) significance
   level.
4. Calculate the value of the test statistic and state your
   conclusion.
# +END_exampletoo


# +BEGIN_exampletoo
<<exa-prop-test-pvalue-A>> Suppose \(p = \mathrm{the\ proportion\ of\
students}\) who are admitted to the graduate school of the University
of California at Berkeley, and suppose that a public relations officer
boasts that UCB has historically had a 40% acceptance rate for its
graduate school. Consider the data stored in the table =UCBAdmissions=
from 1973. Assuming these observations constituted a simple random
sample, are they consistent with the officer's claim, or do they
provide evidence that the acceptance rate was significantly less than
40%? Use an \(\alpha = 0.01\) significance level.

Our null hypothesis in this problem is \(H_{0}:\,p = 0.4\) and the
alternative hypothesis is \(H_{1}:\,p < 0.4\). We reject the null
hypothesis if \(\hat{p}\) is too small, that is, if
\begin{equation}
\frac{\hat{p} - 0.4}{\sqrt{0.4(1 - 0.4)/n}} < -z_{\alpha},
\end{equation}
where \(\alpha = 0.01\) and \(-z_{0.01}\) is 
#+BEGIN_SRC R :exports both :results output pp 
-qnorm(0.99)
#+END_SRC

#+RESULTS:
: [1] -2.326348

Our only remaining task is to find the value of the test statistic and
see where it falls relative to the critical value. We can find the
number of people admitted and not admitted to the UCB graduate school
with the following.

#+BEGIN_SRC R :exports both :results output pp 
A <- as.data.frame(UCBAdmissions)
head(A)
xtabs(Freq ~ Admit, data = A)
#+END_SRC

#+RESULTS:
#+BEGIN_example
     Admit Gender Dept Freq
1 Admitted   Male    A  512
2 Rejected   Male    A  313
3 Admitted Female    A   89
4 Rejected Female    A   19
5 Admitted   Male    B  353
6 Rejected   Male    B  207
Admit
Admitted Rejected 
    1755     2771
#+END_example

Now we calculate the value of the test statistic.

#+BEGIN_SRC R :exports both :results output pp 
phat <- 1755/(1755 + 2771)
(phat - 0.4)/sqrt(0.4 * 0.6/(1755 + 2771)) 
#+END_SRC

#+RESULTS:
: [1] -1.680919

Our test statistic is not less than \(-2.32\), so it does not fall
into the critical region. Therefore, we /fail/ to reject the null
hypothesis that the true proportion of students admitted to graduate
school is less than 40% and say that the observed data are consistent
with the officer's claim at the \(\alpha = 0.01\) significance level.

# +END_exampletoo


# +BEGIN_exampletoo
<<exa-prop-test-pvalue-B>> We are going to do Example
[[exa-prop-test-pvalue-A]] all over again. Everything will be exactly the
same except for one change. Suppose we choose significance level
\(\alpha = 0.05\) instead of \(\alpha = 0.01\). Are the 1973 data
consistent with the officer's claim?

Our null and alternative hypotheses are the same. Our observed test
statistic is the same: it was approximately \(-1.68\). But notice that
our critical value has changed: \(\alpha = 0.05\) and \(-z_{0.05}\) is
# +END_exampletoo

#+BEGIN_SRC R :exports both :results output pp 
-qnorm(0.95)
#+END_SRC

#+RESULTS:
: [1] -1.644854

Our test statistic is less than \(-1.64\) so it now falls into the
critical region! We now /reject/ the null hypothesis and conclude that
the 1973 data provide evidence that the true proportion of students
admitted to the graduate school of UCB in 1973 was significantly less
than 40%. The data are /not/ consistent with the officer's claim at
the \(\alpha = 0.05\) significance level.

What is going on, here? If we choose \(\alpha = 0.05\) then we reject
the null hypothesis, but if we choose \(\alpha = 0.01\) then we fail
to reject the null hypothesis. Our final conclusion seems to depend on
our selection of the significance level. This is bad; for a particular
test, we never know whether our conclusion would have been different
if we had chosen a different significance level.

Or do we?

Clearly, for some significance levels we reject, and for some
significance levels we do not. Where is the boundary? That is, what is
the significance level for which we would /reject/ at any significance
level /bigger/, and we would /fail to reject/ at any significance
level /smaller/? This boundary value has a special name: it is called
the /p-value/ of the test.

#+BEGIN_defn
The /p-value/, or /observed significance level/, of a hypothesis test
is the probability when the null hypothesis is true of obtaining the
observed value of the test statistic (such as \(\hat{p}\)) or values
more extreme -- meaning, in the direction of the alternative
hypothesis[fn:fn-pvalue].
#+END_defn

[fn:fn-pvalue] Bickel and Doksum \cite{Bickel2001} state the
definition particularly well: the \(p\)-value is "the smallest level
of significance \(\alpha\) at which an experimenter using the test
statistic \(T\) would reject \(H_{0}\) on the basis of the observed
sample outcome \(x\)".

# +BEGIN_exampletoo

Calculate the \(p\)-value for the test in Examples
[[exa-prop-test-pvalue-A]] and [[exa-prop-test-pvalue-B]].

The \(p\)-value for this test is the probability of obtaining a
\(z\)-score equal to our observed test statistic (which had
\(z\)-score \(\approx-1.680919\)) or more extreme, which in this
example is less than the observed test statistic. In other words, we
want to know the area under a standard normal curve on the interval
\((-\infty,\,-1.680919]\). We can get this easily with
# +END_exampletoo


#+BEGIN_SRC R :exports both :results output pp 
pnorm(-1.680919)
#+END_SRC

#+RESULTS:
: [1] 0.04638932

We see that the \(p\)-value is strictly between the significance
levels \(\alpha = 0.01\) and \(\alpha = 0.05\). This makes sense: it
has to be bigger than \(\alpha = 0.01\) (otherwise we would have
rejected \(H_{0}\) in Example [[exa-prop-test-pvalue-A]]) and it must also
be smaller than \(\alpha = 0.05\) (otherwise we would not have
rejected \(H_{0}\) in Example [[exa-prop-test-pvalue-B]]). Indeed,
\(p\)-values are a characteristic indicator of whether or not we would
have rejected at assorted significance levels, and for this reason a
statistician will often skip the calculation of critical regions and
critical values entirely. If (s)he knows the \(p\)-value, then (s)he
knows immediately whether or not (s)he would have rejected at /any/
given significance level.

Thus, another way to phrase our significance test procedure is: we
will reject \(H_{0}\) at the \(\alpha\)-level of significance if the
\(p\)-value is less than \(\alpha\).

#+BEGIN_rem
If we have two populations with proportions \(p_{1}\) and \(p_{2}\)
then we can test the null hypothesis \(H_{0}:\,p_{1} = p_{2}\). In
that which follows,
- we observe independent simple random samples of size \(n_{1}\) and
  \(n_{2}\) from Bernoulli populations with respective probabilities
  of success \(p_{1}\) and \(p_{2}\),
- we write \(y_{1}\) and \(y_{2}\) for the respective numbers of
  successes from each of the two groups,
- we estimate \(p_{1}\) and \(p_{2}\) with \(\hat{p}_{1} = y_{1}/n_{1}
  \) and \(\hat{p}_{2} = y_{2}/n_{2}\), while we estimate the pooled
  probability \(\hat{p}\) with \((y_{1} + y_{2})/(n_{1} + n_{2}\), and
- finally, \[z = \frac{\hat{p}_{1} - \hat{p}_{2}}{\sqrt{\hat{p}(1 - \hat{p})\left( \frac{1}{n_{1}} + \frac{1}{n_{2}} \right)}}. \]
#+END_rem


#+NAME: tab-ztest-two-sample-prop
#+CAPTION[Hypothesis tests, difference in population proportions, large sample]: Hypothesis tests, difference in population proportions, large sample.
| \(H_{0}\)         | \(H_{a}\)                | Rejection Region                  |
|-------------------+--------------------------+-----------------------------------|
| \(p_{1} = p_{2}\) | \(p_{1} - p_{2} > 0\)    | \(z > z_{\alpha}\)                |
| \(p_{1} = p_{2}\) | \(p_{1} - p_{2} < 0\)    | \(z < -z_{\alpha}\)               |
| \(p_{1} = p_{2}\) | \(p_{1} - p_{2} \neq 0\) | \( \vert z \vert > z_{\alpha/2}\) |


Example.



**** How to do it with \(\mathsf{R}\)

The following does the test.

#+BEGIN_SRC R :exports both :results output pp 
prop.test(1755, 1755 + 2771, p = 0.4, alternative = "less", 
          conf.level = 0.99, correct = FALSE)
#+END_SRC

#+RESULTS:
#+BEGIN_example
 
	1-sample proportions test without continuity correction

data:  1755 out of 1755 + 2771, null probability 0.4
X-squared = 2.8255, df = 1, p-value = 0.04639
alternative hypothesis: true p is less than 0.4
99 percent confidence interval:
 0.0000000 0.4047326
sample estimates:
        p 
0.3877596
#+END_example

Do the following to make the plot.


*Remark*: In the above we set =correct = FALSE= to tell the computer
that we did not want to use Yates' continuity correction (reference
BLANK).  It is customary to use Yates when the expected frequency of
successes is less than 10. (reference BLANK) You can use it all of the
time, but you will have a decrease in power. For large samples the
correction does not matter.

**** With the \(\mathsf{R}\) Commander

If you already know the number of successes and failures, then you can
use the menu =Statistics= \(\triangleright\) =Proportions=
\(\triangleright\) =IPSUR Enter table for single sample...=

Otherwise, your data -- the raw successes and failures -- should be in
a column of the Active Data Set. Furthermore, the data must be stored
as a "factor" internally. If the data are not a factor but are numeric
then you can use the menu =Data= \(\triangleright\) =Manage variables
in active data set= \(\triangleright\) =Convert numeric variables to
factors...= to convert the variable to a factor. Or, you can always
use the =factor= function.

Once your unsummarized data is a column, then you can use the menu
=Statistics= \(\triangleright\) =Proportions= \(\triangleright\)
=Single-sample proportion test...=

** One Sample Tests for Means and Variances
:PROPERTIES:
:CUSTOM_ID: sec-One-Sample-Tests
:END:

*** For Means

Here, \(X_{1}\), \(X_{2}\), ..., \(X_{n}\) are a \(SRS(n)\) from a
\(\mathsf{norm}(\mathtt{mean} = \mu,\,\mathtt{sd} = \sigma)\)
distribution. We would like to test \(H_{0}:\mu = \mu_{0}\).

- Case A: :: Suppose \(\sigma\) is known. Then under \(H_{0}\),
   \[
   Z = \frac{\overline{X} - \mu_{0}}{\sigma/\sqrt{n}} \sim \mathsf{norm}(\mathtt{mean} = 0,\,\mathtt{sd} = 1).
   \]
   
   #+NAME: tab-ztest-one-sample
   #+CAPTION[Hypothesis tests, population mean, large sample]: Hypothesis tests, population mean, large sample.
   | \(H_{0}\)         | \(H_{a}\)            | Rejection Region                  |
   |-------------------+----------------------+-----------------------------------|
   | \(\mu = \mu_{0}\) | \(\mu > \mu_{0}\)    | \(z > z_{\alpha}\)                |
   | \(\mu = \mu_{0}\) | \(\mu < \mu_{0}\)    | \(z < -z_{\alpha}\)               |
   | \(\mu = \mu_{0}\) | \(\mu \neq \mu_{0}\) | \( \vert z \vert > z_{\alpha/2}\) |

- Case B: :: When \(\sigma\) is unknown, under \(H_{0}\),
   \[
   T = \frac{\overline{X} - \mu_{0}}{S/\sqrt{n}} \sim \mathsf{t}(\mathtt{df} = n - 1).
   \]
   #+NAME: tab-ttest-one-sample
   #+CAPTION[Hypothesis tests, population mean, small sample]: Hypothesis tests, population mean, small sample.
   | \(H_{0}\)         | \(H_{a}\)            | Rejection Region                        |
   |-------------------+----------------------+-----------------------------------------|
   | \(\mu = \mu_{0}\) | \(\mu > \mu_{0}\)    | \(t > t_{\alpha}(n - 1)\)               |
   | \(\mu = \mu_{0}\) | \(\mu < \mu_{0}\)    | \(t < -t_{\alpha}(n - 1\)               |
   | \(\mu = \mu_{0}\) | \(\mu \neq \mu_{0}\) | \( \vert t \vert > t_{\alpha/2}(n - 1\) |

#+BEGIN_rem
If \(\sigma\) is unknown but \(n\) is large then we can use the
\(z\)-test.
#+END_rem

# +BEGIN_exampletoo

In this example we
1. Find the null and alternative hypotheses.
2. Choose a test and find the critical region.
3. Calculate the value of the test statistic and state the conclusion.
4. Find the \(p\)-value.

# +END_exampletoo


#+BEGIN_rem
- Another name for a \(p\)-value is /tail-end probability/. We reject
  \(H_{0}\) when the \(p\)-value is small.
- The quantity \(\sigma/\sqrt{n}\), when \(\sigma\) is known, is
  called the /standard error of the sample mean/. In general, if we
  have an estimator \(\hat{\theta}\) then \(\sigma_{\hat{\theta}}\) is
  called the /standard error/ of \(\hat{\theta}\). We usually need to
  estimate \(\sigma_{\hat{\theta}}\) with
  \(\hat{\sigma_{\hat{\theta}}}\).
#+END_rem

**** How to do it with \(\mathsf{R}\)

I am thinking =z.test= @@latex:\index{z.test@\texttt{z.test}}@@ in
=TeachingDemos=, =t.test= @@latex:\index{t.test@\texttt{t.test}}@@ in base
\(\mathsf{R}\).

#+BEGIN_SRC R :exports both :results output pp 
x <- rnorm(37, mean = 2, sd = 3)
z.test(x, mu = 1, sd = 3, conf.level = 0.90)
#+END_SRC

#+RESULTS:
#+BEGIN_example

	One Sample z-test

data:  x
z = 1.2964, n = 37.000, Std. Dev. = 3.000, Std. Dev. of the sample mean = 0.493,
p-value = 0.1948
alternative hypothesis: true mean is not equal to 1
90 percent confidence interval:
 0.8281414 2.4506151
sample estimates:
mean of x 
 1.639378
#+END_example

The =RcmdrPlugin.IPSUR= package \cite{RcmdrPlugin.IPSUR} does not have
a menu for =z.test= yet.

#+BEGIN_SRC R :exports both :results output pp 
x <- rnorm(13, mean = 2, sd = 3)
t.test(x, mu = 0, conf.level = 0.90, alternative = "greater")
#+END_SRC

#+RESULTS:
#+BEGIN_example

	One Sample t-test

data:  x
t = 1.2904, df = 12, p-value = 0.1106
alternative hypothesis: true mean is greater than 0
90 percent confidence interval:
 -0.05472113         Inf
sample estimates:
mean of x 
 1.072399
#+END_example

**** With the \(\mathsf{R}\) Commander

Your data should be in a single numeric column (a variable) of the
Active Data Set. Use the menu =Statistics= \(\triangleright\) =Means=
\(\triangleright\) =Single-sample t-test...=

*** Tests for a Variance

Here, \(X_{1}\), \(X_{2}\), ..., \(X_{n}\) are a \(SRS(n)\) from a
\(\mathsf{norm}(\mathtt{mean} = \mu,\,\mathtt{sd} = \sigma)\)
distribution. We would like to test \(H_{0}:\,\sigma^{2} =
\sigma_{0}\). We know that under \(H_{0}\), \[ X^{2} = \frac{(n -
1)S^{2}}{\sigma^{2}} \sim \mathsf{chisq}(\mathtt{df} = n - 1).  \]
Table here.

# +BEGIN_exampletoo

Give some data and a hypothesis.
- Give an \(\alpha\)-level and test the critical region way.
- Find the \(p\)-value for the test.
# +END_exampletoo

**** How to do it with \(\mathsf{R}\)
I am thinking about =sigma.test=
@@latex:\index{sigma.test@\texttt{sigma.test}}@@ in the
=TeachingDemos= package \cite{TeachingDemos}.

#+BEGIN_SRC R :exports both :results output pp 
sigma.test(women$height, sigma = 8)
#+END_SRC

#+RESULTS:
#+BEGIN_example

	One sample Chi-squared test for variance

data:  women$height
X-squared = 4.375, df = 14, p-value = 0.01449
alternative hypothesis: true variance is not equal to 64
95 percent confidence interval:
 10.72019 49.74483
sample estimates:
var of women$height 
                 20
#+END_example

** Two-Sample Tests for Means and Variances
:PROPERTIES:
:CUSTOM_ID: sec-Two-Sample-Tests-for-Means
:END:

The basic idea for this section is the following. We have
\(X\sim\mathsf{norm}(\mathtt{mean} = \mu_{X},\,\mathtt{sd} =
\sigma_{X})\) and \(Y\sim\mathsf{norm}(\mathtt{mean} =
\mu_{Y},\,\mathtt{sd} = \sigma_{Y})\) distributed independently. We
would like to know whether \(X\) and \(Y\) come from the same
population distribution, in other words, we would like to know:
\begin{equation}
\mbox{Does }X\overset{\mathrm{d}}{=}Y?
\end{equation}
where the symbol \(\overset{\mathrm{d}}{=}\) means equality of
probability distributions.  Since both \(X\) and \(Y\) are normal, we
may rephrase the question:
\begin{equation}
\mbox{Does }\mu_{X} = \mu_{Y}\mbox{ and }\sigma_{X} = \sigma_{Y}?
\end{equation}
Suppose first that we do not know the values of \(\sigma_{X}\) and
\(\sigma_{Y}\), but we know that they are equal,
\(\sigma_{X}=\sigma_{Y}\). Our test would then simplify to
\(H_{0}:\,\mu_{X} = \mu_{Y}\). We collect data \(X_{1}\), \(X_{2}\),
..., \(X_{n}\) and \(Y_{1}\), \(Y_{2}\), ..., \(Y_{m}\), both simple
random samples of size \(n\) and \(m\) from their respective normal
distributions. Then under \(H_{0}\) (that is, assuming \(H_{0}\) is
true) we have \(\mu_{X} = \mu_{Y}\), or rewriting, \(\mu_{X} - \mu_{Y}
= 0\), so
\begin{equation}
T = \frac{\overline{X} - \overline{Y}}{S_{p}\sqrt{\frac{1}{n} + \frac{1}{m}}} = \frac{\overline{X} - \overline{Y} - (\mu_{X} - \mu_{Y})}{S_{p}\sqrt{\frac{1}{n} + \frac{1}{m}}}\sim\mathsf{t}(\mathtt{df} = n + m - 2).
\end{equation}

*** Independent Samples

#+BEGIN_rem
If the values of \(\sigma_{X}\) and \(\sigma_{Y}\) are known, then we
can plug them in to our statistic:
\begin{equation} 
Z = \frac{\overline{X} - \overline{Y}}{\sqrt{\sigma_{X}^{2}/n + \sigma_{Y}^{2}/m}};
\end{equation}
the result will have a \(\mathsf{norm}(\mathtt{mean} = 0,\,\mathtt{sd}
= 1)\) distribution when \(H_{0}:\,\mu_{X} = \mu_{Y}\) is true.
#+END_rem

#+BEGIN_rem
Even if the values of \(\sigma_{X}\) and \(\sigma_{Y}\) are not known,
if both \(n\) and \(m\) are large then we can plug in the sample
estimates and the result will have approximately a
\(\mathsf{norm}(\mathtt{mean} = 0,\,\mathtt{sd} = 1)\) distribution when
\(H_{0}:\,\mu_{X} = \mu_{Y}\) is true.
\begin{equation} 
Z = \frac{\overline{X} - \overline{Y}}{\sqrt{S_{X}^{2}/n + S_{Y}^{2}/m}}.
\end{equation}
#+END_rem

#+BEGIN_rem
It is often helpful to construct side-by-side boxplots and other
visual displays in concert with the hypothesis test. This gives a
visual comparison of the samples and helps to identify departures from
the test's assumptions -- such as outliers.
#+END_rem

#+BEGIN_rem
WATCH YOUR ASSUMPTIONS.
- The normality assumption can be relaxed as long as the population
  distributions are not highly skewed.
- The equal variance assumption can be relaxed as long as both sample
  sizes \(n\) and \(m\) are large. However, if one (or both) samples
  is small, then the test does not perform well; we should instead use
  the methods of Chapter [[#cha-resampling-methods]].
#+END_rem

For a nonparametric alternative to the two-sample \(F\) test see
Chapter [[#cha-Nonparametric-Statistics]].

#+NAME: tab-two-t-test-pooled
#+CAPTION[Rejection regions, difference in means, pooled t-test]: Rejection regions, difference in means, pooled t-test.
| \(H_{0}\)           | \(H_{a}\)              | Rejection Region                             |
|---------------------+------------------------+----------------------------------------------|
| \(\mu_{D} = D_{0}\) | \(\mu_{D} > D_{0}\)    | \(t > t_{\alpha}(n + m - 2)\)                |
| \(\mu_{D} = D_{0}\) | \(\mu_{D} < D_{0}\)    | \(t < -t_{\alpha}(n + m -  2)\)              |
| \(\mu_{D} = D_{0}\) | \(\mu_{D} \neq D_{0}\) | \( \vert t \vert > t_{\alpha/2}(n + m - 2)\) |


*** Paired Samples

#+NAME: tab-two-t-test-paired
#+CAPTION[Rejection regions, difference in means, pairled t-test]: Rejection regions, difference in means, pairled t-test.
| \(H_{0}\)           | \(H_{a}\)              | Rejection Region                             |
|---------------------+------------------------+----------------------------------------------|
| \(\mu_{D} = D_{0}\) | \(\mu_{D} > D_{0}\)    | \(t > t_{\alpha}(n_{D} - 1)\)                |
| \(\mu_{D} = D_{0}\) | \(\mu_{D} < D_{0}\)    | \(t < -t_{\alpha}(n_{D} - 2)\)               |
| \(\mu_{D} = D_{0}\) | \(\mu_{D} \neq D_{0}\) | \( \vert t \vert > t_{\alpha/2}(n_{D} - 1)\) |


**** How to do it with \(\mathsf{R}\)

#+BEGIN_SRC R :exports both :results output pp 
t.test(extra ~ group, data = sleep, paired = TRUE)
#+END_SRC

#+RESULTS:
#+BEGIN_example

	Paired t-test

data:  extra by group
t = -4.0621, df = 9, p-value = 0.002833
alternative hypothesis: true difference in means is not equal to 0
95 percent confidence interval:
 -2.4598858 -0.7001142
sample estimates:
mean of the differences 
                  -1.58
#+END_example

** Other Hypothesis Tests
:PROPERTIES:
:CUSTOM_ID: sec-Other-Hypothesis-Tests
:END:

*** Kolmogorov-Smirnov Goodness-of-Fit Test
:PROPERTIES:
:CUSTOM_ID: sub-Kolmogorov-Smirnov-Goodness-of-Fit-Test
:END:

**** How to do it with \(\mathsf{R}\)

#+BEGIN_SRC R :exports both :results output pp 
with(randu, ks.test(x, "punif"))
#+END_SRC

#+RESULTS:
: 
: 	One-sample Kolmogorov-Smirnov test
: 
: data:  x
: D = 0.0555, p-value = 0.1697
: alternative hypothesis: two-sided

*** Shapiro-Wilk Normality Test
:PROPERTIES:
:CUSTOM_ID: sub-Shapiro-Wilk-Normality-Test
:END:

Most of the small-sample procedures we have studied thus far require
that the target population be normally distributed.  In later chapters
we will be assuming that errors from a proposed model are normally
distributed, and one method we will use to assess the model's adequacy
will be to investigate the plausibility of that assumption. So, you
see, determining whether or not a random sample comes from a normal
distribution is an important problem for us.  We have already learned
graphical methods to address the question, but here we will study a
formal hypothesis test of the same.

The test statistic we will use is Wilk's \(W\), which looks like this:
\begin{equation}
W = \frac{\left(\sum_{i = 1}^{n} a_{i}x_{(i)} \right)^{2}}{\sum_{i = 1}^{n}(x_{i} - \overline{x})^{2}},
\end{equation}
where the \(x_{(i)}\)'s are the order statistics (see Section BLANK)
and the constants \(a_{i}\) form the components of a vector
\(\mathbf{a}_{1\times\mathrm{n}}\) defined by
\begin{equation}
\mathbf{a}=\frac{\mathbf{m}^{\mathrm{T}}\mathbf{V}^{-1}}{\sqrt{\mathbf{m}^{\mathrm{T}}\mathbf{V}^{-1}\mathbf{V}^{-1}\mathbf{m}}},
\end{equation}
where \(\mathbf{m}_{\mathrm{n}\times1}\) and \(\mathbf{V}_{\mathrm{n}
\times \mathrm{n}} \) are the mean and covariance matrix,
respectively, of the order statistics from an \(\mathsf{mvnorm}
\left(\mathtt{mean} = \mathbf{0},\,\mathtt{sigma} =
\mathbf{I}\right)\) distribution.  This test statistic \(W\) was
published in 1965 by (you guessed it): Shapiro and
Wilk\cite{Wilk1965}.  In contrast to most other test statistics we
know, we reject \(H_{0}\) if \(W\) is too /small/.

**** How to do it with \(\mathsf{R}\)

#+BEGIN_SRC R :exports both :results output pp 
with(women, shapiro.test(height))
#+END_SRC

#+RESULTS:
: 
: 	Shapiro-Wilk normality test
: 
: data:  height
: W = 0.9636, p-value = 0.7545

** Analysis of Variance
:PROPERTIES:
:CUSTOM_ID: sec-Analysis-of-Variance
:END:

*** How to do it with \(\mathsf{R}\)

I am thinking 
#+BEGIN_SRC R :exports both :results output pp 
with(chickwts, by(weight, feed, shapiro.test))
#+END_SRC

#+RESULTS:
#+BEGIN_example
feed: casein

	Shapiro-Wilk normality test

data:  dd[x, ]
W = 0.9166, p-value = 0.2592

--------------------------------------------------------------------- 
feed: horsebean

	Shapiro-Wilk normality test

data:  dd[x, ]
W = 0.9376, p-value = 0.5264

--------------------------------------------------------------------- 
feed: linseed

	Shapiro-Wilk normality test

data:  dd[x, ]
W = 0.9693, p-value = 0.9035

--------------------------------------------------------------------- 
feed: meatmeal

	Shapiro-Wilk normality test

data:  dd[x, ]
W = 0.9791, p-value = 0.9612

--------------------------------------------------------------------- 
feed: soybean

	Shapiro-Wilk normality test

data:  dd[x, ]
W = 0.9464, p-value = 0.5064

--------------------------------------------------------------------- 
feed: sunflower

	Shapiro-Wilk normality test

data:  dd[x, ]
W = 0.9281, p-value = 0.3603
#+END_example

and
#+BEGIN_SRC R :exports code :results silent 
temp <- lm(weight ~ feed, data = chickwts)
#+END_SRC
and 
#+BEGIN_SRC R :exports both :results output pp 
anova(temp)
#+END_SRC

#+RESULTS:
: Analysis of Variance Table
: 
: Response: weight
:           Df Sum Sq Mean Sq F value    Pr(>F)    
: feed       5 231129   46226  15.365 5.936e-10 ***
: Residuals 65 195556    3009                      
: ---
: Signif. codes:  0 '***' 0.001 '**' 0.01 '*' 0.05 '.' 0.1 ' ' 1

Plot for the intuition of between versus within group variation.

#+NAME: between-versus-within
#+BEGIN_SRC R :exports both :results graphics :file fig/hypoth-between-versus-within.ps
y1 <- rnorm(300, mean = c(2,8,22))
plot(y1, xlim = c(-1,25), ylim = c(0,0.45) , type = "n")
f <- function(x){dnorm(x, mean = 2)}
curve(f, from = -1, to = 5, add = TRUE, lwd = 2)
f <- function(x){dnorm(x, mean = 8)}
curve(f, from = 5, to = 11, add = TRUE, lwd = 2)
f <- function(x){dnorm(x, mean = 22)}
curve(f, from = 19, to = 25, add = TRUE, lwd = 2)
rug(y1)
#+END_SRC

#+NAME: fig-between-versus-within
#+CAPTION[Between group versus within group variation]: \small A plot of between group versus within group variation.
#+ATTR_LaTeX: :width 0.9\textwidth :placement [ht!]
#+RESULTS: between-versus-within
[[file:fig/hypoth-between-versus-within.ps]]


#+NAME: some-F-plots-HH
#+BEGIN_SRC R :exports both :results graphics :file fig/hypoth-some-F-plots-HH.ps
old.omd <- par(omd = c(.05,.88, .05,1))
F.setup(df1 = 5, df2 = 30)
F.curve(df1 = 5, df2 = 30, col='blue')
F.observed(3, df1 = 5, df2 = 30)
par(old.omd)
#+END_SRC

#+NAME: fig-some-F-plots-HH
#+CAPTION[Some \(F\) plots from the \texttt{HH} package]: \small Some \(F\) plots from the \texttt{HH} package.
#+ATTR_LaTeX: :width 0.9\textwidth :placement [ht!]
#+RESULTS: some-F-plots-HH
[[file:fig/hypoth-some-F-plots-HH.ps]]

** Sample Size and Power
:PROPERTIES:
:CUSTOM_ID: sec-Sample-Size-and-Power
:END:

The power function of a test for a parameter \(\theta\) is
\[
\beta(\theta)=\mathbb{P}_{\theta}(\mbox{Reject }H_{0}),\quad -\infty < \theta < \infty.
\]
Here are some properties of power functions:
1. \(\beta(\theta)\leq\alpha\) for any \(\theta\in\Theta_{0}\), and
   \(\beta(\theta_{0})=\alpha\). We interpret this by saying that no
   matter what value \(\theta\) takes inside the null parameter space,
   there is never more than a chance of \(\alpha\) of rejecting the
   null hypothesis. We have controlled the Type I error rate to be no
   greater than \(\alpha\).
2. \(\lim_{n\to\infty}\beta(\theta)=1\) for any fixed
   \(\theta\in\Theta_{1}\). In other words, as the sample size grows
   without bound we are able to detect a nonnull value of \(\theta\)
   with increasing accuracy, no matter how close it lies to the null
   parameter space. This may appear to be a good thing at first
   glance, but it often turns out to be a curse, because this means
   that our Type II error rate grows as the sample size increases.

*** How to do it with \(\mathsf{R}\)

I am thinking about =replicate= @@latex:\index{replicate@\texttt{replicate}}@@
here, and also =power.examp= @@latex:\index{power.examp@\texttt{power.examp}}@@
from the =TeachingDemos= package \cite{TeachingDemos}. There is an
even better plot in upcoming work from the =HH= package \cite{HH}.

#+NAME: power-examp
#+BEGIN_SRC R :exports both :results graphics :file fig/hypoth-power-examp.ps
power.examp()
#+END_SRC

#+NAME: fig-power-examp
#+CAPTION[Plot of significance level and power]: \small This graph was generated by the =power.examp= function from the =TeachingDemos= package. The plot corresponds to the hypothesis test \(H_{0}:\,\mu=\mu_{0}\) versus \(H_{1}:\,\mu=\mu_{1}\) (where \(\mu_{0}=0\) and \(\mu_{1}=1\), by default) based on a single observation \(X\sim\mathsf{norm}(\mathtt{mean}=\mu,\,\mathtt{sd}=\sigma)\). The top graph is of the \(H_{0}\) density while the bottom is of the \(H_{1}\) density. The significance level is set at \(\alpha=0.05\), the sample size is \(n=1\), and the standard deviation is \(\sigma=1\). The pink area is the significance level, and the critical value \(z_{0.05}\approx1.645\) is marked at the left boundary -- this defines the rejection region. When \(H_{0}\) is true, the probability of falling in the rejection region is exactly \(\alpha=0.05\). The same rejection region is marked on the bottom graph, and the probability of falling in it (when \(H_{1}\)  is true) is the blue area shown at the top of the display to be approximately \(0.26\). This probability represents the /power/ to detect a non-null mean value of \(\mu=1\). With the command the =run.power.examp()= at the command line the same plot opens, but in addition, there are sliders available that allow the user to interactively change the sample size \(n\), the standard deviation \(\sigma\), the true difference between the means \(\mu_{1}-\mu_{0}\), and the significance level \(\alpha\). By playing around the student can investigate the effect each of the aforementioned parameters has on the statistical power. Note that you need the =tkrplot= package \cite{tkrplot} for =run.power.examp=.
#+ATTR_LaTeX: :width 0.9\textwidth :placement [ht!]
#+RESULTS: power-examp
[[file:fig/hypoth-power-examp.ps]]

#+LaTeX: \newpage{}

** Exercises

#+LaTeX: \setcounter{thm}{0}
